
% ************************** Thesis Abstract *****************************
% Use `abstract' as an option in the document class to print only the titlepage and the abstract.
\begin{abstract}
The neoliberal consensus in British socioeconomic policy over the past fifty years has led to a state of ‘capitalist realism’, a prevailing sense that capitalism is the only viable political economic system and that it is impossible to imagine or construct an alternative. This has been intensified by the onset of austerity policies after the 2008 financial crisis, which introduced financial cuts that reduced social welfare provision and induced negative affective shifts. These changes were drastic for youth support services which saw a 70\% reduction in spend, much to the detriment of their service users. 

Capitalist realism has been under-explored as an analytic for our contemporary politics. This thesis explores capitalist realism in theory and practice through an exploration of the experiences of young people and frontline workers in the third sector after the advent of austerity. Conducting ethnography, participatory action research and speculative design work with three third sector organisations (referred to in composite as \textit{The Charity}), I present the experiences of young people perceived to be vulnerable and frontline youth and social workers, finding that they all feel anxious, confused, distrustful and powerless and express these differently. To describe how these experiences are created and sustained by capitalist realism, I develop the grounded theory of justification practices, classification practices, and discursive accumulation, and describe how programme evaluation leads to the production of vulnerability as a commodity.

The thesis turns towards understanding how design methods can be used to mitigate, resist, and construct alternatives to capitalist realism. I present two design responses to capitalist realism, entitled \textit{It’s Our Future} and \textit{fractured signals}, developing a methodology that enacts speculation through designed objects to create novel possibilities and tie these to an activist praxis. Analysing the impacts of these projects, I present the concept of speculative praxis to describe how design can be used to take experimental actions against capitalist realism and the foreclosure of the future it enacts. 
\end{abstract}
