\begin{publication}

The following people have contributed to the work presented in this thesis:
\begin{itemize}
    \item \textbf{Rebecca Nicholson and James Hodge}. Though the participatory film-making project we collaborated on is not discussed at length, it is mentioned in passing in several chapters. Rebecca  developed the \textit{Gig Academy} method that we adapted within the project. Both Rebecca and James acted as subject matter experts in the project, on audio and education, and film-making respectively. 
    \item \textbf{Daniel Parry}. Daniel Parry was an extensive collaborator throughout this project, primarily for the \textit{It's Our Future} and \textit{fractured signals} projects. Daniel contributed towards the development of the design ideas in both projects, and was the primary visual designer for both projects. Generally, I provided the direction for the visual designs whilst Daniel created them. Centrally, this includes all marketing materials for \textit{It's Our Future}, the cards used in \textit{It's Our Future}, the \textit{It's Our Future} manifesto, the cards used in \textit{fractured signals}, and the guide used in \textit{fractured signals}. All laser cut materials were designed and created by me (the signalfinder and the divining board).
    \item \textbf{Sean Peacock}. Sean and I worked closely during our PhDs and sat alongside each other for much of the process. Working in similar areas, many of our ideas informed each other's informally. Sean and I wrote a paper together \citep{cutting_making_2021}, and ideas from this paper feature within this thesis - centrally, within chapter \ref{ch:3}'s discussions of ethical practice, chapter \ref{ch:4}'s use of the concept of slippages, and chapter \ref{ch:6}'s exploration of classification systems. The work presented in this thesis is substantively different from the published paper, but is significantly inspired by that work.  
    \item \textbf{Erkki Hedenborg}. Erkki and I worked closely during the early part of our PhD and shared a research interest around the use of Foucauldian biopolitics in the digital technology design space, which led to the publication of a paper \citep{cutting_can_2019}. The ideas contained within this paper do not directly reappear in this thesis, but the paper drew on research conducted with Small Steps and draws conclusions about biopolitics that are similar to this thesis. None of Erkki's writing features in any form in the thesis. 
    \item \textbf{The \textit{It's Our Future} team.} The \textit{It's Our Future} team consisted of Daniel Parry, Emily Barker, Sean Peacock, Mohaan Biswas, Velvet Spors, Sara Armouch, Daniel Lambton-Howard, Adam Parnaby, Ian Johnson and Rebecca Nicholson. They contributed to the intellectual development of ideas in the project presented in chapter \ref{ch:7} and as the project was truly collaborative, these certainly feature within the thesis. Members of the team acted as researchers, designers, and facilitators in service of this project.
\end{itemize}
Research presented in this thesis has been published in the following peer-reviewed locations:
\begin{itemize}
    \item \textbf{Kieran Cutting} (2022). "Towards speculative praxis: Finding the politics in speculation with Deleuze and design", \textit{Speculative Geographies: Ethics, Technologies, Aesthetics}. Palgrave Macmillan: Singapore. 
    \item \textbf{Kieran Cutting} and Sean Peacock (2021). "Making sense of ‘slippages’: re-evaluating ethics for digital research with children and young people", \textit{Children’s Geographies}.
    \item \textbf{Kieran Cutting} and Erkki Hedenborg (2019). "Can Personas Speak? Biopolitics in design
processes", \textit{Companion Publication of the 2019 on Designing Interactive Systems Conference 2019}. 
\end{itemize}
A significant amount of this research has also been presented at conferences as a work-in-progress:
\begin{itemize}
    \item \textbf{Kieran Cutting} and Lys Stone (2021)."Co-creating oppression: ‘voice’, participation and co-production in the foster care system", \textit{The Society for the Social Study of Science Annual Conference 2021}.
    \item \textbf{Kieran Cutting} (2021). "‘Let us use our power’: moving beyond datafication in technology design", \textit{Data Justice 2020/1: Civic Participation in a Datafied Society.}
    \item \textbf{Kieran Cutting} and Dean Pomeroy (2021). "‘The Never-Born Future: against the technocapitalist collapse of possibility", \textit{British Sociological Association Annual Conference 2021.}
    \item \textbf{Kieran Cutting} (2021). "‘We have always been living through austerity’: rewriting time with justification practices", \textit{British Sociological Association Annual Conference 2021.}
    \item \textbf{Kieran Cutting} (2020). "Chasing the Philosopher’s Stone: In Search of Healing Practices after Austerity", \textit{The Society for the Social Study of Science/European Association for the Study of Science and Technology Annual Conference 2020.}
    \item \textbf{Kieran Cutting} (2020),"Playing our way out of hegemony: a card game to create possible futures", \textit{Royal Geographical Society Annual Conference 2020}.
    \item \textbf{Kieran Cutting} (2020). "Bruises or scars? Justification practices and the legacies of austerity", \textit{British Sociological Association Annual Conference 2020.}
    \item Sean Peacock and\textbf{ Kieran Cutting} (2019). "Making Sense of Slippages: Reflections on doing ethical research with young people", \textit{Royal Geographical Society Annual Conference 2019.}

    \end{itemize}

\end{publication}