\chapter{Methodology}
\label{ch:3}
\section{Introduction}
\label{sec:3-1-intro}

% structure now should be: developing a methodology, data collection and analysis, ethics, liberation and healing

In the previous chapter, I introduced Mark Fisher’s concept of capitalist realism and proposed that austerity acts as both an example of and intensifier of it. I identified its central components as the constant creation of instability, the use of that instability to create new sources of and strategies for capital accumulation, and detailed the way this new accumulative activity becomes transformed into `common sense'. I showed how the production (and consumption) of vulnerability within the third sector is the primary source of new accumulative activity within austerity-intensified capitalist realism. Setting out to explore firsthand how austerity-intensified capitalist realism operates within the third sector, this chapter is an exploration of the methodological foundations of my research. 

In her  book \emph{Meeting the Universe Halfway: Quantum Physics and the Entanglement of Matter and Meaning}, \citet[p. 185]{barad_meeting_2007} introduces her concept of the "ethico-onto-epistemology", which highlights the necessary and material entanglement of ethics, being, and knowing. For Barad, ethics, ontology and epistemology are not separable \citep[p. 90]{barad_meeting_2007}, because our knowledge creation processes (epistemologies) are necessarily entangled with what we believe exists (ontologies), and the processes of determining what we believe exists are also entangled with what we believe \textit{should} exist (ethics). Following Barad, then, this chapter is an exploration of the ethico-onto-epistemological commitments of this thesis, first detailing my methods of data collection (and the onto-epistemology which underpins them), before turning to my data analysis methods, and finally, the way that ethics, liberation and approaches to trauma and healing have underpinned the way that I have conducted this research.

\section{Data collection}
\label{sec:3-2-collection}
At its core, my research is a work of Charmazian grounded theory, a social constructivist methodology that is informed by both Glaser and Strauss' initial formulation of grounded theory and symbolic interactionism \citep{charmaz_constructing_2006}. Although I will detail my approach to grounded theory more in section \ref{sec:3-3-analysis}, it is important to mention it initially as grounded theory mandates a "continual intermeshing of data collection and analysis" \citep[p. 73]{glaser_discovery_2009}.  As such, my approach to data collection has been informed by the way that I analysed data, and my approach to data analysis was informed by how I collected data. Charmazian grounded theory in particular places a large degree of importance on the talk and language that participants use to represent their experiences (owing to its roots in symbolic interactionism), understanding language to be an authentic reflection of a participant's experience of the world \citep{charmaz_constructing_2006, mead_mind_2015}. Yet particularly in an analysis of an hegemonic system like austerity-intensified capitalist realism, there are more factors at play than what can be easily communicated through participants' linguistic self-representation—a whole realm of actions, artefacts, things, and symbols that take on a life of their own. For this reason, I have also drawn upon ethnomethodological and material-semiotic methodologies. Ethnomethodology and ethnographic methods have supported me in paying attention to the significance of physical interactions that my participants made, such as the ways that they handled paperwork or interacted with their computers. Material-semiotic approaches have enriched my understanding of the data by focusing analysis on the relationship between matter and the meanings that are constructed about matter. As detailed in chapter \ref{ch:6} for example, evaluation plays a greater symbolic and discursive role in the lives of those working in charities than the materiality of evaluation (such as paperwork, post-it notes, and web interfaces) might suggest. 

My onto-epistemology, then, is based around first understanding participants' actions and behaviours, before working with participants to understand how they make sense of and represent their own experiences, then working to understand the structural factors that shape these experiences and the symbolic meanings that can be ascribed to these experiences. This onto-epistemology has been the anchor for my overall research design and method selection, focusing on simultaneous observation, action, and design, with a new materialist understanding that the methods I employ within work in fact also produce the experiences that I am attempting to observe. My basic approach has been to observe what is happening in a given site, take some action to see if it changes what is happening within that site, and use design as a tool to explore alternatives to what is happening within these site. Above all of this, I have used grounded theory as an approach to understanding the phenomena of austerity-intensified capitalist realism, and employed the constant comparative method of grounded theory to seek out further data to answer questions that arise out of the research and theory-building process. In this section, I explore my approach to observation, action and design in turn, and the kinds of data that have resulted from this approach. 

\subsection{Observation}
\label{3-2-1-observation}
Observation-based methods are useful for their ability to create a "systematic noting and recording of events, behaviours, and artifacts in the social setting" \citep[p. 139]{marshall_designing_2010}. Observation-based methods are therefore particularly useful in research projects where the research question centres on rich description of phenomena, or to inform the development of a grounded theory, as mine does. As such, I determined early in my research that I was going to use methods of observation to answer research questions 1 and 2 in particular. I chose to employ participant observation based methods because of their ability to support researchers in gaining a rich personal understanding of the setting that they are researching. As Goffman puts it, participant observation consists of "subjecting yourself, your own body, your own personality, and your own social situation, to the set of contingencies that play upon a set of individuals", creating a situation in which the researcher "tak[es] the same crap [their participants have] been taking [in order to] sense what it is they’re responding to"  \citep[p. 125]{goffman_fieldwork_1989}. The premise behind participant observation is that personal experience can lead to a deeper understanding of not just the phenomena itself, but the factors causing that phenomena. My primary observation method has been ethnography. 

% read more on why ethnography and put into below
Ethnography is variously considered a research method, a research paradigm, and a way of writing about research \citep{bate_whatever_1997}. Ethnographic approaches generally rely upon a researcher entering into a community, attempting to gain rapport with members of the community, and establishing a degree of competence of the systems of knowledge employed within a fieldsite which confers upon them an insider status \citep{atkinson_ethnographic_2017}.  Traditional approaches to ethnography often took the form of an applied anthropology, intended to create knowledge (directly or indirectly) for colonial administrators about the behaviour of colonised or indigenous communities \citep{baba_end_2005}.  This form of extractive knowledge production positioned the colonial power as a baseline for "civilized" normality, and interpreted the behaviours of colonised people as "savage", "barbaric", or "primitive" \citep{hsu_rethinking_1964}. Although ethnography and anthropology have been reckoning with this legacy (e.g. \citet{manning_constructing_2016}), ethnography has often been used as a study of deviance from a norm, rather than a more general approach to observation.  

The emergence of organisational ethnography considerably expanded the range of acceptable field sites and research subjects for `applied anthropology', producing a more sociologically-inflected ethnography. Pioneered by Lloyd Warner, the Hawthorne studies of the 1920s and 1930s began to conceptualise the workplace as a society of its own, showing the utility of ethnography as a method for research that did not take place in colonial sites \citep{neyland_organizational_2007}. Organisational ethnographies brought the benefits of the ethnographic method to "bureaucratically structured formal organisations" \citep{watson_making_2012} such as businesses, charities, or community groups, in order to investigate some aspect of their behaviour. One of the greatest strengths of organisational ethnography is in uncovering the mundane content of organisational life, understanding the everyday interactions that underpin processes or systems which might otherwise seem opaque or inacessible \citep{bate_whatever_1997}. For example, in Latour and Woolgar's \textit{Labatory Life} \citeyearpar{latour_laboratory_1986}, the authors construct an account of the contingency of scientific knowledge, detailing how seemingly objective scientific ‘facts’ are rooted in their historical moment and circumstances surrounding their creation. Turning ethnographic analysis onto knowledge creation itself in this way alters the traditional relationship between observer and observed. More recently, organisational ethnography has begun to consider the importance of theory-building, showing how individual organisations reflect the values, practices, and behaviours of a wider social organisation \citep{watson_making_2012}. 

Despite the methodological innovations brought on by organisational ethnographies, they tend to remain bounded by a single spatial site. Single site ethnographies are intentionally limited, constrained to the creation of understanding through the insights that can be gained through a single site. Yet as complexity and scale increase, as they do with understanding the dynamics of austerity-intensified capitalist realism, a single site may not be sufficient. Multi-sited ethnography, on the other hand, looks to examine the "circulation of cultural meanings, objects, and identities in diffuse time-space" \citep[p. 97]{marcus_ethnography_1995} by tracing a system or cultural formation "across and within multiple sites of activity" with the intent of understanding the ways that individuals lifeworlds are enmeshed with the systems they exist within. Although my research has been considerably informed by organisational ethnographic approaches, my work has not been bounded by a singular space or site. I have conducted research with three different organisations, each based in different parts of the United Kingdom. I \textit{could} consider my ethnography through these multiple geographies, or through its temporality, as an ethnography of how particular spatio-temporal configurations of organisations in the children's social care system experienced a stretch of time at the end of the 2010s, witnessing the ghosts and legacies of austerity, and finishing with the early days of the COVID-19 pandemic. Yet both of these segment my fieldwork in a way that it refused to be in reality, with each fieldsite slipping into each other at times. The most convincing construction of my fieldsite is instead an ethnography of infrastructure. 

Following \citet{star_ethnography_1999}, ethnographies of infrastructure are multi-sited ethnographies in which the fieldsite is an entire infrastructure rather than a spatially or temporally bounded site. As infrastructures are fundamentally relational and modular \citep{star_steps_1996}, ethnographies of infrastructure must follow the relations within the infrastructure to their logical ends. My ethnography has traced the infrastructure of the children’s social care system from the perspective of different charities who support young people perceived to be vulnerable. In order to best understand this infrastructure, I have traced its relations, observing the functioning of the infrastructure from different positions and perspectives. I have worked with care-experienced young people, social workers, youth workers, and managers; I have worked with small, medium-sized, and large charities; I have worked with local authorities and with public sector arms-length organisations. Most importantly, though, as Star recommends, I have followed my "ethnographer’s nose twitch" \citep[p. 610]{leigh_star_this_2010} to determine what to explore next, tracing the phenomena and its relations as it began to become clearer to me. Ethnography has acted as an art of noticing for me, drawing attention to different aspects of the infrastructure, and highlighting alternating moments of harmony and dissonance depending upon how the infrastructure is paid attention to \citep[p. 24]{tsing_mushroom_2017}. Multi-sited ethnography of infrastructure is like turning an object with many sides to try to see all of its aspects.  

Within the different vantage points I have occupied within the infrastructure of the children's social care system, I have conducted a multitude of focused ethnographies. Focused ethnographies are similar to organisational ethnographies, but can be conducted part-time, feature shorter visits to fieldsites than conventional ethnographies, but employ a more intensive set of data collection methods and a greater scrutiny of data through analysis \citep{knoblauch_focused_2005}. Additionally, focused ethnographies tend to deal with specific problems in distinct social contexts \citep{wall_focused_2014} rather than a more holistic ethnography of an entire organisation. As such, the use of focused ethnography suited my research well, allowing me to understand how the infrastructure of the children's social care system (and thus austerity-intensified capitalist realism) operated through specific organisations, rather than interrogating each organisational dynamic that emerged. Here, utilising the grounded theory strategy of theoretical sampling \citep[p. 96]{charmaz_constructing_2006} enabled me to continuously filter and taper what I was collecting data about. 

For each of the organisations I worked with, my approach differed. For one organisation, I based myself in their offices part-time, working there for around half of every week for approximately six months in order to observe, build relationships, and help out. In another organisation, I spent around a month at a time embedded in the organisation, briefly moving to the same county as they were based in and working from their offices every single day, attending in-person meetings and being physically present to observe their interactions, get to know them, and work on projects with them. Finally, another organisation that I worked with had no physical presence, and instead was a remote working nationally-focused team, punctuated by focused, in-person residential weekends. The digitally connected nature of this work meant that I never truly left the field, despite these periods of `focused' ethnography. I remained in consistent digital and telephone contact with my collaborators even when I have not been in the field. Ethnography of infrastructure suited the care system well, as even once I had stepped back from the field (or was forced to, due to the onset of the COVID-19 pandemic), the infrastructure presented itself in new ways, through my social media feeds, the experiences of communities close to me, and my lingering professional interest in supporting young people perceived to be vulnerable. In the present day, my life remains entangled with the care system: I am friends with care-experienced people, youth workers, and social workers; I worked within the care system within my freelance design practice and then later, in my worker's co-op; I have helped facilitate the development of a manifesto for care-experienced people to reclaim their community from charities, and I keep in regular contact with (youth worker) ex-participants digitally as we have become friends. 

% add in a section about ethnomethodology? about how this drew attention to the practices of participants, rather than just their talk, and the role of objects? 

\subsection{Action}
\label{sec:3-2-2-action}
% maybe make section about 'praxis' - is a point that I am making that I created grounded theory from praxis, then use that theory to generate new praxis? 
My entanglement with the care system has also been intensified by my use of action-centered methods, drawing from traditions such as action research, participatory action research and research justice. I have used action-centered methods for both methodological and moral/political reasons. Methodologically, action-centered methods suited my research questions well, as the construction of a meaningful alternative to capitalist realism requires that such alternatives be trialled and understood in practice. Yet morally and politically, the nature of my research—working with young people perceived to be vulnerable, and youth and social workers who are supporting them—has also meant that it felt inappropriate to me \textit{not} to use action-centered methods. It would have felt immoral to observe people, asking questions about their experiences of what may be some of the hardest times of their lives, and then not offer them support, resources, and practical tools to identify changes that they want to make in their worlds and begin to make that change.

Particularly when investigating an infrastructure, ethnography is not sufficient to understand all of its aspects. As Bowker and Star explain in \textit{Sorting Things Out} \citep{bowker_sorting_1999}, ethnography's very nature means that we cannot see what our participants do not see; our observations recreate the exclusions that exist in our participants' worlds, in the same way that their worlds create exclusion by following the functioning of the infrastructure they exist within. Using some form of action-centered methodology, then, helps to address this, by putting participants into new situations, providing them with opportunities to do something about the exclusions that are created by their infrastructures. I have drawn from three different traditions of action-centered research in my work—action research, participatory action research and research justice—using methods from each as appropriate. 

My guiding tenet in considering action-centered forms of research is Marx's \citeyearpar[p. 173]{marx_selected_2000} eleventh thesis on Feuerbach, "the philosophers have previously only interpreted the world; the point... is to change it.". Citing this well-known adage may seem trite, but it has supported me in identifying the most pertinent aspects of action-centered approaches and methods and uniting them in my research. If a method could help me or my participants to interpret or change the world, it is suitable. I diverge from the eleventh thesis in one way: I maintain that interpretation, observation and description are themselves a form of action or `change'. Understanding interpretation as a mere reflection of the world assumes that there is a direct relationship between representations that are created of phenomena and the phenomena itself. This neglects to account for the way that all representations are performative and must be actively constructed by their observer, as Barad notes in \textit{Meeting the Universe Halfway} \citep{barad_meeting_2007}—representation is always a \textit{doing}. As such, I consider my ethnographic methods to be interlinked with my action-centered research practice, and have encouraged my participants to consider description, clarification and articulation a form of action, too.

% the sentence 'three forms of AR' below is new and may need tweaking
This aligns well with a methodology built around action research. Instantiated by Kurt Lewin and developed by thousands of practitioners since, the basic tenet of action research is a spiral of "planning, action, and fact-finding about the result of the action" \citeyearpar[p. 38]{lewin_action_1946}. In Lewin’s view, these steps constitute "rational social management", and can develop a form of research practice which can variously "help the practitioner", "lead… to social action", and "research… the conditions and effects of various forms of social action" \citeyearpar[pp. 34–35]{lewin_action_1946}. These three forms of action research are often confused, leading to disciplinary conflict about what constitutes legitimate action research. For the purposes of this research, though, I draw upon all three of these forms—considering these in turn to be reflexive practice, research which builds social movements, and research about how to build social movements. In conducting action research, practitioners develop their reflexivity, and engage in a process through which they can both learn about their practice and make change in the world. \citet{burns_applying_2017} describe the process of doing action research through reflexive practice using the "Kolb cycle" of experiencing, reflecting, thinking and acting, followed by another cycle of experiencing, reflecting, thinking and acting. Action research has been extensively used in education, health, and organizational management contexts and can be used to situationally understand problems and develop practice-based solutions to these.

% the swantz reference is crap and needs changing in zotero - the chapter is supposedly 2008 in a 2006 book???? but its 2007??? wtf 

As construed by Lewin and operationalised by the Tavistock Institute, early action research often acted as an experimental research method that could support organisations to operate more efficiently \citep{neumann_kurt_2005}. Although action research in the Tavistock's work (and reflected in their journal, \textit{Human Relations}) was essential to the development of science and technology studies (STS) methods and practices, it did not contain an explicit political framing except the loose idea of `social action' that Lewin had proposed. In contrast to this, participatory action research (PAR) began to be developed as a research approach explicitly founded upon the needs and liberation of "the oppressed". PAR is committed to a liberatory knowledge-making practice\citep{reason_handbook_2006}, with the same intellectual movement and tensions giving rise to critical pedagogy and participatory design.

In contrast to action research, PAR is more explicitly concerned with the community-based and collaborative inquiry. \citet{fals_borda_origins_2001} identifies three central challenges of PAR at the time of its development: the relations between science, knowledge and reason; the dialectics of theory and practice; and the tensions between subject and object. PAR seeks to blur the boundaries between each of these areas. It drew upon and transcended a variety of disciplines and methodological approaches, such as social psychology, Marxism, anarchism, phenomenology and classical theories of participation. PAR centers the idea of science as socially constructed, and Fals Borda explains how the nascent movement used Lewin’s action research as a way to justify itself as a legitimate practice and methodology. Moreover, PAR calls for a breakdown between subject and object in research, viewing those involved in research as ‘thinking-feeling-persons’—or humans—rather than merely considering them for their participation (i.e. ‘participants’). As such, participatory action research is committed to collaborative inquiry around the interests of the oppressed, facilitating people to explore questions that matter to them and their own lived experience, towards liberation. In more recent work \citep[p. 1]{mcintyre_participatory_2007}, these principles are synthesised into:
\begin{quote}
\begin{itemize}
    \item a collective commitment to investigate an issue,
    \item a desire to engage in self and collective reflection around the issue,
    \item a joint decision to engage in individual and collective action that leads to a beneficial solution, [and]
    \item the building of alliances between researchers and participants throughout the research process.
\end{itemize}
\end{quote}
The work of Paulo Freire was instrumental to the early development of PAR methods. Though PAR and Freire's critical pedagogy are distinct, they mutually inform each other and share a common root. Freire argues that the dominant model of education is a banking model, where the action afforded to a learner is only "as far as receiving, filing, and storing the deposits" \citep[p. 72]{freire_pedagogy_2000}, reproducing attitudes and practices which mirror the oppression within society at large. Such a pedagogy prevents the development of critical consciousness, a state whereby people are able to relate problems to structural causes, and "create a new situation" through "transforming action" \cite[p. 47]{freire_pedagogy_2000}. Critical pedagogy is therefore easily aligned to action research and PAR, using similar cycles of action and reflection and centering the needs of those with lived experience of oppression \citep{serpa_political-pedagogical_2020}. For Freire, these cycles of "reflection and action upon the world in order to transform it" \cite[p. 51]{freire_pedagogy_2000} constitute praxis, a situated and theoretically-driven form of action. However, this theory-driven action is not a strict application of theory, but an enmeshment of theory and practice. Deleuze posits that theory and praxis have a "fragmentary and partial" relationship to each other, whereby praxis is a "network of relays from one theoretical point to another" and theory "relays one praxis to another" \citep[p. 206]{deleuze_desert_2004}. Theory and praxis are thus mutually constitutive and are constantly changing one another.
%jolivette1 - [location287 
%j2 - 288

Research justice is a framework for conducting research that advocates for research praxis centered on liberation and justice. It draws attention to the ways that the dominant modes of defining `legitimate' knowledge are themselves oppressive, operating from "a paternalistic position of assumed superiority that has been unsuccessful in producing meaningful reforms and social justice for indigenous nations and communities of color" \citep{jolivette_research_2015}. Research justice therefore draws attention towards underrepresented forms of knowledge (such as experiential, cultural and spiritual knowledge traditions) \cite[p. 4]{asad_prefigurative_2019}, and makes clear that invalidating the legitimacy of these knowledge traditions creates harm for many communities that are often the subjects of academic knowledge production. In response to this, research justice proposes a research praxis that facilitates:
\begin{quote}
\begin{itemize}
\item access to information (not just misinformation and outside expert research but what they truly seek and deserve) that impacts their lives; 
\item ability to define what is valid ‘knowledge,' as well as methods to produce this; 
\item capacity to produce their own knowledge; 
\item capacity to use all forms of knowledge; 
\item and control over all stages of the `knowledge lifecycle'—from producing, analyzing, interpreting, packaging and deploying knowledge—on an equal footing with all other institutions in society. \citep{jolivette_research_2015}
\end{itemize}
\end{quote}
Research justice asks us to consider a research praxis that asks what \textit{else} research can be other than the production of knowledge. In this way, researchers can become academic `accomplices' \citep{asad_academic_2019} that support communities and challenge the structures of injustice and oppression created by much research practice.  Research justice can therefore be prefigurative, focused on the transformation of social relationships (towards more just, liberating and healing relationships), redistributing material, skills-based, and network-based resources to those who need or want it, and building counter-structures that can replace the "harmful institutions, policies and practices that impact our community collaborators" \citep[p. 13]{asad_prefigurative_2019}.

My research has blended together these different action-centered approaches to construct a research praxis that is just, liberating and, by being centered on the creation of possibility, is more amenable to a trauma-centered framework. I have worked to embrace action research in all three of Lewin's original forms—as reflexivity, movement-building, and research about how to build movements. I am committed to acting as an academic accomplice, placing the needs of a community first, with my research following those needs, attempting to dismantle power hierarchies, redress historic inequalities, and change structural power dynamics in the process. Finally, I have attempted to support the people that I have worked with to create change in whatever ways felt appropriate to them. This has included developing tools, resources and encounters through my research that either help them to articulate the issues that exist for them under austerity-intensified capitalist realism, or to imagine and begin to construct an alternative to it. This has created tensions at times, which are explored more fully in section \ref{sec:3-4-ethics}. 

My research is characterised by the mutual enmeshment of observation and action (and in turn, theory and practice) as praxis. Although I have worked with managers, frontline workers and young people, I have tried to listen to and support their needs according to who holds the least power at any one given time. For example, the needs of care-experienced young people (e.g. "I need this form of support") may run counter to the needs of frontline workers (e.g. "I need this visit to be easy"), which may further run counter to the needs of managers (e.g. "I need my workers to be proactive and young people to feel supported in the ways that our project plan set out"). In this case, I have always worked towards the needs of young people; in the case of workers and managers alone, I have worked towards the needs of workers. 

\subsection{Design}
\label{3-design}
%here
% what im trying to say is that O A and D are all important aspects, and that design both reflects and prompts further action 
Observation, action and design have been interwoven and complementary methodologies since the emergence of participatory design practices \citep{kyng_setting_1991}. The three methodologies can be seen as constructing or reflecting different aspects of people's experiences: for example, \citet{frascara_user-centered_2002} suggests that what people say, do, and make (corresponding roughly to observation, action, and design) each communicate different kinds of knowledge—explicit, observable, tacit, or latent knowledge, in turn. Yet design is unique, as it is its own form of knowledge creation and also a specific form of action. As with the Kolb cycle of action research—in which future plans are based upon reflection around a previous cycle of action—designed artifacts are both an embodiment of knowledge and a catalyst for further knowledge creation. Designed artifacts are therefore both mediating objects and meaningful tools within a user's practice \citep{kyng_setting_1991} which can tell us more about their social world and the semiotics of materials within their world. 

Research around the design and use of information systems and digital technologies has a rich history of using design and action-oriented research approaches in complementary ways. Beginning in the early 1970s and crystallizing into an emergent design and research agenda in the early 1980s, co-operative design developed out of technology research and design in Scandinavian countries. In its earliest form, the "collective resource" approach, this most often took place with workers and trade unions: including the Nordic Iron and Metal Worker's Union \citep{nygaard_iron_1979}, various Swedish and Danish trade unions (in DEMOS and DUE projects respectively),  the Nordic Graphic Workers' Union (in the UTOPIA project, \citep{ehn_collective_1987, sundblad_utopia_2011}), and with Norwegian nurses (in the Florence project \citep{bjerknes_user_1995}). Eventually, co-operative design became a formalised research agenda and design methodology based around the empowerment of users to "fuller participation and cooperation" \citep{kyng_setting_1991}. In this co-operative design approach, "doing became a primary activity" with the interactive use of interface mock-ups by designers and participants helping to identify key moments of breakdown in a system's design. This approach suggested that design itself is a form of action which surfaces a specific kind of knowledge. Moreover, although this approach used methods similar to action research, it considered itself distinct because action research "bends more towards outcomes" where co-operative design explicitly foregrounds the value of creation (through design), and shapes both "the process and the product" through the application of design methods \citep{greenbaum_design_1993}.

As co-operative design practice disseminated outward from Scandinavian countries, it began to find a base in the United States in the form of participatory design. Participatory design was viewed as an adaptation of the processes and methods of co-operative design to a non-Scandinavian context. Compared to Scandinavia, for example, the United States experienced comparatively low levels of unionisation and an increasingly hostile climate for the building of workplace democracy under the Reagan administration \citep{greenbaum_design_1993}. In this context, then, participatory design advocates suggested that American participatory design required "home-grown characteristics", which focused on "more active involvement of users and developers in the design process" rather than "alter[ing] the power relations" \citeyearpar[p. 7]{greenbaum_design_1993}— which might slowly build a movement for workplace democracy, essentially the inverse of the Scandinavian practice which was already founded upon a basis of workplace democracy.

A set of principles are common to both co-operative design and participatory design. Though these may not be explicitly stated, they form the outline of what co-operative or participatory design praxis and projects are oriented towards. Adapted from \citet{greenbaum_participation_2012}, these principles are:
\begin{itemize}
\item equalising power relations—developing methods of supporting those who are marginalised, oppressed, or less powerful,
\item situation based actions—working in deeply embedded ways, directly with people, in the context that is being designed for,
\item mutual learning—in which participants and designers find common ground and exchange skills and experiences,
\item tools and techniques—that help participants to express their needs, visions of the future, or tacit knowledge,
\item alternative visions about technology—which challenge the dominant or conventional perspectives or imaginaries around technology is, can be, or should be, and
\item democratic practices—the building of equity, democracy and justice both within the context of design and society more generally.
\end{itemize}

Since the development of participatory design in the United States, there have been concerns about it losing its political foundations. \citet{bjerknes_user_1995} called for a renewed emphasis on the issue of democracy within systems design in their paper "User Participation and Democracy", and Beck's "P for Political" suggested that "participation is not a sufficient condition for changing power relations" as "forms of participation exist and presently thrive that do not question, but further, dominant power patterns around the development of IT" \citeyearpar[p. 82]{beck_p_2002}. These issues have persisted—and perhaps intensified—into contemporary participatory design practice. In an analysis of the content of presented at the \textit{Participatory Design Conference} from in its inception in 1996 to 2012 \citep{halskov_diversity_2015, basballe_early_2016}, Halskov, Hansen and Basballe identify a growing subtlety to the political content of papers, slowly transforming from an explicit politics to a focus onan improvement or optimisation of a specific experience for participants. In 2018, Bodker and Kyng identified that a great deal of participatory design work was beginning to focus on the here-and-now, had low technological ambitions, consisted of "do-gooding", and reduced politics merely to ethics \citep[p. 14]{bodker_participatory_2018}. They suggested a repoliticisation of participatory design—to focus on participatory design that "matters", centered on areas where "dramatic, potentially negative, changes are underway", in which researcher-activists co-operate with partners to develop the research agenda and create a high impact vision to counteract that change, and where that impact is safeguarded through democratic control. 

Yet participatory design is not the only political form of design. Most often, participatory design projects employ an action research approach—and which may either reach towards some radical social goal (such as in the case of the earlier Scandinavian work), or foreground issues of power and participation in its composition or process (such as \citet{vines_configuring_2013}). Other forms of political design take a different approach to instantiating their politics. Some approaches (such as prefigurative design and design justice) seek to foreground the development of practical methods to bring about social change, whilst others focus on what the construction of artefacts can uniquely bring to a political design practice (such as critical design and adversarial design).  These approaches are not sufficient in isolation, but developing a design praxis which incorporates these and the lessons of participatory design can provide a foundation for meaningful social change to occur through the linked processes of observation, action, and design.   

Critical and adversarial design are approaches which pay specific attention to the production of artefacts which can themselves produce further action in the world. In contrast to commercial or industrial design practice, critical design exists to explore the possibility of using design processes, interactions, and methods for non-commercial or product-driven purposes, to "stimulate discussion and debate amongst designers, industry, and the public"  \citep[p. 58]{dunne_design_2001} through the process of materialising and experiencing them \citep{malpass_critical_2019}. In doing so, critical designs seek to problematise current social and political formations through the creation of artefacts which embody alternative formations of the world \citep{bardzell_reading_2014}. These designs are not merely critiques of existing social worlds, but can be constructive attempts to build something new by creating defamiliarising experiences for those that interact with the artefact \citep{blythe_imaginary_2018}. In a similar vein, adversarial design advocates for the production of artefacts which "call attention to the contestational relations and experiences aroused through the design thing and the way it expresses dissensus" \citep[p. 7]{disalvo_adversarial_2012}. Whilst many critical designs are adversarial, adversarial design draws specific attention to the way designed artefacts assemble an agonistic politics, potentially capable of challenging hegemony (e.g. the hegemony of austerity-intensified capitalist realism) through the creation of novel articulations of social issues. 

In contrast to the focus on artefacts within critical and adversarial design, prefigurative design instead suggests that the politics and sociotechnical arrangement of the world after a desired social change has been made can be achieved through the design \textit{process} itself \citep{asad_creating_2017}. More generally, prefiguration refers to how political movements can embody the "forms of social relations, decision-making, culture and human experience" that are the aim of the movement through their "ongoing political practice" \cite[p. 7]{boggs_marxism_1977}. Prefigurative design, then, builds on this approach but emphasizes a deliberate commitment to materiality, too—design should not only raise awareness of systemic injustice, as in critical or adversarial design, but "actively address and challenge them" \citep[p. 99]{asad_prefigurative_2018}. Prefigurative design is grounded in anarchist politics and is thus closely related to anarchist conceptions of human-computer interaction, which foreground questions of how to embed anti-oppression, liberation and dignity through technology research and design, and ultimately build counter-power to contest oppressive structures \citep{keyes_human-computer_2019, asad_prefigurative_2019}. 

Prefigurative design therefore acts as a form of anti-oppressive design \citep{smyth_anti-oppressive_2014} in which designers seek to create social justice and counter the efforts of the matrix of domination \citep{collins_black_2002}. In this way, prefigurative design also rests upon the design justice principles, developed by the Design Justice Network (and explored in depth in \citet{costanza-chock_design_2020}'s \emph{Design Justice}). These principles in many ways solidify and update the ideals of participatory, critical and prefigurative design, focusing on: 
\begin{itemize}
\item sustaining, healing, and empowering communities, seek liberation from exploitative and oppressive systems,
\item centering the voices of those directly impacted,
\item prioritising design's impact on the community,
\item viewing change as emergent from an accountable, accessible, and collaborative process,
\item understanding the designer as facilitator,
\item viewing everyone as an expert based on their own lived experience,
\item sharing design knowledge and tools with communities,
\item seeking sustainable, community-led and controlled outcomes,
\item working towards non-exploitative solutions that reconnect us to the earth and to each other, and
\item looking for what is already working in a community before seeking new design solutions.
\end{itemize}

Another form of constructive-and-critical design also bears relevance to these approaches: speculative design. Speculation as a practice is anchored in the imagining of previously unanticipated futures so that "futures and past might manifest themselves differently" \citep[p. 1]{gatehouse_hauntology_2020}. Speculative design then is centered around "critical reflection through future narratives... often mediated through objects" \citep{forlano_ethnographies_2013}. In many ways, speculative design functions as a development of critical design, amending the critical design practice to specifically focus on how to "collectively define a preferable future" \cite[p. 6]{dunne_speculative_2013}, lending itself to the design praxis that is being assembled in this section. This form of speculative design also makes the most of design's discursive and adversarial potentials, dealing with the ambiguous and liminal space of the potentially-fictional, exploring how audiences make sense of "hypothetical possibilities... utopian concepts and dystopian counter-products" \citep[p. 32]{auger_speculative_2013}.

Yet this form of speculative design can still be somewhat limited, often functioning as critical design by another name. \citet[pp. 183–186]{gatehouse_speculative_2020} suggests how speculative design can be made to be more than a reoriented critical design, through the use of a "diffractive gaze" that pays attention to "how difference is made by coming into close contact with the world", "how our practices emerge from our entanglement with world", and how paying attention to details helps designers to find "ways in which they can make a difference". Conceptualising speculative design as a diffractive gaze starts to indicate how design praxis can respond to the issue of austerity-intensified capitalist realism: noticing "small perceptual shifts, momentary glimpses of other ways in which things could matter" and creating spaces of possibility centered around these diffractions. I will return to the matter of speculative design and praxis later in the thesis in chapter \ref{ch:8}, as my proposed response to austerity-intensified capitalist realism. 

My approach to design has drawn upon each of these design methods at different stages. In the earliest stages of my work, I drew upon participatory design methods that prioritised configuring participation, rather than an explicit focus on the politics of design. After working on three participatory design projects and finding the impacts made limited (due to the justification practices described in chapter \ref{ch:6}), I turned towards a more explicitly political approach, drawing upon prefigurative design, critical design, and speculative design to an increasing degree. The majority of my design work has focused on my participants' ability to imagine future scenarios and the kinds of futures they are able to desire and imagine for themselves (as austerity-intensified capitalist realism has significantly limited this). I describe my specific design process more closely in chapters \ref{ch:7} and \ref{ch:8}, where the chapters are focused on my design processes and their outcomes.
%needs a diff title
\subsection{Methods used in data collection}
The overall data collection process for this thesis, then, focused on building relationships with three organisations that occupied different positions in the overall infrastructure of the children's social care system in order to understand how austerity-intensified capitalist realism had affected them. From 2018 to 2020, I maintained an ongoing relationship with these organisations, based in different places around the United Kingdom (until one of these organisations closed, and the pressures brought on by the COVID-19 pandemic meant my relationship with another wound down). I was embedded in these organisations as a critical ethnographer, a participatory action researcher, and a participatory designer.

My research process followed this general pattern:
\begin{enumerate}
\item Identifying an organisation that might be suitable to become embedded within,
\item Making contact with the organisation and agreeing in principle that I can conduct research inside of it,
\item Observing the organisation that I am a part of, becoming a part of its daily functioning,
\item Identifying alongside workers and managers ways that their work practices are counter to what they would like,
\item Making contact with care-experienced young people that use the organisation's services,
\item Talking to care-experienced young people in more depth, identifying the ways that their care-experience has been counter to what they would like,
\item Developing a design intervention on the basis of all of these factors.
\item Observing the organisation in the aftermath of the design intervention.
\end{enumerate}

Across these three years I conducted hundreds of hours of research and design with these organisations, including 14 months of focused ethnography, more than 20 design workshops, five weekend-long residentials, six distinct design projects, and around a dozen interviews, culminating in 1500 pages of fieldnotes across six notebooks. By my last count, I worked with 95 workers and managers and 82 young people with experience of the social care system. 

It is important to note here that for the purposes of length, I focus on only two design interventions in depth in this thesis—the final two design interventions that I conducted. I conducted three more design interventions in my time with these organisations, focusing on different aspects of participatory methods—participatory evaluation, participatory user experience design, and participatory filmmaking. These projects helped me to establish my design approach and get to grip with the design methods I was using, and yielded some insight on how to use design methods against austerity-intensified capitalist realism. They were particularly useful for finding the limits of existing participatory methods, but did not produce many novel design insights and did not significantly support me in answering my final research question, except for establishing that existing methods did not work in their current configurations. The two projects that are detailed in depth (\textit{It's Our Future} in chapter \ref{ch:7} and \textit{fractured signals} in chapter \ref{ch:8}) yielded clearer, greater insight that more succinctly answered my third research question and for that reason only these projects are presented. I conducted the participatory evaluation, user experience design and filmmaking work with the same organisations that I conducted the rest of my research with, alongside the ethnographic research that features in the following chapters. These three projects predominantly took place in 2018 and 2019, before \textit{It's Our Future} or \textit{fractured signals} (featured in chapter \ref{ch:7} and \ref{ch:8} respectively). Additionally, I conducted some semi-structured interviews with children's social care workers in the early months of the 2020 COVID-19 lockdown as a contingency in case I was unable to conduct any further research. These interviews explored their experiences of conducting social care work in the early days of the pandemic and any emergent trends they were noticing in the young people they were supporting. These do not feature in any form in this thesis because I was able to conduct further design research (\textit{fractured signals}) and these would have constituted a digression from my central argument. A full timeline of the research that I undertook for this thesis can be found in appendix \ref{appendix:timeline}.

\section{Data analysis}
\label{sec:3-3-analysis}
%material data collectiona t some point?
My approach to observation, action and design has also been influenced by my approach to data analysis—in particular, Charmazian grounded theory. Grounded theory is a series of "systematic, yet flexible guidelines for collecting and analyzing qualitative data to construct theories `grounded' in the data themselves" \cite[p. 3]{charmaz_constructing_2006}. I selected grounded theory as my method of data analysis early into conducting this research, as its ability to generate high-level, explanatory theories seemed appropriate for the kinds of research questions I was asking. Throughout my (iterative) data analysis processes, I have found the structures of grounded theory incredibly liberating as a way to make clear the relationships between individuals' experiences and the systems and structures which create the conditions for these experiences to arise within. 
%here
Grounded theory began as a methodological innovation to explore the experiences of people who are dying \citep{glaser_awareness_2017}—particularly as at the time, discussions of death and dying were avoided with terminally ill patients. Grounded theory represented a change in the efforts of sociological theory, from the verification of existing theories and hypotheses towards the generation of new theory directly from data grounded in the lived experiences of people \citep{glaser_discovery_2009}. This early work by Glaser and Strauss stood in stark difference to the quantitative research which was becoming the norm at the time, and grounded theory represented a first attempt to defend qualitative research against the claims that it is anecdotal, biased, or not rigorous. In any of its various forms, grounded theory remains rigorous because of its incredibly systematic methods for collecting, analysing, sorting, writing, and theorizing about data. 

Glaser and Strauss took the methodology in divergent directions, with Strauss later working with Juliet Corbin. In distinction to both of these approaches—which can be at times prescriptive—Kathy Charmaz's flavour of grounded theory emphasises flexibility. Instead of "rules, recipes, and requirements", Charmazian grounded theory focuses on a set of guidelines for constructing quality grounded theory that can complement the use of other qualitative methods  \cite[p. 10]{charmaz_constructing_2006}. Because of my own methodological focuses on observation, action and design, Charmazian grounded theory was the most suitable form of grounded theory for me to use, and it allowed the flexibility that my multi-sited multi-method research required. Specifically, Charmazian grounded theory works from a symbolic interactionist perspective, which understands theory to be constructed through interactions with "people, perspectives, and research practices" \citeyearpar[p. 10]{charmaz_constructing_2006} and prioritises the ways that research participants represent their experiences in language as a central aspect of theory-building. Charmaz refers to this as an \textit{interpretive} portrayal of the world, avoiding any claim to objectivity. This focus on language and lived experience was a good match for my work due to its participatory nature. 

Moving beyond a linguistic and interpretive portrayal of the world, though, I have also relied upon material-semiotic approaches (such as those used within the new materialisms and actor-network theory) to examine the interplay of actions, artifacts, and symbols with people's representations of the world. Each material-semiotic approach uses distinct language for how they discuss about the agency of actions, artifacts, and symbols, but each emphasise how non-human assemblages have a huge power and impact on the world and events within it. In Karen Barad's work, for example, this takes the form of "agential realism", which shifts focus from the nature of representations towards the nature of the discursive practices which create these representations \citep[p. 45]{barad_meeting_2007}. Even events or practices which appear to be wholly material are thus understood to have a performative and discursive aspect. For Jane Bennett, this takes the form of the "vital materialism" of "thing-power"—the capacity of "inanimate things to animate, to act, to produce affects dramatic and subtle" \citep[p. 6]{bennett_vibrant_2010}. Finally,  for Bruno Latour, this agency takes the form of "actor-networks", which emphasise the relations between actors and networks, and highlights their role as mediators that can transform or translate what they come into contact with. I do not rely prescriptively on one of these approaches, but instead follow their overlapping focus on the power of actions, artifacts, and symbols as they feature in my research. Material-semiotic approaches lend an important discursive analytic to my work which is blended with grounded theory in chapter \ref{ch:6}. 

The key components of a grounded theory study are:
\begin{itemize}
    \item simultaneous involvement in data collection and analysis,
    \item the development of analytic codes and categories from data (rather than codes deducted from a preconceived hypothesis),
    \item the use of the constant comparative method to make comparisons between data, incidents, and phenomena,
    \item each stage of data collection and analysis driving the development of a theory forward,
    \item the writing of memos to elaborate categories and themes within the data, to define relationships between different elements of those themes, and identify gaps for the next iteration of data collection,
    \item theoretical sampling, wherein the data is sampled for clearer/fuller development of the theory rather than representation of a given population,
    \item conducting the literature review after the fieldwork and theory development has already got underway. (adapted from \citet[pp. 5–6]{charmaz_constructing_2006}).
\end{itemize}
 
My research process has closely followed Charmaz's guide here, with amendments only for methodoloigcal specifics. Through the methods described in section \ref{sec:3-2-collection}, I collected data including fieldnotes, semi-structured interviews, photographs, and constructed or designed artefacts, diagrams, and annotated paperwork. Data was inductively analysed in the field to inform further data collection within that field visit, and then analysed in greater detail at the conclusion of each field visit where it could be more easily integrated into existing data. Both of these constituted their own action-reflection cycles, with the post-field visit analysis forming the foundation for the next field visit or design intervention. Throughout the collection and analysis processes, I also developed a number of diagrams or design artefacts to help make sense of what I was analysing. In grounded theory terms, this could be considered a kind of design-led memo-writing process, in which I used the techniques of design to better clarify aspects of the emergent theory—such as relationships between actors, the influence of certain actions, or to evaluate the efficacy of methods used. Throughout my data collection and analysis processes, I made use of the constant comparative method to make sense of how new data diverged from previously collected data, to make comparisons between the different organisations within my research, and to isolate specific instances of unique behaviours or phenomena.

Along with the design-led memos, I wrote more conventional grounded theory memos consisting of thick descriptions and to elaborate on relationships between people, things, or phenomena. After the conclusion of field work, I conducted a more thorough analysis on all of the collected data and began theoretically sampling in order to clarify the emergent theory that I was developing. In practice, I went through my entire corpus of data and wrote a post-it note for each individual unit of meaning. I managed to clarify a number of initial categories with some memos to explain relationships and impacts. I then transferred these data, codes and memos to the digital whiteboard tool Miro to get a better sense of the relationship between different data and concepts within the theory. This enabled me to develop more sophisticated memos as I could quickly move between different scales—the micro level of personal experiences and interactions and the meso level of system-wide changes. Having developed my grounded theory (elaborated in chapters \ref{ch:5} and \ref{ch:6}), I brought it to participants to see how they worked with it and if it helped them to make sense of their experiences, an essential aspect of participatory research. Every participant I presented the theory to immediately understood the content of the theory, feeling that it reflected their experiences and helped to explain some of their experiences over the last few years.

The grounded theory of justification practices was used to develop the design interventions in this research, with each of the methods developed intended to target an aspect of the phenomena specifically. This featured in a nascent form within \emph{It's Our Future}, and I then iterated on the methodological insights that \emph{It's Our Future} gave me in order to develop the \emph{fractured signals} project and the methods of speculative praxis. This process is described in greater detail in chapter \ref{ch:8}. Finally, further observation and action-centered data was collected during these design interventions, at which point data saturation was reached, as the behaviour of things relating to the theory could be reliably understood and no novel aspects could be identified (except for as they related to the COVID-19 pandemic, which hit during the final stages of this research). 

% maybe clarify data saturation

% do i need to add something about in vivo coding 


\section{Ethical research with people who have experienced oppression or trauma}
\label{sec:3-4-ethics}
Ethical practice is an essential aspect of any methodology but can be under-valued in writing about research. In this section I detail some of the ethical tensions I have encountered in my research, and ways that I have responded to these. These responses have not always been adequate; indeed, as chapter \ref{ch:4} details, some of these ethical tensions constitute slippages \citep{cutting_making_2021} which cannot be easily ethically reconciled with the state of knowledge production in contemporary academia. Although these responses have not always been adequate, however, they were always the best I could do at that moment, centering care and arising out of a commitment to being present and connected in the ways that the people I was working with needed at the time. It is important to note though that meaningfully ethical research practice with people who have experience of oppression and trauma must go beyond the regular commitments of ethical practice, seeking to work towards liberation and to create an open, trusting environment that supports people to discuss experiences that may intersect with traumatic experiences. In this section, I deal with both more general ethical concerns and the need for specific commitments to the development of an approach to ethical research with people with experiences of oppression and trauma. I separate these into issues that arise during the course of doing research, and those that arise in the course of writing about research. 

\subsection{Doing ethical research with people who have experienced oppression or trauma}
I have conducted the research for this thesis with people who exist at the intersection of multiple perceived and actual vulnerabilities, working with children and young people, people with experience of the social care system, and people who might exist precariously due to the the way their organisation employs them. These people often have experience of other issues perceived as vulnerabilities, such as homelessness,  addiction or drug use, parental or intimate partner abuse, mental health problems, refugee status, neurodivergence or disability. This is not to say that any of these experiences or identities necessarily render the people that I have worked with any more vulnerable; the organisations I have worked with, however, would certainly perceive them to be vulnerable by virtue of these experiences and identities. As mentioned in the previous chapter, vulnerability is an incredibly wide category that expands as the behaviour of those classified as vulnerable changes, and so I am not seeking to explore the classification of vulnerability itself here. What is important to the conduct of my work is that the organisations I have worked with perceive these people as vulnerable. 

Conducting research alongside perceived vulnerability comes with a multitude of increased ethical questions at both the procedural level, and the practical or relational level. The research described in this thesis followed Newcastle University’s guidelines for ethical research and received institutional ethical approval by the Faculty of Arts and Humanities. As part of this process, I was made to engage with an institutional ethics process that focuses on the classification of a large number of potential vulnerabilities and the ways that my research might interface with these, such as whether or not my participants would be children (they were), required gatekeepers (they did), whether sensitive topics would be discussed (they were), whether participants would experience prolonged or repetitive testing (a question that is impossible to establish at the outset of a project for participatory and embedded researchers), or whether participants would experience pain or more than mild discomfort (they didn’t). Although responding to these questions sufficiently is important, procedural ethics systems tend to function mostly as risk classification processes  (as I explored with Sean Peacock in our 2021 paper "Making sense of slippages"\citeyearpar{cutting_making_2021}). It is not enough to simply consider a risk classification process sufficient for ethical practice; these potential vulnerabilities must be considered from a practical and relational perspective that acknowledges the mess and complexity of embedded and participatory research.

The majority of the participants I conducted research with were children and young people, though most were over the age of 16. I began working with these children and young people through my partner organisations, and I always followed my partner organisations’ lead when it came to the establishment of what ethical practice should look like in that circumstance. For example, my consent forms contained no space for the consent of a parent or carer; each organisation that I worked with acted on the belief that a parent or carer had given consent for their child to attend the organisation’s activities (in writing), so any activity that the organisation deemed to be in that child’s interest was consented to by their parents or carers. This left the actual decision of consent and participation to the young people themselves, creating a rare space of direct empowerment in children’s social care, where a young person's choice is held in high regard and actually matters. If they wanted to participate in a session, they could; if they didn't want to, they didn't participate. 

To recruit participants, a member of the organisation would explain to potential participants what the research would entail and why they might want to participate (usually the ability to develop a skill, and use their experiences to help make or design something, along with whatever support the organisation would usually offer). Then, at the beginning of a project, I would introduce the research or design process, what we were doing, and why, and offer opportunities to ask questions. Once I was certain that everyone understood the project, the contents of the information sheet, and the contents of the consent form, each participant signed the consent form, dated it, and provided a pseudonym of their own choosing. At any point, a participant could withdraw consent—or simply choose not to participate in certain activities. During one project, for example, a participant chose not to participate because she had barely slept the night before and was becoming irritated by the other attendees. She took a nap and then rejoined the group later in the day. After they had participated in my research, participants were given a window of three months in which they could contact me to have their data removed from the project. They were also told that if they wanted to, they could contact me after the three months and I would do my best to remove their data from the project, but that no assurances could be made at this point. No-one contacted me at any point. I did have several interesting conversations with care-experienced young people about what I planned to do with their data, what would happen as a result of it, and the politics of knowledge creation more generally.

Newcastle University’s institutional ethical approval processes draws attention towards the discussion of "sensitive topics". In many ways, this was the intention of my research—discussing experiences that may have been difficult, particularly pertaining to living or working in the children’s social care system. The implication of questions such as these are that sensitive topics should be avoided, or if it is unavoidable, that the risks around the discussion of sensitive topics should be mitigated. Rather than eliminating risk, however, I wish to focus on an alternate value set here: foregrounding care and the possibility of healing through creating trusting relationships and a non-judgmental space that supports people to talk about difficult experiences. As engagement in my research might lead to discussion of topics that participants struggle to talk about, it felt important to me not just to let people leave the research or withdraw consent but to actively create spaces that they felt comfortable, safe, and cared for within. The facilitation practice I developed throughout my PhD research helped me to support people during this. Though it is not a particular focus at any point during this thesis, most of my participants actively felt that participating in my research gave them an ability to advocate for themselves and self-reflect in a way that they didn’t often have an opportunity to do, in large part because they were able and encouraged to talk about ‘sensitive topics’—such as homelessness, the difficulty of doing care work when things are precarious in your personal life, or the difficulties of running a large charity—without fear of judgment or reprisal. 

I conducted my research in a trauma-centered way. As I touch on in chapter \ref{ch:6}, I am sceptical of the current dominance of so-called ‘trauma-informed’ approaches, which often are provided in the form of single day training sessions that allow an organisation to claim status as 'trauma-informed' without interrogating the ways in which their own practice might need to shift in order to change the experiences of the people they work with. When I speak of a trauma-centred methodology, I am talking about creating a research program that creates genuinely safe spaces for people who might (or might not) have lived experience of trauma to reflect on their experiences. This enhances our conceptions of what care, consent, and consideration might look like in the research process. In my case, when I was conducting interviews, for example, I always encouraged participants to choose environments that they felt comfortable and safe in. Any questions which were centered on potentially difficult experiences were framed in open, participant-centered language, such as "If you feel comfortable talking about it, I’d love to hear more about what that was like for you". Framing questions in this way allowed participants to structure the flow of the research, diving deeper into things they felt comfortable discussing and painting only broad strokes of things that they didn’t. 

Both academic research practice and trauma-informed approaches (particularly as they were employed in the organisations I conducted research within) favour an approach to traumatic experiences and safeguarding that explicitly avoids disclosure. It can be suggested that practitioners encourage that participants with lived experiences of trauma to think outside of themselves when answering our questions or engaging in activities—to avoid participants saying "well, when this happened to me, I…". Whilst this approach stems from a desire to not pressure participants into sharing their experiences, it removes participants' ability to potentially experience some healing through sharing. As \citet[p. 56]{carless_narrating_2016} notes, "to listen as a distanced, neutral, disembodied Other [can] provide... a very different, more one-dimensional, simplistic, superficial understanding" that can prevent people who have experienced trauma from sharing openly. It removes the opportunity for the rich detail that people may give when reflecting on their own lives, which is a central benefit of doing phenomenological work.

Another frequent issue in workplace ethnographies is the perception that the ethnographer might be a spy for management \citep{baum-talmor_its_2019}. This is particularly exacerbated in larger or more precarious organisations, or in situations where the trust between management and general staff is weak, and the ethnographer has joined the organisation at the behest of management. In my research, this was not a key concern of any of my participants. With one of my organisations, the person who contacted me and who was my main relationship in the organisation was a general worker, who I quickly built a trusting relationship with. This happened in another of the organisations too, and though that person did not remain my central partner throughout my work, the endorsement of this worker meant that my relationship with this organisation started with an assumption of trust. In one organisation, I was brought in through the management side of the organisation, but in this situation the team did not yet exist, so I was present for the formation of the team and again, I quickly built a relationship with the general worker who was going to act as the project co-ordinator. All of the people that I worked with knew that I was going to take a highly anonymised approach to writing about this data (highlighted later in this chapter), and had complete confidence that anything they said wouldn’t travel to any other member of staff—particularly their managers. All interviews and workshops with workers took place in settings that either deliberately excluded their managers—in order to create safety for them to speak about their working conditions—or deliberately included them, in order to allow people to discuss things they wanted to change about their work in a safe way.  

There is a significant ethical tension around the use of incentives with people who marginalised, oppressed, or otherwise have low access to resources. The prevailing assumption here is that the provision of too significant an incentive might prove coercive, as if a prospective participant might not have a lot of money, they might feel they \textit{need} to participate in a research project. Yet on the other hand, to ask a tremendous amount of emotional (and in the case of some design projects, physical) labour of people, to talk about experiences they may have had which they might have found difficult, and not provide any kind of incentive for them or recognition of this experience disrespects and delegitimises their expertise. Particularly in care-experienced spaces, there has been a move towards recognising the tendency of organisations and individuals with high resource-access to commodify the experiences of care-experienced people. As such, I followed my partner organisations’ regular practices on each project. With two of the three organisations, no financial incentive was given for participation. With the other, financial incentives were offered (per the organisation's usual policy) in the form of £25 Love2Shop vouchers, which were preferred over money or any other kind of incentive by the care-experienced young people that I worked with as they were less likely to intefere with the money they might receive from Universal Credit \footnote{Whilst technically vouchers should be reported through Universal Credit too, it was widely acknowledged in all of my fieldsites that this rarely happens.}. In \emph{fractured signals}, participants were offered also offered Love2Shop vouchers (£30) for participating. Participants in \emph{fractured signals} were professionals who worked with care-experienced people in some way, and the vouchers in this situation were mostly a recognition of the time they gave up for the project. 

Rather than focus on the offer of financial incentives, throughout my work I tried to instead create capacity-building opportunities, which would offer participants opportunities to build new skills, try new things, or make something that felt personally meaningful them. Following the approach to critical pedagogy described earlier in this chapter, I created spaces throughout my research that gave participants to both question the power systems they found themselves in and build practical skills. For example, in one of the design projects which do not feature in detail throughout this thesis, I used participatory design methods and critical pedagogical approaches to teach care-experienced young people the technical and storytelling skills to create a documentary film about their experiences of life story work. Participants in the project received no direct incentive, but would learn the camera, audio, and story skills to make a film with professional-grade equipment and receive lunch, drinks and snacks across the five days that we were working together for. Similarly, in \emph{It’s Our Future}, there was no direct incentive for participants (except for train travel, paid for by the charity partner), but during the workshop they would be surrounded with new ideas and an environment that helped them to imagine and begin to build a future that addressed their needs and desires. Recognition and capability-building formed a key aspect of my approach. 

\subsection{Writing about research with people who have experienced oppression or trauma}
%maybe matters of care? Ways of knowing, theories and concepts have ethico-poltiical and affective effects on the perception and re-figuration of matters of fact and sociotechnical assemblages, on their material-semiotic existences' 
Writing about research with people who have experienced oppression or trauma requires care and consideration. Particularly because of the aforementioned commitments of this research—to the participatory construction of knowledge and to social change—it is important for you to know \emph{who} is writing this thesis. My positionality in relationship to this research has shifted throughout my fieldwork—which is apt, as subjective research such as this is a dynamic process that demands self-reflection \citep{rose_situating_1997}. I am not care-experienced. I have at no time lived through the children’s social care system. Prior to starting this research, I had only a small intersection with the care system: in the third year of my undergraduate degree I worked as a Brand Manager for the social work graduate scheme Frontline, which provides social work training to graduates of other degrees, much like the TeachFirst graduate scheme\footnote{Arguably, Frontline is part of the system of discursive accumulation I identify in \ref{sec: 6-7-performativity}}. My work mostly consisted of putting branded bike seat covers on bike seats and distributing free pens and teabags. It wasn’t a huge intersection with social care, but it was my first (knowing) intersection. I reflected after I finished this job that I had slightly known the social care system growing up—partly through a friend, who went to live with her grandparents after her parent became abusive—and partly through a spectre in my own childhood home, a fear from my parents that I might "say too much". There was nothing particular going on in my childhood home that my parents should have been worried about me talking about—we were poor, struggling to make ends meet, and we had the usual concerns of people who are poor. Yet my parents always had a paranoia, fearing that if I disclosed some insufficiency of parenting to someone outside of the house, "children’s services will come and take you away". I recognised this same fear immediately in many of the care-experienced young people I researched with who were parents, who had this same fear. Through this experience, I think that I learned to be afraid of being too articulate about what was happening.

Through doing this research and writing this thesis, I have learned to be comfortable with clear articulation, of naming what is known but unarticulated. Although I have not experienced the children’s social care system myself, then, I was familiar with its vague shape and form well before I began this research, and consider myself an ally to care-experienced people. Through the course of my research, I realised that researching the care experience was like finding a series of bruises I hadn’t realised were there. When I was younger, I was in two emotionally abusive relationships which left me traumatised and  emotionally reactive. As I mentioned, my family were (and are) poor, and my relationship with them is complicated and marked by this intergenerational cycle of poverty. I am not care-experienced and am not claiming in any way to understand the exact nature of care-experience. Through the course of my research, though, I began to realise that I shared some experiences with care-experienced people that helped me to understand where they were coming from more easily than some others might have. 

Neither of my parents finished school; my mum gave birth to my brother aged seventeen and my dad was in hospital through most of his twenties. Before them, there was a generation of cafe workers, cleaners, and power plant workers. Before that, farmers and bakers. My family and upbringing were thoroughly working class as a result. I am the first person in my family to go to university and to know that privilege. I went to a grammar school, and then to a Russell Group university, and then to another. I have been fortunate enough to be surrounded by people who believed in me, who would always give me any support I needed out of a belief that I was "gifted". So with some of the people I have conducted research with, I have been wracked with an intense sense of survivor’s guilt—a sense that if things had just been slightly different, if one sliding door had closed a little earlier, this could have been me. Learning to sit with the tension of these sliding doors has been instructive. In the conduct of my research, I would often feel the need to alter my accent—from the clearly-enunciated sort-of-received-pronounciation that I speak with in day-to-day life, back towards something that approximates an older version of my Medway-inflected-estuary-accent. It’s rougher, it’s harder, it’s more full of slang, and it has commanded much more respect and understanding in my fieldsites (and particularly with young people) than my "normal accent" ever has done—which is of course itself a construction, built through years of increasingly-more-middle-class-socialisation.

Suffice to say, then, I have a complicated relationship to care-experience and the object of my study. In a lot of ways, researching the care-experience has allowed me to return to myself, to re-assess my roots and how circumstance, support, and privilege has shaped me into the person that I am today, from similar roots to many of the people that I work with. Yet in the present, it remains true that I am an academic at a university, socialised in all of the ways that you might expect someone to be after nine years of proxmity to a university. My body places me firmly in that same seat of power—I am a white non-binary (yet masculine presenting) person conducting research with people who are oppressed, using methods such as ethnography that have historically been used to other. My whiteness and masculinity has likely shaped this research to a great degree—in many situations, I embody people’s expectations of what a researcher might look like, and this will have conditioned their responses to me, for better or worse. In some cases, this might mean they have responded more warmly to me, understanding the function that I am there for and being happy to spend some time with me. In others, this might have mean they have responded more coldly, hostile to a class of people whose job consists of expropriating knowledge from communities where it is already held in common. In the former circumstance, practitioners have often taken me seriously and had more time for me than they might otherwise have done; in the latter circumstance, young people have taken a while to build trust with me, and I have often leaned upon some of my other identities—designer, or sort-of-youth-worker, perhaps. 

The matter of voice is exceedingly important in this thesis, especially as it concerns people who have experienced oppression or trauma, and it is centred on supporting them to think differently about their futures. As such, the issues of who speaks, who listens, and what can be spoken about remain essential. Some of my earliest research in this thesis concerned the very idea of what it means to be listened to for care-experienced people, and this has subsequently shaped my research approach. No matter how I try to preserve voice throughout this thesis, though, I remain its narrator, and so it should be noted explicitly here that even when I attempt to give over narrative space in the thesis to the voices of others (as in chapter \ref{ch:5}), I am committing a sleight-of-hand, as I retain the power of ascertaining what matters and what does not. I have attempted to approach this responsibility sensitively, giving more space to things that people have spoken more about, and following my research participants’ wants and needs to shape what I actually explored. 

To the actual material of voice: I have attempted, where possible, to preserve accents and dialect words. Throughout this thesis, I have worked with people from across the United Kingdom, and so worked amongst a variety of types of voice. Broadly-accented Geordies, softly-spoken Southerners, fast-speaking Mancunians and people from the West Country with a dialect word for everything have all featured in my work, and I have tried to preserve the tone of their voices where possible. In line with this, I have tried to use as much of my participants' own language as possible, and used in-vivo coding. In line with Charmazian grounded theory, I believe that participants' own language can help to reveal a great deal about a phenomena (or their experience of that thing, at least)—much more than by papering over their words with bland abstractions.   

Textually, I have attempted to carve spaces for participants to tell their own stories throughout the text of the thesis. If a participant told me a story about their life or something that they had done or had happened to them, I have used as much of their own words as possible to re-tell that same story in-text if it is relevant to the overall story of the thesis. In line with this, I also had all of my participants choose their own pseudonyms. Particularly for care-experienced people, it felt important to give them an opportunity to have agency and ownership over their self-representation, as often this is taken away from them. At times, this has resulted in pseudonyms that may sound strange or peculiar—like L.TUKZOMBIE, who features in chapters \ref{ch:5} and \ref{ch:6}. By giving the choice to people in how they wish to be represented, I have hoped to create an environment where their consent and agency are actively respected.

One of the biggest ethical tensions that I encountered through my work was the simultaneous use of a critical ethnographic method and a participatory design approach. Critical ethnography often necessitates a kind of distance, a willingness to look beyond the individual towards the (infra)structural factors that maintain a situation, whilst participatory design instead can have a tendency to fetishise the individual and the language they use to represent their experiences as an unproblematic good. Contemporary participatory design approaches might suggest that researchers co-author publications with collaborators, or allow them editorial control of work which goes out into the world. On the other hand, critical ethnography might require us to be exceptionally critical of the actions that an individual takes, and relate that to the structural factors the motivate that action—which can be at odds with giving individuals involved editorial control or collaboratively publishing. Although I have engaged in these practices, resolving the tension between these two aspects of my work has been exceedingly difficult.

The way that I have attempted to reconcile this tension is as follows: my commitments within my participatory design practice are always to the person with the least power, which most often is the children and young people that I have worked with. I am committed to researching the care system and the structural factors that keep many care-experienced people in precarious situations, with significant difficulty in their futures due to intergenerational oppression. Sometimes, individuals that I have worked with—usually workers and managers—might be working counter to this goal through no individual fault of their own, attempting to exist within a system of structural power relations that condition their actions, which effectively coerces them into less-than-ideal situations. Indeed, this is the focus of much of my thesis. In these situations, whilst I am still on the side of the worker, they are expressing a facet of the structural power relations rather than existing as wholly themselves. As such, I cannot be fully committed to participatory work with them in this fashion. It is of utmost importance to talk about the bad, unethical, dangerous, or value-unaligned practice that is taking place inside of many charities that provide support services to children and young people in order to best support these young people.

That is not to say, however, that I am hositle to workers or managers, or intend to publish an exposé identifying certain charities whose practice leaves something to be desired. The organisations I have worked with throughtout my fieldwork merely express a set of behaviours that are likely to arise due to the structural conditions they find themselves in under austerity-intensified capitalist realism. As such, I have made the choice throughout this thesis—in honour of my commitments to both critical ethnography and participatory design—to create a semi-fictional composite of these organisations. In many large-scale research projects such as censuses, noise is actively added into datasets in order to protect the privacy of individuals whilst being able to discuss wider ramifications. The semi-fictional composite I construct throughout this thesis should be thought of as an example of this noise. Although I talk about three different field sites in this thesis, I have transposed some workers, events, and places between the organisations. Moreover, rather than considering them as three distinct charities, I discuss this through the composite organisation "The Charity" and discuss their work as individual projects inside of this larger charity. The Charity should be viewed as an expression of the infrastructure of the children's social care system and explicitly does \emph{not} refer to any singular charity. I introduce The Charity and its projects (Small Steps, Building Bridges and Seabird) substantively in chapter \ref{ch:4}. This enables me to speak openly about the practices of the organisations I have worked within and the experiences of workers and young people without compromising the privacy and confidentiality of my participants or the genuinely good work their organisations do. All events, quotations, and experiences recounted within this thesis are real; the composite merely exists to provide a buffer of safety and privacy  around the organisations I have worked within. 

\section{Conclusion}
\label{section:6-5-conc}
In this chapter, I have outlined my methodological approach. I described the ethico-onto-epistemological commitments of this thesis through my adoption of Charmazian grounded theory and material-semiotic onto-epistemologies, and described my data collection and data analysis methods in turn. I highlighted the interconnection of observation, action and design, and described how I used this hybrid methodology throughout my research in order to work with three organisations and almost 200 people across three years. I detailed my use of Charmazian grounded theory as an analytic for the processing of data, and my use of the constant comparative method and theoretical sampling in my data analysis process, which took place throughout my research. Finally, I described the ethical challenges and commitments of my research, exploring the need for a more well-developed approach to ethical research when working with people who have experiences of oppression or trauma in both the conduct of research and writing about research.

Having articulated my methodological commitments and approach, this thesis can now proceed to exploring how I enacted this methodology in order to learn more about the nature of austerity-intensified capitalist realism, the production of vulnerability within the children's social care system, and the experiences of young people perceived to be vulnerable and workers alike. In the next chapter, I will present The Charity in greater depth, introducing the Small Steps, Building Bridges, and Seabird projects and the key individuals and themes within each in order to set the scene for the rest of the thesis. After this, the rest of the thesis will consider how the experiences of young people perceived to be vulnerable and workers in the care system are affected by austerity-intensified capitalist realism, and how the tools and methods of design can be used to imagine and build coherent alternatives to this.