\chapter{Methodology}
\label{3}

\section{Introduction}
\label{3-intro}

% structure now should be: developing a methodology, data collection and analysis, ethics, liberation and healing


In the previous chapter, I introduced Mark Fisher’s concept of capitalist realism as a way to understand our contemporary sociopolitical moment. I identified its central components as the constant creation of instability, the use of that instability to create new sites of and strategies for capital accumulation, and detailed the way this new accumulative activity transformed into common sense. I showed the ways that austerity acts as an intensifier of capitalist realism, and how the production of vulnerability and the ‘digital turn’ led to the creation of a culture in the public and third sector focused on the use of service design. Resisting capitalist realism and constructing alternatives to it thus become increasingly difficult, because of its hegemony and tendency towards precorporation. In this chapter, I detail the methodological underpinnings that I have used to explore the question of how to construct meaningful alternatives to capitalist realism. 

% n.b. below: the stories that are told about people, objects and behaviours are imaginaries; stories of people's lives are... life stories? 
% ref Brene for research-as-storytelling
Capitalist realism's hegemony can be understood as an hegemonic \textit{story} about the state of the world and how people operate within the world. Part of constructing meaningful alternatives to capitalist realism, then, must also be about the creation and telling of new and different stories. This thesis contains many stories. Some are the stories of people’s lives; their experiences, and how they understand and make sense of those experiences. Some are the stories that are told \textit{about} people, objects, and their behaviours - like the stories that are told about the behaviour and lives of people who are perceived to be vulnerable, or the stories that are told about the 'innovative' potentials of digital technologies. In the latter part of the thesis, I also explore methods of  crafting new stories which might counter the functioning of capitalist realism, detailing how these new stories can material effects and affects. Unifying all of these, though, is the story of the thesis itself - the \textit{research story}, which brings together the stories of capitalist realism, people's lives, the stories told about people, the methods of telling new stories, and where we might go from here. These stories often sit uncomfortably together, creating irreconcilable tensions. Each of these stories could be thought of as holding a candle in a very dark place - illuminating one thing just well enough that you can see it in detail. The research story, then, attempts to bring enough candles to enough dark places so that we might see the walls of the room we are in: understand our environment (capitalist realism), what it is made of (how capitalist realism maintains dominance), and how we might go about finding a door (how to construct meaningful alternatives). This chapter is the account of how I came to hear the stories in this thesis, how I make sense of those stories, and the commitments I make to research-as-storytelling. 

In her 2007 book *Meeting the Universe Halfway: Quantum Physics and the Entanglement of Matter and Meaning*, Karen Barad introduces her concept of the “ethico-onto-epistemology”, which highlights the necessary and material entanglement of ethics, being, and knowing. For Barad, ethics, ontology and epistemology are not separable (2007, p. 90), because our knowledge creation processes (epistemologies) are necessarily entangled with what we believe exists (ontologies), and the processes of determining what we believe exists are also entangled with what we believe \textit{should} exist (ethics). This chapter is structured through an exploration of the ethico-onto-epistemological commitments of this thesis, first detailing my methods of data collection (and the onto-epistemology which underpins them), before turning to my data analysis methods, and finally, the drive for ethics, liberation and healing that underpins this research. 

% It is also worth noting at this stage that many of the contributions of this thesis are methodological. As such, this chapter should be understood more as a foundation than a complete artefact. This is the approach that outlines my work more generally, and in chapters seven and eight, I explore some of the methods I have developed to fulfil these commitments in more depth, outlining the ways that speculation can be used for capacity-building and prefigurative service design.

\section{Data collection methods}
\label{3-collection}
% maybe add page number
At its core, my research is a work of Charmazian grounded theory, a social constructivist methodology that is informed by both Glaser and Strauss' initial formulation of grounded theory and symbolic interactionism \cite{charmaz_constructing_2006}. Although I will detail my approach to grounded theory more in \ref{3-analysis}, it is important to mention it now because grounded theory mandates "simultaneous involvement in data collection and analysis" (maybe cite glaser and strauss 1967?). As such, my approach to data collection has been informed by the way that I analysed data, and my approach to data analysis was informed by how I collected data. Charmazian grounded theory in particular places a large degree of importance on the talk and language that participants use to represent their experiences (cite Mead?), understanding them as authentic reflections of their experience of the world. Yet particularly in an analysis of a system of control like capitalist realism, there are more factors at play than what can be easily represented by individuals' linguistic representations of their experiences and the world - a whole realm of actions, artefacts, things, and symbols that take on a life of their own. For this reason, I have also drawn upon ethnomethodological and material-semiotic methodologies. Ethnomethodology and ethnographic methods have supported me in paying attention to the significance of physical interactions that my participants made, such as the ways that they handled paperwork or interacted with their computers. Material-semiotic approaches have enriched my understanding of the data by focusing analysis on the relationship between matter and the meanings that are constructed about matter. As detailed in \ref{6} for example, 'evaluation' takes on a greater deal of agency and symbolism to the lives of those working in charities than the matter of evaluation (such as paperwork, post-it notes, web interfaces) might suggest. 

My onto-epistemology, then, is based around understanding participants' own experiences of reality, the structural factors shaping these experiences, the meanings that are ascribed to these experiences, and the behaviours and actions that underpin all of these. This onto-epistemology has been the anchor for my overall research design and method selection, focusing on simultaneous observation, action, and design. My basic approach has been to observe what is already happening in a given site, take some action to see if it changes what is happening within that site, and use design as a tool to explore alternatives to what is happening within these site. Above all of this, I have used grounded theory to inform the phenomena that I am paying attention to, and the constant comparative method to seek out further data to answer questions that arise out of the research and theory-building process. In this section, I explore my approach to observation, action and design in turn, and the kinds of data that have resulted from this approach. 

\subsection{Observation}
\label{3-observation}

Observation-based methods are useful in research for their ability to create a “systematic description of events, behaviours, and artifacts in the social setting chosen for study” (Marshall and Ross, 1989). Observation-based methods are therefore particularly useful in research projects where the research question centres on rich description of phenomena, or to inform the development of a grounded theory, as mine does. As such, I determined early in my research that I was going to use methods of observation to answer research questions 1 and 2 in particular. I chose to employ participant observation based methods because of their ability to support researchers in gaining a rich personal understanding of the setting that they are researching. As Goffman puts it, participant observation consists of “subjecting yourself, your own body, your own personality, and your own social situation, to the set of contingencies that play upon a set of individuals”, creating a situation in which the researcher “tak[es] the same crap they’ve been taking [in order to] sense what it is they’re responding to”  \citep{goffman_fieldwork_1989}. The premise behind participant observation is that personal experience can lead to a deeper understanding of not just the phenomena itself, but the factors causing that phenomena. My primary observation method has been ethnography. 

% read more on why ethnography and put into below
Ethnography can equally be considered a research method, a research paradigm, and a way of writing about research \citep{bate_whatever_1997}. Traditional approaches towards ethnography have relied upon a researcher entering into a community, attempting to gain rapport with members of the community, and establishing a ‘naive competence’ (cite?) with the fieldsite which confers upon them an insider status.  These traditional approaches to ethnography often took the form of an 'applied anthropology' intended to create knowledge for colonial administrators about the behaviour of colonised or indigenous communities \citep{baba_end_2005}.  This form of extractive knowledge production positioned colonising forces (such as the Britain or the United States) as a baseline for ‘normality’, interpreting behaviours which exist outside of them as ‘primitive’, ‘backward’, or 'deviant' [cite]. 

The emergence of organisational ethnography considerably expanded the range of acceptable field sites and research subjects for 'applied anthropology', producing a more sociologically-inflected ethnography. Pioneered by Lloyd Warner, the Hawthorne studies of the 1920s and 1930s began to conceptualise the workplace as a society of its own, showing the utility of ethnography as a method for research that did not take place in 'exotic' colonial sites \citep{neyland_organizational_2007}. Organisational ethnographies brought the benefits of the ethnographic method to "bureaucratically structured formal organisations" \citep{watson_making_2012} such as businesses, charities, or community groups, in order to investigate some aspect of their behaviour. One of the greatest strengths of organisational ethnography is in uncovering the mundane content of organisational life, understanding the everyday interactions that underpin processes or systems which might otherwise seem opaque or inacessible \citep{bate_whatever_1997}. For example, in Latour and Woolgar's \textit{Labatory Life} \citep{latour_laboratory_1986}, the authors construct an account of the contingency of scientific knowledge, detailing how seemingly objective scientific ‘facts’ are rooted in their historical moment and circumstances surrounding their creation. More recently, organisational ethnography has begun to consider the importance of theory-building, showing how individual organisations reflect the values, practices, and behaviours of a wider social organisation \cite{watson_making_2012}. 

% add in george marcus multi sited ethnography 
Despite the methodological innovations brought on by organisational ethnographies, they tend to remain bounded by a single spatial site. Although my research has been considerably informed my organisational ethnographic approaches, my work has not been bounded by a singular space. I have conducted research with three different organisations, each based in different parts of the United Kingdom. I \textit{could} consider my fieldsite through these multiple geographies: an ethnography of the functioning of capitalist realism in the children's social care system across multiple geographies, for instance. I might also consider my fieldsite through its temporalities: an ethnography of how organisations in the children's social care system experienced a stretch of time at the end of the 2010s, witnessing the ghosts and legacies of austerity, and finishing with the early days of the COVID-19 pandemic. Yet both of these segment my research artificially, and assume a unity between the different kinds of organisation that I conducted my ethnographies within. The most convincing construction of my fieldsite is instead an ethnography of infrastructure. 

Following \cite{star_ethnography_1999}, ethnographies of infrastructure are multi-sited ethnographies in which the fieldsite is an entire infrastructure rather than something that a spatially or temporally bounded site. As infrastructures are fundamentally relational (do I cite again), ethnographies of infrastructure must follow the relations within the infrastructure to their logical ends. My ethnography has traced the infrastructure of the children’s social care system from the perspective of different charities who support young people perceived to be vulnerable. In order to best understand this infrastructure, I have traced its relations, observing the functioning of the infrastructure from different positions and perspectives. I have worked with care-experienced young people, social workers, youth workers, and managers; I have worked with small, medium-sized, and large charities; I have worked with local authorities and with arms-length organisations. Most importantly, though, as Star recommends, I have “followed my ethnographer’s nose twitch” to determine what to explore next, tracing the phenomena and its relations as it began to become clearer to me. Ethnography has acted as an art of noticing for me, drawing attention to different aspects of the infrastructure, and highlighting alternating moments of harmony and dissonance depending upon how the infrastructure is paid attention to \citep[24]{tsing_mushroom_2017}. It is like turning an object with many sides to try to see all of its aspects.  

Within the different parts of the infrastructure that I have occupied, I have conducted a multitude of focused ethnographies. Focused ethnographies are similar to organisational ethnographies, which may often be conducted part-time, and typically feature shorter visits to fieldsites than conventional ethnographies would, but employ a more intensive set of data collection methods and a greater scrutiny of data through analysis \cite{knoblauch_focused_2005}. Additionally, focused ethnographies tend to deal with specific problems in distinct social contexts \cite{wall_focused_2014}. As such, the use of focused ethnography suited my research well, allowing me to understand the way the infrastructure of the children's social care system operated through specific organisations, rather than interrogating each organisational dynamic that emerged. Here, utilising the grounded theory strategy of theoretical sampling \cite{charmaz_constructing_2006, 96} enabled me to continuously filter and taper what I was collecting data about. 

For each of the organisations I worked with, my approach differed. For one organisation, I based myself in their offices part-time, working there about half of the week in order to observe, help out, and offer support in regards to my specific role in the organisation. In another organisation, I spent around a month at a time embedded in the organisation, living in the same county as them and working from their offices every single day, attending in-person meetings and being physically present to observe their interactions, get to know them, and work on projects with them. Finally, another organisation that I worked with had no physical presence, and instead was a remote working nationally-focused team. The digitally connected nature of this work meant that I never truly ‘left’ the field, despite these periods of 'focused' ethnography. I remained in consistent digital and telephone contact with my collaborators even when I have not been in the field. The care system does not stop existing when I am not observing it; indeed, it seems to have only become ever-more-present. In the present day, my life is wrapped around the care system: I am friends with care-experienced people, youth workers, and social workers; I work within the care system within my freelance design practice; I have helped facilitate the development of a manifesto for care-experienced people to reclaim their community from charities, and I keep in regular contact with ex-participants through WhatsApp and Zoom. My ethnography has not been limited to my ‘focused’ time in the field, nor bounded by spaces or times: I have become entangled with the infrastructure of the care system.

% add in a section about ethnomethodology? about how this drew attention to the practices of participants, rather than just their talk, and the role of objects? 

\subsection{Action}
\label{3-action}
% maybe make section about 'praxis' - is a point that I am making that I created grounded theory from praxis, then use that theory to generate new praxis? 

My entanglement with the care system has also been intensified by my use of action-centered methods, drawing from traditions such as action research, participatory action research and research justice. I have used action-centered methods for both methodological and moral/political reasons. Methodologically, action-centered methods suited my research questions well, as the construction of a meaningful alternative to capitalist realism requires that such alternatives be trialled and understood in practice. Yet morally and politically, the nature of my research questions (or site?) has also meant that it felt inappropriate to me to \textit{not} use action-centered methods. The majority of my research has taken place with people who may be considered to be vulnerable in some way. It would have felt morally bankrupt to observe people, asking questions about their experiences of what may be some of the hardest times of their lives, and then not offer them support, resources, and practical tools to identify changes that they want to make in their worlds and begin to make that change.

Particularly when investigating an infrastructure, ethnography is not sufficient to understand all of its aspects. As Bowker and Star explain in \textit{Sorting Things Out} \citep{bowker_sorting_1999}, ethnography's very nature means that we cannot see what our participants do not see; our observations recreate the exclusions that exist in our participants' worlds, in the same way that their worlds create exclusion by following the functioning of the infrastructure they exist within. Using some form of action-centered methodology, then, helps to address this, by putting participants in new situations, providing them with opportunities to do something about the exclusions that are created by their infrastructures. I have drawn from three different traditions of action-centered research in my work (action research, participatory action research and research justice), drawing on each as it felt appropriate. 

% (footnote? As I go on to develop in the next section, design has also functioned as a form of action but I will not expand upon that here

My guiding tenet in considering action-centered forms of research is Marx's eleventh thesis: 
"the philosophers have previously only interpreted the world. The point, however, is to change it." (Marx Feurbach 1888). 
Citing this well-known adage may seem trite, but it has supported me in identifying the most pertinent aspects of action-centered approaches and methods and unite them in my research - if a method has helped me or my participants to interpret or change the world, it is suitable. I diverge from the eleventh thesis in one way: I maintain that interpretation, observation and description are themselves a form of action or 'change'. Understanding interpretation as a reflection of the world assumes that there is a direct relationship between representations that are created of phenomena and the phenomena itself. This neglects to recognise that all representations are performative and must be actively constructed, as Barad notes in \textit{Meeting the Universe Halfway} \cite{barad_meeting_2007} - representation is always a \textit{doing}. As such, I consider my ethnographic methods a part of my action-centered research practice, and have encouraged my participants to consider description, clarification and articulation a form of action, too.

% the sentence 'three forms of AR' below is new and may need tweaking
This aligns well with a methodology built around action research. Instantiated by Kurt Lewin and developed by thousands of practitioners since, the basic tenet of action research is a spiral of “planning, action, and fact-finding about the result of the action” \citep{lewin_action_1946}. In Lewin’s view, these steps constitute “rational social management”, and can develop a form of research practice which can variously “help the practitioner”, “lead… to social action”, and “research… the conditions and effects of various forms of social action”. These three forms of action research are often confused, leading to disciplinary conflict about what constitutes legitimate action research (cite?). For the purposes of this research, though, I draw upon all three of these forms - reflexive practice, research which builds movements, and research about how to build movements. In conducting action research, practitioners develop their reflexivity, and engage in a process through which they can both learn about their practice and make change in the world. Burns (20XX) summarizes the process of doing action research using the “Kolb cycle” of planning, acting, observing and reflecting, leading into another cycle of planning, acting, observing and reflecting (see figure X). Action research has been extensively used in education, health, and organizational management contexts and can be used to situationally understand problems and develop practice-based solutions to these.

% the swantz reference is crap and needs changing in zotero - the chapter is supposedly 2008 in a 2006 book???? but its 2007??? wtf 
As construed by Lewin and operationalised by the Tavistock Institute, early action research often acted as an experimental research method that could support organisations to operate more efficiently\cite{neumann_kurt_2005}. Although action research in the Tavistock's work (and reflected in their journal, \textit{Human Relations} was essential to the development of STS methods and practices, it did not contain an explicit political framing except the loose idea of 'social action' that Lewin had proposed. In contrast to this, participatory action research (PAR) began to be developed as a research approach explicitly founded upon the needs and liberation of "the oppressed", whoever they might be, and. PAR committed to a liberatory knowledge-making practice\cite{reason_handbook_2006}. This same set of tensions and priorities also gave rise to critical pedagogy and participatory design.

In contrast to action research, PAR is more explicitly concerned with the community-based and collaborative inquiry. Fals Borda \citep{2001} identifies three central challenges of PAR at the time of its development: the relations between science, knowledge and reason; the dialectics of theory and practice; and the tensions between subject and object. PAR seeks to blur the boundaries between each for these areas.   It drew upon and transcended a variety of disciplines and methodological approaches, such as social psychology, Marxism, anarchism, phenomenology and classical theories of participation. PAR centers the idea of science as socially constructed, and used Lewin’s action research as a way to justify itself as a legitimate practice and methodology [ref fals borda?]. Moreover, it calls for a breakdown between subject and object in research, viewing those involved in research as ‘thinking-feeling-persons’ - or humans - rather than merely considering them for their participation (i.e. ‘participants’). As such, participatory action research is committed to collaborative inquiry around the interests of the oppressed, facilitating people to explore questions that matter to them and their own lived experience, towards liberation. 	

The work of Paulo Freire has also been instrumental in developing PAR methods. Though PAR and Freire's critical pedagogy are different, they mutually inform each other and share a common root. Freire argues that the dominant model of education is a banking model, where the action afforded to a learner is only "as far as receiving, filing, and storing the deposits" \cite{freire_pedagogy_2000, 72}, reproducing attitudes and practices which mirror the oppression within society at large. Such a pedagogy prevents the development of critical consciousness, a state whereby people are able to relate problems to structural causes, and "create a new situation" through "transforming action" \cite{freire_pedagogy_2000, 47}. Critical pedagogy is therefore easily aligned to action research and PAR, using similar cycles of action and reflection and centering the needs of those with lived experience of oppression (Serpa et al., 2020). For Freire, these cycles of "reflection and action upon the world in order to transform it" \cite{freire_pedagogy_2000, 51} constitute praxis, a situated and theoretically-driven form of action. However, this theory-driven action is not a strict application of theory, but an enmeshment of theory and practice. Deleuze posits that theory and praxis have a "fragmentary and partial" relationship to each other, whereby praxis is a "network of relays from one theoretical point to another" and theory "relays one praxis to another" \cite{deleuze_desert_2004, 206}. Theory and praxis are mutually constitutive and are constantly changing one another.

Research justice is a framework for conducting research that advocates for research praxis centered on liberation and justice. It draws attention to the ways that the dominant modes of defining 'legitimate' knowledge are themselves oppressive, operating from "a paternalistic position of assumed superiority that has been unsuccessful in producing meaningful reforms and social justice for indigenous nations and communities of color" \cite{jolivette_research_2015, location 287}. Research justice therefore draws attention towards underrepresented forms of knowledge (such as experiential, cultural and spiritual knowledge traditions) \cite{asad_prefigurative_2019, 4}, and makes clear that invalidating the legitimacy of these knowledge traditions creates harm for many communities that are often the subjects of academic knowledge production. In response to this, research justice proposes a research praxis that facilitates:

\begin{itemize}
\item access to information (not just misinformation and outside expert research but what they truly seek and deserve) that impacts their lives; 
\item ability to define what is valid ‘knowledge,' as well as methods to produce this; 
\item capacity to produce their own knowledge; 
\item capacity to use all forms of knowledge; 
\item and control over all stages of the ‘knowledge lifecycle'—from producing, analyzing, interpreting, packaging and deploying knowledge—on an equal footing with all other institutions in society. \cite{jolivette_research_2015 location 288}
\end{itemize}

Research justice asks us to consider a research praxis that asks what \textit{else} research can be other than the production of knowledge. In this way, researchers can become academic 'accomplices' (Asad, academic accomplices 2019) that support communities and challenge the structures of injustice and oppression created by much research practice.  Research justice can therefore be prefigurative, focused on the transformation of social relationships (towards more just, liberating and healing relationships), redistributing material, skills-based, and network-based resources to those who need or want it, and building counter-structures that can replace the "harmful institutions, policies and practices that impact our community collaborators" \cite{asad_prefigurative_2019 13}.

My research has blended together these different action-centered approaches to follow a just, liberating and healing research praxis. I have worked to embrace action research in all three of Lewin's original forms - as reflexivity, movement-building, and research about how to build movements. I am committed to acting as an academic accomplice, placing the needs of a community first, with my research following those needs, attempting to dismantle power hierarchies, redress historic inequalities, and change structural power dynamics in the process. Finally, I have attempted to support the people that I have worked with to create change in whatever ways felt appropriate to them. This has included developing tools, resources and encounters through my research that either help them to articulate the issues that exist for them under austerity-intensified capitalist realism, or to imagine and begin to construct an alternative to it. This has created some tensions at times, which are explored more fully in \ref{3-ethics}. 
% actually should be 3-ethics-critical-ethnography or something

My research is characterised by the mutual enmeshment of observation and action (and in turn, theory and practice) as praxis. Although I have worked with managers, frontline workers and young people, I have tried to listen to and support their needs according to who holds the least power at any one given time. For example, the needs of care-experienced young people (e.g. "I need this form of support") may run counter to the needs of frontline workers (e.g. "I need this visit to be easy"), which may further run counter to the needs of managers (e.g. "I need my workers to be proactive and young people to feel supported in the ways that our project plan set out"). In this case, I have always worked towards the needs of young people; in the case of workers and managers alone, I have worked towards the needs of workers. 

In practice, my research process has followed this general pattern:
\begin{enumerate}
\item Identifying an organisation that might be suitable to become embedded within,
\item Making contact with the organisation and agreeing in principle that I can conduct research inside of it,
\item Observing the organisation that I am a part of, becoming a part of its daily functioning,
\item Identifying alongside workers and managers ways that their work practices are counter to what they would like,
\item Making contact with care-experienced young people that use the organisation's services,
\item Talking to care-experienced young people in more depth, identifying the ways that their care-experience has been counter to what they would like,
\item Developing a design intervention on the basis of all of these factors.
\item Observing the organisation in the aftermath of the design intervention.
\end{enumerate}

It is important to note here that for the purposes of length, I focus on only two design interventions in depth in this thesis. I conducted three more design interventions in my time with these organisations, focusing on different aspects of participatory methods - participatory evaluation, participatory user experience design, and  participatory filmmaking. The two projects that are detailed in depth - \textit{It's Our Future} and \textit{fractured signals} - are exemplary of my general research approach. As the only aspect of this process that I have not yet turned towards, then, I will now address the ways that design functions within my methodology. 

\subsection{Design}
\label{3-design}

% what im trying to say is that O A and D are all important aspects, and that design both reflects and prompts further action 
Observation, action and design have been interwoven and complementary methodologies since the emergence of participatory design practices (see e.g. Greenbaum setting the stage). The three methodologies can be seen as constructing or reflecting different aspects of people's experiences: for example, Sanders (2002 from UCD) suggests that what people say, do, and make (corresponding roughly to observation, action, and design) each communicate different kinds of knowledge  - explicit, observable, tacit, or latent knowledge, in turn. Yet design is unique, as it is its own form of knowledge creation and also a specific form of action. As with the Kolb cycle of action research - in which future plans are based upon reflection around a previous cycle of action - designed artifacts are both an embodiment of knowledge and a catalyst for further knowledge creation. Designed Things are therefore both mediating artifacts and meaningful tools within a user's practice (Greenbaum setting the stage) which can tell us more about their social world and the semiotics of materials within their world. 
% in this section I briefly review the design approaches that have influenced my own design praxis. 

\subsubsection{The connections of design to observation and action}

% Design and observation - design anthropology/ethn, RtD, design inquiry
* design has a history tied up with anthopology/ethnography -\textbf{ design as a way of knowing }- RtD, design inquiry
% 	Design inquiry sits closely alongside design ethnography as an approach to ROTKOHLROTKOHL 
% More recent work has focused on…. +ref Reworking the Gaps between Design and Ethnography (Khovanskaya et al., 2017)
% 	Design ethnography draws attention to the process of constructing designed artefacts and Things [cite]. They might be seen as a synthesis of observation and action. Observations about a phenomena can be instilled in designed artefacts through action. In this way,

Research around the design and use of information systems and digital technologies has a rich history of using design and action-oriented research approaches in complementary ways. Beginning in the early 1970s and crystallizing into an emergent design and research agenda in the early 1980s [cite?], co-operative design developed out of technology research and design in Scandinavian countries. In its earliest form, the 'collective resource' approach, this most often took place with workers and trade unions: including the Nordic Iron and Metal Worker's Union (Nygaard 1979), various Swedish and Danish trade unions (in DEMOS and DUE projects respectively),  the Nordic Graphic Workers' Union (in the UTOPIA project, (Ehn and Kyng, Computers and Democracy 1987; Sundblad 2010)), and with Norwegian nurses (in the Florence project, Bjerknes and Bratteteig 1995). Eventually, co-operative design became a formalised research agenda and design methodology based around the empowerment of users to 'fuller participation and cooperation' (Setting the Stage). In this co-operative design approach, 'doing became a primary activity' with the interactive use of interface mock-ups by designers and participants helping to identify key moments of breakdown in a system's design (setting the stage, again). This approach suggested that design itself is a form of action which surfaces a specific set of knowledges. Moreover, although this approach used methods similar to action research, it considered itself distinct because action research 'bends more towards outcomes' where co-operative design explicitly foregrounds the value of creation (through design), and shapes both 'the process and the product' through the application of design methods (A design of one's own). 

\subsubsection{Participatory design and politics}

As co-operative design practice disseminated outward from Scandinavian countries, it began to find a base in the United States in the form of 'participatory design'. Participatory design was viewed as an adaptation of the processes and methods of co-operative design to a non-Scandinavian context. Compared to Scandinavia, for example, the United States experienced comparatively low levels of unionization and an increasingly hostile climate for the building of workplace democracy under the Reagan administration (A design of one's own). In this context, then, PD advocates suggested that American participatory design required 'home-grown characteristics' (a design of one's own), which focused on 'more active involvement of users and developers in the design process' rather than 'alter[ing] the power relations' (a design of one's own) - which might slowly build a movement for workplace democracy, essentially the inverse of the Scandinavian practice.

A set of principles are common to both co-operative design and participatory design. Though these may not be explicitly stated, they form the outline of what co-operative or participatory design praxis and projects are oriented towards. Adapted from Greenbaum and Loi (2010), these principles are:
\begin{itemize}
\item equalising power relations - developing methods of supporting those who are marginalised, oppressed, or less powerful,
\item situated based actions - working in deeply embedded ways, directly with people, in the context that is being designed for,
\item mutual learning - in which participants and designers find common ground and exchange skills and experiences,
\item tools and techniques - that help participants to express their needs, visions of the future, or tacit knowledge,
\item alternative visions about technology - which challenge the dominant or conventional perspectives or imaginaries around technology is, can be, or should be, and
\item democratic practices -the building of equity, democracy and justice both within the context of design and society more generally.
\end{itemize}

Since the development of participatory design in the United States, there have been concerns about it losing its political foundations. Bjerknes and Bratteteig called for a renewed emphasis on the issue of democracy within systems design in their 1995 paper 'User Participation and Democracy' (B/B 1995), and Beck's 2002 paper "P for Political" suggested that 'participation is not a sufficient condition for changing power relations' as 'forms of participation exist and presently thrive that do not question, but further, dominant power patterns around the development of IT' (Beck, 2002). These issues have persisted - and perhaps intensified - into contemporary participatory design practice. In an analysis of the content of presented at the \textit{Participatory Design Conference} from in its inception to 2012 (periodized as 1990-1998, 2002-2012)\cite{halskov_diversity_2015, basballe_early_2016}, Halskov, Hansen and Basballe identify a growing subtlety to the political content of papers  slowly transforming from an explicit politics to an improvement or optimisation of a specific experience for participants. In 2018, Bodker and Kyng identified that a great deal of participatory design work was beginning to focus on the here-and-now, had low technological ambitions, consisted of 'do-gooding', and reduced politics merely to ethics (Bodker and Kyng 2018). They suggested a repoliticisation of participatory design - to focus on participatory design that 'matters', centered on areas where 'dramatic, potentially negative, changes are underway', in which researcher-activists co-operate with partners to develop the research agenda and create a high impact vision to counteract that change, and where that impact is safeguarded through democratic control \cite{bodker_participatory_2018}. 

\subsubsection{Critical and prefigurative design}

Yet participatory design is that it is not the only political form of design. Most often, participatory design refers to design projects which employ an action research approach - and which may either reach towards some radical social goal [cite an example?], or which foregrounds issues of power and participation in its composition or process \cite{vines_configuring_2013}. Other forms of political design take a different approach to instantiating their politics. Some approaches (such as prefigurative design, design justice and anarchist HCI) seek to foreground the development of practical methods to bring about social change, whilst others focus on what the construction of artefacts can uniquely bring to a political design practice (such as critical design and adversarial design).  These approaches are not sufficient in isolation, but developing a design praxis which incorporates these and the lessons of participatory design can provide a foundation for meaningful social change to occur through the linked processes of observation, action, and design.   

Critical, and adversarial design are approaches which pay specific attention to the production of artefacts which can themselves produce further action in the world. In contrast to commercial or industrial design practice, critical design exists to explore the possibility of using design processes, interactions, and methods for non-commercial or product-driven purposes, to 'stimulate discussion and debate amongst designers, industry, and the public' (Dunne and Raby 2001, 58) through the process of materializing and experiencing them (Malpass 2019). In doing so, critical designs seek to problematize current social and political formations through the creation of artefacts which embody alternative formations of the world \cite{bardzell_reading_2014}. These designs are not merely critiques of existing social worlds, but can be constructive attempts to build something new by creating defamiliarizing experiences for those that interact with the artefact \cite{blythe_imaginary_2018}. In a similar vein, adversarial design advocates for the production of artefacts which 'call attention to the contestational relations and experiences aroused through the design thing and the way it expresses dissensus' \cite{disalvo_adversarial_2012}. Whilst many critical designs are adversarial, adversarial design draws specific attention to the way designed artefacts assemble an agonistic politics, potentially capable of challenging hegemony (for example, the hegemony of capitalist realism) through the creation of novel articulations of social issues. 

In contrast to the focus on artefacts within critical and adversarial design, prefigurative design instead suggests that the politics and sociotechnical arrangement of the world after a desired social change has been made can be achieved through the design \textit{process} itself \cite{asad_creating_2017}. More generally, prefiguration refers to how political movements can embody the 'forms of social relations, decision-making, culture and human experience' that are the 'ultimate goal' of the movement through their 'ongoing political practice' \cite{boggs_marxism_1977, 7}. Prefigurative design, then, builds on this approach but emphasizes a deliberate commitment to materiality, too - design should not only raise awareness of systemic injustice, as in critical or adversarial design, but 'actively address and challenge them' \cite{asad_prefigurative_2018}. Prefigurative design is grounded in anarchist politics and is thus closely related to anarchist conceptions of human-computer interaction, which foreground questions of how to embed anti-oppression, liberation and dignity through HCI research, and ultimately build counter-power to contest oppressive structures \cite{keyes_human-computer_2019,asad_prefigurative_2019}. 

Prefigurative design therefore acts as a form of anti-oppressive design (Smyth and Dimond 2014), in which designers seek to create social justice and counter the efforts of the matrix of domination \cite{collins_black_2002}. In this way, prefigurative design also rests upon the design justice principles, developed by the Design Justice Network (and explored in depth in Sasha Costanza-Chock's \textit{Design Justice} (Costanza-Chock, 2020). These principles in many ways solidify and update the ideals of participatory, critical and prefigurative design, focusing on: 
\begin{itemize}
\item sustaining, healing, and empowering communities, seek liberation from exploitative and oppressive systems,
\item centering the voices of those directly impacted,
\item prioritizing design's impact on the community,
\item viewing change as emergent from an accountable, accessible, and collaborative process,
\item understanding the designer as facilitator,
\item viewing everyone as an expert based on their own lived experience,
\item sharing design knowledge and tools with communities,
\item seeking sustainable, community-led and controlled outcomes,
\item working towards non-exploitative solutions that reconnect us to the earth and to each other, and
\item looking for what is already working in a community before seeking new design solutions (Design Justice Network Principles).
\end{itemize}

\subsubsection{Towards a speculative design praxis}

Another form of constructive-and-critical design also bears relevance to these approaches: speculative design. Speculation as a practice is anchored in the imagining of previously unanticipated futures (cite 41 in Cally's paper), so that 'futures and past might manifest themselves differently' \cite{gatehouse_hauntology_2020}. Speculative design then is centered around 'critical reflection through future narratives... often mediated through objects' \cite{forlano_ethnographies_2013}. In many ways, speculative design functions as a development of critical design, amending the critical design practice to specifically focus on how to 'collectively define a preferable future' \cite{dunne_speculative_2013, 6}, lending itself to the design praxis that is being assembled in this section. This form of speculative design also makes the most of design's discursive and adversarial potentials, dealing with the ambiguous and liminal space of the potentially-fictional, exploring how audiences make sense of 'hypothetical possibilities... utopian concepts and dystopian counter-products' \cite{auger_speculative_2013}.

Yet this form of speculative design is still somewhat limited, often functioning as critical design by another name. As suggested in 'A Hauntology of Participatory Speculation' and elaborated upon in her PhD thesis, Cally Gatehouse explores the potentials of speculative design to be more than a reoriented critical design, acting as a diffractive gaze that notices 'how difference is made by coming into close contact with the world', 'how our practices emerge from our entanglement with world', and how 'pay[ing] attention to the details' helps designers to find 'ways in which they can make a difference' (Gatehouse, 20XX, thesis). Conceptualising speculative design as a diffractive gaze starts to indicate how design praxis can respond to the issue of capitalist realism: noticing 'small perceptual shifts, momentary glimpses of other ways in which things could matter' and creating openings, spaces of possibility centered around these diffractions (Gatehouse, 20XX, thesis). 

Understood through the lens of diffraction, speculation and speculative design are deeply embodied practices, that begin to transform the world just by engaging in them. Speculation's embodied and action-oriented nature has led to the development of specific speculative design approaches that consider the role of experience. Speculative enactment, for example, focuses on an individual's experience of and interaction with a given speculative setting (Elsden, 2017), and focuses on the meaningful social and emotional outcomes that a participant in speculation might experience through their interaction with speculative materials and settings. For Elsden et al., these speculative enactments achieve an authentic outcome within the somewhat-fictional space of a speculative encounter. Similarly, for DiSalvo, speculative interventions offer an opportunity for participants in speculation to experiment with different potentials, to support them in determining whether they might to live in a given future, or whether they are comfortable with what that future might suggest about their present \cite{disalvo_irony_2016: 140}.

My overall data collection approach, then, can be described as a speculative praxis. It is built around the three prongs of observation, action and design that I have described, paying special attention to issues of power, participation, liberation and healing.  


% Because of speculative design's ability to allow people to materially and experientally experiment with different potentials, then, I have adopted a speculative \textit{praxis} in my work. This builds on all other aspects of my methodology:



% slippages as speculative gaze?!


% prefigurative speculative participatory speculation = speculative praxis 

% +how form can be radical – ‘form itself is alterity’ re: Malabou 

  
% move towardss Thinging, infrastructuring and service design 
% In this context, practitioners of early PD have begun to adopt a new approach:
% Things - 'assembly around matters of concern'  - design thus acts as a synthesis of observation, action, expeirnece and other knowledges - a designed artifact is the assemblage of all of the things that glue that matter of concern together; the people, the ideas, the products, the places
% this links well to deleuze's notion of the assemblage...
% matters of care? Bellacasa

% service design?
% Increasingly, practitioners are calling for a unification or meeting of participatory design and service design...
% SD uses UCD methods...
% As mentioned in \ref{the-digital-turn}, one of the central elements of the digitalisation agenda within the UK Government was the promotion of "a service culture" (Lane Fox, 2010), which sought to install service design approaches as standard practices in the research, design, and delivery of public services. Arguably, this can be seen as the realisation of a


% =* these speculative approaches are vastly different from the normative agile design approaches frequently taken within service design 




% As I have already mentioned, service design has become increasingly important with the advent of digitalisation initiatives and the onset of austerity. I have used design-centric methods as part of my research process; viewed in its broadest possible sense, I have used design as a practice of planning, facilitating, infrastructuring, reifying and *making* things [cite] or ideas which have supported the people I have worked with to imagine alternatives and attempt to instantiate them. This kind of making is not purely about the construction of artefacts, but a way of understanding the world differently through design. This neatly maps onto the concepts of research through design and design inquiry, both of which see design as a valuable research tool. Research through design is ROTKOHLROTKOHL whilst design inquiry is ROTKOHLROTKOHL. 







\section{Data analysis (maybe incl something on material data collection)}
\label{3-analysis}
As explained in \ref{3-collection}(?), my approach to observation, action and design has also been influenced by my approach to data analysis - in particular, Charmazian grounded theory. Grounded theory is a series of "systematic, yet flexible guidelines for collecting and analyzing qualitative data to construct theories 'grounded' in the data themselves" \cite[3]{charmaz_constructing_2006}. I selected grounded theory as my method of data analysis early into conducting this research, as its ability to generate high-level, explanatory theories seemed appropriate for the kinds of research questions I was asking. Throughout my (iterative) data analysis processes, I have found the structures of grounded theory incredibly liberating as a way to make clear the relationships between individuals' experiences and the systems and structures which create the conditions for these experiences to arise within. 

Grounded theory began as a methodological innovation to explore the experiences of people who are dying (cite awareness of dying) - particularly as at the time, discussions of death and dying were avoided with terminally ill patients. Grounded theory represented a change in the efforts of sociological theory, from the verification of existing theories and hypotheses towards the generation of new theory directly from data grounded in the lived experiences of people (cite discovery of grounded theory 1967). This early work by Glaser and Strauss stood in stark difference to the quantitative research which was becoming much more common at the time, and grounded theory represented a first attempt to defend qualitative research against the claims that it is anecdotal, biased, or not rigorous. In any of its various forms, grounded theory remains rigorous because of its incredibly systematic methods for collecting, analysing, sorting, writing, and theorizing about data. 

Glaser and Strauss took the methodology in divergent directions, with Strauss later working with Juliet Corbin. In distinction to both of these approaches - which can be at times prescriptive - Kathy Charmaz's flavour of grounded theory emphasises flexibility. Instead of "rules, recipes, and requirements", Charmazian grounded theory focuses on a set of guidelines for constructing quality grounded theory that can complement the use of other qualitative methods  \cite{charmaz_constructing_2006, 10}. Because of my own methodological focuses on observation, action and design, Charmazian grounded theory was the most suitable form of grounded theory for me to use, and it allowed the flexibility that my multi-sited multi-method research required. Specifically, Charmazian grounded theory works from a symbolic interactionist perspective, which understands theory to be constructed through interactions with "people, perspectives, and research practices" and prioritises the ways that research participants represent their experiences in language as a central aspect of theory-building. Charmaz refers to this as an \textit{interpretive} portrayal of the world, avoiding the 'God-trick' of assumed objectivity present in many other theoretical works (Haraway, ??). This focus on language and lived experience was a good match for my work due to its participatory nature. 

Moving beyond just an interpretive interpretive portrayal of the world, though, I have also relied upon material-semiotic approaches (such as those used within the new materialisms and actor-network theory) to examine the interplay of actions, artefacts, things and symbols with people's representations of the world. Different material-semiotic approaches use different language for how they discuss about the agency of actions, artefacts, things and symbols, but each emphasise the power of non-human assemblages to have a drastic impact on the world and events within it. In Karen Barad's work, for example, this takes the form of 'agential realism', which shifts focus from the nature of representations towards the nature of discursive practices (Barad, 20XX, 45). Even events or practices which appear to be merely material are thus understood in their performative and discursive aspects. For Jane Bennett, this takes the form of the 'vital materialism' of 'thing-power' - the capacity of "inanimate things to animate, to act, to produce affects dramatic and subtle" (Bennett, 6). Finally,  for Bruno Latour, this agency takes the form of 'actor-networks', which emphasises the relations between actor-networks and their role as mediators that can transform or translate what they come into contact with. I do not rely prescriptively on one of these approaches, but instead the unified focus on the power of actions, artefacts, things and symbols as they feature in my research. 

% 46 106 131-136 179 Reassmelbing the social 
% instead of attempting to seek causal explanations, acotr-network theory seeks incredibly thick descriptions which seek to describe everything within the actor-entwork; to add something outside this as an 'explanation' is merely to understand that the description has not gone far enough 

The key components of a grounded theory study are:
\begin{itemize}
    \item simultaneous involvement in data collection and analysis,
    \item the development of analytic codes and categories from data (rather than codes deducted from a preconceived hypothesis),
    \item the use of the constant comparative method to make comparisons between data, incidents, and phenomena,
    \item each stage of data collection and analysis driving the development of a theory forward,
    \item the writing of memos to elaborate categories and themes within the data, to define relationships between different elements of those themes, and identify gaps for the next iteration of data collection,
    \item theoretical sampling, wherein the data is sampled for clearer/fuller development of the theory rather than representation of a given population,
    \item conducting the literature review after the fieldwork and theory development has already got underway. (adapted from \cite[5-6]{charmaz_constructing_2006}.
\end{itemize}
 

My research process has closely followed Charmaz's guide here, with amendments only for methodoloigcal specifics. Through the methods described in \ref{3-collection}, I collected data including fieldnotes, semi-structured or unstructured interviews, photographs, constructed or designed artefacts, diagrams, and annotated paperwork (examples of each can be found in Appendix CHANGEME). Data was partially inductively analysed in the field to inform further data collection within that field visit, and then analysed in greater detail at the conclusion of each field visit where it could be more easily integrated into existing data. Both of these constituted their own action-reflection cycles, with the post-field visit analysis forming the foundation for the next field visit or design intervention. Throughout the collection and analysis processes, I also developed a number of diagrams or design artefacts to help make sense of what I was analysing. In grounded theory terms, this could be considered a kind of design-led memo-writing process, in which I used the techniques of design to better clarify aspects of the emergent theory - such as relationships between actors, the influence of certain actions, or to evaluate the efficacy of methods used. Throughout my data collection and analysis processes, I made use of the constant comparative method to make sense of how new data diverged from previously collected data, to make comparisons between the different organisations within my research, and to isolate specific instances of unique behaviours or phenomena.

Along with the design-led memos, I wrote more conventional grounded theory memos consisting of thick descriptions and to elaborate on relationships between people, things, or phenomena. After the conclusion of the majority of field work, I conducted a more thorough analysis on all of the collected data and began theoretically sampling in order to clarify the emergent theory that I was developing. In practice, I went through my entire corpus of data and wrote a post-it note for each individual unit of meaning. I managed to clarify a number of initial categories with some memos to explain relationships and impacts. I then transferred these data, codes and memos to the digital whiteboard tool Miro to get a better sense of the relationship between different data and concepts within the theory. This enabled me to develop more sophisticated memos as I could quickly move between different scales - the micro level of personal experiences and interactions and the meso level of system-wide changes. Having developed my theory (of justification practices, elaborated in \ref{6}), I brought this to participants to see how they worked with it and if it helped them to make sense of their experiences, an essential aspect of participatory research. Every participant I presented the theory to immediately understood the content of the theory, feeling that it reflected their experiences and helped to explain some of their experiences over the last few years.

The grounded theory of justification practices was used to develop the design interventions in this research, with each of the methods developed intended to target an aspect of the phenomena specifically. This featured in nascent form within It's Our Future; this allowed the insights gained from using the methods in It's Our Future and the further development that the theory underwent to allow fractured signals to be a highly targeted intervention. This process is described in greater detail in \ref{8}. Finally, further observation and action-centered data was collected during these design interventions, at which point data saturation was reached, as the behaviour of things relating to the theory could be reliably understood and no novel aspects could be identified. 

% maybe clarify data saturation

% do i need to add something about in vivo coding 

% is there something about me being an outcome of the process - honing my instincts as a tool of data analysis? 

\section{Ethics, liberation and healing}
\label{3-ethics}
Ethical practice is an essential aspect of any research approach but is often under-valued in writing about research [cite]. To appropriately foreground this importance, I am turning towards the ethical practice underpinning my research first. In order to discuss this ethical practice, though, I must first give a brief summary of the fieldwork I have undertaken. This is covered in more depth in chapter four, but is discussed here to illustrate the broad shape of my work and to give context to some of the ethical decisions I have made.

From 2018 to 2020, I worked with three organisations that provide support services to children and young people that they perceived to be vulnerable. I maintained an ongoing relationship with these organisations, based in different places around the United Kingdom, throughout the research (until one of these organisations closed, and the pressures brought on by the COVID-19 pandemic meant my relationship with another wound down). I was embedded in these organisations as a critical ethnographer, a participatory action researcher, and a participatory designer. I was seeking to understand the ways that austerity-intensified capitalist realism had affected them and changed how they *do* the care work of supporting children and young people, and how that had changed young people’s experiences of care. Simultaneously, I was designing projects and provocations alongside them that would help them and the young people they work with to imagine different futures.  Across these three years I conducted hundreds of hours of research and design with these organisations, including 14 months of focused ethnography, more than 20 design workshops, five weekend-long residentials, six distinct design projects, and around a dozen interviews, culminating in over 1500 pages of fieldnotes.  

In this section I detail some of the ethical tensions I have encountered in my research, and ways that I have responded to these. These responses have not always been adequate; indeed, as chapter four details, some of these ethical tensions constitute ‘slippages’ which *cannot* be ethically reconciled with the state of knowledge production in contemporary academia. Although these responses have not always been adequate, however, they have been the best I could do at that moment, always centering care and arising out of a  commitment to being present and connected in the ways that the people I was working with needed at the time. It is important to note though that meaningfully ethical research practice with people who are oppressed and who have experienced trauma must go beyond the regular commitments of ethical practice, into liberation and healing. In this section, I first deal with the more general ethical concerns -  separated into those that arise during the course of *doing* research, and those that arise in the course of writing about  research. Then, I turn towards the need for commitments to a liberation and healing-centred research practice in addition to a generalised ethical practice.

\subsection{Doing ethical research with oppressed or traumatised people}
I have conducted my research with people who exist at the intersection of multiple perceived vulnerabilities, working with children and young people, people with experience of the social care system, and working with people who might exist precariously due to the the way their organisation employs them. These people often have experience of other things  perceived as vulnerabilities, such as homelessness,  addiction or drug use, parental or intimate partner abuse, mental health problems, refugee status, neurodivergence or disability. This is not to say that any of these experiences or identities do *necessarily* render the people that I have worked with any more vulnerable; the organisations I have worked with, however, would certainly perceive them to be vulnerable by virtue of these experiences and identities. As mentioned in the previous chapter, vulnerability is an incredibly wide category that expands as the behaviour of those classified as vulnerable changes, and so I am not seeking to perpetuate this classification here. What is important to the conduct of my work is that the organisations I have worked with perceive these people as vulnerable. 

Conducting research alongside perceived vulnerability comes with a multitude of increased ethical questions at both the procedural level, and the practical or relational level. The research described in this thesis followed Newcastle University’s guidelines for ethical research and received institutional ethical approval by the Faculty of Arts and Humanities. As part of this process, I was made to consider a large number of potential vulnerabilities and the ways that my research might interface with these, such as whether or not my participants would be children (they were), required gatekeepers (they did), whether sensitive topics would be discussed (they were), whether participants would experience prolonged or repetitive testing (a question that is impossible to answer for embedded researchers), or whether participants would experience pain or more than mild discomfort (they didn’t). Although responding to these questions sufficiently is important, as Sean Peacock and I mentioned in our 2021 paper “Making sense of slippages”, procedural ethics systems often function mostly as risk classification processes (Cutting and Peacock, 2021). It is not enough to simply consider a risk classification process sufficient for having ‘done’ ethics; these potential vulnerabilities must be considered from a practical and relational perspective that acknowledges the mess of embedded and participatory research.

The majority of the participants I conducted research with were children and young people, though almost all were over the age of 16. I began working with these children and young people through my partner organisations, and I always followed my partner organisations’ lead when it came to the establishment of what ethical practice should look like in that circumstance. For example, my consent forms contained no space for the consent of a parent or carer; each organisation that I worked with acted on the belief that a parent or carer had given consent for their child to attend the organisation’s activities (in writing), so any activity that the organisation deemed to be in that child’s interest was consented to by their parents or carers. This left the actual decision of consent and participation to the young people themselves, creating a rare space of direct empowerment in children’s social care - where what a young person *wants* to happen happens. If they want to participate, they can; if they don’t, they won’t. To recruit participants, for example, a member of the organisation would explain to potential participants what the research would entail and why they might want to participate (usually the ability to develop a skill, and use their experiences to help make or design something, along with whatever support the organisation would usually offer). Then, at the beginning of a project, I would introduce the research or design process, what we were doing, and why, and offer opportunities to ask questions. Once I was certain that everyone understood the project, the contents of the information sheet, and the contents of the consent form, each participant signed the consent form, dated it, and provided a pseudonym of their own choosing. At any point, a participant could withdraw consent - or simply choose not to participate in certain activities. During a film-based project, for example, one participant chose not to participate because she had barely slept the night before and was becoming irritated by the other attendees. She took a nap and then rejoined the group later in the day. After they had participated in my research, participants were given a window of three months in which they could contact me to have their data removed from the project. They were also told that if they wanted to, they could contact me after the three months and I would do my best to remove their data from the project, but that no assurances could be made at this point. 

Newcastle University’s institutional ethical approval processes draws attention towards the discussion of ‘sensitive topics’. In many ways, this was the intention of my research - discussing experiences that may have been difficult, particularly pertaining to living or working in the children’s social care system. The implication of questions such as these are that sensitive topics should be avoided, or if it is unavoidable, that the risks around the discussion of sensitive topics should be mitigated. Rather than eliminating risk, however, I wish to focus on an alternate value set here: care and the possibility of healing. As engagement in my research might lead to discussion of topics that participants struggle to talk about, it felt important to me not just to let people leave the research or withdraw consent but to actively create spaces that they felt comfortable, safe, and cared for within. **could be a tangent: In facilitation practice, there is the idea of a container…** Though I do not focus on it in particular throughout the bulk of this thesis, most of my participants actively felt that participating in my research gave them an ability to advocate for themselves and self-reflect in a way that they didn’t often have an opportunity to do, in large part *because* they could talk about ‘sensitive topics’ - such as homelessness, the difficulty of doing care work when things are precarious in your personal life, or the difficulties of running a large charity  - without fear of judgment or reprisal. 

I conducted my research in a trauma-centered way. As I will expand on later in the thesis, I am sceptical of the sudden dominance of so-called ‘trauma-informed’ approaches, which often are provided in the form of single day trainings that allow an organisation to claim status as ‘trauma-informed’ without interrogating the ways in which their own practice might need to change in order to change the experiences of the people they work with. When I speak of a trauma-centred methodology, I am talking about the shaping of a research program that creates safe spaces (**containers**) for people who might (or might not) have lived experience of trauma to reflect on their experiences. This enhances our conceptions of what care, consent, and consideration might look like in the research process. In my case, when I was conducting interviews, for example, I always encouraged participants to choose environments that they felt comfortable and safe in. Any questions which were centered on potentially difficult experiences were always framed in open, participant-centered language, such as “If you feel comfortable talking about it, I’d love to hear more about what that was like for you”. Framing questions in this way allowed participants to structure the flow of the research, diving deeper into things they felt comfortable discussing and painting only broad strokes of things that they didn’t. 

Both academia and trauma-informed approaches (particularly as they were employed in the organisations I conducted research within) favour an approach to trauma, abuse and safeguarding that explicitly tries to avoid disclosure. It is suggested that we encourage that participants with lived experiences of trauma think outside of themselves when answering our questions or engaging in activities - to avoid participants saying ‘well, when this happened to me, I…’. This approach is fundamentally unethical. It stems from a desire to not pressure participants into sharing their experiences, but it removes the ability to potentially experience some healing through sharing your story with others, which can be of huge benefit to people with trauma experiences (ref). It removes the opportunity for rich detail that people may give when reflecting on their *own* lives, which is a central benefit of doing phenomenological work. This stems from a desire to avoid legal responsibility and culpability: organisations such as charities and universities would prefer to avoid dealing with a safeguarding disclosure, pretending that if the disclosure doesn’t happen on their watch then there might not be any unsafe things going on. (Is there a bell hooks quote on silence?)

Another frequent issue in workplace ethnographies is the perception that the ethnographer might be a spy for management (ref). This is particularly exacerbated in larger organisations or situations where the trust between management and general staff is weak, and the ethnographer has joined the organisation at the behest of management. In my place, this was not an issue. With one of my organisations, the person who contacted me and who was my main relationship in the organisation was a general worker, who I quickly built a trusting relationship with. This happened in another of the organisations to, and though that person did not remain my central partner throughout my work, the ‘endorsement’ of this worker meant that my relationship with this partner started with an assumption of trust. In one organisation, I was brought in through the management end of things, but in this situation the team did not yet exist, so I was present for the formation of the team and again, I quickly built a relationship with the general worker who was going to act as the co-ordinator. All of the people that I worked with knew that I was going to take a highly anonymised approach to writing about this data (highlighted later in this chapter), and had my complete confidence that anything they said wouldn’t travel to any other member of staff - particularly their manager. All interviews and workshops with workers took place in settings that either deliberately *excluded* their managers - in order to create safety for them to speak about their working conditions - or deliberately *included* them, in order to allow people to discuss things they wanted to change about their work in a safe way.  

There is a huge ethical tension around the use of incentives with people who marginalised, oppressed, or otherwise have low access to resources. The prevailing assumption here is that the provision of too significant an incentive might prove coercive (ref), as if a prospective participant might not have a lot of money, they might feel they *need* to participate in a research project. Yet on the other hand, to ask a tremendous amount of emotional (and in some cases physical) labour of people, to talk about experiences they may have had which they might have found difficult, and not provide any kind of incentive for them to do that disrespects and delegitimises the expertise of these experiences. Particularly in care-experienced spaces, there has been a move towards recognise the tendency of organisations and individuals with high resource-access to commodify and (make small) the experiences of care-experienced people whilst also being unable to function without them. As such, I followed my partner organisations’ regular practices on each project. With two of the three organisations, no financial incentive was given for participation. With the other, financial incentives were offered (per the organisations usual policy) in the form of £25 Love2Shop vouchers, which were preferred over money or any other kind of incentive by the care-experienced young people that I worked with as the receipt of them had no effect on the amount of money they might receive from Universal Credit. This was a longstanding practice in this organisation and was also instituted for my work with them. In *fractured signals*, participants were offered also offered Love2Shop vouchers (£30) for participating. Participants in *fractured signals* were professionals who worked with care-experienced people in some way, and the vouchers in this situation were mostly a recognition of the time they gave up for the project. 

Rather than focus on the offer of financial incentives, throughout my work I tried to instead create capacity-building opportunities, which would offer participants opportunities to build new skills, try new things, or make something that felt personally meaningful them. Amartya Sen’s notion of capability theory suggests that ROTKOHL. Martha Nussbaum builds on this, proposing that there are ROTKOHL central capabilities that people need the opportunity to enact in order to have a flourishing life, including ROTKOHL. I kept the central capabilities in mind whilst creating methods which followed Freire’s notions of ‘critical pedagogy’. For Freire, critical pedagogy is a consciousness-raising approach that seeks to build a critical awareness of the power structures that underlie the world… I did this but then also created environments to build material skills. For example, in one of my projects (the design outputs of which are not mentioned heavily throughout this thesis), I set about using participatory design methods and critical pedagogical approaches to teach care-experienced young people the requisite technical skills to create a documentary film about their experiences of life story work, a therapeutic approach that aims to ROTKOHL. Participants in the project received no direct incentive, but would learn the camera, audio, and story skills to make a film with professional-grade equipment and receive lunch, drinks and snacks across the five days that we were working together for. Similarly, in *It’s Our Future*, there was no direct incentive for participants (except for train travel, paid for by the charity partner), but during the workshop they would be surrounded with new ideas and an environment that helped them to imagine and begin to build a future that addressed their needs and desires. Lunch was also provided. 

\subsection{Writing about research with oppressed or traumatised people}
To understand the story of this research, the ways it has been crafted, shaped, and told, you must know a little about this story’s narrator. ~Following \cite{rose_situating_1997}, I present my positionality in relationship to this research, which has shifted throughout the project.~ First and foremost, it must be said that I am *not* care-experienced. I have at no time lived through the children’s social care system. I have a small history with it: in my third year of my undergraduate degree I worked as a ‘Brand Manager’ for the social work graduate scheme “Frontline”; my work mostly consisted of putting branded bike seat covers on bike seats and distributing free pens. It wasn’t a huge intersection with social care, but it was my first (knowing) intersection. I reflected after this job that I had *slightly* known the social care system growing up - partly through my friend, who went to live with her grandparents after her mum’s mental health declined rapidly and she became emotionally abusive - and partly through a kind of spectre in my own home. Growing up, my parents never wanted me to *say too much*. There was nothing particular going on in my house that they should have been worried about me saying - we were poor,  we had the usual concerns of people who are poor, and that’s it. Yet there was always a paranoia to my parents, fearing that if I disclosed some insufficiency of parenting to someone outside of the house, “children’s services will come and take you away”. I learned not to say too much, I think.

Through doing this research and writing this thesis, I have learned to be comfortable with saying too much - to ‘say the quiet part out loud’, to name what is known but unarticulated. So although I have not experienced the children’s social care system myself, I have known its vague shape and consider myself an ally to care-experienced people. Through the course of my research, I realised that researching the care-experience was like finding a series of bruises I hadn’t realised were there. When I was younger, I was in a few emotionally abusive relationships which left me traumatised and left me emotionally reactive for years. As I mentioned, my family were (and are) poor, and my relationship with them is complicated and marked by an intergenerational cycle of poverty, not having enough, being unable to advocate for your needs, and trauma. I am not care-experienced and am not claiming in any way to understand the exact nature of care-experience. Through the course of my research, though, I began to realise that I shared some experiences with care-experienced people that helped me to understand where they were coming from more easily than some others might have. 

Neither of my parents finished school; my mum had my brother aged seventeen and my dad was in hospital through most of his twenties. Before them, there was a generation of cafe workers, cleaners, and power plant workers. Before that, farmers and bakers.  My roots are as working class as working class could get - in the entire history of my family there has never been *enough*. As a result, I am the first person in my family to go to university and to know that privilege. I went to a grammar school, and then to a Russell Group university, and then to another. I have been fortunate enough to be surrounded by people who believed in me, who would always give me any support I needed out of a belief that I was ‘gifted’. So with some of the people I have conducted research with, I have been wracked with an intense sense of survivor’s guilt - a sense that if things had just been slightly different,  if one sliding door had closed a little earlier, this could have been me. I am aware that this is a little self-centered, but it was one of the most recurrent feelings I experienced throughout my research. In the conduct of my research, for example, I would often feel the need to alter my accent - from the clearly-enunciated sort-of-received-pronounciation that I speak with in day-to-day life, back towards something that approximates an older version of my Medway-inflected-estuary-accent. It’s rougher, it’s harder, it’s more full of slang, and it has commanded much more respect and understanding in research environments than my ‘normal accent’ ever has done - which is of course itself a construction, built through years of increasingly-more-middle-class-socialisation.

Suffice to say, then, I have a complicated relationship to care-experience and the object of my study. In a lot of ways, researching the care-experience has allowed me to return to myself, to re-assess my roots and how circumstance, support, and privilege has shaped me into the person that I am today, from similar roots to many of the people that I work with. Yet in the present, it remains true that I am an academic at a university, socialised in all of the ways that you would expect someone to be after seven years of university. My body places me firmly in that same seat of power - I am a white man conducting research with people who are oppressed, using methods such as ethnography that have historically been used to other and abstract. My whiteness and masculinity has likely shaped this research to a great degree - in many situations, I embody people’s expectations of what a researcher might look like, and this will have conditioned their responses to me, for better or worse. In some cases, this might mean they have responded more warmly to me, understanding the function that I am there for and being happy to spend some time with me. In others, this might have mean they have responded more coldly, hostile to a class of people whose job consists of expropriating knowledge from communities where it is already held in common. In the former circumstance, practitioners have often taken me seriously and had more time for me than they might otherwise have done; in the latter circumstance, young people have taken a while to build trust with me, and I have often leaned upon some of my other identities - designer, or sort-of-youth-worker, perhaps. 


% ~Notes on positionality in case I need to add any in~
% Positionality - haraway and harding, situated knowledges 
% Facets of the self
% The role of multiple selves - showing how the researcher influences the data and thus what becomes knowledge 

As a sequence of stories, the matter of voice is exceedingly important. As I have just mentioned - this is a work in which too much must be said - where things which have often gone unsaid, unheard, or unlistened to must be spoken. The question of *who* tells, who listens, and about what can be spoken is exceedingly important. Some of my earliest research in this thesis concerned the very idea of what it means to be listened to for care-experienced people, and this has subsequently shaped my research approach. No matter how I try to preserve voice, though, I remain the narrator of this thesis, and so it should be noted explicitly here that even when I attempt to give over narrative space in the thesis to the voices of others, this is still a sleight-of-hand committed by me; I retain the power of ascertaining what matters and what does not. I have attempted throughout to approach this responsibility sensitively, giving more space to things that people have spoken more about, and following my research participants’ wants and needs to shape what I actually explored. 

To the actual material of voice, then: I have attempted, where possible, to preserve accents and dialect words. Throughout this thesis, I have worked with people from across the United Kingdom, and so worked amongst a variety of types of voice. Broadly-accented Geordies, softly-spoken Southerners, fast-speaking Mancunians and [an adjective for Cornishpeople] have all featured in my work, and I have tried to preserve the tone of their voices where possible. In line with this, I have tried to use as much in-vivo language as possible, as there is power to using an individual’s own words for their own situation [cite GT], which can reveal more about what creates a phenomena than by papering over their words with bland abstractions.   

Textually, I have attempted to carve spaces for participants to tell their own stories. If a participant told me a story about their life or something that they had done or had happened to them, I have tried to use their own words to re-tell that same story in-text if it is relevant to the overall story of the thesis. In line with this, I also had all of my participants choose their own pseudonyms. Particularly for care-experienced people, having agency and ownership over self-representation is important [cite]. At times, this has resulted in pseudonyms that may sound strange or peculiar - like L.TukZombie - or that may be close to or similar to someone’s real name. By giving the choice to people in how they wish to be represented, I am creating an environment for them where their consent and agency is actively respected.

One of the biggest ethical tensions that I encountered through my work was the simultaneous use of a critical ethnographic method and a participatory design approach. Critical ethnography often necessitates a kind of critical distance, a willingness to look beyond the individual towards the structural factors that maintain an issue, whilst participatory design instead almost fetishes the individual and their talk as an inherent good. Contemporary participatory design approaches might suggest that we co-author publications with collaborators, or allow them editorial control of work which goes out into the world. On the other hand, critical ethnography might require us to be exceptionally critical of the actions that an individual takes, and relate that to the structural factors the motivate that action - which can be at ends with giving individuals involved editorial control or collaboratively publishing. Although I have engaged in these practices, resolving the tension between these two aspects of my work has been exceedingly difficult.

The way that I have attempted to reconcile this tension is as follows: my commitments within my participatory design practice are ultimately to the children and young people that I have worked with first and foremost. I am committed to researching the care system and the structural factors that keep many care-experienced people in precarious situations, with significant difficulty in their futures due to intergenerational oppression. Sometimes, individuals that I have worked with - usually workers and managers - might be working counter to this goal through no individual fault of their own, but because of the structural power relations that condition their actions, effectively coercing them into less-than-ideal situations. In these situations, whilst I am still on the side of the worker, they are expressing a facet of the structural power relations rather than existing as themselves. As such, I cannot be committed to participatory work with them in this fashion. It is of utmost importance to talk about the bad, unethical, dangerous, or value-unaligned practice that is taking place inside of many charities that provide support services to children and young people.

That is not to say, however, that I wish to throw them under the bus, to publish an expose saying ‘these charities are awful and you shouldn’t work with them’. The charities I have worked with merely express a set of behaviours that are likely to arise due to the structural conditions we find ourselves in, under austerity-intensified capitalist realism. As such, I have made the choice throughout this thesis - in honour of my commitments to both critical ethnography and participatory design - to fictionalise the organisations that I have worked with. Instead of talking about them discreetly as three different field sites, I am aggregating them, finding archetypal factors across people in different roles, and writing about fictional individuals who act as composites of multiple actually-existing people. The organisation featured throughout this thesis, then, is The Charity (substantively introduced in \ref{4}. There is no alternative to ethically write about this work, potentially posing far too much of a danger to practitioners that I have worked with over the past four years, that I know, love, and care about. Fictional approaches to ethnography have existed ROTKOHL [all ethnography is fiction… some more on storytelling and ethnography?]

If you are reading this thesis and you work in the charity sector, or the space of children’s social care, pay extra attention. By fictionalising these organisations into the composite *The Charity*, I am not just protecting those that I have worked with; instead, I am trying to tell *the general story* rather than the specifics. In this way, I am following in the tradition of Marx;  *Capital, Vol 1* (18XX) is about the conditions of working of English industrial and agricultural labourers, but at the beginning of the book, Marx warns the German reader, who might be comforting themselves that things in Germany are nowhere near as bad that “De te fabula narratur! (the tale is told of you!)”.  If you work in the charity sector or children’s social care, this tale is told of you too. *The Charity* may very well be your organisation, and you must ask yourself what you can do to unpick the roots of capitalist realism within it. 


% (Add in Rinehart, 1998, and Brown, 201X - Marcuse) - all ethnography is fiction 


% **Have I adequately covered these**
% * beyond procedural ethics 
% * working with young people
% * //do I need to talk more about anonymity itself?
% * *add in word about confidentiality - its there just be more explicit*
% * knowledge production? Research justice - where does this go 

% original methodology on ethics here 
% (Maybe cite engaging with social work, an introduction)
 Deeply considering what constitutes ethical practice with people who are oppressed  or have had traumatic experiences necessitates a consideration of what ethical practice actually *means* in these contexts. A key distinction here is to be made between equality, equity and liberation. An illustration that is frequently used to demonstrate the distinction between these terms features three people at a baseball game, staring over a fence to see the game. Equality consists of treating everyone the same, giving them the same support and opportunities. In the illustration, each person - of varying heights - is given a box to stand on to see over the fence. The tallest person can see, but the middle and smallest person still cannot. Equality-entered practice thus remains unethical, because it perpetuates existing inequalities, imagining everyone’s needs to be the same. In contrast, equity-centred practice consists of addressing these inequalities by giving people what they need. In the case of the baseball example, the tallest person does not require a box to stand on, the middle requires one box, and the smallest requires two. This gives them support that is relevant to their individual needs. Equitable practice, then, is good at ‘giving people a seat at the table’, potentially creating the conditions for participation by taking stock of existing inequalities. Yet this still does not consider the structural factors that create this situation in the first place.

Liberation, then, goes one step further, considering these structural factors and what created them. In the case of the baseball example, then, liberation takes away the fence, giving each person a view of the game without the need for any additional support. It is possible that we could even go further than this - perhaps true liberation or empowerment looks like being free to participate in the game itself. That being said, the important aspect for the conduct of my work is this focus on liberation as an approach that takes account of structural factors - such as capitalist realism, austerity, or cycles of intergenerational poverty created by an entrenched neoliberalism - and understands and accounts for people’s behaviours in this context. 
% ****	critical pedagogy****

The other side to creating a methodology actively centered on liberation in this context, then, is a focus on healing and integration. Many of the people that I have worked with have experienced trauma, and this has significantly impacted their lives and the way that they relate to the world. Bessel van der Kolk’s *The Body Keeps the Score* gave form to a new way of working with people who have experienced trauma and how we might understand their experiences. van der Kolk suggests that trauma - defined as ‘too much, too soon, too quickly’ (by who?) becomes embedded in the body, changing the functioning of the autonomic nervous system and thus the way that people who have experienced trauma (and been unable to integrate these experiences) continue to relate to the world. 






%  *need for situational ethical praxis that sees ethical research as a constant relational practice*






% Design approach - creation of speculation and novelty to critique and resist capitalist realism 


% # Data analysis and writing
% ## 




% ---

% //emergence
% * How do we make it not happen? (deterritorialization?)
% * Doing deterritorialisation in practice is a kind of speculation
% * What is speculation?

% //why design?
% Against alienation. Making things as a strategy to move against capital.
% Speculation


% ## Participatory work
% Co-production etc
% Critical pedagogy
% Counterpower
% PAR / ?
% Emergent strategy 
% Trauma-centered? 

% ## Speculation and deterritorialisation
% * what is speculation 
% * Doing deterritorialisation in practice is a kind of speculation

% ## Interaction and service design
% Design 
% Part of ‘the digital’ 
% PD

% Structuring practice 
% Against alienation. Making things as a strategy to move against capital.

% Design justice





