\chapter{Setting the scene}
\label{4}

\section{Introduction}
\label{4-intro}

In the previous chapter, I detailed my methodological approach, focusing on the intersections of ethnography, action, and design, and detailed how a speculative approach to design research might open new junctures in the tightly territorialized frame of capitalist realism and the way it conditions action. In the following chapters, I will explore what living and working in the care system is like under austerity-intensified capitalist realism, and detail what a speculative approach to working against this might look like. Before that, though, we must find an opening into this world, to learn about our characters and our setting. What happens here? What are these people like? 

In this chapter, I will introduce The Charity, its many sprawling projects, and the people that I have come to know through those projects. As explained in the previous chapter, The Charity is a fictionalised composite of the organisations I have worked with over the past four years. Throughout this chapter, I 'follow [my] ethnographer’s nose twitch' (after \cite{leigh_star_this_2010}), paying attention to my intuitions about things that don’t \textit{quite} feel right. This is structured around three moments in which my attention was brought towards a 'slippage' \citep{cutting_making_2021} that was occurring, between the assumed and stated values of the practitioners I was working with and what was \textit{actually} happening. In the following two chapters, I introduce the theory of justification practices, which describes the  changes in behaviour, practice and affect that have been brought about in the care system due to austerity-intensified capitalist realism. This chapter describes my first noticings of justification practices in action.

\section{Introducing The Charity}

The Charity are a huge, sprawling organisation with so many employees that they nay have lost count themselves. They work in every corner of the United Kingdom, across all of its regions and nations, and are devoted to work with children and young people that they consider to be vulnerable. The Charity are old - at least as old as the modern idea of a charity, though some say they can trace their history back to the 19th and 18th centuries. They try to empower their teams and projects across the country to make decisions for themselves, as far as is possible. A project being run by The Charity in the Lake District, for example, might be vastly different to one that running in Bristol - as local control and local decision-making reign supreme. Except, of course, when they don’t - like when a new manager attempts to centralise power and control. It never works for long, but this tension between central control and local autonomy are a constant feature for The Charity.

The Charity is funded in a number of ways. Much of their work is commissioned - normally by a local authority outsourcing some of their statutory responsibilities - whilst other work is grant-funded, giving The Charity an opportunity to develop novel approaches to working with young people perceived to be vulnerable. On top of this, there is also work funded by 'unrestricted finance' -  money given by members of the public that allows them to work on new and 'innovative' projects, or strategically prioritise certain areas of work. This three different funding streams give rise to entirely distinct working experiences. People working on commissioned projects might find that they are only allowed to deliver what the project has been contracted to, making acting outside of this exceedingly difficult. Grant-funded projects may appear to have more flexibility, but the strict bureaucracy around evaluation and reporting often means their hands are just as tied as commissioned project workers. Projects funded by unrestricted finance easily have the largest degree of flexibility - but their 'strategic' nature often means they have to operate under the watchful gaze of senior central teams. 

This means that people who work for The Charity are hugely varied. The majority of people I worked with in The Charity tended to be youth workers or social workers. Yet all sorts of workers exist in an organisation so vast. There are researchers, HR, admin, and finance staff, business managers, mental health workers of different forms, and ex-civil servants in senior managerial roles. Most of the people I worked closely with, though, came from conventional charity backgrounds, either primarily having worked in a helping or caring profession, or making a drastic switch at some point to do that. Some were professionally trained, pursuing social work qualifications at university before joining The Charity, whilst others did their training on the job, pursuing youth work qualifications through working with The Charity, or joining whilst on placement from their course. 

% maybe swap locations? i.e. building bridges in sw?
Although other areas, projects, or aspects within The Charity may show up within this thesis, I have worked primarily with three main projects: Small Steps, Building Bridges, and Seabird. Small Steps is a project focused on developing research and policy to support young people with experience of homelessness in the North East of England. Building Bridges is a project based in the North West of England, focused on supporting care-experienced young people to identify ways that they want to change the care system and providing them with resources to work towards making those changes. Finally, Seabird is a project providing youth and social work services for care-experienced young people in the South West of England. As projects within The Charity, all three projects had some level of awareness of each other, but there was not necessarily a huge degree of interaction between them. In the rest of this chapter, I will introduce you to members of each project in turn, and the slippages that gradually became apparent as I spent more time with each. 

\section{Noticing slippages}

Along with introducing you to the members of each project I have worked with, I want to show how each of these projects began to gesture towards the barely-perceptible network of power relations that defined the context they operated within. To do this, I will be using the concept of \textit{slippages}. \cite{fanon_black_1986} first detailed this as an ethical slippage (\textit{un glissement ethique}), using the concept to describe the ways that the moral values of white France were able to take hold in the consciousness of Black Martinicans . For Fanon, it was important to understand how the values and priorities of one world could be transported to another world in which they did not fit, and resulted in oppression. Martinicans came to inherit the colonial French value system that 'black is bad, immoral, and sinful, while white is good, virtuous, and pure' \citep[11]{sullivan_ethical_2004}, meaning that Martinicans came to consider morally understanding people to be white. These values 'slip' into subconscious thought and action, often diffused through media, and without much conscious attention being brought towards it. 

My use of the concept builds on Fanon's, and Sullivan's characterisation of this. In my rendering \citep{cutting_making_2021}, slippages refer to the moments in conducting research in which everything suddenly appears to be different: you thought you were researching one thing, but something that feels suddenly out of place redirects your attention and your nose twitches. You feel yourself - or your participants - slipping from one valueset to another, or notice a widening bifurcation between a person or project's stated or implicit values and their actions. You begin in one worldview, imaginary, or ideology, and feel yourself pulled into another. You may just as easily slip back - but once you have seen the presence of that other world, it is due diligence to 'follow the phenomenon' (Cite whitaker phd 2015). In essence, slippages are the first noticings which allow you to become aware of the affects that your participants are constantly experiencing or the environments they must navigate.

In this section, I describe three slippages happening across the three different projects I worked with, and use these as an opportunity to explain the dynamics of the project, the people involved with it, and the slippage that underpinned my research with them. The Small Steps made me aware of \textit{how} austerity-intensified capitalist realism begins to affect projects and organizations. Then, the Building Bridges slippage made me aware of the ways that  managers may feel that their hands are tied, even if they want something to happen Finally, with Seabird, the slippage began to make clear to me how these mindsets  become embedded within projects and workers, indicating some of the machinery of the network of power. 

\subsection{From peer support to product}
Small Steps was the first project by The Charity that I was introduced to, and was my introduction to them as an organization. When I first meet them in 2018, they are a newly-independent project, just starting to find their feet with their own small team. Until 2017, they had been hosted within another project in The Charity, but it became apparent that they needed to spin off into their own project if they wanted to continue their work on research-led policy advocacy for young people with experience of homelessness. Their staff consists of Shelly, the Director; Colin, the Regional Project Officer; Randall, the Communications Intern; and Ronnie, the Fundraising Officer. I soon come to learn that these job titles are incredibly malleable - Ronnie mostly does the team's finance and admin; Shelly does most of the grant writing; Colin does \textit{everything} that involves direct work with young people and Randall is 'encouraged' to do way more than he's meant to because Shelly thinks it will help him after his degree, which he is in the middle of. They also have a few infrequent volunteers and projects run with volunteers from a local university. 

I first meet Small Steps when they contact my research group: they have recently received some funding to redesign their website and develop an app to increase access to their resources for young people who might at risk of becoming homeless. Their current website at the time is mostly geared towards professionals, but they found through their website's analytics that many of their users were much younger, and therefore may be a young person at risk of homelessness seeking support or resources. They heard that my research group sometimes helps charity projects out with digital things, and....


 After our first meeting - with Randall and Colin - it becomes clear that they would be up for a more human-centered design process; understand that they don't necessarily need 'an app', and are excited about the potentials of introducing digital technology into this space. They're a tad sceptical as to whether Shelly will feel the same or not, but they tell me they'll propose it to her nonetheless.

% They're right: Shelly was a bit sceptical...



\subsection{"He took us for ice cream in the park..."}
% Building Bridges are a large, national youth services organisation in the United Kingdom. Their work engages with young people in every way and aspect of their lives, in both times of stability and need. Their entire range of services span ...

% The Director for Impact, Michael, is a tall bald man with an energetic demeanour. Within his role - which he's fairly new to - he is tasked with delivering impact for his part of the Central Services Innovation Fund, the Care System.

% The Central Services Innovation Fund is a pot of money set aside by Rocksteady to innovate in three care areas where they have identified a need for 'doing things differently'.

% Michael's pet project within Rocksteady at the moment is a scheme called 'Bridges'. Bridges is to be a scheme that puts young people who have care experience at the forefront of influencing the Central Services Innovation Fund and how they do things differently within the Care System.

% I find out that Michael used to be Colin from Small Steps' manager in their old jobs. Colin speaks fondly of Michael, of his vision, his incredible drive, how well he works and how brilliantly he treats staff.

\subsection{Seabird}

% Seabird are a local charity based in the South West of England, providing youth and social work services to young people in and leaving care.