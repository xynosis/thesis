\chapter{Setting the scene}
\label{}

\section{Introduction}
\label{sec:}

In the previous chapter, I detailed my methodological approach, focusing on the intersections of ethnography, action, and design, and detailed how a speculative approach to design can research might open new junctures in the tightly territorialized frame of capitalist realism and the way it conditions action. In the following chapters, I will explore what living and working in the care system is like under austerity-intensified capitalist realism, and detail what a speculative approach to working against this might look like. Before that, though, we must find an opening into this world: learn about our characters and our setting. What happens here? What are these people like? 

In this chapter, I will take you on a journey that follows the broad shape of  the journey I have been on over the past three or four years. I will introduce you to **The Charity** and its many, sprawling projects, and some of the brilliant - and troubling - people that I have worked with. Throughout this story, I “follow [my] ethnographer’s nose twitch” (after Leigh Star, XXX), listening to my intuition about things that just don’t quite feel right. Having started with the intent of exploring the design of digital peer support systems for young people perceived to be vulnerable (a need that I was approached by The Charity with), this chapter recounts the story of how my work shifted, changed and contorted as I began to listen to what was *actually* there. In the next two chapters I will introduce the theory of *justification practices*, which describe the kinds of changes in behaviour, practice, and affect that have occurred in the care system and what sustains them. In this chapter, I describe the soil of justification practices: what they operate within and where they come from. 


% Peer support - the product 
% Ice cream moment - The Meeting
% somethign with carefree



% Context
% This thesis engages substantively with four organisations: Small Steps, Rocksteady, Seabird, and Wallenhead Leaving Care Service. These organisations all represent vastly different elements of the young-people-in-perceived-situations-of-vulnerability assemblage. Small Steps is a youth homelessness research and policy charity based in the North East of England; Rocksteady is a large, national youth services charity; Seabird is a charity that provides youth and social services to young people with care experience in the South West of England; and Wallenhead Leaving Care Service are a local authority-run service for young people leaving the care system. These four organisations intersect of their own accord, due to the complexity of the youth services assemblage. Yet they also have vastly different views of the systems they operate within: Rocksteady are national, well-established, tens of years old; Small Steps are a product of post-austerity Britain, whilst Seabird benefited from pre-financial crash security for youth work. Whilst these three organisations are charities, with varying degrees of nearness to private companies, Wallenhead Leaving Care Service is a local authority-run service, operating within constraints set explicitly by the authority rather than being a commissioned service.

% ...

% Rocksteady's scheme 'Bridges' brings together 8 practitioners (of varying professions) and 16 young people within the care system to focus on how to change the care system for the better.

% ## Small Steps
% Small Steps are a youth homelessness research and policy organisation based in the North East of England. They have a small team - only three to four paid staff at any one time - and a small army of volunteers, who are interested in the work that they do and want to help them make a difference, in some way. When I first start working with Small Steps, their paid staff are Shelly, the Director; Colin, the Regional Project Officer; Randall, the Communications Intern; and Ronnie, the Fundraising and Admin Officer.

% I first meet Small Steps when they contact our Lab: they have recently got some funding to develop an app and redesign their website, and in speaking to one of their trustees, have heard that Open Lab sometimes helps charities out with these processes. They don't know much about us, and we know very little about them. After our first meeting - with Randall and Colin - it becomes clear that they would be up for a more human-centered design process; understand that they don't necessarily need 'an app', and are excited about the potentials of introducing digital technology into this space. They're a tad sceptical as to whether Shelly will feel the same or not, but they tell me they'll propose it to her nonetheless.

% They're right: Shelly was a bit sceptical...

% ## Rocksteady
% Rocksteady are a large, national youth services organisation in the United Kingdom. Their work engages with young people in every way and aspect of their lives, in both times of stability and need. Their entire range of services span ...

% The Director for Impact, Michael, is a tall bald man with an energetic demeanour. Within his role - which he's fairly new to - he is tasked with delivering impact for his part of the Central Services Innovation Fund, the Care System.

% The Central Services Innovation Fund is a pot of money set aside by Rocksteady to innovate in three care areas where they have identified a need for 'doing things differently'.

% Michael's pet project within Rocksteady at the moment is a scheme called 'Bridges'. Bridges is to be a scheme that puts young people who have care experience at the forefront of influencing the Central Services Innovation Fund and how they do things differently within the Care System.

% I find out that Michael used to be Colin from Small Steps' manager in their old jobs. Colin speaks fondly of Michael, of his vision, his incredible drive, how well he works and how brilliantly he treats staff.

% ## Seabird

% Seabird are a local charity based in the South West of England, providing youth and social work services to young people in and leaving care.
