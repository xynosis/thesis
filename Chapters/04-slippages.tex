\chapter{Setting the scene}
\label{4}

\section{Introduction}
\label{4-intro}

In the previous chapter, I detailed my methodological approach, focusing on the intersections of ethnography, action, and design, and detailed how a speculative approach to design research might open new junctures in the tightly territorialized frame of capitalist realism and the way it conditions action. In the following chapters, I will explore what living and working in the care system is like under austerity-intensified capitalist realism, and detail what a speculative approach to working against this might look like. Before that, though, we must find an opening into this world, to learn about our characters and our setting. What happens here? What are these people like? 

In this chapter, I will introduce The Charity, its many sprawling projects, and the people that I have come to know through those projects. As explained in the previous chapter, The Charity is a fictionalised composite of the organisations I have worked with over the past four years. Throughout this chapter, I 'follow [my] ethnographer’s nose twitch' (after \cite{leigh_star_this_2010}), paying attention to my intuitions about things that don’t \textit{quite} feel right. This is structured around three moments in which my attention was brought towards a 'slippage' \citep{cutting_making_2021} that was occurring, between the assumed and stated values of the practitioners I was working with and what was \textit{actually} happening. In the following two chapters, I introduce the theory of justification practices, which describes the  changes in experiences, affects and practices that have been brought about in the care system due to austerity-intensified capitalist realism. This chapter describes my first noticings of justification practices in action.

\section{Introducing The Charity}

The Charity are a huge, sprawling organisation with so many employees that they nay have lost count themselves. They work in every corner of the United Kingdom, across all of its regions and nations, and are devoted to work with children and young people that they consider to be vulnerable. The Charity are old - at least as old as the modern idea of a charity, though some say they can trace their history back to the 19th and 18th centuries. They try to empower their teams and projects across the country to make decisions for themselves, as far as is possible. A project being run by The Charity in the Lake District, for example, might be vastly different to one that running in Bristol - as local control and local decision-making reign supreme. Except, of course, when they don’t - like when a new manager attempts to centralise power and control. It never works for long, but this tension between central control and local autonomy are a constant feature for The Charity.

The Charity is funded in a number of ways. Much of their work is commissioned - normally by a local authority outsourcing some of their statutory responsibilities - whilst other work is grant-funded, giving The Charity an opportunity to develop novel approaches to working with young people perceived to be vulnerable. On top of this, there is also work funded by 'unrestricted finance' -  money given by members of the public that allows them to work on new and 'innovative' projects, or strategically prioritise certain areas of work. This three different funding streams give rise to entirely distinct working experiences. People working on commissioned projects might find that they are only allowed to deliver what the project has been contracted to, making acting outside of this exceedingly difficult. Grant-funded projects may appear to have more flexibility, but the strict bureaucracy around evaluation and reporting often means their hands are just as tied as commissioned project workers. Projects funded by unrestricted finance easily have the largest degree of flexibility - but their 'strategic' nature often means they have to operate under the watchful gaze of senior central teams. 

This means that people who work for The Charity are hugely varied. The majority of people I worked with in The Charity tended to be youth workers or social workers. Yet all sorts of workers exist in an organisation so vast. There are researchers, HR, admin, and finance staff, business managers, mental health workers of different forms, and ex-civil servants in senior managerial roles. Most of the people I worked closely with, though, came from conventional charity backgrounds, either primarily having worked in a helping or caring profession, or making a drastic switch at some point to do that. Some were professionally trained, pursuing social work qualifications at university before joining The Charity, whilst others did their training on the job, pursuing youth work qualifications through working with The Charity, or joining whilst on placement from their course. 

% maybe swap locations? i.e. building bridges in sw?
Although other areas, projects, or aspects within The Charity may show up within this thesis, I have worked primarily with three main projects: Small Steps, Building Bridges, and Seabird. Small Steps is a project focused on developing research and policy to support young people with experience of homelessness in the North East of England. Building Bridges is a project based in the North West of England, focused on supporting care-experienced young people to identify ways that they want to change the care system and providing them with resources to work towards making those changes. Finally, Seabird is a project providing youth and social work services for care-experienced young people in the South West of England. As projects within The Charity, all three projects had some level of awareness of each other, but there was not necessarily a huge degree of interaction between them. In the rest of this chapter, I will introduce you to members of each project in turn, and the slippages that gradually became apparent as I spent more time with each. 

\section{Noticing slippages}

Along with introducing you to the members of each project I have worked with, I want to show how each of these projects began to gesture towards the barely-perceptible network of power relations that defined the context they operated within. To do this, I will be using the concept of \textit{slippages}. \cite{fanon_black_1986} first detailed this as an ethical slippage (\textit{un glissement ethique}), using the concept to describe the ways that the moral values of white France were able to take hold in the consciousness of Black Martinicans . For Fanon, it was important to understand how the values and priorities of one world could be transported to another world in which they did not fit, and resulted in oppression. Martinicans came to inherit the colonial French value system that 'black is bad, immoral, and sinful, while white is good, virtuous, and pure' \citep[11]{sullivan_ethical_2004}, meaning that Martinicans came to consider morally understanding people to be white. These values 'slip' into subconscious thought and action, often diffused through media, and without much conscious attention being brought towards it. 

My use of the concept builds on Fanon's, and Sullivan's characterisation of this. In my rendering \citep{cutting_making_2021}, slippages refer to the moments in conducting research in which everything suddenly appears to be different: you thought you were researching one thing, but something that feels suddenly out of place redirects your attention and your nose twitches. You feel yourself - or your participants - slipping from one valueset to another, or notice a widening bifurcation between a person or project's stated or implicit values and their actions. You begin in one worldview, imaginary, or ideology, and feel yourself pulled into another. You may just as easily slip back - but once you have seen the presence of that other world, it is due diligence to 'follow the phenomenon' (Cite whitaker phd 2015). In essence, slippages are the first noticings which allow you to become aware of the affects that your participants are constantly experiencing or the environments they must navigate.

In this section, I describe three slippages happening across the three different projects I worked with, and use these as an opportunity to explain the dynamics of the project, the people involved with it, and the slippage that underpinned my research with them. The Small Steps made me aware of \textit{how} austerity-intensified capitalist realism begins to affect projects and organizations. Then, the Building Bridges slippage made me aware of the ways that  managers may feel that their hands are tied, even if they want something to happen Finally, with Seabird, the slippage began to make clear to me how these mindsets  become embedded within projects and workers, indicating some of the machinery of the network of power. 

\subsection{From peer support to product}
Small Steps was the first project by The Charity that I was introduced to. When I first meet them in 2018, they are a newly-independent project, just starting to find their feet with their own small team. Until 2017, they had been hosted within another project in The Charity, but it became apparent that they needed to spin off into their own project if they wanted to continue their work on research-led policy advocacy for young people with experience of homelessness. Their staff consists of Shelly, the Director; Colin, the Regional Project Officer; Randall, the Communications Intern; and Ronnie, the Fundraising Officer. I soon come to learn that these job titles are incredibly malleable - Ronnie mostly does the team's finance and admin; Shelly does most of the grant writing; Colin does \textit{everything} that involves direct work with young people and Randall is 'encouraged' to do way more than he's meant to because Shelly thinks it will help him after his degree, which he is in the middle of. They also have a few infrequent volunteers and projects run with volunteers from a local university. 

I first meet Small Steps when they contact my research group: they have recently received some funding to redesign their website and develop an app to increase access to their resources for young people who might be at risk of becoming homeless. Their then-website was mostly geared towards professionals, but they found through their website's analytics that many of their users were much younger, and therefore might be a young person at risk of homelessness seeking support or resources. They heard that my research group is known for helping charities out with digital projects, and after an initial meeting, it's agreed that we will work together exploring the design of digital peer support systems for young people with experience of homelessness. Randall and Colin are excited about the potentials of something less transactional than the original project idea, but are sceptical as to whether Shelly will agree. 

I meet Shelly at Small Steps' office across town the next day. Their offices are my first taste of the world of charities that support young people perceived to be vulnerable. In an otherwise grey and impersonal office, colourful memories of previous projects line the walls. One is a mosaic, the word 'HOME' made up of the word 'hope' written over and over. Another is a linocut print of a home torn in two. Another, a collage of letters cut from magazines, reading 'family is as noisy as a traffic jam'. A whiteboard containing the team's schedule looms over the room, whilst stacks of paper fill every free nook and cranny of the office. Every project office I enter after this is similar: grey walls strewn with memories of previous projects and paperwork bursting at the seams. Shelly greets me enthusiastically, shares a little of the history of the organization with me, and we get to talking about my project. She likes the idea of exploring digital peer support, but doesn't quite see how it helps Small Steps right now. She's willing to give it a go, but has worries about moderation, about things 'going bad', and wants to ensure that the young people using the platform aren't too 'risky'. Her concerns strike me as a little odd, as there was no plan to actually create a technology to support digital peer support at this stage, though - just to explore the design. I don't give it a second thought at the time.

A pattern continues over the next few months, though: Colin, Randall and I agree something important concerning the project, we tell Shelly about it, she agrees, and then a week later wants us to do something entirely different. We agree that data privacy and security will be integral throughout the project, as we want the young people we're working with to feel comfortable that they have control over their own information. After a few weeks, Shelly begins making a case that we should ask for consent to allow their information to be included in the project's research. 

"It's just too good an opportunity to miss," she says. "This could really help us to get really young-person informed research."

We don't come to a consensus on what to do - and it doesn't feel too essential at the time as there is still no system being actively developed as part of this project. 

One evening, after I helped Colin out with a young people's workshop he'd been running, I checked in with him. "You look exhausted, mate, you alright?"

"Aye, it's fine, it's just... Shelly's got me doing everything, man. I'm here, there and everywhere, and she wants us to do a blog about the workshops I've been running, and I've still gotta be up and down the M1 every week and there's just not enough time, y'know?"

"Yeah, that sounds like a lot. I noticed you've been doing a lot lately."

"It'll be fine, it's just - you know, you need to be careful. She means the best but Shelly'll have every pound of flesh she can get. You'll be next!"

I knew what Colin meant. It had felt like Shelly had been wanting more and more from me lately. First it was borrowing some equipment from the research group, which I was happy to help with. Then it was asking me if I would do some evaluation for them on their new project. The boundaries of research like this are so malleable, so you never know when too much is too much. Is it building rapport? Is it helping the organisation? Is it what you necessary for the work to go further? I knew it had started to go too far, though, when Shelly had asked me to develop a funding bid for the organisation. It was a fund that I had access to because of my connection to the university, but as a junior researcher at the time I was way out of my depth. I found myself working right down to the day of the deadline to develop a bid for a wide-ranging early intervention project. My role seems to stretch further and further. 

"I think Shelly might already have every pound of flesh from me!" Although I brushed it off, Colin's comments stuck with me. Shelly had been hounding him for those blogs for \textit{ages}, but it wasn't clear why she wanted them. Sure, it makes sense for a project to write about its work, but Shelly seemed to care more about Colin writing about the work than actually doing it. This was the first slippage I noticed: the values of the project seemed to be becoming increasingly secondary to whatever made the project look good. Small Steps claimed to be young person-led, but most of their recent work seemed to consist of asking young people about already-predefined topics. Instead of trying to understand young people's needs and priorities, they delivered priorities to them fully formed and asked young people to comment on or endorse them. Colin had to write reports and blogs despite his role being focused on direct work with young people.

The most curious of all was Shelly, who seemed to spend all of her time writing funding proposals, which were more often than not rejected. It was odd because this was meant to be Ronnie's job. I asked her about it one day, and she told me:

"I'll tell you what it is. First day of me job, Shelly asks us to get started on this funding proposal - for the website stuff, why you're here. And I said to her, I can get on with that Shelly, but I don't know all that much about digital technology and stuff, so I might need a second pair of eyes at some point, just to check it's all making sense. And she took it off us, she said, 'oh, don't mind, I'll do it'. And she took it. But I was only asking for a second look, I wasn't saying I couldn't do it. I've worked with all the big funds, like right now, I'm two days a week at one of them. I'm not bragging, but I'm an expert at this. But that was the last funding proposal I ever saw. She's had me on admin ever since. And it's no wonder we don't get any bloody funding, she means well but she doesn't know what she's doing!"

By this time I was coming towards the end of my work with Small Steps, but Ronnie had made the slippage clear: although this was a project full of people that genuinely cared about the young people they were working with, Shelly had become so worried about the project not receiving funding that the team's labour had to be directed towards that. Colin's blogs were an attempt to document their work so that funders believed in their ability to deliver it. Shelly wouldn't let Ronnie prepare funding bids because she was worried that she didn't have the skill to secure funding. Shelly's concern about 'risky' young people was about them endangering the project's reputation, and the want to use young people's data from my project was to strengthen their claim that they were young people-led. I understood the project's need for funding - they were new and needed to become established. Yet I wasn't quite aware why it was making Shelly behave in the ways that it was. 

\subsection{"He took us for ice cream in the park..."}
Although they are based in the North West, Building Bridges is actually a national project, funded by some of The Charity's unrestricted finance. It was set up in 2018 by Michael, after he was appointed Innovation Leader for the Care System strand of The Charity's Central Services Innovation Fund. He came into the role from senior project leadership in another organization - where he was actually Colin from Small Steps' manager. He has a calm and measured demeanour, but every now and then his eyes will light up and an idea will really set him ablaze. When I first meet him, Building Bridges is just an idea still coming together in his mind. I join the project's early development team, and before long, they have recruited a project co-ordinator, Karen, and a research lead, Mandy. Although other people work in and around the project, the four of us form the core team. 

The key idea behind Building Bridges is bringing together groups of 'bridges' - a frontline worker who has a good relationship with two care-experienced young people who may or may not have an existing relationship with each other. These bridges will be drawn from across the country and will work on a 'personal goal' and  a 'community goal' together. The personal goal is intended to be something that they identify for themselves and want to change about where they are in life - and Building Bridges will help them with this. The community goal, on the other hand, will be about identifying something that they want to change about the care system in their local area based on their own experiences. The program will last a year, and will be structured around four residentials, where the bridges will meet each other, do outdoor activities that help to build their confidence, share experiences with likeminded people and get ideas for their goals. 

In the first year of the project, Karen manages to recruit eight bridges. These bridges are as follows:
[Manchester: Frank/Macaulay/Jack]
[Black Country: Sadie/Kim/Harley
[Lanarkshire: Jonathan/Dylan/Lee]
[Cornwall: Trudy/Benn/Marta]
[Lancashire: Rebecca/Jake/Callum]
[Brent: Grace/Wayne/Stephen]
[Derbyshire: ????/Jo/Rosie]
[Bristol: Abi/Stacey/Ant]

There is no common theme amongst the workers and young people who join Building Bridges. When I meet him, Frank is a personal advisor at his local authority, a role essentially grounded in checking up on the young people he's working with every few weeks and helping sort whatever needs they have at the time. Soon after our first residential, he moved into a role in another organisation helping to support young people with their mental health. [Sadie] on the other hand leads her area's Duke of Edinburgh scheme for care-experienced young people. [Trudy] works within two different projects inside The Charity, in social work and participation roles. [Abi] has been supporting her young people with a recent scheme called the Individual Health Allowance, designed to fund care-experienced young people to meet self-identified health needs. The young people are just as varied: [Marta] is training to become a youth worker herself, and could maybe even be a worker on the project, whilst it is a success for [Jack] to have even made it to the residential. [Jo] is a parent in her early twenties, but [Kim] is a 17 year old at college. [Ant] struggles with his mental health a lot, and it causes him a lot of day-to-day difficulty, but [Dylan] is the president of his college's students' union. These are not simple juxtapositions of 'good experiences' and 'bad experiences'; there are struggles and successes inside of each of these people's lives, and to explain each of their lives in depth would take up far too much space in this chapter. 

The first year of Building Bridges has its ups and downs, which is to be expected. Karen can't attend the first residential due to health issues. Several of the young people make close friends with each other and them some fall out. They decide on personal goals, like learning to drive, stopping smoking, going to the gym, attending each of the residentials. They start to build community goals, like campaigning for free public transport for care-experienced young people, making a podcast about mental health, and reducing the stigma care-experienced young people experience in the care system by reforming the way foster placements handle bad behaviour. Some are huge, difficult goals, whilst some are small and contained.For the third residential, we chose a different venue to hopefully inject some different energy, but there is chaos, as a party of school children are booked in at the same time as us.  The Building Bridges young people decide they don't want to be seen as children and refuse to take part in a lot of our activities. Some of the young people are caught smoking weed, which is to be expected in these sorts of residentials - normally the sort of thing you turn a blind eye to as long as it doesn't affect people's participation in the work and everyone stays safe. This time, they are caught giving weed to a young person with a history of psychosis, and Michael gets scared, worried that the project might be liable for something exceedingly bad happening. The young person is fine, but I notice the shift in Michael from this point.

A few weeks later, Karen, Michael and I are meeting at my office to talk about how we want to handle evaluation and redesign of the Building Bridges scheme. Building Bridges is designed to be a seven year project that iterates as it goes on, so learning lessons from the pilot year is hugely important. Before the meeting, Karen and I have a chat and decide to put up a united front: the evaluation and redesign is where my research and expertise is meant to be put to work primarily. Up until this point, I've been doing the 'observation' and 'action' parts of my methodology, but I'm yet to make it to the 'design'. I've developed a plan for the redesign, involving workshops with young people and the frontline workers both together and separately, to understand what worked for them and what didn't, and some higher-level team workshops where we consider how different aspects of the service design work together. For an hour and a half, Michael lets us speak basically uninterrupted, presenting the plan. Then he says: "Folks, I see you've put a lot of work into this, but I think we need to move on from the current year and start recruiting the next lot. We need to get the next one going by September at the latest." It is April at the time.

We have a break shortly after, and Karen and I regroup. We're both stunned. Michael has always been the biggest champion of making sure we do things differently, of working in a way that makes care-experienced young people feel powerful. What could he be thinking to want to skip the redesign? How can we start a second year of the project before we've even finished the first? We resolve to hold the line, and insist that it would be a huge mistake to skip or even rush the redesign process. We head back in and hold the line together, explaining to Michael that perhaps we could streamline the redesign and evaluation a little, but that it's essential to making sure the project can achieve its aims. He lets us talk uninterrupted for a long time. It begins to feel as if we're having some management training used on us, as if the unsaid mantra is 'when giving news to your project team that they won't want to hear, let them explain their side of it fully before reasserting how you want things to happen'. After a while, Karen falters; she explained to me afterwards that she didn't want Michael to think that she wasn't capable of the quick turnaround to the new program. I hold the line a while longer, trying to understand why Michael is pursuing this, as it doesn't fit with the vision of who I thought Michael was. 
%there's quotes from an old version of ch 5 that i can add to this
Eventually, Michael lets slip that he's been experiencing some pressure from his manager Ben recently. Despite the Central Services Innovation Funds being delivered for a guaranteed seven years, his manager was wanted to see outcomes and outputs, markers of the success that the project is having. Although he's tried to hold this back for a long time, arguing that we won't see results for years as this is a long-term intervention, the pressure on him is increasing. His manager wants him to start a new year of the project to get the numbers of young people taking the program up. It all starts to make sense now, and the slippage becomes clear. Despite his values and intentions being in the right place, he is becoming increasingly unable to fend off the growing pressures on him, and in this meeting, he is offloading that pressure onto us - cutting time where he can in the form of the evaluation and redesign and starting the next project straight away. I can tell that he knows it's not what is best to do for the project, but he also knows that if he carries on  with little to show in terms of results - and increasing risks, like with the rebellion at the last residential and the incident with the young person who experiences psychosis - then he might lose some of the influencing power he has over areas of the Central Services Innovation Fund.

He can tell that the conversation isn't necessarily going anywhere, though. He suggests that we pause the conversation and get a change of scenery, walking around the park near my office. We have more lighthearted conversation and I again see the Michael I know - a man who is clearly reluctant to have to be responsible for the changes to our plans, and who wants to do everything to support these young people. Years later, I asked Karen what she remembered about this incident. She tells me:

"It was mad, wasn't it? Like we had this big argument and then he took us for ice cream in the park, like he were our dad or something."

There was no ice cream in the park that day, but Karen's memory had accurately captured the mood: it felt as if he was issuing an apology for not being there in the way we needed him to, like an absent father might. A slippage had become clear: Michael didn't \textit{want} to have to be this person, to be doing the things he was doing - but he was doing those things because of pressures being placed on him. I wondered if it was just him, or if this was a normal part of the machinery of power in the modern charity sector.

\subsection{Developing 'best practice'}
I was introduced to Seabird as the Building Bridges project got started. They were a partner project in Building Bridges, and Michael and Mandy thought it might be useful for me to get to know Seabird better. My first contact within Seabird was Trudy, who was social work trained but at that time leading a participation project within the organisation. Seabird had been running for around 15 years when I started working with them. Overall, the project employed around thirty people, divided into two streams of work: youth work-led projects and social work-led projects. The youth work was mostly grant-funded, and headed up by [Tori], whilst the social work was a service commissioned by the local authority, headed up by [Stacey]. The project was led by [Mari], a former youth worker and foster parent herself. In my time with Seabird, I met and worked closely with multiple members of the youth work team and a few of the social work team. 

% add to below
% Interesting thing 5: yesterday in the offices of the charity, I'm sitting next to the lead youth worker and we get talking about technology. we're talking about tv, about scrapheap challenge, he used to love that when he wasy ounger and is making his son watch it, he also used to love robot wars 'but I don't like robots anymore'. Goes on to talk about the Boston Dynamics robots, how people kick them etc and the Russians are now outfitting them with guns, eventually starts talking about how 'they're our evolution' and soon we'll end up fighting a war with AI.
% Later, he's talking about... some experiment he heard of where if you put a grain of rice in a small amount of water and neglect it/be nasty to it, it goes foul but if you do the same are nice to it, care for it, then it.. grows well? I don't quite know.
% AND FINALLY he talks about a similar thing about snowflakes - he notices a snowflake on my jumper, says about how all snowflakes are different and he heard about a thing where if you freeze a droplet of water after saying 'love' or 'gratefulness' or similar things to it, it freezes in the most aesthetically pleasing pattern, whereas if you shouted it it freezes in sort of a mess.
% "It's all about resonance, isn't it."

I would spend around a month at a time working exclusively from Seabird's offices. My desk was next to [Alex]'s. [Alex] was a fairly senior youth worker, having worked for Seabird for about ten years. He was in charge of their peer mentoring program, and although he was always up for a laugh, he had a bit of a strange side. Throughout my time with Seabird, [Alex] would often discuss conspiracy theories that he earnestly believed, or talk about pseudoscientific theories. He was incredibly sceptical of the robots being developed by Boston Dynamics, and often spoke about Masaru Emoto's 'research' that showed human consciousness had an affect on the molecular structure of water. For context, one of Masaru Emoto's claims is that water that had been spoken to in a caring, affirming way whilst it froze would freeze in an aesthetically pleasing way, whilst water that had been verbally abused as it froze would form 'ugly' ice crystals. [Alex] liked to use this as a justification for the biological importance of his work as a youth worker - that he was helping to shape the biological structures of the young people he worked with. Of course, this is a basis for understanding that there is a neurobiological component to the building and maintenance of healthy relationships - but it certainly isn't based off of the way that ice crystals form.

I also worked closely with [Kate], the team's schools' liaison; [Laura], who tended to float around different projects in her time with Seabird; and [Claire], a youth worker who had begun to take on the project's participation work with young people. In my early days with Seabird, I asked the team to fill my calendar with everything, and ask me to do anything. This meant that I ended up attending a really varied range of meetings, events, and workshops. I began to work most closely with [Claire], as she was interested on my thoughts as to how best shape the project's new participation work with young people. One day, [Claire] asked if I wanted to attend a 'learning lunch' that she was going to. It was about an hour away but she said she would drive me, and that it might be interesting to me to see how the local authority tried to spread good practice.  

We arrived at a damp building that had once been a school, then a SureStart centre, and which was now living an in-between life: occasionally an events space, occasionally empty. Despite being hosted by the local authority, the event mostly consisted of a lecture from one particular organisation - SafeFutures. SafeFutures were an organisation that had been recently founded in response to constantly-growing Child and Adolescent Mental Health Services (CAMHS) waiting lists; their remit was to do something different, develop new practice, and then eventually, try to make systems change within the space of children and young people's mental health. The event had primarily been marketed as being about e-safety, though, so I was beginning to feel a little confused. I understood the relationship between digital safety and (in particular, child and adolescent) mental health, but I noticed the beginning of a slippage occurring: this was not the event I had thought it was going to be, but this seemed a very familiar set of interactions to everyone around me. I sat back and paid attention. 

The event on e-safety becomes primarily about the need for children and young people to become resilient. This is a concept that is frequently thrown around in work with children and young people perceived to be vulnerable. Whilst it is true that children and young people who score as resilient on outcomes frameworks tend to have better life experiences, this tends to miss the point that resilience is a capacity that is developed within a community: it originates from ecological and systems work that focuses on the ability of a system to return to a equilibrium state from crisis. 
Like a lot of what the person leading the session from SafeFutures was saying, this focus on resilience \textit{does} have a basis in research - just not necessarily in the way that it was being deployed. Individual people cannot become resilient without the system around them becoming more resilient; individual resiliency depends directly on a system's resiliency. 

I begin to notice that the person delivering the lecture has a pattern that he follows in his presentation of information. He will give an incredibly complex explanation of a concept - like resilience, or trauma - and then make out that it's far too hard to understand, and give the audience an incredibly simplified version that misses a lot of the important nuance. He proceeds to give the most complex definition of resilience I've ever heard, then disregards it in favour of explaining that we can think of resilience like Tigger (from Winnie the Pooh; he bounces back) or like Iron Man (able to push things away, and call on his friends for help when he needs it). Neither of these account for the systems-level adaptations necessary for the development of resilience.

The only time he doesn't perform this pattern of presenting the complex definition, disregarding it as too complex, and then giving a too-simple definition is when he explains the neurobiology of a child's developing brain. He uses specialist terms with no explanation. He describes in minute detail the way that Adverse Childhood Experiences impact neurological development. He cites research for the only time in his lecture to support his argument. It appears to me to be a kind of legitimacy work, of justifying the overall takeaway message that 'young people need to be more resilient (in both the physical and digital worlds), otherwise their Adverse Childhood Experiences might affect their neurological development'. There is no room for contestation, or questioning, which strikes me odd as an event billed as a 'learning lunch'.

After the event, [Claire] and I catch up about it. She tells me that she didn't think the event was particularly good, and I'm briefly encouraged, thinking that she might have had a similar experience to me - before she explains that it was because she'd heard the exact same presentation four times before. The slippage becomes clear: even if you were hesitant the first time you heard some of these concepts, by the fourth time, it will have made some kind of lasting imprint. Although they're marketed as events to disseminate best practice, these learning lunches begin to function as a mechanism of control: shaping the space of what is considered legitimate in the world of youth and social work. 

Before the presentation, [Claire] and I had been speaking about resilience, and she had described it in almost the same words as the person from SafeFutures did later in the presentation, as bouncing back, and calling on others. She describes some training she went on recently to become a trauma-informed practitioner, in which she saw a photograph of 'the DNA of people with Adverse Childhood Experiences'. According to her, the photo showed the DNA 'coming apart at the ends', beginning to fray. This isn't how the epigenetics of trauma works. Yes, traumatic experiences leave biomarkers on individual's DNA that may alter the mechanism of expressing that DNA, but the DNA doesn't 'fray'. I'm reminded of [Alex]'s reference to Masaru Emoto's ice crystals. It's starting to become clear to me that a very strange version of neurobiological research is being deployed in youth work settings, disseminated as 'best practice' and delivered in a way that doesn't give any space to critically assess the information. Practitioners who have no specialist scientific training - and who have no real \textit{need} for scientific training - are being asked to learn and work with complex science that is being misrepresented through simplification. Whether it's the fraying DNA, the ugly ice crystals, or Tigger bouncing back, I begin to realise that attempting to understand the neurobiology of trauma and attachment is becoming a necessary endeavour for youth workers working with young people perceived to be vulnerable - that it is becoming an important justification for \textit{why} their work is important.
% add footnote to val gillies brave new brains - is the work only valid if it has a neuro component

\section{Conclusion}

In this chapter, I introduced The Charity, a composite entity that works with children and young people across the United Kingdom, and three of their projects: Small Steps, Building Bridges, and Seabird. I introduced significant people in these projects, such as Shelly and Colin; Michael, Karen and [Sohila]; and [Alex], [Claire], Trudy, [Tori] and [Mari]. Through three vignettes charting the slippages I experienced in my time with these organizations, I have presented the journey that I went on through the course of this research. Importantly, these slippages showed me how organizations that otherwise have values grounded in being young person-led are gradually led away from this focus. Small Steps showed me how organizations begin to focus more on the acquisition of further funding than delivering work they were already funded to do; Building Bridges showed me how even the most young people-led and 'innovative' managers can have their hands tied by organizational constraints; and Seabird showed me how these practices begin to take hold, disseminated through events for 'best practice' whilst deploying increasingly complex justifications for why their work with young people is valuable. 