\chapter{A grounded theory of justification, classification and discursive accumulation practices}
\label{ch:6}

\section{Introduction}
\label{sec:6-1-intro}
In the previous chapter, I detailed the changes brought about by austerity-intensified capitalist realism to workers providing services to young people perceived to be vulnerable, and to the young people receiving that support themselves. These changes have led to affects and experiences marked by powerlessness, distrust, isolation, confusion and anxiety. Yet before turning towards my third research question to understand how it might be possible to intervene meaningfully in this system using the tools and methods of design, it is first important to understand the mechanics of how austerity-intensified capitalist realism functions in practice, how it creates and sustains these negative experiences, and how this relates to the accumulative strategy of the production of vulnerability. This is the focus of my second research question.

The systemic roots of these changes have been described in chapter \ref{ch:2}—the introduction of fiscal austerity, the changes made to youth services funding, and the new importance given to evaluation within the ‘value for money’ agenda. In this chapter, I construct a grounded theory to describe the changes in practice to The Charity I have witnessed which underpin these experiential shifts—focused in turn on justification, classification, and discursive accumulation. Justification practices describe the new labour that workers within The Charity have to undertake that is focused on justification, constantly contorting their actually-existing practice with young people to justify their work or its value to funding organisations and other bodies. Classification practices describe the system that incentivises or mandates these justifications, functioning as a tightly bounded classification system, where funders, policy organisations and thinktanks engage in attempts to delimit and classify the kinds of work that organisations can do, and the kinds of people that organisations can engage with, which manifests in the form of control practices described in the previous chapter. Finally, the looping effects of these justification and classification practices result in organisational performativity that saturates the third sector. Taken together, these practices describe \emph{how} vulnerability is produced and consumed as a commodity in order to stabilise capitalist realism after austerity.  I end the chapter by discussing the potential challenges of beginning to intervene in such a tightly bounded justification and classification system that incentivises performativity and discursive accumulation practices which lead to the production of vulnerability. I also highlight other contexts that justification and classification practices might exist in, as a way to signpost these contexts for further research. 

\section{The turn towards evaluation}
\label{sec:6-2-turn-towards}
In 2011, the House of Commons Education Committee released a report on youth service provision that led to a slew of changes to how youth services were operated, funded, and evaluated. Although evaluation has had a role to play within youth service provision, the \citeyear{house_of_commons_education_committee_services_2011} report acted as a point of intensification, with a call for a "common measurement framework" (p. 4). This common measurement framework came in the form of the government's "A framework of outcomes for young people" \citep{mcneil_framework_2012}, developed by the Catalyst Consortium (The Young Foundation, National Council for Voluntary Youth Services, the National Youth Agency and Social Enterprise UK). The Framework articulated a clear idea for how organisations should conceptualise their work: in terms of extrinsic/intrinsic and individual/social outcomes.

The framework also outlined several tools and technologies which were already being used by organisations who delivered youth services. I encountered many of the tools and technologies mentioned and language used in the report in frequent use throughout my fieldwork: the distinction made between outcomes, activities and outputs; the use of Triangles Consultings' Outcomes Star to evaluate an individual's change in outcomes over time; use of the database technology VIEWS to track individuals' outcomes; the use of theories of change and logic models to explain intended impact; references to baselines, benchmarks, indicators and monitoring; and discussions of "resilience", "risky behaviours" and "protective factors". Like most agenda-setting documents, the framework both set a standard for youth services measurement and evaluation and also reflected the practice of the time. As such, it is important to stress that the processes of justification, classification, and the turn to discursive accumulation I describe in this chapter are not necessarily new: they are the latest form of forces that have existed much longer (due to neoliberal policy changes), and were intensified by austerity. Similarly, the framework amplified what was already there. 

The framework drew upon a recent history of evidence-based policy-making. As \citet[pp. 1–2]{edwards_early_2016} highlight, the turn towards evidence-based policymaking "appears to be indisputably desirable and unquestionable commonsense". Yet this indisputability is a function of capitalist realism itself—part of the creation of hyperreality explored in \ref{subsec:new-sites-strategies-of-accumulation}. Whilst material-semiotic ontologies understand all representations to be subjectively produced, the capitalist realist tendency towards hyperreality attempts to position representations of the world as entirely unmediated. "Evidence" and "phenomena" become equated, as evidence becomes positioned as neutral and directly observable. \citet[p. 1]{edwards_early_2016} note that this assumption is rooted in a "modernising, new manageralist approach to governance in which social values and moral issues are reduced to technical rationality". This turn to evidence-based policymaking also saw a movement towards early years intervention in youth and social work, out of a belief that these are critical moments for a child's development. Whilst a child's early life is undoubtedly important for their development, it is not the \emph{only} critical developmental period. As \citet[p. 8]{edwards_early_2016} highlight, "neuroscientific evidence... is not necessarily called upon its for its actual explanatory capacity, but for its persuasive value". In a later article, the same researchers suggest that the alliance made between the social sciences and the life sciences here is dangerous, because of the way that biological explanations are employed without the assumptions inherent within them being explored or explicated \citep{gillies_brave_2016}. This obscures alternative explanations and prevents critical reflection of the science being employed in these explanations. This same process plays out in in the context of austerity, with persuasion—or justification of an intervention's value—becoming the intention of evidence-based evaluation. 

Following the framework, services were required to conceptualise their work in terms of outcomes and outputs and use tools like those provided in the framework to create a case as to why (and sometimes how) their intervention works. If they didn't, they risked punitive measures from funders, or the possibility of not receiving funding at all. Tessa (the manager of Small Steps) told me that this was incredibly different to how things used to be. Tessa described the early 2000s (when Small Steps began) as a time of "magic and luck, but also definitely waste". She and a few others had an idea for a new project so they went to their local authority with that idea alone: a youth work service specifically for care-experienced young people. At the time, care-experienced young people were being turned away from `universal services' because of the perception that they have difficult additional needs. The authority agreed to fund her if she could provide an itemised budget for the programme; she did so, and the project was funded. In retrospect, she told me that she had realised that the itemised budget hadn't equalled the total: she had budgeted for £13,000 but the items only totalled £11,000. Perhaps this is an example of the "huge waste" that Tessa described that was part of this time of "magic and luck". Yet without this so-called waste, Small Steps might not exist today. Under the deeply structured evaluation and justification regime that is in place now, it is impossible to imagine a project being funded with even £10 unaccounted for, let alone £2,000.

Other workers spoke about this process—the introduction of austerity and the changes in youth service funding—in a similarly evocative way. A worker who I met when she was leading training around young people's rights and entitlements explained to me:
\begin{quote}
It feels like it's all about the money now. The government's really limited with what they fund now... and the whole system is like a pressure cooker about to explode. It's only a matter of time.
\end{quote}
Similarly, a man who worked for a regional mental health charity told me that:
\begin{quote}
The public sector had its heart ripped out, and in its place... it's all budget driven now.
\end{quote}
These two images succinctly describe the scene presented in the previous chapter. Managers, workers and young people are isolated, disconnected, disempowered and stuck in ever-more-precarious situations. Taken together, these images present a system that simultaneously has had its heart ripped out and is at a critical moment, about to explode. The care system continues operating, but perhaps lifelessly so, fulfilling the tasks it needs to, going through the motions to tick its boxes and move young people around placements. At the same time, pressure continues to build, edging over close to destruction or crisis. Simultaneously, then, the care system is inert, routine and crisis-fuelled. 

If the care system under austerity-intensified capitalist realism is lifeless and inert whilst it exists on the edge of crisis, then evaluation processes are the way it appears to be functioning normally. As I will detail shortly, evaluation has taken a central role in the management of youth and social work, as all other activities—even the delivery of actual youth and social work—become secondary to producing a good evaluation. Without good evaluation, organisations risk losing their funding, and therefore might be unable to offer any services, to do any good. As a result, time, money and labour becomes directed towards evaluation. Yet as I argue throughout the chapter, evaluation processes in their current form are responsible for a great deal of bad practice circulating in the care system, and, more than this, lead to the circulation of ill-produced knowledge about what good practice might be. The value for money agenda and evidence-based funding therefore lead to a system which values funding projects that can produce high-quality evaluations rather than funding projects that can produce high-quality practice. I refer to these changes as justification practices—the practices that The Charity developed in order to produce high-quality representations of their work in the form of evaluations, the practices through which they justify the value for money of their projects. In the next section, I describe the components that make up justification practices.

\section{The components of justification practices}
\label{sec:6-3-components}
First and foremost, justification practices require labour devoted to evaluation. This adds an additional pressure to small organisations, who do not have for specialist evaluators: evaluation becomes part of everyone's role, whether they have any expertise in it or not. Larger organisations might be able to afford the labour of specialist evaluators (either as long-term staff or as contractors). Inside of The Charity, only Building Bridges had a dedicated and trained evaluator; both Small Steps and Seabird had to rely on the unskilled labour of their youth and social workers to do evaluation work. At a rudimentary level, evaluation in children's social care revolves around the production of outputs, and the observation of changes in outcomes. Outputs refer to the products of a project—what is being delivered, to how many people, for how long, and the material artefacts that can prove the existence of these outputs. For example, for a mental health project,  The Charity might deliver an hour-long session weekly over three months, and at the end of it have made a film documenting individuals' progress throughout the three months. The film could be considered an output\footnote{Some projects might refer to `activities' as `outputs' also. This is not strictly correct and represents a manipulation of evaluation (whether wilfully through or a lack of skill)}. At the beginning of those three months,  participants might be asked a series of questions about how they are feeling about themselves, their life, and the situations they are in. This is "the baseline". Throughout the project,  participants might be repeatedly asked these same questions—perhaps with the help of a tool such as the Outcomes Star—and by the end of the project, The Charity will have a clear picture of how participants' understanding of themselves, their life and the situations they are in have changed. These are their outcomes, and what funders tend to care most about is what changes in outcomes a project can deliver for as little money as possible. Outputs and outcomes are mutually enmeshed: outputs can become evidence in an evaluation process, and a change in outcomes can become an output of its own accord.

Before outcomes can be identified, however, a project must choose suitable evaluation methods and explanatory tools. These may include tools mentioned within the Framework for Young Peoples' Outcomes—such as a theory of change or logic model, which describe the impact a given project intends to make (or, at a higher level, the entire team or organisation intends to make). These involve mapping forwards or backwards in order to ensure that a project's activities and outputs contribute towards the project's intended impact. Sometimes this happens the other way around, and when a project has already been delivered, the team will look back at it, trying to justify why what they did matters, how it made the changes that it did, and how that constitutes `value for money'. Sticking to a theory of change after it has been made is a rare thing—so this retrospective evaluation approach often happens even if there was a theory of change in place at the beginning of the project. Creating a theory of change retrospectively is a great deal easier than doing so beforehand: teams can simply look at the changes they have made, wrap an explanation around why those changes have occurred, and they appear to be capable of delivering a project which makes all of the change they intended to make. This is not a bad approach in itself to learn about the unintended impacts of a programme's activities and how they might support people in unexpected ways; yet it is a subtle manipulation of the explanatory tools given to organisations. In this section, I will look in closer detail at outcomes and outputs, and show how work practices inside of The Charity have changed to accommodate the need to capture these.

\subsection{Outcomes}
\label{subsec:6-3-1-outcomes}
When a project is established, the team putting it together will likely identify some ‘outcomes’ that they would like to improve in the participants of its project. These may be identified according to some already-existing explanatory mechanism (such as an organisational theory of change), or may be bespoke to the project. The framework \citep[p. 7]{mcneil_framework_2012} presents these outcomes as `hard'—based on "tangible `results' such as educational achievement, participation in training, exclusion from school, offending or challenging behaviour"—or soft—such as "social and emotional capabilities", which are deemed harder to assess. Hard outcomes tend to be quantitatively assessed, as they refer to a binary state—they either are happening or are not. Soft outcomes, on the other hand, are primarily qualitatively assessed and seen as incredibly subjective. Multiple workers within The Charity expressed to me that they felt that funders thought that "soft outcomes counted less", though a few workers did note that they felt things were improving recently (in 2019/2020). If a project's funding does support the evaluation of soft outcomes, though, then there is a need to determine how these outcomes will be measured. Soft outcomes require a rudimentary understanding of qualitative research practice—how do we measure? What do we measure? How often do we measure? How can we develop proxy measures for emotional self-regulation, for example? 

Workers in The Charity acknowledged that though they understood these measures to be necessary to assess the (relative) success of a project, they knew them to be "deeply, deeply flawed". This type of measurement work doesn't feel natural to many care workers, such as Tina, who in my time working with her wanted to "find a fun way to assess" different outcomes. It didn't feel right to her to simply stick a survey on the end of a project (which was the implied alternative), so it felt important to her that she develop a method of measuring that suited her youth work practice. This tension between good youth work practice and good evaluation practice existed in many of The Charity's projects, as the extraction of information purely for the purposes of evaluating the project does not inherently benefit the young person they are working with. As such, sometimes projects are faced with difficult choices between good evaluation practice and good youth work practice. Particularly in situations of greater trauma, risk or perceived vulnerability, there may not be "a fun way" to evaluate something. Colin spoke of a project he had worked on that had an outcome focused on a young person's self-understanding of their risk factors for sexual abuse or exploitation. There is no "fun way" to evaluate this. The opposite of the "fun way", however, can also be difficult as it exposes young people to the bureaucracy of The Charity and can feel impersonal and extractive. 

Because the evaluation process is directly linked to an organisation being able to demonstrate that they provide value for money, the measurement of outcomes becomes a significant element in the delivery of a given project. Multiple times throughout my engagements with Small Steps and Building Bridges, for example, I witnessed workers being told to "nail down what their outcomes are" or to "familiarise [themselves] with the outcomes". Here we see two complementary views on outcomes—as something to be affixed to a project, or as a tool to discipline workers' practice with. The practice that takes place within the project is fundamentally disconnected from outcomes and their measurement; as such, the outcomes that are measured are always speculatively chosen. This can often be an act of convenience, choosing outcomes that might be the easiest to evidence or that seem most likely to change with little intervention. For these workers, evaluation work is hard and requires a great deal of focused attention. Workers who spent more time working closely with evaluation schemes, though, were able to "put [their] outcomes head on" as Jake often did, conjuring appropriate outcomes for a given project from their head. The selection and measurement of outcomes is thus a specific mindset that managers are more easily able to enter than frontline workers, as it is tied closely to managers' bureaucratic functions.%The measurement of outcomes becomes a discrete practice which has little connection to frontline youth and social work.

Some workers struggle to adopt this practice or to see how it would benefit them to focus on this. For these workers,  their work is the delivery and planning of youth and social work interventions, and the measurement of outcomes is secondary to the delivery of meaningful care for young people. Yet this means that when they are forced to engage in evaluation processes or outcomes measurement, there can be a great deal of confusion about outcomes. Karen repeatedly struggled to tell the difference between "what an outcome is, [and] what an output is" as when she joined her role at Building Bridges it hadn't been something she had to engage with previously. Elsewhere, I witnessed this same confusion result in managers having to constantly clarify certain terms to frontline workers and how and when to apply them. The distinction between bureaucratic evaluation concepts and frontline practice is what led to events such as the `learning lunch' described in \ref{subsec:4-seabird}, held by organisations who intended to help workers "make the link [from their practice] to what is measured". Evidently, time, labour, and money is being spent across the social care system to discipline workers to understand what outcomes and outputs are.

\subsection{Outputs}
\label{subsec:6-3-2-outputs}
Outputs are the traces a project leaves in the world, the reification of its activities into objects that can be referred to independently. Evaluation processes produce a series of outputs, which typically culminate in a project evaluation report. Before the report is produced, however, a smaller series of outputs might be produced. These might include other formal documents, such as interim monitoring reports, but can also be as simple as the capture of photos and videos or the production of artefacts on paper by young people themselves. These smaller outputs might be used as evidence of a change in outcomes or merely to demonstrate the project's delivery.  Alongside more evaluative outputs, there has also been a move in The Charity towards creative outputs, public exhibitions, and "celebration events". Each of these attempt to surface a project's activity to a more public audience. These types of outputs might be short co-produced documentaries, the public display of artistic work created by young people, or something akin to an awards ceremony. Outputs like this can create positive feedback loops where more outputs or further change in outcomes (and demonstrations of `impact') are made. For example, the creation of a documentary might lead to screenings of the documentary which influence frontline workers' practice, or which young people bring their friends to, potentially expanding the project's reach. The documentary could be reported on by a local newspaper, or featured in a research project. These secondary outputs expand the project's speculative impact. Outputs may then be used in future funding proposals, as evidence of the successful practice of a given team. As such, the traces of one project can help to give rise to a new project.

In recent years (2017–2020), The Charity has particularly favoured more visually-oriented outputs for three reasons: skill, authenticity, and shareability. Firstly, visual outputs can require less skill to produce than evaluation outputs. Facilitated by the prevalence of camera-enabled smartphones, every worker with a smartphone is able to take photos or videos of their practice. Rather than training workers in how to conduct evaluation work, it is much easier and cheaper to simply instruct them to capture everything. This was reflected in a session I once ran with Seabird exploring what technologies they felt would be useful in the future of the care system; all of the workers unanimously agreed that a technology which could do "automated video taking—so workers don’t have to" was a key feature they'd want in any technology. Likewise in Small Steps, Shelly would frequently tell Colin and Randall to take photos and videos whenever they went out to work with young people or visit other organisations, to create documentation of what they were doing and spending their funding on. 

The need to document visually often overrode good practice—Small Steps did not provide work phones for their staff, and as such, all photos that Colin and Randall took were stored on their personal phones, embedded into whatever digital ecosystem they already stored photos in (sometimes locally on their phones, sometimes using cloud-based storage such as iCloud, Google Photos, or Dropbox). I even found myself folded into these visual documenting practices— at Michael's request, on the second and third Building Bridges residential, I spent the weekend filming the group's activities, to demonstrate to the project's funders how their funding was being spent. I stored these on an SD card which I gave to Michael. After the weekend, Michael uploaded these videos to an open Google Drive folder to share with the funders. Soon after, it became clear that these videos had been shared with one of the young peoples’ care service. Whilst this could be tenuously be seen as legal, it was certainly a breach of the young peoples’ privacy, as they had consented to the funders seeing these videos but not their care service. The lack of skill needed to visually document work can undermine the autonomy and agency that these organisations are trying to build with young people, in favour of performing a positive image of their work. 

Secondly, visual outputs are favoured because they appear to be unmediated, authentic representations of the world. Whilst there is an understanding that evaluations require the subjective curation of outcomes and data collection methods to support these, visual outputs appear to be self-evident. A picture taken of some young people working together and supporting each other at an event appears to be irrefutable evidence of this collaborative atmosphere. Yet just like evaluation methods, visual outputs are also subjectively produced, and can be even more flexibly deployed because of how easy it is to decontextualise them. It is impossible to know outside of the context of production how staged a photo or video might be, and how representative it might be of an event. As part of my work with Seabird, I ran a participatory film-making project that resulted in the co-production of a documentary. The final documentary was a great short film that showed young people's difficulties with gaining access to their care records and the potential benefits of life story work. Yet the process of creating the film was chaotic, exhausting, and involved at least a week of most of the young people we were working with not wanting to significantly participate because they had stayed up late the previous night or had other important things happening in their life at the time. The film does not tell \emph{that} story. The imagined authenticity of visual outputs helps organisations to cultivate a positive image of their work and manage or boost their reputation. Visual outputs can be easily decontextualised in support of any message the organisation is trying to advance.

Finally, this performativity is amplified by the shareability of visual outputs. Many projects within The Charity felt pressured to share their visual outputs. Sometimes this might be on a website or a blog post, but this might also be on social media. Of course, because of the sensitive nature of the work The Charity does (and the fact that it primarily works with children and young people under the age of legal responsibility), social media posts about its work mostly cannot feature photographs of young people showing their faces. Instead, workers will find creative ways to take photos without revealing the identities of young people—having them turn away from the camera, hide their face, or cover their faces with emoji stickers. This pressure comes from a desire to improve or maintain The Charity's reputation, in an attempt to ensure they will be seen as high quality providers of youth or social work. In an abstract sense, this is an attempt by The Charity to imply that they provide value for money through the mere presence of their work on social media. There is something contradictory within the need to make visual outputs shareable, though: in hiding the young people's faces, often the photos they post can appear incredibly generic. In an attempt to prove their work's authenticity, they must obscure much of it.

Visual outputs are not the only way that outputs are prioritised or made `legitimate' however. There is also a focus on securing accreditation to add onto the projects that The Charity runs. These might include things such as the Arts Award, any of the courses offered within AQA's `unit awards', the Duke of Edinburgh award, or the Sports Leader award. These accreditations are often imagined to bring benefits to the young people who achieve them—a clear and externally validated sign that they have been working towards something important and valuable. Young people are often sold on the idea that this will help them to secure employment, because it will look good on a job application. Yet in reality, these accreditation schemes can be of greater benefit to the organisations who facilitate them. The number of young people who have achieved awards or accreditation through their work with The Charity becomes an easy shorthand for them to explain their impact—for example, by explaining that "we have had 6 young people complete their gold Duke of Edinburgh this year, and 30 young people achieved their bronze Arts Award", The Charity gestures towards an externally validated metric of success to indicate the way they provide value for money. This can lead to the addition of accreditation schemes into projects primarily for their power as indicators of impact, rather than for young people's benefits. During my fieldwork, Rebecca at Seabird was assigned the task of finding out about some changes to some accreditation schemes that were upcoming and feed back to the rest of the group. She proudly reported back that she had found out that "basically,... you can fit accreditation into anything". I witnessed this firsthand with Building Bridges—even at the point where we were constructing the project, it had no content and was just an outline of a scheme,  conversations were being had about about what kind of accreditation could be put into it: perhaps we should try the Arts Award, because its criteria is so open, or maybe we need something more significant or bespoke. Or maybe the idea of having to do work would put young people off.  This was also reflected in Seabird, when the project team were meeting with an external organisation about an idea for a potential new project. Halfway through the meeting, after discussing some of the details, Jake interjected:
\begin{quote}
The other thing—sorry guys—Bronze Arts Award? It sounds like a perfect fit!    
\end{quote}
Once Small Steps and the collaborating organisation had established that the Bronze Arts Award might be suitable, there came a discussion of who would get to `count' these as successes. Jake explained that they didn't "see any reason why it can't count for both of us",  mentioning that it would "tick some boxes" for Small Steps if it did. Within this, there was no consideration of whether the Bronze Arts Award complemented the project, enhanced the offer, or was something that young people who might participate wanted to do, evidence, and complete. Instead it was just added onto the scheme as an afterthought, to be counted as a clear output for everyone involved.

In its worst case, visually-led outputs can affect the core of the organisation itself. When Small Steps were struggling to attract funding, they decided to embark upon a "rebrand". Shelly felt that the reason that they weren't receiving funding was that the "look and feel" of the organisation must be outdated because other organisations' branding and visual design was much sleeker and "young person friendly". They worked with a design agency to rebrand the entire organisation. Colin was made to run workshops with young people with experience of homelessness to determine which colours they preferred, which logos they liked, and there was even discussion of renaming the charity at one point. This redesign process took months and required significant labour, and  when the new branding was in place, nothing changed. There was no difference in engagement from either young people or funders after the redesign. Shelly became preoccupied with the organisation's branding because it acts quite literally as the organisation's `image'. Again, the value for money agenda has led to a change in work practices towards the curation and management of appearances. Moving towards visual outputs can encourage charities to undertake work which has no substantive relationship to their aims, and instead prioritises work that appears to be significant but has little material impact.

In all of these cases, the tendency towards visual outputs has made the work that The Charity does performative. Rather than having \emph{done} good practice, it is more important to \emph{show} the idea of good practice. Photos and videos require little expertise to take, appear to be unmediated representations of the world, and can be shared on social media or in other outputs. Securing accreditation gives The Charity externally verifiable ways to refer to the impacts they make. Redesigning an organisation through a `participatory' design process can make it seem as if a project's identity is driven by the young people they work with. Despite the appearance of impact or change here though, these practices divert from the organisation's core purpose, of making material change for young people perceived to be vulnerable. They add additional labour to all frontline youth and social work practice, and rarely is there the level of skill or additional capacity within an organisation to accommodate. I will return to this later in the chapter, in my exploration of the performative impacts of justification and classification practices in section \ref{sec: 6-7-performativity}. 

\section{How evaluation work gets done}
\label{sec: 6-4-eval-work}
Having established what outcomes and outputs \emph{are}, in this section I will examine the content of evaluation work, relaying experiences from across the three projects of The Charity that illustrate how different kinds of evaluation work gets done. In this section, I draw extensively on Latour's approach in \emph{Laboratory Life} \citep{latour_laboratory_1986}, both in written approach and content. In \emph{Laboratory Life}, Latour describes how "the daily activities of working scientists lead to the construction of facts" \citep[p. 40]{latour_laboratory_1986}, as I am similarly concerned with how the daily activities of youth and social workers lead to the construction of "evaluations" and "value for money". Latour seeks to understand how "the costly apparatus... and activities of the bench space combine to produce a written document, and why... these documents [are] so highly valued by participants" \citep[p. 48]{latour_laboratory_1986}, and how rather than being reports about the conduct of his participants, these documents actually become the primary product of their work. In this chapter, I am exploring a similar transformation process to Latour and present these components of evaluation practice similarly. Because of the high stakes given to evaluation practices within the third sector, evaluation reports similarly become its product. 

Latour describes this transformation process as relying upon "inscription devices", any tool or technology which "can transform a material substance into a figure or diagram" \citep[p. 51]{latour_laboratory_1986}. Inscriptions are seen to have a direct relationship to the object or phenomena the inscription has been made about. In Latour's case, these inscriptions are used as the starting point for the production of scientific research papers. In the case of the third sector under austerity-intensified capitalist realism, these inscriptions are primarily outcomes\footnote{Confusingly, outputs can also act as inscriptions, as outputs can be redeployed as evidence, whilst also trying to be positioned as a complete record in and of themselves. For clarity, though, I will focus primarily on outcomes in this section.}.  If austerity-intensified capitalist realism relies upon the value for money agenda and the production of vulnerability, understanding the transformation a given instance of  youth and social work practice undergoes into an outcome (that can be used in an output) is integral to comprehending how austerity-intensified capitalist realism works. This section therefore focuses on how  these inscriptions get made, and the inscription devices that support their production.

I present two distinct versions of this transformation process: a skilled transformation and an unskilled transformation. The unskilled transformation is the sort of evaluation that is conducted by projects within The Charity with limited capacity, knowledge of evaluation, or lack of funding. Generally, this means smaller projects like Small Steps and Seabird, conducting evlauations with staff members who have not collected data before and have little understanding of good research and evaluation practice. The skilled transformation is conducted by larger projects within The Charity, national charities, local authorities, and government bodies. These sorts of transformation use the size and capital-power of the organisation to hire highly skilled evaluation professionals and utilise their expertise to their advantage. In both cases, the data underpinning the evaluation tends to be highly contingent and subjectively deployed, but each pose similar problems to the wider system that underpins the provision of care to young people perceived to be vulnerable.

\subsection{Unskilled evaluation}
\label{subsec:6-4-1-unskilled-eval}
As mentioned, unskilled evaluation tends to take place in organisations which either lack capacity, knowledge of good evaluation practices, or which lack stable or abundant funding. Because of this lack of expertise, the process of creating data on young people's outcomes are highly contingent and subjective. As explored in the previous chapter, these workers are overworked and struggling to cope, and will often have received little training in evaluation, research or data management. Additionally, the inconsistency of the types of roles they are doing within their work means that their workloads tend not to take account of the high volume of data-creation they must engage in, and so it tends to become a low priority task, done only when it has to be. Data is created with little consideration to how it might be used—as one worker once described to me, they were told the most important thing was to play "the numbers game"—working with enough participants for a project to achieve its intended outcomes. I sat with Rebecca and Charlie, a pair of workers from Seabird, as they were evaluating a young person’s behaviour over the course of a weekend residential. Rebecca was new at the time, and Charlie was trying to teach her the ropes of evaluation and using their outcomes tracking system. Charlie implored Rebecca to not give the young person "top whack… yet":
\begin{quote}
Charlie: She’s definitely a 4 [out of 5]. Don’t put her as a 5.

Rebecca: I don’t know, I think she’s a 5. She really came out of her shell over the weekend and joined in with everyone by the end.

Charlie: No, she’s a 4. She’s fat, but she knows she’s fat. So she’s a 4. You’re too nice.\footnote{Although  this somewhat fatphobic comment may seem unrelated to the evaluation, the pair had been previously discussing the young person's struggles to share food and difficulties with emotional eating.}
\end{quote}
In the process of conducting evaluations for young people, workers like Rebecca are being taught to think about how to demonstrate a young person’s progress over the course of a project moreso than evaluating an individual instance of their behaviour. Although Rebecca was new to her role, she had been a trained youth worker for years and had previously been working as a teacher that used youth work methods in her classroom—she wasn't out of touch with the way that young people feel or act. Charlie, more experienced with how things worked in the current outcomes measurement regime, wanted to ensure there was room for improvement—because then Small Steps could take credit for that improvement. This is also an incredibly arbitrarily subjective process with no specific consideration for how data collection is happening.

Imagine the outcome being measured was `teamwork and co-operation skills'. Without a plan for how the team was going to collect data about how young people are doing teamwork and co-operation, then the data becomes (by default) the observations of the workers who are running sessions. These could be a good data collection method, as the professional expertise of the people who are working with young people could be a reliable indication of how a young person is doing. Yet consider everything that was mentioned in the previous chapter about the struggle of workers within this system. The likelihood of a given project having sufficient staff to run a residential or session, keep people on task, deal with issues as they come up, and effectively observe the individual teamwork and co-operation skills of all of the young people participating a session is incredibly low. Moreover, by not having a clear and defined plan for data collection, the team must rely on their own partial understanding of the situation, and actively exclude the possibility for young people to self-advocate. Something may have happened immediately prior to the session that caused them to contribute less, or they may have felt something during a session that was emotionally difficult for them. Without collecting data from young people themselves, there is no way to know about these contextual factors. A worker might leave a session like this thinking that the young person's score in this outcome is decreasing—all because they were unaware of the context. Finally, all of this contingent evaluation practice is compounded by the measurement of outcomes typically being left until a while after a session, as they typically take place at night or at the weekends. This might mean a worker turns up to work on a Tuesday (after taking time off in lieu on a Monday because they worked on the weekend), and has to attempt to recall how each young person behaved at a session at the weekend when they were simultaneously delivering the session. It is easy to see how the discussion merely becomes a debate about whether someone "is a 4" or "is a 5". 

The unskilled nature of this kind of evaluation can also result in unsuitable forms of measurement or evaluation being used, or applied poorly. In Colin's previous job, for example, he worked with young people who were perceived to be at risk of being groomed or sexually exploited. They would work with young people over a period of a few months, and they would complete an Outcomes Star after each weekly session. One of these outcomes was centered on how likely the young person thought they were to be at risk of sexual exploitation. Over the course of the programme, young people would tend to `improve' in this outcome, that is, by the end they would think there more at risk than when they started. This was a key intention of the intervention—to make young people who were vulnerable to these harms aware of just how susceptible they might be, and give then give them a toolkit to deal with these risks. Yet due to this outcome being poorly conceptualised, a cursory view at this data might instead suggest that engaging in the programme makes young people more likely to be sexually exploited. Colin told me of a few occasions when senior management, not understanding that the project wanted participants to have a greater awareness of the risks they were experiencing, had repeatedly demanded explanations for these declining outcomes—and every time, the project team would have to explain that the change in outcomes was a positive thing. A lack of data and research or evaluation literacy can therefore cause deep strategic problems for workers or  managers.

Although unskilled evaluation practice leads to incredibly contingent data collection and potentially poor usage of this data, this is not to lay blame at the feet of individual workers. As explored in the previous chapter, they are under considerable amounts of pressure with an ever-increasing workload, and they are undoubtedly doing the best they can with what they have. Despite this, it results in a poor-quality dataset that might otherwise appear to be of an adequate quality. I will return to the troubling implications of this happening across the wider system later in the chapter.

\subsection{Skilled evaluation}
\label{subsec:6-4-2-skilled-eval}
Skilled evaluation is undertaken by organisations that have sufficient funding available to them to pay external organisations, employ specialist evaluation officers, or train their existing staff. Mostly, this tends to be national charities, local authorities, and government bodies. Within The Charity, this was undoubtedly Building Bridges, who had an evaluation officer attached to the project since its initial design. At the time, Mandy remarked on how peculiar this was, and thanked Michael for the opportunity to get involved with a project at this early stage, rather than having to evaluate a project at its end when data collection or monitoring activities cannot be added in post-hoc. Organisations that can pay specialist evaluators are in the opposite situation to those doing unskilled evaluation: here, it is a few people's job to do all of the evaluation work; whilst there, it is everyone's job to do some evaluation work. One advantage of practitioner-led evaluation, despite its flaws, is that there is always a clear link to frontline practice. The person evaluating tends to be the person who did the work with the young person, and so they can rely upon the informal observations they have gained over time about how that young person acts and how they might have changed. In skilled evaluation, this link is broken. Skilled evaluators tend towards becoming a kind of bureaucrat of the kind described in chapter \ref{ch:2}; their job is to purely to collect, receive, and disseminate information, or to control the flow of information or data. As such, they are highly powerful actors in what has become an incredibly information-dependent system.

Skilled evaluators represent a professionalisation of evaluation as a labour activity. Although this is not a new phenomenon as a result of austerity, it is worthy of comment because of the degree to which evaluation under austerity strongly draws on ideas of legitimacy, verification, and authenticity. The professionalisation process itself seeks to create some actors who can be considered `legitimate' (professional and skilled evaluators) and some who can be considered `illegitimate' (unskilled evaluators). This is described by \citet[p. 796]{oconnell_paperwork_2011} as a rite of passage that those undergoing the professionalisation process must go through, involving "rituals of verification" and which results in "game playing, dissatisfaction and disaffection". Skilled evaluators must know the latest trends in evaluation practice, the tools that everyone is using, the new way to present logic models or evaluation schemes, and must be able to put on their "outcomes head" (as Jake could) at a moment's notice. The process of professionalisation brings about "game playing" behaviour by requiring individuals to become familiar with the tools, techniques, ideas and symbols of evaluation, and to be capable of performing the best quality evaluation with the circusmtance they are given. In contrast to unskilled evaluation, then, the danger of skilled evaluation is that they are able to use the full toolkit of evaluation methods to tweak data collection and analysis processes to present the best version of a project. They are able to subjectively pick from existing data, or ask the right questions that will give them the answers they (or funders, or policymakers) want to hear.

I worked extremely closely with Mandy, a skilled evaluator who worked in The Charity's head office and who worked on Building Bridges and other projects within the organisation. Mandy complained to me that she hated dealing with many of these other projects data because of the inconsistency between them:
\begin{quote}
What am I supposed to do with that? The data produced by different services isn't uniform, it's inconsistent on how they've collected data for different metrics—they've picked and chosen different outcomes from god-knows-where and they may or may not have run their own evaluation, probably in a completely different way than we would have suggested they do it. There's no sense in it whatsoever.
\end{quote}
These services sound like they have conducted evaluation in an unskilled way. Yet Mandy acted as another stage in the transformation process—perhaps acting as an inscription device herself—transforming the content of a service's evaluation or data collection into something that could be usable. Mandy's job is to find the right angle to present a piece of data from. When writing evaluation reports for the entire organisation, then, Mandy is able to draw on a multitude of projects from across the country when trying to make the case for the organisation's brilliant practice. The project The Charity was running in Leicester can't be used as a case study because of some issues with data collection? No issue—Mandy might be able to find something in Somerset, or Kent, perhaps.  Having such a vast range of sources and resources means that professional evaluators are always able to construct a more convincing evaluation than a smaller charity or unskilled evaluator could. It is not just the implicit knowledge that professional evaluators carry, or the networks and relationships they may maintain with policy organisations that makes them produce better evaluations: it is also the fact that they have a disproportionately large source of data, and the capacity and funding to do something with it. As Mandy explained to me at the end of this conversation:
\begin{quote}
We have no reason to think we do anything better than anyone else. The data isn't there to support it. But we have to pretend we do, otherwise we'd never get funded.
\end{quote}
The professional evaluator is effectively a bureaucrat, (mostly) disconnected from frontline practice and sometimes missing vital context. A significant issue with this is that as mentioned in the previous section, the data being recorded and collected by frontline workers can be incredibly contingent representations of an individual's state at one particular moment. These representations remain static regardless of how dynamic and shifting an individual's life might be. In the hands of skilled evaluators, these representations become biopolitical, producing knowledge that is "legitimate" by way of its professionalised processing \citep{foucault_power_2002}. The disconnect between the evaluator and the practice that has taken place result in these representations having more value placed on them. A worker might have a genuinely transformative relationship with a young person, where the young person knows they can rely on the worker and can open up about anything to them. Yet it is only the inscriptions this worker makes about this relationship that has an onward path towards the skilled evaluator. Suppose they chose not to fill in a form at the end of a session because they were trying to listen to the young person in a genuine, authentic and meaningful way. Because of this, they wait until they are back in the office to write their notes from the session. In that time, they have already forgotten much about what has happened. Their workload allocation model doesn't actually take into account the fact they have to feedback on every session, and they're managing a caseload of 25 to 30 young people, who they have a statutory responsibility to see once a month, more than one a day. The worker scribbles down some hurried notes and goes to their next session. A box on an online database might merely read "difficult at first but engaged eventually". Those hurried notes are the only traces made of their interactions with the young person—and they are all the professional evaluator has to work with. Good practice may instead just end up looking like average or even subpar work. The evaluator may not be able to ask more about what happened to fill in the gaps—because they may not have contact with the worker, there may be no indication of who recorded the data, the worker may have left (as worker turnover in the care system is so high), or the worker may not even remember more details about the session. The good practice of the worker patiently listening to the young person they were working with doesn't end up forming part of the evaluation outside of "difficult at first but engaged eventually". 

The opposite of this is also true—bad practice can actually appear to be good practice. This gap between what is and what appears to be speaks to a gap between discourse and reality. At the discursive level, a project or organisation may appear to be doing very genuinely good things, using all of the latest buzzwords, claiming to have a deeply participatory approach and supporting young people through some of the hardest times in their lives. Yet in reality, this may not have ever happened—a skilled evaluator may have been able to spin the limited input of three young people (often this might even be the same three young people engaging in every event) into the appearance of genuine co-production. Many of the slippages presented in chapter \ref{ch:4} can be explained by this—the want or even need to appear one way despite  practice not existing to support that.

For example, Building Bridges described their project as co-produced, aiming to "embed the voice of young people in all that we do". This was an aim for the project before it began and continues to be a basis on which it is evaluated to this day. In the evaluation report of the first year of the project, Mandy noted that Building Bridges "made experts and workers feel like The Charity was really interested in hearing from them". Whilst this feeling may have been authentic, due to my involvement with the project I know just how often participation and co-production were deliberately curtailed at some stages. For example, on multiple residentials, Michael had a tendency to appear to open a discussion on what young people would like to happen. In reality, he always had a desired endpoint; he would keep discussion going until the group had reached the answer he wanted, or  he would make some reason that things had to be the way that he suggested. Sometimes this was for things as inconsequential as the time the group agreed to go to bed, but at other times this was more instrumental. Take, for instance, Michael's refusal (in section \ref{sec:4-building} to devote time to the redesign of the programme. That is a key moment that participation and genuine co-production might have taken place. Yet Michael actively blocked this due to the pressures on him for results from his managers. Despite this, the evaluation spends a great deal of time explaining the positives in Building Bridges' approach, and makes no reference to moments such as this, except with the ambiguous reminder that "Proving that the team is listening to feedback from participants is crucial". Much of the practice that took place within Building Bridges \emph{was} genuinely good, and the programme has consistently got better at co-production over time—but this is precisely the point, that even in a programme that is doing well, skilled evaluation can make something that occupied months of meetings can disappear into a single line.

\section{Justification practices and the production of vulnerability}
\label{sec:6-5-justification-practices}
Austerity-intensified capitalist realism and the value for money agenda has increased the work that needs to be done within the third sector without increasing the capacity available to do that work. As such, workers are experiencing high workloads, feeling as if there is too much to be done, having to do work that they have no experience of performing, whilst managing new emotional labour and having to contend with the control techniques of their managers. To demonstrate and perform value for money, The Charity and other organisations in the sector must conduct evaluations of their work. Projects and organisations that have little capacity or scale will perform unskilled evaluations, using the labour of workers who have not previously conducted evaluation work. Projects and organisations that have more funding will perform skilled evaluations, using the labour of a professionalised class of evaluation workers. Unskilled evaluations rely on incredibly contingent and subjectively produced data with little consideration of the knock-on effects of that data. Skilled evaluations use the expertise of professional evaluators to select outcomes and outputs that will tell the most effective story of the organisation's work. Both unskilled and skilled evaluations rely on the production of outcomes and outputs. Evaluation work functions as a transformation process, where the frontline work of youth and social work practice is transformed through a series of inscriptions ('data collection') and is gradually translated into an evaluation or other similar output that justifies the funding the project received and demonstrates the ways the project has constituted value for money. By directing labour towards the production of these justifications (as they lead to the possibility of further funding), The Charity and other similar organisations direct their practice away from young people, leading to them feeling as if they are not being listened to, and not receiving the right support. As a result of this, workers and young people alike feel anxious, confused, distrustful and powerless. If the production of vulnerability is a key new source of accumulative activity for austerity-intensified capitalist realism, then justification practices are the strategy through which this site can effectively produce (and consume) the commodity of `vulnerable people'. 

All of these experiences are underpinned and punctuated by justification practices. When Shelly prevented Ronnie from contributing to any funding proposals (despite it being her role), she did so because she was terrified that Ronnie would do something wrong with the proposal,  which might prevent Small Steps from being successful in their funding bid and therefore constituting a threat to the organisation. Shelly clearly felt that she was best placed to do the justificatory labour of explaining the organisation's practice. Tina was constantly overwhelmed and burnt out in the time that I worked with her, but felt the need to keep constantly busy because she was terrified that everything might come crashing down around her if she stopped. The majority of Tina's work consisted of justification practices—collecting feedback, launching surveys, collating survey responses, writing up session reports, and even leading a funding bid. Ricardo felt anxious and disoriented because when he tried to access support, he was thrown around different bureaucratic systems filled with workers who didn't understand his needs and who could conceptualise his support only in terms of outcomes. He was seen only in terms of the outcomes that he might generate, which could in turn create a justification about the labour The Charity did, rather than the support he needed.

Justification practices stabilise austerity-intensified capitalist realism. The labour being done by workers and managers alike is creating `vulnerable people' as a new commodity through the inscriptions of practice into outcomes and the justification claims made about these. The deployment of biological explanations by workers (such as in chapter \ref{ch:4}) for why their practice is successful represents an appeal to some kind of external validating authority—imagining that the work is valuable because of the `verifiable' biological changes it is making happen, rather than focusing on the `soft' outcomes of changes in social and emotional health. The use of justification practices within the third sector directs labour towards the production of evaluation-representations of the relative value of a given intervention. They are a dynamic set of practices—akin to capitalist realism's precorporation—though they may often make use of static data to make their arguments. They constantly transform in order to remain relevant to whims of funders—deploying a neurobiological explanation one year, then highlighting The Charity's "trauma-informed" practice another year, before moving onto discussing its "co-production" practice. Justification practices ultimately represent a performative and discursive layer that sits on top of actually-occurring youth and social work.

Above everything, The Charity and other services engaged in this work want to continue existing. These projects often begin with good intentions. Whilst the practice of the project team might shift to contort to the whims of funders, the project team (especially managers) cling onto these good intentions to reassure themselves that they are doing something valuable—that it is better to deliver some kind of supportive intervention rather than not being able to support people at all. Because the need to continue existing guides their work, justification practices emerge in the gap between good intentions and good practice. Justification practices can be seen when a project like Small Steps applies for and receives funding for something they "wouldn't normally do" (recall Randall's "That was never really what we were about!" from the previous chapter), but need the funding to keep them afloat. We see justification practices when a worker is less interested in listening to a young person and more interested in "ticking a box" or "getting their paperwork signed". We see justification practices when an evaluation report transforms a struggling or unsuccessful programme into something that "needs more work, but shows promising beginnings". 

Justification practices are not limited to charities; even local authorities who run in-house social support services enact justification practices. Their funder is the government, and they use justification practices to stabilise their relationship with the state, maintaining the appearance of value for money at whatever cost, even if they are struggling. Once, towards the end of Small Steps's funding, I was asking some professionals if they knew of any funds Small Steps might be able to apply to. I was cautioned by one of the professionals I asked that sharing the team's struggles "wasn't the kind of info that most organisations would want shared with their funders... it might be worth thinking about how you phrase this kind of message in future. One of the things that makes funding harder to get for organisations is if they have a reputation for financial insecurity". If a project admits that they can't sustain themselves, they \emph{create} a situation in which they cannot sustain themselves. Better to perform that everything is fine, to justify the work the project is doing and subtly direct the evaluation to make it seem as if the project is doing well, and just needs some work. 

The control techniques that are used by managers to attempt to maintain a dynamic and high workload create the environment that leads to young people's negative experiences of care. This is not to suggest that managers of charities are solely responsible for the trauma that care-experienced young people experience within the care system. However, this is a clear demonstration of how young people are unable to access the support that they actually need. Far from helping or healing young people, services often end up entrenching the harm that has been done. In his new role as a personal advisor for a local authority, Colin caught up with me and explained that he wasn't happy with some aspects of his job:
\begin{quote}
They keep saying that I'm spending too much time with [a young person Colin was working with who had a significant disability]. But I have no time to do any of the proactive stuff. Like, I know he's going to have problems with his new hob so I could just get onto that, but I can't because I'm being told it's too much. So instead of any actual support instead we just give them money. What's the point in that? It's just teaching them to be materialistic. And then we get mad at them and say they're criminals for stealing things from shops—but we're teaching them that money's the only way they can get support. It's not on.
\end{quote}
In his new role, as in Small Steps, Colin was directed away from good frontline practice by his managers. If there is a direct line between the control techniques employed by managers in the form of justification practices and young people's poor experiences of care, then it is worth interrogating further how and why managers act the ways that they do. In the next section, I will turn to the classification practices that lead to justification practices and that structure the behaviour of managers and organisations more widely, in response to the value for money agenda brought in by austerity-intensified capitalist realism.

\section{Classification practices}
\label{sec:6-6-classification-practices}
To understand the distinction between justification and classification practices, it is useful to draw a parallel to classical mechanics. Within classical mechanics, all things are assumed to be reducible to an elementary unit, and comprehending the way an object is moving merely requires understanding the laws of motion that are operating on that unit or `atom' \citep{capra_web_1996}. The value for money agenda can be thought of as inducing a similar process of atomisation and an attempt to find similar `laws of motion'. The value for money agenda seeks to understand how the most change can be induced for the lowest cost; as such, it becomes important to understand what kinds of intervention work for which kinds of people under which conditions. The focus on outcomes is the atomisation of the practice of youth and social work—an attempt to break it down into individual comprehensible components, in order to explain the exact results of a given project. Rather than good practice taking place and workers being unable to explain what had changed, the splitting of changes into outcomes forces workers to be granular about what changes have taken place. Justification practices then act as the `laws of motion'—an attempt to explain \emph{why} this kind of practice has been able to successfully make these changes. 

By contrast, classification practices represent the atomisation of the category of `the vulnerable' itself. If justification practices seek to establish what changes can be made in a given project and why, classification practices seek to establish who this intervention could be successful for, and what different types of intervention might have in common. If the production of vulnerability is the new accumulative activity of austerity-intensified capitalist realism, classification practices seek to expand this activity through the creation of ever-more sub-categories of vulnerable people, so that there is the need to constantly `innovate' and develop new practices within youth and social work. As explained in chapter \ref{ch:2}, classifications are "spatial, temporal, or spatio-temporal segmentation[s] of the world" \citep[p. 110]{bowker_sorting_1999} and classification systems are a "set of boxes\ldots{} into which things can be put to then do some kind of work—bureaucratic or knowledge production". These classification systems are "tied to a particular set of coding practices... [and] reflect the conflicting, contradictory motives of the sociotechnical situations that gave rise to them" \cite[p. 64]{bowker_sorting_1999}.  Although classification systems may attempt to be totally inclusive, it is impossible for them to do so, and in attempting to identify points of comparability across all classifications, variation will always persist.  This variation is further intensified due to the looping effects of categories as they are applied to humans, which change the characteristics and behaviour of those being classified. As \citet[p. 369]{hacking_looping_1996} explains, when classifications are applied to humans, they "change how we can think of ourselves\ldots{} our sense of self-worth, even how we remember our own past". As people who have been classified begin to behave differently, the classifications that have been applied to them rarely shift, but instead, the classifications tend to become more expansive as "kinds are modified, revised classifications are formed, and the classified change again, loop upon loop" \citep[p. 370]{hacking_looping_1996}.

Under austerity-intensified capitalist realism, then, a dual classification system is operating to both categorise types of vulnerable people and types of interventions that work to support different types of vulnerability. Justification practices provide the groundwork for classification practices to take place. Projects might be classified, for example, by the types of outcomes they support changes within, types of activities involved, or reasons why the intervention is successful. These can be further disambiguated into the kinds of vulnerable person the project might support. One could imagine, for example, classifications based on age (such as children's social care and adult social care), classifications based on type of need (such as substance use support or mental health support), or classifications based on experiences a person has had (such as experience of the social care system, or experience of homelessness). Although there can  be useful and practical reasons to make distinctions between these groups, often one person may fall into many of these simultaneously. Many of the people I have worked with throughout my fieldwork, for example, have been both care-experienced and experienced homelessness; have been children who aged out into adult services; and have needed support for both their mental health and their use of substances. 

The arbitrary distinction between these classifications serves two purposes simultaneously. First, these distinctions create increasingly granular categories for the provision of support. Although a given person may experience all of the issues I have mentioned above and more, a service that is funded only to support them as it relates to their experience of homelessness may be limited to providing support that directly improves outcomes related to homelessness. The restrictiveness of this support can lead to people receiving the `wrong support' mentioned in the previous chapter. Although a young person may work with a service around their employment needs, the reasons they are unable to access work or training opportunities may be related to their mental health, which in turn might be underpinned by substance use problems, homelessness, or money worries. Without supporting these issues, there may be no improvement in an outcome relating to employment, but if project funding is too tightly tied to delivery of certain outcomes, it may be impossible to support a young person with these other needs. When managers succeed in getting a project funded, they receive a classification system in the form of the funder's own worldview, and this can significantly limit the kinds of work they can do—and in turn, can encourage them to limit their own workers' practice more.  Secondly, these distinctions suggest that different types of practice are needed for each different type of vulnerable person that exists, encouraging the development of new projects to establish what good practice might look like in each classification, thus encouraging new accumulative activity through `innovation'. In the rest of this section, I will explore classification practices through the lens of identity-based classification. In section \ref{sec: 6-7-performativity}, I will turn towards this second purpose of classification as it relates to the performative and discursive aspects of justification and classification practices\footnote{It is worth mentioning that practice has improved a little in relation to classification since this fieldwork took place—there is greater acknowledgement of the interrelation of experiences and issues, and a better understanding of complexity and systems. This is used to some extent as its own justification practice, but it is useful to note that this has begun to change.}.

\subsection{Identity-based classification}
\label{subsec:6-6-1-identity}
Identity-based classification (and funding which limited practice on the basis of identity-based classifications) is perhaps the most common classification within the children's social care system. These are not identities that people ascribe to themselves; instead, they are identities assigned to them by workers, organisations, and funding systems. For example, a service may be funded to work with `care-experienced young people', `young people with experience of homelessness' or `young people with experience of the criminal justice system'. The funding may explicitly state that a service is to engage a certain number of people of one of these `types' and that they will be evaluated on the basis of this engagement. In reality, however, these people may be one and the same, particularly as there are significant pathways between care-experience, homelessness and experience of the criminal justice system (for instance, a quarter of people who experience homelessness are care-experienced \citep{mackie_nations_2014}). In part, this is because care-experienced young people are likely to experience frequent interruptions in support which leave them without somewhere stable to live, and also because of early criminalisation, as disputes or arguments in care settings lead to the police being called rather than a de-escalation of the conflict. Yet funding dictates who a service may work with and what they may work with them on. This leads to a necessary identity-based classification within youth social service provision—rather than being able to see a young person as having diverse and varied experiences and needs, services must support people on the basis of one facet of their identity.

These classification practices have material and lived effects on the lives of young people attempting to access support. On the one hand, this may result in a service having to twist the truth, as Nellie had to when working with care-experienced young people on a project to help them bypass barriers to work. She explained to me that she would be talking to a young person:
\begin{quote}
about, you know, anxiety or depression or whatever it is. And then I have to write that down in a way that it's a barrier to work or, you know, so I'd have to just put a certain spin on it. They might not have mentioned work—they might not even be thinking about that, or college. But I have to explain to them that, you know, I'm just going to write down that this is something you're experiencing, and it's possibly a reason why you're not able to hold down a job or access college.
\end{quote}
Although it is true that people may be unable to access or maintain work because of their anxiety or depression, in this case Nellie was focused on finding ways that the funds her role was supported by could help her to do the work that she needed to. The way that she flexibly reconfigured the young person's problems to be framed in terms of how they access or maintain work or educational opportunities is a product of the classifications at the heart of the funding scheme. This is of course, another way that the contingency of evaluation is exploited, but it is primarily an example of how classification practices constrain practice.

Many of the apparently-rigid classifications employed here use medico-legal jargon, leading to a framing of people that is deficit-focused, and which can result in the use of stigmatising language. It is common in youth work practice to refer to young people as "YP" in both written and spoken communication, a simple contraction used for ease. Yet young people are aware of the use of phrases such as "YP" being used about them. One young person that I spoke to had understood the word to be "wipey" (YP pronounced phonetically) and was offended that workers would refer to him as such, assuming it to be pejorative. Similarly, many young people I have worked with objected to being referred to as "NEET" (the acronym for not in employment, education or training). Workers may refer to young people as such because the focus of their project is to work with young people who are not in employment, education, or training—thus working with "NEET young people". This kind of reductionism can lead to young people feeling that workers do not care about other aspects of their lives or see them as a whole person. This was reflected to me by L.TUKZOMBIE who expressed to me that he "ain't no case" when discussing the language of "caseloads" in social care. Being reduced to a singular aspect of their lives can be incredibly damaging for young people perceived to be vulnerable.  

If a funding scheme is established on a stigmatising classification, there will be significant harm incurred to those who are a part of the project. Yet as we have seen, punitive funding regimes can make workers who may well know better than to use this stigmatising language begin to use it in their practice, for fear of eventually losing funding. It may not be the worker themselves that has this fear; it may be their manager, who enacts and lives this fear through the control techniques described in the previous chapter, or it may be a product of their professional environment, through participation in events such as the learning lunch also described in the previous chapter. A particularly prevalent instantiation of this is referring to young people as "risky". Unconditional positive regard, a hallmark of youth and social work practice, would suggest that a young person cannot \emph{be} risky; they may engage in higher-risk behaviours, or be at a greater risk of exploitation, but the professional classification of "risky" instead responsibilises the young person themselves rather than seeking to understand the contextual factors that makes them appear "risky". Young people can be made to feel that they are at fault for their material circumstances because of the classifying language that is used in a project they are a part of.

For example, I briefly worked with Firhampton County Council on a participatory filmmaking project about life story work as part of my work with Seabird. The senior social workers within the council explained what kind of imagery they were looking for from the project, each taking a deficit-led approach. The project was supposed to be about how life story work can be of benefit to care-experienced young people, and was meant to be an honest reflection of young people's experiences of life story work. They suggested imagery  such as a person being gradually built up with LEGO, or drawings of "half a person" being reunited with their other half through the use of life story work. Whilst the workers had intended the point here to be about how life story work can be therapeutic through the restoration of lost knowledge, the visuals they were advocating all suggested that care-experienced young people would be incomplete or less than if they didn't receive life story work—which was most of the young people in Firhampton at the time, hence the film being commissioned. This can also affect who becomes involved with a project. Repeatedly throughout Building Bridges—both in its first year and before it began—Michael referred to the kind of young people that he wanted to work with on the programme as "rough and raw", or "tricky customers", rather than working with young people who are always a part of participation initiative. Whilst Michael's intention here was good—he wanted to avoid tokenism—by conceptualising these young people as rough, raw and tricky, he promoted the idea that they are in somehow less than, unable to keep up with the trappings of liberal bourgeois society. Despite Michael's intent, throughout the programme he repeatedly favoured the less "tricky" of the participants—being more willing to advance the missions of the young people who could present their experiences in a way that was more legible and palatable to the system he was working in.  

This rigid classification of who can be the focus of a planned project or intervention also leads to the stratification or specialisation of knowledge and practice. One day in a Seabird team meeting, we were joined someone from another charity who had received some funding to do a project with care-experienced young people. They explained their idea and was hoping to be able to work The Charity to run the programme. When a member of the team questioned the guest on an aspect of what they wanted to do, they were completely clueless:
\begin{quote}
Oh, well, I don't know about engaging care leavers... that's not something we really do. You guys know much more about that, so I think we'd be looking to you on that one.
\end{quote}
After she left, the team were  incredulous that the guest couldn't answer their fairly basic question. Yet it was not necessarily her fault; the woman from the other organisation was merely doing her job—which didn't focus on working with care-experienced young people. She regarded that as something which required specialist knowledge for a specialist set of problems, rather than understanding how her own practice might be transferable—or even understanding care-experienced young people as humans. As part of this novel accumulative activity, the current funding regime makes out that work with different groups of people is necessarily significantly specialised. Whilst there may be some additional considerations when working with a care-experienced young person to working with someone without care experience, this is not and should not be the defining feature of the practice. The current funding regime instead makes out that all areas of work are highly specialised, and if you are intending to engage a group of people outside of your core expertise then there are specific methods and paradigms you should be following to work in this way. 

This identity-based classification's genesis in a neoliberal funding system based on "value for money" results in young people being viewed in a financialised way. As I mentioned, Michael referred often to the idea of the young people he wanted to work with as "tricky \emph{customers}". He also referred to the work of Building Bridges as a programme of "investing in young people", whilst Mandy, Shelly, and Tessa variously referred to the activities undertaken by The Charity as "products". This is a fairly transparent way of conceptualising the work of service providers: they deliver products to customers, and invest in those customers. Young people perceived to be vulnerable here are being conceptualised in financial terms, articulating quite clearly that they act as a site of accumulative activity for The Charity. In its most limited view, The Charity can be thought of as being paid to work with young people perceived to be vulnerable, then producing the product of young people who might be slightly less vulnerable—but never \textit{not} vulnerable.

\section{Performativity and discursive accumulation}
\label{sec: 6-7-performativity}
Austerity-intensified capitalist realism and the value for money agenda have led to an atomisation of youth and social work practice, attempting to break practice down into individual comprehensible units and classify both the recipients of this practice and the practice itself into distinct categories with minimal overlap. Justification practices produce explanatory models of what youth and social work interventions \emph{are} and why they are successful (or unsuccessful), whilst classification practices lead to constraints being placed on the work that takes place with young people perceived to be vulnerable. As mentioned earlier in this chapter, this change in practice towards justification and classification practices represents a professionalisation of the discipline. Both youth and social work experienced some degree of professionalism prior to the advent of austerity, but austerity and the value for money agenda brought with them an increasing degree of conditionality of funding which increased the need for professionalisation, in order to `prove' the efficacy of interventions. As indicated in section \ref{subsec:6-4-2-skilled-eval}, the professionalisation of evaluation work and the restructuring of the labour of youth and social work around this leads to an increased performativity within the social care system. 

As conceptualised by \citet[p. 173]{butler_gender_2011}, acts may be considered performative if "the essence or identity that they otherwise purport to express are fabrications manufactured and sustained through corporeal signs and other discursive means". Here, we may consider youth and social work practice under austerity-intensified capitalist realism to be performative through its focus on the production of the `fabrications' of evaluations and other justificatory and classificatory technologies. The use of fabrication here should not be taken to suggest that there is no relationship to material reality contained within these evaluations; instead, the fact that they are fabricated can be used to understand that there is a contingent and material process that has gone into their creation. They are not and cannot be unmediated reflections of material reality, but instead are a refraction of the work that has taken place through the lens of austerity-intensified capitalist realism and the individual whims and interests of funders, who have a structuring role within the system. Butler indicates that performative acts are made through the "regularized and constrained repetition of norms" \citep[p. 95]{butler_bodies_2014}, which the process of professionalisation and its associated rituals helps to create. For Butler, this ritualism ensures that the performance is not a singular event, but a continual re-production "under and through constraint, under and through the force of prohibition and taboo, with the threat of ostracism and even death compelling the shape of the production". In the case of the production of vulnerability through austerity-intensified capitalist realism, these constraints, taboos, and potential ostracism is the threat of a charity being unable to continue practicing because of its inability to receive funding. If The Charity stepped too far outside of the acceptable range of discursive expressions (about the kind of work it does, the reasons it works, or the kinds of people it is supporting), it might find itself unable to secure future funding. 

These processes began well before the fiscal austerity of the 2010s, and were a tool of the neoliberal-capitalist realism resonance machine that has been a feature of politics since the late 1970s; austerity merely intensified the use of these tools and their impacts to the extent that these have now become the dominant processes. \citet[p. 225]{ball_teachers_2003}'s description of performativity in teaching identified these same processes at work, drawing attention to how these fabrications become "embedded in and are reproduced by systems of recording and reporting on practice", excluding that which did not fit the narrative of what was to be represented, and making the organisation legible to the state through a continually-improving `best practice', underpinned by `robust procedures', on the lookout for `what works'. These processes  now dominate in the practice of those working with young people perceived to be vulnerable. I have indicated prior to now that this led to a new source of accumulation (the production of vulnerability) and a new strategy for accumulation (the constant diversification of more types of vulnerable people and new ways to work with them, through justification and classification practices). What Butler and Ball's interventions can help us to identify is that this accumulative activity is fundamentally discursive. 

Discursive accumulation can be near-endless, as the surplus value that is being extracted from accumulative activity is performative. As long as novel claims can be made about the practice being undertaken, the site of accumulation can expand infinitely. I refer to this as the creation of `best practice' \footnote{I am distinguishing here between `best practice' (practice that performs goodness) and genuinely best practice, which I will refer to without scare quotes}. Best practice may include practices which lead to genuinely good material impacts, but `best practice' is more focused on circulating a particular kind of practice around a given system for the benefit of an individual organisation or project. This can make `best practice' difficult to critique, as it can appear to be an unquestionable good, but the hyperreality this creates obscures the contradictions that much `best practice' contains. In this section, I will briefly explore co-production and innovation as examples of discursive `best practice' that allow The Charity to obscure much worse examples of practice.

\subsection{Constructing `best practice' through co-production and innovation}
All of the projects within The Charity claim that their work is co-produced, regardless of the presence of any actually co-produced content within a given project. Claiming that a project is co-produced generates an additional discursive value on top of the project itself. In every one of Small Steps' funding proposals that I saw (which were most of the ones that were submitted in the time that I worked with them), Small Steps claimed to value co-production highly and that all of their work took a co-production approach. This was also stated in multiple places on their website. Yet this claim was a significant distance from reality. The project certainly believed in the values of co-production and putting young people at the heart of everything that they did, but their small size and the limitations in practice I have explored in this chapter and the previous chapter simply meant that they had no ability to do co-production practice. The only regular group of young people they had was a single group they worked with one evening every few weeks (for a few months), and instead of interrogating the power imbalance between Colin (who ran the sessions) and the young people, they mostly presented them with things for consultation. There was little ability for the young people to set the agenda of the sessions (and the young people seemed to hold little desire to do so, too). This is not to suggest that Small Steps' practice was bad—they worked effectively with the young people they met with, supported them well, and generally people had a positive experience. Yet there was a significant gap between the consultative practice Small Steps took ("What do you think of this? Would you change something about it?") and meaningful co-production.

A similar process happened with Building Bridges. Because of Building Bridges'\ comparatively larger resource, there were more significant attempts to do genuine co-production throughout the programme. At  the inception of the project, for example, two young people fed into the design of the overall programme, and advised on some decisions that affected its structure. This was an afterthought, however, and was mostly due to the young people being present in the building when the meeting was taking place. Nonetheless, various aspects of the programme were set up for better co-production practice: I supported with an attempt at more participatory evaluation, the structure of the residential weekends was set up to allow for maximal divergence according to the wants and needs of participants, and the overall content of the programme ultimately depended upon young people themselves. Yet at several critical stages this was undermined by Michael in particular. In  section \ref{sec:4-building}, I described Michael's unwillingness to support the participatory redesign of the programme. If Michael truly believed in meaningful co-production practice, he would would have believed that redesigning the programme based on the experiences of those who have actually participated in it is more important than running another iteration of the project with no changes. Yet as he was being pressured by his managers to produce outcomes for the project, he put aside his co-production practice. Nonetheless, the project was described as taking a co-production approach during and after this point. Having anything that looks even remotely close to participatory activity can allow organisations to gesture towards co-production as something that they do. It might not stand up to close scrutiny, but it could be good enough for the depth of attention that is paid to such claims. 

Although co-production was the primary example of `best practice' circulating at the time that I was conducting fieldwork, other examples existed contemporaneously with co-production and have come to prominence since then. `Trauma-informed' practice was one of these. As with co-production, this appears to be an unquestionable good—it is certainly good practice for organisations such as The Charity to be working with an awareness of trauma and how people with traumatic histories might respond to a given programme. Yet the status of `trauma-informed' was primarily conferred by notable individuals in a project going on a training programme. During my fieldwork with The Charity, senior  leaders and managers from within the organisation (such as Michael and Tessa) were sent on a "trauma-informed training programme" for three days. After the programme completed, The Charity referred to itself as trauma-informed in much of its media. Although leaders attending a training programme is a useful step towards genuine trauma-informed practice, a more nuanced understanding of trauma-informed practice might suggest it requires the changing of organisational structures, building of a new organisational culture, or an audit and change of many of the organisation's practices.  Instead, The Charity performed the idea of a trauma-informed organisation without much of the action to support this. Other `best practice' ideas that circulated throughout my fieldwork included `digital transformation' (using and creating digital technologies being seen as an unquestionable good, and `innovative' practice), and `transformational change', which eventually morphed into `systems change'. As with other examples of `best practice', systems change is a practice with a great deal of history and conceptual weight behind it. How systems change might be deployed as `best practice' is often significantly diminished from this, and instead functions at the bureaucratic level of the creation of representations. 

Following the emergent trends that circulate within the third sector around these ideas might be considered to be `innovative' practice. Constantly chasing `best practice', whatever that might be at any one time, can lead to an organisation being considered as `innovative'. Although innovation might appear to be about the production of the new, as \citet[p. 331]{suchman_problematizing_2000} has described, innovation practice is often a deeply conservative project, dedicated to the "reproduction of existing organizational and economic orders" whilst discursively appearing to be about embracing change and novelty. Here, the existing order being produced through innovation practice in the third sector is austerity-intensified capitalist realism itself. Through the creation and evaluation of new practices to undertake with service users, there is a constant production of new strategies of accumulation, and organisations such as The Charity are able to either undertake work or produce representations which perform these new `best practices'. The constant pull to innovate in order to be successful within austerity-intensified capitalist realism encourages organisations to continually undertake new kinds of projects, even if they have identified ways of working that are successful and of benefit to the young people they are working with. This ensures a looping effect through the constant creation of instability within third sector organisations such as The Charity, with organisations taking on too much work, leading to organisational lack of capacity, individuals feeling overworked, and the creation of the experiences and affects described in the previous chapter. 

\section{Conclusion}
\label{sec:6-8-conclusion}
In this chapter, I constructed a grounded theory which describes how and why the experiences explored in chapter \ref{ch:5} came about. I described the turn towards evaluation within the third sector, as a result of the post-financial crisis value for money agenda, and indicated how frontline youth and social work practice is broken down into components as outcomes and outputs. Evaluation work takes place (either by unskilled or skilled evaluators) to describe how and why these changes in outcomes have taken place. Justification practices use these evaluations to produce `vulnerable people' as a commodity, and classification practices act to encourage the continual diversification of this practice, preventing it from remaining static. Finally, I described the performativity that is created by these justification and classification practices, and the discursive form of accumulation that it results in, which relies on the production of `best practice' and innovation.

This chapter has answered my second research question, "How does capitalist realism operate to make these experiences and affects possible (or probable)?" by establishing that that austerity-intensified capitalist realism makes these experiences and affects possible (or probable) through the move towards the "value for money" agenda and youth service provision centered on the production of evaluations. These lead to an atomisation of youth service provision into discrete outcomes and outputs and evaluations that describe how practice creates these outcomes and outputs, which act as justifications of the programme's practice to funders. In order to establish for whom this practice works, potential participants are classified on the basis of their identities. Finally, these justification and classification practices result in performativity and a focus on discursive accumulation at the expense of high-quality practice. In the rest of this thesis, I will turn towards my third research question, "How can the tools and methods of design be used to respond to the challenges presented by capitalist realism?. In the next chapter, I will establish the requirements of a design methodology to resist or challenge austerity-intensified capitalist realism, based on the insights presented in this and the preceding chapter. I will then describe my first attempt to use design tools and methods to respond to the challenges of austerity-intensified capitalist realism, justification and classification practices, and the performative third sector environment in the form of the project \textit{It's Our Future}. 

