\chapter{How does capitalist realism work? Austerity and the production of vulnerability}
\label{ch:2}

% Key question: what is capitalist realism, how has it come to prominence and to structure our everyday experiences, and how does it work? What are its connections to capital, technology and vulnerability?
%change new sites to new sources
\section{Introduction}
\label{sec:2-intro}

This chapter introduces capitalist realism (as conceptualised by \citet{fisher_capitalist_2009}), and explores its relevance to austerity and the third sector. In particular, it elaborates on Fisher's initial conceptualisation by using work by \citet{shonkwiler_reading_2014} to unpack capitalist realism's theoretical basis, before introducing the notion of `austerity-intensified capitalist realism' to explore how capitalist realism has operated in practice (in the context of the third sector after the advent of austerity). This chapter provides the foundation for the rest of the thesis by justifying the third sector as a site for field exploration of capitalist realism, and the use of design methods to explore alternatives to capitalist realism. 

I structure the chapter around \citet{shonkwiler_reading_2014}'s three central components of capitalist realism, re-interpreted here as:
\begin{itemize}
    \item         the creation of instability through changing economics, 
    \item the creation of new sources of and strategies for capital accumulation, and 
    \item the transformation of these experiences into a "common sense" or hegemony that operates at both the individual and structural level. 
\end{itemize}

I explore each of these components in turn and identify how austerity is both exemplary of capitalist realism and an intensifier of it. Applying this understanding of capitalist realism to the third sector, I locate a gap in capitalist realism's explanation of austerity; it is simple to see how austerity has created instability through changing economics, and research has shown how this instability has been transformed into a felt sense of the impossibility of change. It is more difficult, however, to identify the sources of and strategies for accumulation that austerity creates. I propose that the instabilities and existential sense of resignation that austerity creates are part of its new accumulative activity, and that the missing aspect here is an understanding of how the production of vulnerability has become a new site of capital accumulation.

\section{The components of capitalist realism}
\label{sec:2-components}

In his 2009 book \emph{Capitalist Realism}, Mark Fisher describes the titular concept as "the widespread sense that\ldots{} capitalism [is] the only viable political and economic system\ldots{} [and] it is now impossible even to imagine a coherent alternative to it" \citep[2]{fisher_capitalist_2009}. Acknowledging its similarity to Jameson's notion of postmodernism as "the cultural logic of late capitalism" \cite{jameson_postmodernism_1991}, Fisher suggests that capitalist realism represents an intensification of these forces. Although capitalist realism and Jameson's postmodernism stem from the same root, Fisher argues that "the processes\ldots{} have now become so aggravated and chronic that they have gone through a change in kind" \cite[7]{fisher_capitalist_2009}. The links Fisher draws between these shows that for Fisher, capitalist realism is a material, aesthetic, and affective force, just like Jameson's postmodernism. 

To understand capitalist realism as a material, aesthetic, and affective force, \citet[6]{shonkwiler_reading_2014} identify three central components to capitalist realism - the ability of capital to:
\begin{quote}
\begin{itemize}
\item  constantly revolutionize its sources of and strategies for accumulation, developing new configurations of activity;
\item   have an economic, social and affective life that has vast consequences for our lived experiences;
\item  and to transform this constant change and lived experience into a widely accepted brand of Gramscian common sense'.
\end{itemize}
\end{quote}

In order to develop an understanding of what capitalist realism is and how it works, in the rest of this section I turn to each of these components. First, I explore the economic, social, and affective life of capitalist realism by detailing the instability created by a changing economics. I describe the changes in work and labour patterns over the last fifty years through the lenses of Fordist and post-Fordist work, the Foucauldian disciplinary society and the Deleuzian society of control. I locate this project of transforming work as one of the core missions of neoliberalism. Then, I turn towards understanding capitalist realism's accumulative activity, exploring Marx, Luxemburg, and Harvey's conceptions of accumulation, the bureaucratisation brought on by the advent of post-Fordist work, and detail how this creates a hyperreality that is the source of the majority of capitalist realist accumulation. Finally, I identify the methods through which this is transformed into an hegemonic `common sense', where new desires are anticipated and incorporated into capitalist realism, or are \emph{precorporated} from their inception. This ensures capitalist realism's hegemony and reduces the possibility of resistance to it.

\subsection{Instability created by changing economics}
\label{subsec:2-instability}

\subsubsection{Fordism and the disciplinary society}
\label{subsubsec:fordism-and-the-disciplinary-society}

The story of capitalist realism begins with the Fordist workplace and the affect of boredom elicited by the `disciplinary society'. In the middle of the 20th Century, Fordist models of work and labour were dominant, characterised by work that was "routine, hierarchical, mind-deadening, mechanical [and which] tied people to one task\ldots{} for the rest of their lives" \citep[51]{horgan_lost_2021}. Coined by Antonio Gramsci in his \emph{Prison Notebooks}, Fordism describes a kind of work which emerged after the First World War in America founded modelled after the Ford company's production lines. This work was characterised by "assembly-line production, managerial hierarchy and technical control" and relied upon Taylorist models of scientific management and rationalisation, "which simplified necessary operations, eliminated others, and radically routinized, deskilled, and intensified labor" \citep[34]{antonio_new_2000}. In compensation for this labour, Fordist employees were given higher wages than were typical for the time. According to Gramsci, this is because the labour was more "wearing and exhausting than elsewhere" \citep[311-312]{gramsci_selections_2007} due to their focus on efficiency, and Ford believed that to convince employees to perform the labour, "coercion [had to be] \ldots{} combined with persuasion and consent" \citep[310]{gramsci_selections_2007}. Fordism became the dominant model of work in the aftermath of the Second World War as a result of the "class compromise" \citep[X]{harvey_brief_2007} that was brokered between Capital and labour. 
% add another sentence from Harvey about class compromise

The dominance of Fordist work coincided with the "disciplinary society" reaching its peak of activity. In \emph{Discipline and Punish}, Michel Foucault studies the techniques through which the state has enacted
punitive measures against its populace - in essence, how states have created and used the technologies of punishment and discipline. Moving through "torture", "punishment", "discipline", and "prison", Foucault charts the decline of "the great spectacle of physical punishment" \citep[14]{foucault_discipline_1977} and its transformation into more understated punitive measures. Though this is often claimed to be a "humanization" \citep[7]{foucault_discipline_1977} of punishment, Foucault shows that punishment has merely become nuanced and refined in order to maintain its dominance. The punitive methods of a given society or epoch are a "political technology of the body" which tells us about the "power relations and object relations" \citep[24]{foucault_discipline_1977} in that society. Even if techniques of punishment do not make use of the spectacle of violent physical punishment, Foucault shows that the object of punishment is always "the body and its forces, their utility and their docility, their distribution and their submission" \citep[25]{foucault_discipline_1977}. Foucault's project, then, is essentially a "history of the present" \citep[31]{foucault_discipline_1977} or a genealogy, through which he attempts to construct an understanding of how the body is affected and subjected by the techniques of power to punish, and "its bases, justifications and rules" \citep[23]{foucault_discipline_1977}.

Foucault argues that the move away from physical punishment resulted in punishment moving inwards, facilitating a more subtle and implicit use of punishment as a technology. When sovereigns wielded power, they used torture and corporal punishment to make a clear "spectacle", but humanising and reforming groups argued that this was immoral. Foucault states that as early capitalist society emerged, punishment "shift[ed] the object and change[d] the scale", resulting in the development of new tactics to wield power that were "more subtle but also more widely spread in the social body" \citep[89]{foucault_discipline_1977}. The principles of punishment, then, became more regularised, universalised and homogenised - both to reduce their economic and political cost and also create a sense of necessity about them. Rather than baring the sovereign's mark of power through corporal punishment, the body of the punished becomes "property of society" \citep[109]{foucault_discipline_1977}, and the creation of discipline "produces subjected and practised bodies, `docile' bodies" which "dissociates power from the body". In essence, then, "disciplinary coercion establishes in the body the constricting link between an increased aptitude and an increased domination" \citep[138]{foucault_discipline_1977}. As punishment moves inwards and becomes the concerns of a more general social body, individuals are made to desire their own subjection, as the society around them rewards the exercise of greater discipline whilst removing more and more of their individual agency and power.

The `disciplinary society' then is constituted by enclosure (explored further in section \ref{new-strategies-of-accumulation}), partitioning, individualisation, and ranking (placement within a hierarchy). Fisher suggests that these elements created spaces - such as the factory, the school, the hospital, and the prison - through which discipline is coercively cultivated and encouraged. The advent of Fordist work, then, worked to strengthen the disciplinary society. By employing Taylorist techniques such as segmentation of activities into their constituent parts, encouraging repetition and routine, Fordist work and the disciplinary society elicit affects of exhaustion and boredom. Although surveillance is inherent within the disciplinary society, Fisher notes that the place of surveillance does not actually need to be occupied, as being uncertain of whether one is being observed makes people "constantly act as if [they] are always about to be observed \citep[56]{fisher_capitalist_2009}. Fordist work and the disciplinary society, then, create self-disciplining subjects, who are dissatisfied, tired, and made to desire their own subjection.

% \emph{\emph{* note: Control and post-Fordism create a subject which actively desires the subjection and boredom of discipline and Fordism, because it is predictable, familiar, and allows desires to be achieved - as opposed to the debtor-addict figure of Control society}}

\subsubsection{Neoliberalism and resonance machines}
\label{neoliberalism-and-resonance-machines}

Although Fordist work and the disciplinary society are successful at creating self-disciplining subjects who further the interests of Capital, the negative affects they elicit are likely to also create subjects who desire societal change. This dominance was consolidated through the co-option of unions and labour leaders. Fisher argues that these "traditional representatives of the working class \ldots{} found Fordism rather too congenial" \citep[38]{fisher_capitalist_2009} because its stability gave them a permanent role, a professional position as antagonists towards Capital. In doing so, labour representatives essentially traded their "aspirations for stakeholder rights in capital and shared control of the labour process"   \citep[37]{antonio_new_2000} for higher wages and stable employment. As a result, unions became increasingly ineffective during this time (the late 1970s and early 1980s). Fisher suggests that at this time unions did "little to advance the hopes of the class they purportedly represented" \citep[38]{fisher_capitalist_2009}, and began to resist attempts for social change in case it jeopardised their comfortable position. Taking the collective negative affects elecited by Fordist work alongside the failure of these traditional representatives of the working class to support social change, an affective atmosphere emerged of bored, tired workers desiring something `new'. Without progressive social change to fulfill this desire, neoliberalism emerged to capitalise on it.
% check whether this is an AA or SoF

%does harvey talk about ford or dis soc
When this desire for change was hit by massive waves of unemployment and inflation in tandem, the ground was prepared for neoliberal reform.  David Harvey argues that post-war, a "class compromise" \citep[10]{harvey_brief_2007} emerged between Capital and labour, which had been advanced as an attempt to guarantee peace and economic growth. Yet when this "stagflation" \citep[12]{harvey_brief_2007} hit, a crisis of capital accumulation (explored more in the next section) occurred, jeapordising the stability of this arrangement. Harvey argues that "to have a stable share of an increasing pie is one thing" but that when growth rates collapsed and stagflation began in the 1970s, "upper classes everywhere felt threatened" \citep[39]{harvey_brief_2007}, especially as socialist alternatives to embedded liberalism began to gain ground across Europe and South America. The neoliberal project emerged, then,with two central aims: to "realise a theoretical design for the reorganisation of international capitalism" and to "re-establish the conditions for capital accumulation and to restore the power of economic elites" \citep[45]{harvey_brief_2007}. In practice, the latter aim reigned supreme, dispensing with parts of the neoliberal model which didn't quite help the bourgeoisie to achieve their aims of increased (and unlimited) capital accumulation. The former aim instead became "a system of justification and legitimation" \citep[19]{harvey_brief_2007} through which neoliberal (and neoliberalising) rhetoric could flow.
%it's interesting that we have 'justification and legitimation' called out here like this. this is the basis for JP, arguably.

Neoliberalism in theory, then, is ordered around the base idea that the "position of the individual [is] progressively undermined by extensions of arbitrary power" taken by the state, and that state interventionism (as modelled in Keynesian economic theory) and centralised state economic planning (as modelled in the socialisms and communisms of the day) prevented individuals enacting their agency and so-called `freedom'. In particular, neoliberalism valued "private property and the competitive market" above all else, and argued that a decline of belief in these was leading not only to the decay of society but to economic ruin, too \citep{mont_pelerin_society_about_2005}. In the views of neoliberals such as Milton Friedman and Friedrich Von Hayek, the state would necessarily be wrong on matters of investment and capital accumulation because the information available to the state "could not rival that contained in market signals" \citep[49]{harvey_brief_2007}. Thus they argued for a restructuring of the economy and state to prioritise economic freedom, individual property rights, the rule of law, and free trade.

Neoliberalism proposes a minimal state that only intervenes in limited situations - "to defend the nation against foreign enemies, to prevent coercion by some individual by others, to provide a means of deciding upon our rules, and to adjudicate disputes" \citep[53]{connolly_fragility_2013}. In this way, it attempts to model itself after classical liberal economies, using the state to protect the functioning of the market and leaving the rest of society to its own ends. Where neoliberalism differs though is in an understanding - or underestimation, as the case may be - of how elastic these criteria can be and thus how incoherent its ideology becomes in practice. The sorts of actions neoliberal governments end up taking - such as deregulation, dismantling collective bargaining structures and trade unions, dismantling social welfare systems, privatising public services, lowering tax income, and creating systems that encourage foreign direct
investment \citep[52]{harvey_brief_2007} do not fit with the image of a minimal state. What instead results is a "selectively active" state \citep[21]{connolly_fragility_2013} which will intervene in the market in order to "defend the rights of private property, individual liberties, and entrepeneurial freedoms" \citep[49]{harvey_brief_2007} no matter what form of state action this requires.

The selectively active neoliberal state - which seeks the restoration of the bourgeoisie's power after the post-war class compromise - wants the state to "inject market processes into new zones" constantly \citep[21]{connolly_fragility_2013}, such as schools, prisons, healthcare, public transport, logistics, and social care. This understanding of neoliberalism as-it-is thus seeks a state which constantly acts to "maintain the preconditions of market vitality" \citep[60]{connolly_fragility_2013}. Hayek understands this to require a cultural action: the creation and maintenance of "public attitudes and state practices" that allow neoliberal beliefs to flourish \citep[57]{connolly_fragility_2013}. The idea of freedom within neoliberalism then "degenerates into a mere advocacy of free enterprise" creating a "fullness of freedom for those whose income, leisure and security need no enhancing" \citep[265]{polanyi_great_1944}. Because of this, neoliberalism constantly pursues the standardisation and regularisation of everything, resulting in the "pursuit of a nation of regular individuals who have internalized market norms" \citep[53]{connolly_fragility_2013}, who are easier to govern successfully and more able to resist challenges to the neoliberal ideology. The mechanism through which this operates is a biopolitics, creating regularised individuals as its prime unit of understanding. Neoliberal capitalism thus gains "a significant supporting infrastructure through ideological hegemony, state action, neoliberal jurisprudence, schools, and the internalized market virtues of participants" \citep[62]{connolly_fragility_2013}.

The rigid, mechanical structures of disciplinary Fordist work played easily into the hands of neoliberal organisations (such as the Mont Pelerin Society and the Institute of Economic Affairs), economists (such as Friedrich Hayek), and politicians (such as Margaret Thatcher and Ronald Reagan). Coming to prominence in the late 1970s and early 1980s, these groups tried to argue that the current economic and social order was broken, that society and the economy needed something else, and that `something else' was neoliberalism. Neoliberalism essentially positioned itself as the answer to the great question of the day: feel as if you have no freedom in your job and that you're being kept from your potential? That's because we need freer markets that let everyone do what they want to, away from the overextension of the power of the state and the stable, arbitrary antagonism of the unions. Neoliberals around the world - but most importantly for this thesis, in the United Kingdom - rose to power on the basis of this rhetoric.

In many histories and theories on how neoliberalism operates, there is a suggestion that neoliberalism merely proposed an attractive argument to people who were feeling negative affects, they were convinced, and as a result, neoliberals were able to rise to power. In order to better understand how capitalist realism works and takes hold (as it is built upon neoliberalism), I want to make a moment here to  expand upon the process mechanisms of how this might happen,  with reference to Deleuze and Guattari's ideas of deterritorialization and (re)territoritialization\footnote{The Americanisation of "deterritorialization" and "(re)territorialization" is used throughout as these tend to be how these ideas are referred to in wider litature.}. In \emph{A Thousand Plateaus}, \citet{deleuze_thousand_1987} portray deterritorialization as an experience of the oncoming horizon of "the new" - as the ideas of neoliberalism were to people who were dissatisfied with their jobs and lives at the time. Deleuze and Guattari describe how, when a novel element enters a system, the system may attempt to incorporate it, or may reject it entirely. If the system attempts to incorporate it, a moment of deterritorialization occurs, expanding the range of possible classificiations or meanings that could be given to the system. In turn, this is always accompanied by a (re)territorialization, which contracts these potentials, creating new and different boundaries. 

For Deleuze and Guattari, one of the central functions of capitalism is the "generalised decoding of flows" \citep[153]{deleuze_anti-oedipus:_1983} - a mass deterritorialization. As such, Capital is constantly searching for ways to incorporate new elements and (re)territorialize them with its own agenda of accumulation and value-extraction. I would argue that what happens at the neoliberalizsing moment, then, is that the genuine desire for change, liberation, for freedom from the crushing boredom of Fordist work that has been built by years of negative affects at the hands of Capital itself is deterritorialized. People's base desire for something different and better is taken back to its clearest form - freedom - and in this deterritorializing moment, neoliberal capitalism (re)territorializes in its own agenda of freedom through work, private property, and individual enterprise. Neoliberal capitalism deterritorializes desires and (re)territorializes in its own values, which colour and change the desire. Before, you wanted freedom. Now, you just want to be an entrepreneur, or start your own business - because that is what freedom now means.

This deterritorialization and (re)territorialization doesn't just happen abstractly - it operates through the creation of "resonance machines". Connolly describes resonance machines as being composed of "parties who hold overlapping political-economic theories" \citep[68]{connolly_fragility_2013}. Resonance machines are sets of actors who share compatible beliefs. They do not have to believe the same thing, but they work together to enlarge the reach of their shared ideas. Resonance machines then "amplif[y] the sites and modes of inculcation" for each ideology. As such, we can understand capitalist realism as a mode and constituency that is distinct from neoliberalism, but which has benefited from neoliberal actions and participates in a resonance machine with it. Similarly, the political reality of the past  fifteen years (since the 2008 financial crisis), which has featured oscillating amounts of neoconservativism and securitisation is not the same as neoliberalism or capitalist realism, but does also participate in a resonance machine with them. These ideologies make each other more possible, creating the conditions (or preconditions) for each other to go further, or to metamorphose into new states. Austerity-intensified capitalist realism as I continue to talk about it throughout this thesis therefore can be understood as a resonance machine composed of capitalist realism, neoliberalism, and contemporary politics. 

Neoliberalism brought with it a change in the nature of work and everyday life. In contrast to the mundanity of Fordist work, the post-Fordist work of neoliberalism "promised to be flexible, exciting, fast-paced, based on team-work, and full of variety". For Fisher, post-Fordist labour is marked by "flexibility", "nomadism" and "spontaneity" \citep[28]{fisher_capitalist_2009}, a kind of decentralisation that becomes hard to resist because it appears to be an uncontestable good in the face of the freedom and the "new" of neoliberalism. This flexibility is a (re)territorialization of the desire for freedom in the Fordist workplace. Although ostensibly workers appeared to have more "freedom", post-Fordist work removed the possibility of stability, consistency, and permanence. This destroyed traditional sites of labour power (by removing the consistency and predictability of the Fordist workplace) and could be summed up, Fisher argues, by the slogan "no long term" \citep[36]{fisher_capitalist_2009}. Where once workers could build skills and move (albeit slowly) through the hierarchies of an organisation, "now they are required to periodically re-skill as they move from institution to institution, from role to role" \citep[36]{fisher_capitalist_2009}. Nothing is ever - or can ever - be over.

Fisher suggests that this transformation of economic and social life under capitalist realism also changes the nature of power, from disciplinary society to `control society'. In his 1992 "Postscript on the Societies of Control", Deleuze suggests that "societies of control" have taken hold, in which power and punishment are distributed through the "modulations" of control \citep[4]{gilles_deleuze_postscript_1992}. In contrast to enclosures, which act as defined spaces, the modulations of the Control society ensures that things "continuously change from one moment to the other", just as in post-Fordist work. "Perpetual training" replaces education; "continuous control" replaces examination \citep[5]{gilles_deleuze_postscript_1992}. Deleuze argues that in the disciplinary societies, "one was always starting again", but that in the control societies "one is never finished with anything". No longer is the aim to meet the rigid, strictly-enforced structures of discipline - because you cannot, you won't, it will always be infinitely delayed - but you will still have to try. Fisher suggests that in the control society, "external surveillance is succeeded by internal policing" \citep[22]{fisher_capitalist_2009} whilst feedback mechanisms - their own kind of modulation - are governed by the technological, able to change at a moment's notice. 
%clear link to make here is that feedback mechanisms are the mode through which the control society works - thats what JP are

The true danger of the control society that was brought about by neoliberalism and post-Fordist work is how it transformed the nature of desire. If, in the control society of capitalist realism, you can never be done with anything and are always governed by relations of indefinite postponement, desires can never be fulfilled. If I want something, I may well get that thing, but I will never be able to appreciate the realisation of that desire, immediately wanting something else. Fisher suggests that this leads to a kind of "reflexive impotence", a state in which people "know things are bad, but\ldots{} know they can't do anything about it" \citep[21]{fisher_capitalist_2009}. An affect of boredom and resignation continues to circulate, but no-one feels as if they can do anything about it. In \emph{Anti-Oedipus} \citet{deleuze_anti-oedipus:_1983}, Deleuze and Guattari argue that desire is not a lack, but instead a productive, creative force. Control societies instead arrest our capabilities to produce and create through desire by indefinitely postponing fulfilment of the desire.  This sets the stage for the next aspect of capitalist realism, which deals more significantly with the nature of how Capital operates under capitalist realism: its creation of new sources of and strategies for accumulation, made to temporarily salve these new desires which can never be fulfilled.

%Haven't spoken about how the regularisation that neoliberalism does creates homogeneity which creates stagnation hauntology and nostalgia }}

\subsection{New sites/strategies of accumulation }
\label{subsec:new-sites-strategies-of-accumulation}

The lack of fulfillable desires created by neoliberalism's transformation of the economy set the stage for  new sources of and strategies for capital accumulation, potentially creating products, services and methods of surplus value extraction which were not previously possible. To understand these new sources and strategies, we must first understand how accumulation within capitalism has been theorised. In \emph{Capital},  \citet[736]{marx_capital_1889} claims that "primitive accumulation" functions as the Original Sin myth of capitalism, retold by those aligned with the interests of capital:

\begin{quote}
In times long gone-by there were two sorts of people; one, the diligent, intelligent, and, above all, frugal elite; the other, lazy rascals, spending their substance, and more, in riotous living\ldots{} It came to pass that the former sort accumulated wealth, and the latter sort had at last nothing to sell except their own skins. And from this original sin dates the poverty of the great majority that, despite all its labour, has up to now nothing to sell but itself, and the wealth of the few that increases constantly although they have long ceased to work.
\end{quote}

Marx suggests that capitalists attempt to maintain that those with capital arrived at it simply by frugality. In contrast, Marx claims that the labouring class were actively dispossessed of their land. Private property was created through the enclosure of land, a primitive accumulation whereby land is expropriated from the commons and an exploited class is made to sell their labour  in order to subsist. Primitive accumulation thus acted as "the historical process of divorcing the producer from the means of production", which "transforms\ldots{} the social means of subsistence and of production into capital" and turns direct producers into wage labourers \citep[785]{marx_capital_1889}. 

In \emph{The Accumulation of Capital} \citep{luxemberg_accumulation_2015}, Rosa Luxemburg noted a central problem with Marx's thesis that primitive accumulation is sufficient for describing the functioning of capital: under Marx's view, it appears that "capitalist production would itself realise its entire surplus value, and that it would use the capitalised surplus value entirely for its own needs" \citep[309]{luxemburg_accumulation_2015}. Luxemburg shows that this cannot be the case - as wage labourers receive less value than they create (a necessary condition of capitalism's functioning, via the creation of surplus value), this means that capitalism "is unable to exist by itself", needing "other economic systems as a medium and soil" \citep[447]{luxemburg_accumulation_2015}. According to Luxemburg, then, accumulation becomes "a relationship between capital and a non-capitalist environment" \citep[398]{luxemburg_accumulation_2015}. Herein lies the contradiction of capitalism, then: capitalism "strives to become universal", but is "incapable of becoming a universal form of production" \citep[447]{luxemburg_accumulation_2015} because of the requirement of an "outside" to expand into, to make surplus value productive.

David Harvey casts Luxemburg's analysis of this problem of accumulation as a theory of "underconsumption" - wage labourers are not able to consume enough to make the logic of capital accumulation make sense. By contrast, Harvey casts this problem as one of "overaccumulation". Overaccumulation is a condition in which "surpluses of capital\ldots{} lie idle with no profitable outlets in sight" \citep[149]{harvey_new_2003}. Harvey's response to the problem of overaccumulation is to suggest that accumulation by dispossession then occurs, whereby assets are released at a low cost so that "overaccumulated capital can seize hold of such assets and immediately turn them to profitable use" \citep[149]{harvey_new_2003}. Although accumulation by dispossession retains the idea of an outside into which capital can expand from Luxemburg, Harvey astutely notes that "capitalism necessarily and always creates its own `other'" \citep[141]{harvey_new_2003}. Capitalism can either make use of a pre-existing outside - in the form of colonial expansion, or movement into sectors which have yet to be effectively capitalised - or manufacture an outside through dispossession.

Harvey suggests that primitive accumulation functioned as an accumulation by dispossession - land was taken and enclosed, the population who occupied the land were expelled, and a landless people was created as the enclosed land became private property. Neoliberal privatisation of public services and utilities (in the UK and other neoliberal countries) - such as water, gas and electricity, the postal service, and even housing) also function as accumulation by dispossession. In each of these circumstances, something that was held in common was then enclosed or made private and access to or use of that thing became conditional. Harvey also suggests that accumulation by dispossession can also occur through flooding the market with "cheap raw materials", or "the devaluation of existing capital assets and labour power" \citep[150]{harvey_new_2003}.

Accumulation by dispossession provides a good model for understanding the new sources and strategies for accumulation under capitalist realism. If the key \emph{change} under capitalist realism is its transformation of work and power, then accumulation in the capitalist realist economy must therefore be tied to the structures of post-Fordist work, the control society, and infinitely postponed desires. Fisher argues that one of the key changes in the nature of work is the creation of a class of bureaucrats who can administer, regulate, and maintain the new state of affairs. What emerges in practice is a highly centralised network of managers and bureaucrats who wield power but assume no responsibility for how they use that power. These kinds of bureaucratic procedures appear to "float freely, independent of any external authority" but this in turn means that they face a heavy "resistance to any amendment or questioning" \citep[55]{fisher_capitalist_2009} able to make decisions only by "refer[ring] to decisions that have always-already been made" \citep[53]{fisher_capitalist_2009}. They are agents of the society of control, deferring to a power which is always-present and yet never really there. Although bureaucrats have been present in many different economic arrangements, the post-Fordist bureaucrat is a central strategy for accumulation - creating new sources of accumulation.

These new bureaucrats spend the majority of their labour creating artefacts which hold a symbolic status within the realm of post-Fordist neoliberal work. Across different sectors, this may appear differently - in the public sector, for example, this manifested as New Public Management, whilst in the contemporary technology sector, this has manifested in the form of the Agile software development methodology. These bureaucrats are responsible for the creation or maintenance of "aims and objectives", "outcomes" and "mission statements", functioning as auditor, evaluator, producer and interpreter of symbols. This new form of labour is "geared towards the generation and massaging of representations rather than the official goals of the work itself" \citep[46]{fisher_capitalist_2009}. In this new form of accumulation, anything can become a site of enclosure or debt because this strategy of accumulation depends upon "a complex series of social semiotic signals" \citep[54]{fisher_capitalist_2009}. Capital again created its own outside, but this time, rather than dispossessing land, labour or materials, it dispossessed the representational content of that which it accumulates. Put simply, capital has transcended material artefacts: instead what now becomes most relevant is the aesthetic, ideological or intellectual component of an artefact. Capital begins to accumulate through this network of representations, resulting in an economy in which "real world effects matter only insofar as they register at the level of (PR) appearance" \citep[46]{fisher_capitalist_2009}. In capitalist realism, then, accumulation becomes a strategy of manipulating the appearance of reality rather than (or in addition to) manipulating reality itself.  

There are two central problems with an accumulation that is centered on the creation of representations: the creation of a sense of "hyperreality" and the massive expansion in potential sources of accumulation. In \emph{Capitalist Realism}, Fisher argues that the transition towards post-Fordist bureaucratic accumulation participates in a Baudrillardian "hyperreality" \citep[52]{fisher_capitalist_2009} Bureaucrats create \emph{representations} - which are necessarily contingent slices and mediations of the world. However, many of these representations claim not to be representations - instead claiming to "present reality in an unmediated way" lead not to a direct encounter with reality but to a "hemorrhaging of the Real" \citep[52]{fisher_capitalist_2009}. This creates a hyperreality, in which it becomes impossible to understand what is real in a way that would exist without the creation of a representation, and what is real only by way of the act of creating a representation. Fisher explains this with reference to reality TV shows, fly on the wall documentaries, and political opinion polls - in each of these situations "reality" is not simply accessed but is being actively produced - and so the production of  reality potentially gets in the way of experiencing authentic reality. 

This hyperreality is essential to the sense of necessity produced by capitalist realism. There is a strong visual component to capitalist realism: in order to make people believe that capitalism is the only viable political system and it is futile to imagine or try to build anything other than it, it is important to produce a large number of hyperreal representations of the world. These hyperreal representations try to make claims about the base nature of living in the world that they are simply cannot. In late 2000s and early 2010s media, for example, this took the form of media which was "gritty" or "grimdark" (such as the 2000s \textit{Batman} films, or \emph{Children of Men}. The creators of this media often claim that they are simply representing reality as it actually is, and pundits might proclaim that their work is incredibly authentic. In doing so, these creators show themselves to conform to a capitalist realist view of the world: that as Thomas Hobbes would have it, life is "nasty, brutish and short", and humans are fundamentally selfish creatures. Viewing the world as such, and creating media and bureaucratic representations that claim to be the world "as it is" serves the purpose of destroying the possibility of hope for something other than our current imaginary of the world. As Fisher says, "we are integrated into a control circuit that has our desires and preferences as its only mandate - but those desires and preferences are returned to us, no longer as ours, but as the desires of the big Other [capital]" \citep[53]{fisher_capitalist_2009}.

In tandem with this move towards a representation-centered economy, the potential  of accumulation become massively expanded. If what matters in this new accumulation is representation, not something's material existence, then the constraint of the "old" accumulation are removed. These constraints could be the amount of land available to enclose, production limits of factories, or even the amount of hours in a day to extract value from service workers. The symbolic or representational content of a thing thus becomes worth much more than anything to do with its materiality, and heightens the importance of branding and marketing under capitalist realism. 
%this is weak
If desires are infinitely postponed under capitalist realism, Fisher suggests that its representation-centric methods of accumulation act as a form of "dreamwork", producing a "confabulated consistency which covers over anomalies and contradictions" \citep[64]{fisher_capitalist_2009}, whilst also creating a cultural anxiety about "memory disorders". Fisher refers to capitalist realism as a "condition\ldots{} of ontological precarity" in which "forgetting becomes an adaptive strategy" \citep[60]{fisher_capitalist_2009}. We remember the easy-to-understand, simplistic narrative of events, rather than the complex one.  Here, the hyperreal returns, as everyone knows at some level that they are doing this simplification and "dreamwork", but at the same time finds themselves unable to do so.

If this dreamwork and memory-anxiety goes on for long enough, it creates a cultural affect of nostalgia - a longing for what was. Fisher argues that contemporary culture is "excessively nostalgic" due to the fact that it is "incapable of generating any authentic novelty" \citep[63]{fisher_capitalist_2009} due to its homogeneity and hegemony. The hyperreal dreamwork taking place here leads to a craving for "familiar cultural forms" \citep[63]{fisher_capitalist_2009}, being continuously drawn back to the security of old things. This stagnant, overly nostalgic culture thus is incapable of creating novelty, relying on the old and familiar for the appearance of novelty. There is no particular difference, for example, between modern vaporwave music and the smooth jazz and corporate lounge music being produced in the 1980s and 1990s - except that the smooth jazz and corporate lounge music of the 1980s and 1990s were responding to their own cultural referents, rather than a past culture's. For our purposes, this is important for understanding how  cultural nostalgia and the postpone of desire of the control society can make it hard to imagine a different future - either for the world or for ourselves.

In later work,  Fisher argues this produces a profound sense of hauntology. Hauntology acts as an ontology that is identified with that which is not present, making clear that "everything that exists is possible only on the basis of \ldots{} of absences" \citep[187]{fisher_ghosts_2014}. Fisher extends this concept to the idea of capitalist realism and the collapse of possibility it creates, by arguing that what is doing the haunting in hauntology is not something that actually existed and once was (as in nostalgia) but "the spectres of lost futures" which never materialised \citep[21]{fisher_ghosts_2014}. Whilst these futures that never came to pass might be lost, they sometimes surface themselves through material artefacts, traces that can break through the fugue state of capitalist realist dreamwork. Yet capitalist realism remains an hegemony that transforms desire and experience through these new methods of accumulation and the transformation of social and economic life. In this next section, then, I turn towards understanding how capitalist realism functions as an hegemony, acts with plasticity, and is able to absorb and precorporate resistance.

\subsection{The creation of hegemony and "common sense"}
\label{subsec:the-creation-of-hegemony-and-common-sense}

The final aspect of capitalist realism is its totalising, hegemonic power. As explored in the previous section,  there is a contradiction within Luxemburg's theory of accumulation - capital will continually attempt to become universal, but relies on the existence of an `outside' in order to function. Harvey suggests that in neoliberalism, this outside its created through devaluation, enclosure or by flooding the market with raw materials (another form of devaluation). In contrast, Fisher suggests that capitalist realism engages in \emph{precorporation} order to create its outside. Precorporation describes capital's ability to shape desires as they emerge, so that the form they emerge in can be more easily incorporated into contemporary capitalism. In traditional conceptions of hegemony, it might be argued that capital has a significant capacity to incorporate potentially subversive materials and actors (as in the case of the co-option of union management, for example). Precorporation goes one step further, suggesting that rather capital actively structures and shapes materials and actors them from the outset so that they can more easily fit into the structures of capital. Fisher uses Nirvana as his clearest example of this: though "alternative" and "independent", Nirvana knew well that "nothing runs better on MTV than a protest against MTV" \citep[13]{fisher_capitalist_2009}. Resistance to capital therefore becomes an intrinsic part of capital itself. Understanding how this can be possible is integral to understanding the totalising and hegemonic force of capitalist realism.

Precorporation is made possible because of capital's plasticity. Developed by Catherine Malabou, the concept of plasticity refers to aspects of any system that "allows play within the structure" \citep[44]{malabou_sovereignty_2015}, or that "loosens the frame's rigidity". Plasticity then allows for the preservation of an overall structure whilst recomposing the elements contained within it. Capital - and particularly neoliberal capital - have a huge capacity for plasticity, needing only to retain the accumulation of surplus value in order to continue functioning. So long as accumulation remains possible, capitalism can plasticly reconfigure itself and its elements to any possible form. This plasticity enables precorporation:  under capitalist realism, capital is able to adjust itself, its form, and the external perception of its form in order to appear differently, so that things which might appear subversive are always-already amenable to incorporation by capital. Capitalist realism therefore functions dialectically - its abstract is capital itself, its negative is the possibility of subversion, and its concrete is the incorporation of that resistance into the capitalist system. As explored earlier, this is maps onto the processes of deteritorrialization and (re)territorialization that are inherent to the functioning of capital.

Because capital seeks to become universal yet must generate an outside, the continuation of capitalist realism relies on the creation of "safe" resistance through precorporation. To ensure actors under capitalist realism continue to create incorporatable resistance, precorporation must act as a governmentality. First developed by Foucault, governmentality refers to both the "rationalities, technologies, programmes and identities of regimes of government" and the "mentalities of government" \citep[24]{dean_governmentality_2010} that must be created in order to support this. It widens our understanding of governance as a practice to show how specific modes of thinking become embedded in language and other instruments of power.  They are biopolitical, acting as "processes of power that seek to regulate and control life" \citep[XX]{lemke_biopolitics:_2011}. These processes are expressed diffusely, through structures, technologies and regimes of government which support capital's interests. By interacting with these structures on a regular basis, people who aren't aligned with the interests of capital gradually develop the mentalities of government that a given form of government seeks to create. Put simply, by interacting with structures of power, one becomes liable to think how the power structure demands. The governmentalities of capitalist realism, then, encourage people to notice their desires for something other than the contemporary sociopolitical arrangement, and then express them in ways that continue the current sociopolitical arrangement. 

%nb. This point connects to hegemony and articulation -  should also discuss subjectification and bio politics here**}}Interacting with the power structure requires the acceptance of the frame of reference of the power structure.

For Fisher, precorporation leads to the creation of a feeling of reflexive impotence, where people know that "things are bad\ldots{} [but] they know they can't do anything about it" \citep[24]{fisher_capitalist_2009}. Fisher argues that this feeling of powerlessness is reflexive because it creates a cycle: the recognition that you feel powerless then discourages and disincentivises action. Fisher does not go into a great deal of depth on what creates this reflexive impotence other than the control society. I would instead argue that this is the result of the broader aspects of capitalist realism presented thus far. In essence, the control society and advent of post-Fordist work creates new  and strategies of accumulation that rely upon bureaucratisation and the feeling of hyperreality it creates. This hyperreality encourages people to engage in dreamwork to smooth over capitalist realism's inconsistencies, which in turn precorporates desires which are always-already a part of capitalist realism. Reflexive impotence emerges, then, because of the appearance and feeling of the absolute dominance of capitalist realism. Put simply, this might be summed up as: "Work is always changing yet always boring, the people with power claim that they can't change anything, nothing will change, so there's no point trying to change anthing". It is easy to see how a sense of powerlessness results.

The final element of capitalist realism discussed by \citet{shonkwiler_reading_2014} is its creation of a "common sense", which reflexive impotence is integral to. This idea of a "common sense" is originally invoked by Antonio Gramsci in his discussion of hegemony \citep{gramsci_selections_2007}. In Gramsci's rendering, common sense refers to collectively held opinions that have gathered "a social force" behind them. In contrast to political society, which uses domination as its means of power, Gramsci highlights that total control is achieved through the use power within civil society, which takes the form of hegemony over social and cultural life. As a result, Gramsci argues, ruling classes are able to reform consciousness \citep[]{gramsci_selections_2007} and actively manufacture consent for their governance \citep[39]{harvey_brief_2007}. In the case of capitalist realism, this has taken the form of traditional institutions and media reperpetuating the idea that change is impossible and that life has always been broadly similar to how it is now. 

Whilst Gramsci's conception of hegemony is useful for understanding institutional reproduction of power throughout civil society, I have already brought some understanding of this into capitalist realism through Foucault's notion of governmentalities. Laclau and Mouffe refine hegemony in useful ways, however, by centering on how antagonisms can proliferate around the potential meanings of "floating signifiers" \citep{laclau_hegemony_2001}. For Laclau and Mouffe, all practices that attempt to establish a relationship between elements can be considered an "articulation", which represents a claim about the nature or value of an element, or the relationships between some elements. Hegemonic formations occur when a social and political space becomes "relatively unified" \citep[136]{laclau_hegemony_2001} in its articulations. For example, the word "freedom" may function as a floating signifier - is loose enough to mean many possible contradictory things to many people. Any one of those individual meanings of freedom exists as an articulation. An hegemonic formation would exist when many people become unified on that same articulation being the only possible articulation. To use Deleuze and Guattari's language, floating signifiers are very tightly territorialized - their boundaries are not particularly porous. Yet Laclau and Mouffe make clear that even in a hegemonic formation, there are always competing articulations. No articulation can become fixed forever, which assures  the possibility of resistance. Yet in line with capitalist realism's dynamic, adaptive, and plastic qualities, an hegemonic formation "embraces what opposes it" as any opposing force must accept the basic articulations of a formation to attempt to negate it \citep[139]{laclau_hegemony_2001}. In essence, to critique capitalist realism, one must engage with the terms that capitalist realism sets out, making it easier for it to precorporate resistance.  If something can extract surplus value or otherwise support the economic changes of capitalist realism, it can be successfully incorporated. Capitalist realism, then, is a hegemonic formation resting upon the articulations that: there is no meaningful alternative to (neoliberal) capitalism; there is no possibility of changing the current political, social and economic order; and any negative feeling that arises as a result of this is a sign of something wrong with an individual specifically.

Capitalist realism, then, describes the transformations that have occurred to create massive instability and precarity through a change in the political, social and economic order and a change in the nature of work. It details how these transformations have created new  and strategies for accumulation, and shows how these new mechanisms of accumulation lead to a bureaucratic and powerless culture. This culture slowly stagnates through its focus on representations, and in doing so, begins to become a totalising hegemony, in which any potential resistance to capitalist realism is always-already re-incorporated into it. The cultural material from which people are working under capitalist realism begins to stagnate, and there is a sense that even the very idea of the future is retreating, as hyperreality paints a sense that the world has always been as it currently is, and will always be that way. In the face of complete and totalising power, people feel powerless and unable to change this system. How could one even begin to imagine the possibility of change, when it feels as if there are no successful examples of change, and society tells us again and again that it has always been as it is now? 

The "slow cancellation of the future" \citep[5]{fisher_ghosts_2014} posed by capitalist realism, then, drives us back to the essential question of Deleuze and Guattari's work:
\begin{quote}
How is it that there is always something new? How are novelty and change possible? How can we account for a future that is different from, and not merely predetermined by, the past? \citep[23]{shaviro_deleuzes_2007} 
\end{quote}


\section{Capitalist realism in practice: austerity and the third sector}
\label{sec:capitalist-realism-in-practice-austerity-and-the-third-sector}
If austerity acts as both an example and intensifier of capitalist realism, we would expect to see \citet{shonkwiler_reading_2014}'s components of capitalist realism replicated and extended. Austerity would have to create greater instability through changing economics, create  and strategies for capital accumulation that are entirely novel, and transform these experiences into a `common sense' that made social change or any alternative feel impossible. In this section, I explore what austerity is and the impacts that it has had, particularly within the third sector where some of the largest changes have been felt. I note how the instability and creation of `common sense' of austerity are easy to see, but identify that it is less apparent what the new  and strategies for accumulation are.    

\subsection{What is austerity?}
\label{subsec:what-is-austerity}

In a speech at the Conservative Party spring conference in 2009, David Cameron redoubled his party's commitment to an entirely new approach to governance if they were elected. He claimed:

\begin{quote}
There are deep, dark clouds over our economy, our society, and our whole political system. Steering our country through this storm; reaching the sunshine on the far side cannot mean sticking to the same, wrong course. We need a complete change of direction\ldots{} In this new world comes the reckoning for Labour's economic incompetence. The age of irresponsibility is giving way to the age of austerity. \citep{cameron_age_2009}
\end{quote}

This "age of austerity" was an attempt to signal an end to the prevailing economic ideas of the past fifteen years or so \citep{blyth_austerity_2013}. Although these policies appeared Keynesian in nature, they actually were an attempt to enact a progressive neoliberalism, functioning as an "ambidextrous state" \citep[X]{peck_zombie_2010} where one hand of the state supported social causes and the other  hand of the state developed punitive responses to veer people away from accessing social welfare support.

Austerity measures were developed in response to the 2008 global financial crisis, supposedly as a way to bring public spending under control after the aforementioned "age of irresponsibility". For austerity to make sense, it casts the financial crisis as a "sovereign debt crisis", in which the state spent too much on social welfare and financial support for the poorest in society, spending "wastefully". In response to this, austerity measures were presented as a need for "everyone [to] tighten\ldots{} their belts" \citep[13]{blyth_austerity_2013}. By falsely equating household debt with national debt and public spending deficits, the Coalition government was able to position austerity as a necessary set of measures to be taken after a shared sense of "going too far". Yet the financial crisis was never a sovereign debt crisis - it can be perfectly acceptable for a states to be in a public spending deficit, as much public infrastructure spending is costed on the basis of reaping the benefits of long-term investments. At its core, the financial crisis was a banking crisis that was built upon neoliberal deregulation of the banks and a lack of legal and professional oversight of financial institutions, so that unregulated credit could be easily given to those who had little to no ability to pay it back - essentially, through creating masses of personal debt. Austerity, then, was an answer without a problem.
%**add in Lowndes and Pratchett 2012 which talks about key policies of Coalition government**

%could maybe add more literature
When the Coalition government took power in 2010, the Conservatives and Liberal Democrats quickly set about transforming the state to enact austerity policies. Austerity affected local government, central government, other public sector organisations and even third sector organisations in a multitude of ways. Central to this new age of austerity were plans to cut the budgets of public sector organisations and to incentivise projects and organisations that delivered "more for less". The Coalition government cut spending and changed policies in order to ensure that a new logic of "value for money" reigned supreme.  Here, we can see the enactment of the austerity governmentality: the machineries of governance changed to force the reduction of public spending, and changes in logics and rationalities accompanied it. Whilst no area of public spending in the United Kingdom remained untouched by austerity policies, one of the key areas of financial cuts and policy transformations was services delivered to children and young people \citep{youdell_assembling_2015}.

\subsection{The impacts of austerity}
\label{subsec:the-impacts-of-austerity}

These cuts and policy transformations had both material and affective impacts. Materially, austerity led to a massive decline in funding for public and third sector organisations. Small and medium-sized charities experienced a 16\% and 17\% decline in income, respectively (over the 2008 - 2014 period), and there were larger declines in charity incomes in more deprived areas \citep{clifford_charitable_2017}, despite the narrative that "we're all in this together". The "Big Society" policy agenda that accompanied austerity suggested an enhanced role for voluntary sector organisations would be important, but the massive reduction in income for mid-sized organisations was a huge barrier to this. In some cases this was due to a direct removal of public funding for the charity, but in others it has been linked to a general decrease in charitable donations. \citet{jones_uneven_2016} has highlighted the spatial disparities in the effects of austerity on Voluntary and Community Sector (VCS) organisations, showing how affluent areas have been able to maintain healthy levels of VCS funding whilst deprived areas have not. Yet affluent areas have comparatively less need for VCS organisations' support in the context of austerity, intensifying the negative effects of austerity in areas without high levels of charitable donation. VCS organisations are being asked to do much more with exponentially fewer resources.

Particularly in the case of youth services, service provision has disappeared or become particularly strained. Thousands of youth work jobs were cut and youth centres closed across the country as a result of a 70\% reduction in local authority youth services spend in England since 2010 \citep[9]{ymca_making_2020}. The reduction in funding and loss of jobs also resulted in greater competition between organisations, which tends to favour larger, national organisations and which makes the development of positive relationships between organisations in the third sector more difficult. As funds were reduced and made increasingly conditional, decisions about the allocation of resources began to be made on the basis of "value for money", which in practice often means "the cheapest bid, rather than explicit commitment to support existing work which has demonstrated effectiveness and commitment to local communities" \citep[732]{clayton_distancing_2016}. Most significantly this "austerity localism" has led to a breakdown in trust between the users of support services and voluntary and third sector organisations. This increased sense of distance between the state and organisations, and organisations and the state has created "forms of disconnect between those in power and those who feel on the receiving end of damaging decisions" \citep[737]{clayton_distancing_2016}.

Building on the affective impacts of austerity, \citet{horton_anticipating_2016} highlights austerity's role as an anticipatory politics, an imagined-and-anticipated future that becomes embedded into governmental discourses and highlights how this anticipated future helps to build a "common sense" around the idea of austerity policy. Horton's work focuses in particular on the experiential and affective aspects of austerity, furthering \citet[632]{peck_austerity_2012}'s notion of exploring "the politics of everyday austerity at the street level\ldots{} experienced in daily life, in workplaces, households and the public sphere" by trying to understand what "anticipated service withdrawal" \emph{feels} like. Horton identifies a multitude of different affects around the anticipated future of service withdrawal: feelings of "low mood", "distrust", anxieties about young peoples' futures, and a feeling that people were "just waiting" because "time [is] running out" circulated in response to the horizon of an austere future.

Through its affective life, then, austerity creates a waiting subject that anticipates a negative future. This anticipation of negative affects can be seen as one of the defining factors of an austerity-intensified capitalist realism: it transforms people's experiences and therefore subjectivities. Recent work on austerity has paid increased attention to its affective life, both to describe the affects that circulate as a result of austerity and to understand how these affects change the everyday life of those who experience them. For example, \citet[102]{hitchen_living_2016}  focuses on austerity "shape[s] capacities to feel and act", which draws attention to the contradictions and complexities of austerity as it is actually experienced. Through this lens, Hitchen suggests austerity can be understood as a multiplicity that surfaces in different ways through individuals' everyday lives, as a series of affective atmospheres that "envelop and condition everyday moments and spaces" which shapes everyday life and future imaginaries or anticipations. These atmospheres include "frustration", "disappointment", "anxiety", and "fear". Importantly, these negative affects sit alongside changes to future imaginaries or an understanding of the possibility of action. Hitchen specifically highlights "the presentation of absence" as a limiting factor to action, a perceived lack of possibility which conditions and structures people's capacities. This can lead to both an acceptance of austerity, in order to attempt to "get on with life" and a paralysis in the face of austerity, as people become exhausted with merely trying to "stay afloat". In this way, then, austerity's affective atmospheres "shape bodily capacities to act as they envelope, and therefore influence, subject-object encounters" \citep[117]{hitchen_living_2016}.

Austerity-intensified capitalist realism, then, acts in both the background and the foreground, impacting every single part of life but also failing to surface enough to become an active presence. \citet[195]{raynor_dramatising_2017} also addresses the "diffuse and disparate" ways that the contradictory present absence of austerity surfaces in people's lives. For Raynor, mundane presences and absences bring austerity to the forefront, like the unanswered phone or the empty flower bed. These presences and absences take on a vital life and agency of their own, making apparent the all-consuming power of austerity. Yet this is never a power that can be taken to completion. Much like the Luxemburgian contradiction of accumulation - that it must become universal but requires an outside - Raynor argues that austerity attempts to structure all experience but is always fractured, fragmentary, and partial. As Raynor states, under austerity, things "fall apart" because austerity is always contradictory, paradoxical, incomplete.

\citet{hitchen_affective_2019} picks up on austerity's temporal nature and the ways it shapes experience. For Hitchen, these contradictory and paradoxical affective atmospheres are both "uncanny" and "paranoid", and created an extended experience of austerity's temporality. The uncanny emerges because things that are unknown become entangled with with things that are known. As people begin more familiar with the perpetual unknown, it becomes more familiar and comfortable than the known. It is always bounded by the understanding that when the unknown becomes the known, material circumstances will get worse; thus it is better to exist in the unknown. For example, it feels more comfortable to be waiting for the outcome of a benefits claim than to know that the claim has been unsuccessful. Living with the uncanny then  creates a "paranoid mode of waiting" \citep[13]{hitchen_affective_2019}. Through the repetition of "employee engagement processes" or other such bureaucratic capitalist realist measures designed to engage in dreamwork or memory disorder, Hitchen suggests that "a temporality emerges that looks both forward and backwards: forwards through the unknown knowledge that remains absent, and backwards through knowledge imparted from all previous employee engagement sessions" \citep[12]{hitchen_affective_2019}. In doing so, paranoia is created and shapes people's capacities and understanding of time. Austerity-intensified capitalist realism, then, becomes an unbounded temporality which threatens to transform people's entire understanding of time. Because there is no clear start or end to austerity due to this, austerity-intensified capitalist realism is able to secure hegemony by colonising the past, present, and future, and our imaginaries of which of those are, were, or could be like.

Austerity-intensified capitalist realism exerts its hegemonic power across subjectivity, materiality, affective atmospheres, and temporality. Austerity is able to intensify the effects of capitalist realism because of its self-optimising nature: exhausting people and their capacity to act against it. \citet{harrison_cant_2020} has built upon this to explain why the public response to austerity has been so muted. Harrison blends together civic voluntarism, grievance theory, and policy feedback theory to suggest that those most affected by the lived realities of austerity are those who most lack the resources to take action on austerity. For Harrison, austerity arrived at a time of already decreasing levels of political participation. Because the impacts of austerity have been unequally distributed, responses to it have been also. On the one hand, those who have been affected most severely by austerity lack the material and mental resources to participate in oppositional activity - as we have seen, austerity exhausts and disempowers people. On the other hand, those who are minimally affected by austerity lack sufficient grievance with the policy to mobilise opposition towards it, due to successful government rhetoric around the economic necessity of austerity - "we're all in this together". These factors combine to create significant barriers to resisting austerity, and ensure the maintenance of a high-need, low-resource status quo - cementing austerity's impact.

In addition to this, it is worthwhile considering how austerity has affected how people identify with their own subjectivity and class positioning. \citet{jeffery_classificatory_2019}  have shown that as a result of austerity, people who might traditionally have identified as working class instead feel a sense of "shameful identification" with that identity or indeed a "disidentification" with it, based on a displacement of that identity to others (who represent some kind of negative Other). Jefferey et al. position these as "classificatory struggles" after \citet{tyler_classificatory_2015}, and show the experiential impacts of almost a decade of austerity policies. Not only do people feel shame around working class identity, but they have internalised this affect of shame to the level of personal identity. In large part, this is due to stigmatising media narratives that have come to dominate the public consciousness - or as Gramsci might argue, create a new "common sense". Other work around working class subjectivity has confirmed this. In analysing the response around the Grenfell Tower fire, \citet{shildrick_lessons_2018} develops the concept of "poverty propaganda", which builds consent for a regressive and capital-supported class politics by stigmatising working-class identity. Through TV shows such as \emph{Benefits Street} and \emph{Life on the Dole}, alongside an austerity politics that demonises "shirkers" and "scrounger[s]", the working class subject becomes a figure of shame, a haunting figure that stalks people's experiences, intensifies the effectiveness of capitalist realism, and which disempowers subjects almost entirely.

The material changes brought about by austerity are indicative of its creation of instability through a changed economic situation. Voluntary and community sector organisations are less able to deliver support than prior to austerity, and this impact is concentrated on less affluent areas. Youth services have been reduced by 70\% and there is much greater competition between organisations, leading to a centralisation of service provision and the loss of positive relationships. As a result, there have been significant affective changes which have created a "common sense" around the impossibility of change. Austerity is experienced as anticipatory, creating negative affective atmospheres of frustration, disappointment, and anxiety, alongside paranoia and waiting. What is less clear, though, is how austerity has created new  and strategies of accumulation. In the next section, I will propose that these  and strategies of accumulation can be located in the production of vulnerability in the third sector.

\subsection{The production of vulnerability}
\label{subsec:the-production-of-vulnerability}

The accumulative activities of austerity are not immediately clear. As \citet{blyth_austerity_2013} explains in \emph{Austerity: The History of a Dangerous Idea}, one of the central problems of a global policy of austerity is a crisis of accumulation - it becomes impossible for capital to become productive. Someone has to spend in order for someone else to save. Thus if all states try to cut their growth at the same time,  a global financial stalemate emerges. Blyth invokes Keynes to explain this paradox of thrift: "if we all save at once there is no consumption to stimulate investment" \citep[8]{blyth_austerity_2013}. If an entire country's economy is paying back debt at the same time, then the only way capital can become productive is through exporting more to a state that is still spending (capital expanding into its outside, as per Luxemburg and Harvey). If every country is paying back debt and trying to bring down the amount of public spending at the same time, capital has nowhere to go to become productive. Although austerity hasn't (in most places) created any noticeable level of growth, capital did seem to stabilise in response. The question becomes - if there is global austerity, how was this crisis of accumulation dealt with - and what are the new sources of and strategies for accumulation that developed as a response? In this section, I argue that austerity's creation of insecurity and precarity is itself a new (or revitalised) source of accumulation, through the production and management of vulnerability by third sector organisations. 

Charities that provide support services to people they perceive to be vulnerable have had to completely change the way they operate as a result of austerity policies. Not only have charities become increasingly responsible for the provision of services due to the "Big Society" policy agenda, but they also find themselves operating in an even more competitive environment. Charities find themselves operating in a heavily marketised and financialised environment, competing against each other for tenders for contracts to deliver vital services \citep{buckingham_capturing_2012} or for funding from grantmaking organisations, meaning they must follow these organisations' agendas \citep{clayton_distancing_2016}. As Adams has noted, market-driven governance such as those engendered by neoliberalism, "enable the needy to become a site for the production of capital, generating profits for companies that spring into existence after a disaster", in turn creating an "affect economy" predicated upon the management or amelioration of people's suffering \citep[9]{adams_markets_2013}. Following \citet{lord_profit_2018}, I argue that the new role that charities and social support organisations have taken on as a result of austerity is one of these such enterprises. Austerity operates by "encouraging business intervention into areas where its presence was traditionally limited", adding market-driven rationales, and "subordinat[ing] charity to business" resulting in "the insertion of an ethos of private profit into charity work" \citep[5]{lord_profit_2018}.

Viewed in terms of accumulation, Lord's suggestion that charities have had to become businesslike due to austerity can be understood as a claim that spurred on by austerity, charities have made "the vulnerable" into a site of accumulation. Yet charities are both the producer and consumer of "the vulnerable" as commodity. They create and prepare the commodity of the vulnerable (for others, including other charities or themselves, to make profit from) and they consume that commodity, spending contracted funds, grant funds, or donated funds on `improving' the living situation of "the vulnerable", but only to such an extent that they remain vulnerable. Thus charities under austerity-intensified capitalist realism act cyclically: they prepare the commodity of "the vulnerable", allow others to consume "the vulnerable", and consume "the vulnerable" themselves. In doing so, charities reproduce "the vulnerable". Charities rely upon vulnerable subjects existing in order to continue functioning, which they do in order to support vulnerable subjects.

Under austerity, then, charities and other social justice organisations have become an important part of the accumulative framework. Austerity and other social injustices create a market for "the vulnerable" by worsening social inequalities, and increasing levels of deprivation or marginalisation. Marketised charities attempt to respond to this increased need, but need to do so in a way that conforms to the structures of capitalism and austerity-induced `value for money' policies. As such, they contribute towards the continuation of the very situations that sustain vulnerability. As will be explored in greater depth later throughout this thesis, charities may do things that run counter to their ostensible purpose, but which are necessary for their continued existence. In doing so, one of their primary activities - and one of the primary mechanisms for accumulation under austerity-intensified capitalist realism - becomes the work of classifying people as vulnerable.

According to \citet[110]{bowker_sorting_1999}, a classification is "a spatial, temporal, or spatio-temporal segmentation of the world". A classification system is a "set of boxes\ldots{} into which things can be put to then do some kind of work - bureaucratic or knowledge production". Under austerity-intensified capitalist realism, charities attempt to create classification systems which assert their ability to perform well and suggest that their model of how to work with "the vulnerable" is the best "value for money". When  classifications are applied to humans, however, they create "looping effects" \citep{hacking_looping_1996} which change the characteristics and behaviour of those being classified. Hacking  explains that when we create new ways of classifying people (such as "vulnerable"), we also "change how we can think of ourselves\ldots{} our sense of self-worth, even how we remember our own past". For Hacking, this generates a looping effect, because people who have been classified begin to behave differently; classifications attempt to create standardised segmentations of the world, but because of these changes in self-conception and behaviour due to the act of being classified, "kinds are modified, revised classifications are formed, and the
classified change again, loop upon loop" \citep[370]{hacking_looping_1996}.

\citet{campbell_making_2015} specifically deal with the looping effects of the classification of vulnerability. In their paper, they describe how the figure of the "vulnerable human subject" emerged in line with the
emergence of the field of bioethics. The figure of "the vulnerable" was seen to be a member of a racial minority group from an urban area who was likely to have a low income and likely to have a minimal education. Prior to the emergence of bioethics, people who were incarcerated were frequently seen as an ideal population for long-term research because of their perceived stability. After the emergence of bioethics, "children, prisoners and the `institutionalised mentally infirm'\ldots{} [and] people at a socioeconomic disadvantage, including racial minorities" \citep[16]{campbell_making_2015} began to be seen as unfit for use in research programs, due to the possibility of coercion or exploitation. In surveying evidence of those involved in LSD trials both at the time and through oral histories, Campbell and Stark show that people are able to reorganise their experiences to fit with the new classifications that have been developed to describe them, in order to see their past in a different way. Put simply, those who were once not "vulnerable" came to understand their experiences in light of "vulnerability" once the classification had been applied to them.

Returning to the case of austerity-intensified capitalist realism, then, this internalisation of the accumulative activity of classification results in a further intensification of the power, stigmatisation, and affects of powerlessness that circulate as a result of capitalist realism. "Vulnerability" is a lens through which an individual's capacities might be externally moderated or protected, creating "vulnerable" subjects who must remain passive, or be controlled. Individuals who are supported by charities then might slowly begin to understand themselves in the context of "vulnerability", and may begin to limit their own capacities in response to this, or expect control, paternalism, or passivity. This is the contradiction of charity work under austerity: although ostensibly organisations appear to be helping their service users, they only help them to the point that accumulative activity can continue (i.e. their charity can continue to make money, earn grants, or receive donations as a result of the work), and may limit the capacities or functionings of their service users - or encourage them to limit their own capacities or functionings. Vulnerability is a new and enlarged site of accumulation under austerity.

\section{Conclusion}
In this chapter, I have introduced the concept of capitalist realism, as coined by \citet{fisher_capitalist_2009}. I explored capitalist realism's theoretical basis with reference to \citet{shonkwiler_reading_2014}'s components of capitalist realism, which focus on its economic, social, and affective life, its new sources of strategies for capital accumulation, and the transformation of these experiences into a hegemonic "common sense". I detailed the research and theoretical frameworks that underpin capitalist realism, surveying the move to post-Fordist work, the disciplinary society, and neoliberalism, before detailing Luxemburg and Harvey's conceptions of capital accumulation to understand the differences in capitalist realist accumulation. I used this to explain how capitalist realism becomes hegemonic, through its use of precorporation and creation of reflexive impotence. 

After this, I turned towards understanding capitalist realism in practice, with reference to third sector service provision after austerity. I presented austerity as both an example and intensifier of capitalist realism, first presenting the material changes of austerity as evidence of its creation of economic and social instability, before detailing the way that hegemony and "common sense" show up under austerity through the creation of negative affects such as anxiety and paranoia. Finally, I suggested that the new source of accumulative activity under austerity is the production of vulnerability itself, giving third sector organisations a vital role to play in stabilising the economy post-financial crisis. In producing (and consuming) this vulnerability, I propose that third sector organisations do a disservice to their service users, limiting the efficacy of their work with them in order to continue the existence of the organisation itself. 

Throughout this chapter, I have been exploring my first research question, "what experiences and affects does capitalist realism elicit?". Although I will return to this question in chapters \ref{ch:5} and \ref{ch:6}, this chapter provides a good foundation for understanding what prior research has identified about the experiences and affects elicited by capitalist realism. What has been less clear throughout this chapter is what the new strategies for accumulation used under austerity-intensified capitalist realism are. As a new source of accumulation, the third sector is integral to exploring this, but further research is needed to understand precisely \emph{how} capitalist realism manages to extract surplus value through the production of vulnerability and how the dominance of this order is maintained. Throughout the rest of this thesis, I will be exploring these strategies of accumulation in order to answer my second research question, "how does capitalist realism operate to make these experiences and affects possible (or probable)?" To explore this, I will need to conduct primary fieldwork inside of third sector organisations, to understand how vulnerability is both produced and consumed, and what actions are taken to enable the smooth functioning of capitalist realism.