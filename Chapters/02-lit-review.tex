\chapter{Literature review: capitalist realism, the production of vulnerability, and the digital turn}
 \label{2}

Key question: what is capitalist realism, how has it come to prominence and to structure our everyday experiences, and how does it work? What are its connections to capital, technology and vulnerability?

\section{Introduction}
\label{2-intro}

If capitalist realism is so powerful, how does it work, and how might we reach outside of its bounds? In this chapter, I explore the theoretical background to capitalist realism, before examining what capitalist realism looks like in practice. I identify the three main elements of capitalist realism as the creation of instability through changing economics, the creation of new sites of and strategies for capital accumulation, and the transformation of these experiences into a `common sense' or hegemony that operates at both the individual and structural level. I explore the ways that austerity is both exemplary of capitalist realism and an intensifier of it. Using the third-sector as a case study, I note that it is easy to ascertain the ways that austerity changes economics to create instability, and how it transforms this instability into a sense of necessity, an impossibility of things changing. It is more difficult, however, to identify the sources of and strategies for accumulation that austerity creates. I propose that the instabilities and existential sense of resignation that austerity creates are part of its new accumulative activity, and that the `missing link' here is an understanding of how the \emph{production} of vulnerability has become a new \emph{source} of capital accumulation and the \emph{consumption} of that vulnerability by organisations engaging in `digital' activity has become a new \emph{strategy} for capital accumulation. I close the chapter by proposing that if we are to understand how we can escape the hegemony of capitalist realism (in its contemporary incarnation, intensified by austerity), then we must turn towards an exploration of the third-sector under austerity and the `digital turn' -- and the role of a simplified conception of service design - as a response to this.1

\section{The components of capitalist realism}
\label{2-components}

In his 2009 book \emph{Capitalist Realism}, Mark Fisher describes the
titular concept as ``the widespread sense that\ldots{} capitalism
{[}is{]} the only viable political and economic system\ldots{} {[}and{]}
it is now impossible even to imagine a coherent alternative to it''
(2009, p. 2). Acknowledging its similarity to Jameson's notion of
postmodernism as ``the cultural logic of late capitalism'' (198X, xxii),
Fisher suggests that capitalist realism represents an intensification of
these forces - highlighting that though they stem from the same root,
``the processes\ldots{} have now become so aggravated and chronic that
they have gone through a change in kind'' (2009, p. 7). The links
between capitalist realism and Jameson's postmodernism can help us to
clarify that for Fisher, capitalist realism is a material, aesthetic and
affective force. Capitalist realism is at once incredibly simple and
exceedingly complicated. In order to understand how it could be possible
to act \emph{outside} of capitalist realism -- in an attempt to
prefigure and create new ways of being and doing -- it is first
important to understand it at a much deeper level. Shonkwiler and Berge
(2014, p. 6) identify three central components to capitalist realism -
the ability of capital to:

\begin{itemize}
\item
  constantly revolutionize its sources of and strategies for
  accumulation, developing new configurations of activity;
\item
  have an economic, social and affective life that has vast consequences
  for our lived experiences;
\item
  and to transform this constant change and lived experience into a
  widely accepted brand of Gramscian ``common sense'''.
\end{itemize}

Each of these elements relate to the other two, but there are certain
aspects which must be understood in order to understand the core of
capitalist realism. In the rest of this section, I will detail these
components in turn to explore capitalist realism's materiality,
aesthetics, and affects. First, I turn towards the instability created
by changing economics, and use this to explore Fordist and post-Fordist
conceptions of work, the Foucauldian disciplinary society and the
Deleuzian society of control, and the way that one of neoliberalism's
core missions is the changing of what work \emph{is}. Then, I turn
towards capitalist realism's accumulative activity, exploring Marx,
Luxemburg, and Harvey's conceptions of accumulation, the
bureaucratisation brought on by the move towards post-Fordist work, and
how this creates a hyperreality which is the source of much of
capitalist realist accumulation. Finally, I identify the ways that this
becomes transformed into a hegemonic `common sense' that leads to an
anticipation and a \emph{precorporation} of new desires that become
folded back into capitalist realism itself, ensuring its totalising
abilities.

\subsection{Instability created by changing
economics}
\label{2-instability}

\subsubsection{Fordism and the disciplinary
society}
\label{fordism-and-the-disciplinary-society}

The story of capitalist realism begins with the Fordist workplace and
the profound boredom of the `disciplinary society'. In the middle of the
20th Century, Fordist models of work and labour loomed supreme, with the
majority of work being ``routine, hierarchical, mind-deadening,
mechanical {[}and which{]} tied people to one task\ldots{} for the rest
of their lives'' (Horgan, 2021, p. XX). Coined by Antonio Gramsci in his
*Prison Notebooks*, Fordism describes a kind of work which emerged after
the First World War in America founded modelled after the Ford company's
production lines. This work was characterised by `assembly-line
production, managerial hierarchy and technical control' and relied upon
Taylorist models of scientific management and rationalization, `which
simplified necessary operations, eliminated others, and radically
routinized, deskilled, and intensified labor ` (Antonio and Bonanno,
2000, p. 34). In compensation for this labour, Fordist employees were
given higher wages than were usual for the time. According to Gramsci,
this is because the labour was more `wearing and exhausting than
elsewhere ' (Gramsci, Prison Notebooks, p. 311 - 312), and Ford believed
that to convince employees to perform the labour, `coercion {[}had to
be{]} \ldots{} combined with persuasion and consent' (Gramsci, Prison
Notebooks, p. 310). As a result of the `class compromise' (Harvey, 2005,
p. X) that emerged between Capital and labour in the aftermath of the
Second World War, Fordism became the dominant model of work.

The dominance of Fordist work coincided with the `disciplinary society'
reaching its peak of activity. In *Discipline and Punish*, Michel
Foucault studies the techniques through which the state has enacted
punitive measures against its populace. Moving through `torture',
`punishment', `discipline, and `prison', Foucault charts the decline of
`the great spectacle of physical punishment' (Foucault, XXXX, p. 29
Ebook) and its transformation into more understated punitive measures.
Though this is often claimed to be a `humanisation' (p. 18) of
punishment, Foucault shows that punishment has merely become nuanced and
refined in order to maintain its dominance. The punitive methods of a
given society or epoch are a ``political technology of the body'' which
tells us about the ``power relations and object relations'' (p. 44) in
that society. Foucault shows us that even if techniques of punishment do
not make use of the spectacle of violent physical punishment, that the
object of punishment is always ``the body and its forces, their utility
and their docility, their distribution and their submission'' (p.46).
Foucault's project, then, is essentially a ``history of the present''
(p. 55) or a genealogy, through which he attempts to construct an
understanding of how the body is affected and subjected by the
techniques of power to punish, and ``its bases, justifications and
rules'' (p. 43).

Foucault argues that the movement away from physical punishment
constituted a movement of punishment inwards, through which the power to
punish could be wielded increasingly less explicitly. When sovereigns
wielded power, they used torture and corporal punishment to make a clear
`spectacle'; but the humanising and reforming forces argued that this
was immoral. Foucault states that as early capitalist society emerged,
punishment ``shift{[}ed{]} the object and change{[}d{]} the scale'' and
new tactics were developed in order to wield power that is ``more subtle
but also more widely spread in the social body''. The principles of
punishment, then, became more regularised, universalised and homogenised
- both to reduce their economic and political cost and also create a
sense of necessity about them (p. 144). Rather than baring the
sovereign's mark of power through corporal punishment, the body of the
punished becomes ``property of society'' (p. 176), and the creation of
discipline ``produces subjected and practised bodies, `docile' bodies''
which ``dissociates power from the body''. In essence, then,
``disciplinary coercion establishes in the body the constricting link
between an increased aptitude and an increased domination'' (p. 219).
Individuals are thus made to desire their own subjection, as the society
around them rewards the development of an increased discipline whilst
removing more and more of their individual agency and power.

The `disciplinary society' then is constituted by enclosure (which I
will explore further when detailing the function of accumulation within
capitalism), partitioning, individualisation, and ranking (placement
within a hierarchy). These elements created spaces - such as the
factory, the school, the hospital, the prison - through which discipline
was coercively cultivated and encouraged, with each element of the
system serving to further the others. Fisher argues that Fordist work
worked to strengthen the disciplinary society, through the Taylorist
segmentation of activities into their constituent parts and making them
repetitive, routine, and boring. Speaking of the disciplinary society,
Fisher notes that the place of surveillance (that instils discipline)
does not actually need to be occupied, as not knowing whether or not you
will be observed makes people ``constantly act as if you are always
about to be observed'' (Fisher, CR, p. 56). Fordist work and the
disciplinary society, then, create a significant affect of boredom and
dissatisfaction, exhausted, tired, disciplined bodies, and
self-disciplining subjects who are made to desire their own subjection.

\emph{\emph{* note: Control and post-Fordism create a subject which
actively desires the subjection and boredom of discipline and Fordism,
because it is predictable, familiar, and allows desires to be achieved -
as opposed to the debtor-addict figure of Control society}}

\subsection{Neoliberalism and resonance machines}
\label{neoliberalism-and-resonance-machines}

Because of the demotivating affects elicited by the disciplinary society
and Fordist work, the subjects it creates are likely to desire societal
change. Despite this, Fisher argues that the ``traditional
representatives of the working class - unions and labour leaders - found
Fordism rather too congenial'' (Fisher, CR, p. 38) because its stability
gave them a permanent role, that of antagonists towards Capital. This is
by no means a universal view, though; Antonio and Bonanno (2000, p. 37)
instead suggest that unions tended to co-operate with management during
this period of high union membership, essentially trading their
``aspirations for stakeholder rights in capital, and shared control of
the labour process'' for higher wages and stable employment. Both views,
however, concur that unions became increasingly ineffective during this
time (the late 70s and early 80s). Fisher suggests that at this time
they did ``little to advance the hopes of the class they purportedly
represented'' (CR p. 38) , and began to resist any attempt for change in
case it jeapordized their comfortable position - whether this was
because of the constancy of the antagonism or because they were
comfortable with higher wages and stable employment. In the face of the
left's abandonment of `the new', and due to the profound negative
collective affects that were circulating, a power vacuum was created,
which began to be filled by neoliberalism.

Although David Harvey does not view it from this same angle of Fordist
work and the disciplinary society, he argues that it was a similar set
of social forces which led to the emergence of the neoliberal project.
Harvey argues that post-war, there had been a ``class compromise'' (NL,
p. ? Ch1) between Capital and labour, which had been advanced as an
attempt to guarantee peace and tranquility. Yet in the 1970s, a crisis
of capital accumulation (explored more in the next section) jeapordised
the stability of this arrangement. Harvey argues that ``to have a stable
share of an increasing pie is one thing'' (p.39) but that when growth
rates collapsed and `stagflation' began in the 1970s, ``upper classes
everywhere felt threatened'' (p.39), especially as socialist
alternatives began to gain ground across Europe and South America. The
neoliberal project emerged with two central aims: to ``realise a
theoretical design for the reorganisation of international capitalism''
and to ``re-establish the conditions for capital accumulation and to
restore the power of economic elites'' (p. 45). In practice, the latter
aim reigned supreme, dispensing with parts of the neoliberal model which
didn't quite help the *bourgeoisie* to achieve their aims of increased
(and unlimited) capital accumulation. The former aim instead became ``a
system of justification and legitimation'' through which neoliberal (and
neoliberalizing) rhetoric could flow.

The theory that provided the basis for this rhetoric, then, is ordered
around the base idea that the ``position of the individual {[}is{]}
progressively undermined by extensions of arbitrary power'' (Mont
Pelerin Society, quoted in Harvey) taken by the state, and that state
interventionism (as modelled in Keynesian economic theory) and
centralised state economic planning (as modelled in the socialisms and
communisms of the day) prevented individuals enacting their agency and
so-called `freedom'. In particular, neoliberalism valued ``private
property and the competitive market'' (Mont Pelerin Society, in Harvey)
above all else, and argued that a decline of belief in these was leading
not only to the decay of society but to economic ruin, too. In the views
of neoliberals such as Milton Friedman and Friedrich Von Hayek, the
state would necessarily be wrong on matters of investment and capital
accumulation because the information available to the state ``could not
rival that contained in market signals'' (p. 49). Thus they argued for a
restructuring of the economy and the state to prioritise economic
freedom, individual property rights, the rule of law, and free trade.

Neoliberalism in *theory* calls for a minimal state, that only
intervenes in specific situations - ``to defend the nation against
foreign enemies, to prevent coercion by some individual by others, to
provide a means of deciding upon our rules, and to adjudicate disputes''
(Connolly, 2011, FT, p. 53). In this way, it attempts to model itself
after the classically liberal economies, using the state to protect the
functioning of the market and leaving the rest of society to its own
ends. Where neoliberalism differs though is in an understanding - or
underestimation, as the case may be - of how elastic these criteria can
be and thus how incoherent its ideology becomes in practice. The sorts
of actions neoliberal governments end up taking - such as deregulation,
dismantling collective bargaining structures and trade unions,
dismantling social welfare systems, privatising public services,
lowering tax income, and creating systems that encourage foreign direct
investment (Harvey, NL., p. 52) do not fit with the image of a minimal
state. What instead results is a ``selectively active'' state (Connolly,
FT, p. 21) which will intervene in the market in order to ``defend the
rights of private property, individual liberties, and entrepeneurial
freedoms'', (Harvey, 2005, p. 49) no matter what form of state action
this takes.

The selectively active neoliberal state - the one which seeks the
restoration of the bourgeoisie's power after the post-war class
compromise - wants the state to ``inject market processes into new
zones'' constantly (Connolly,FT p. 21) , such as schools, prisons,
healthcare, public transport, logistics and social care. This
understanding of neoliberalism as-it-is thus seeks a state which
constantly acts to ``maintain the preconditions of market vitality''
(Connolly, FT, p. 60). Hayek understands this to require a cultural
action: the creation and maintenance of ``public attitudes and state
practices'' that allow neoliberal beliefs to flourish (Connolly, FT, p.
57). The idea of freedom within neoliberalism then ``degenerates into a
mere advocacy of free enterprise'' (Polanyi, quoted on 75 Harvey),
creating a ``fullness of freedom for those whose income, leisure and
security need no enhancing''. Because of this, neoliberalism constantly
pursues the standardisation and regularisation of everything, resulting
in the ``pursuit of a nation of regular individuals who have
internalized market norms'' (2011 FT, p. 53), who are easier to govern
successfully and more able to resist challenges to the neoliberal
ideology. The mechanism through which this operates is a biopolitics,
creating regularised individuals as its prime unit of understanding.
Neoliberal capitalism thus gains ``a significant supporting
infrastructure through ideological hegemony, state action, neoliberal
jurisprudence, schools, and the internalized market virtues of
participants'' (Connolly, FT, p. 62).

The rigid, mechanical structures of disciplinary Fordist work played
easily into the hands of a class of people - mostly thinktanks,
academics, and politicians that came to prominence in the late 1970s and
early 1980s - that were trying to argue that the current way things were
being done was broken, that society and the economy needed something
else, and that `something else' should be neoliberalism. Neoliberalism
essentially positioned itself as the answer to the problematics of the
day: feel as if you have no freedom in your job and that you're being
kept from your potential? That's because we need freer markets that let
everyone do what *they* want to, away from the overextension of the
power of the state and the stable, arbitrary antagonism of the unions.
Neoliberals across the world - but centrally for this thesis, in the
United Kingdom - rose to power on the basis of this rhetoric.

It is important at this point to consider exactly what is happening
here. In many histories and theories of the operation of neoliberalism,
this step is skipped over: simply that neoliberalism proposed an
attractive argument to people who were feeling negative affects, they
were convinced, and neoliberalism was able to rise to power. I want to
expand upon an understanding of how this works with reference to Deleuze
and Guattari's ideas of deterritorialization and
(re)territoritialization. In *A Thousand Plateaus*, Deleuze and Guattari
(1987) portray deterritorialization as an experience of the oncoming
horizon of `the new' - as the ideas of neoliberalism were to people who
were dissatisfied with their jobs and lives at the time. When a novel
element enters a system, the system may either attempt to incorporate
it, or reject it entirely. If the system attempts to incorporate it, a
moment of deterritorialization occurs that expands its range of
potential classifications and meanings. This is always accompanied by a
(re)territorialization, which contracts the potentials, creating new and
different boundaries. For Deleuze and Guattari, one of the central
functions of capitalism is the ``generalised decoding of flows) (1983,
p. 153), and is thus constantly searching for ways to incorporate any
new elements and (re)territorialize them with its own agenda of
accumulation and value-extraction. I would argue that what happens at
the neoliberalizing moment, then, is that the genuine desire for change,
liberation, for freedom from the crushing boredom of Fordist work that
has been built by years of negative affects at the hands of Capital
itself is deterritorialized. People's base desire for *something better
than this* is taken back to its more bare and abstract form: and in this
deterritorializing moment, neoliberal capitalism (re)territorializes in
its own agenda of freedom through work, private property, individual
enterprise. Neoliberal capitalism deterritorializes desires and
(re)territorializes in its own values, which colour and change the
desire. Before, you wanted freedom. Now, you just want to be an
entrepreneur, or start your own business - because that is what freedom
now means.

This deterritorialization and (re)territorialization doesn't just happen
in an abstract sense, though, somewhere out in the aether without
grounding in reality. This operates through the creation of resonance
machines. Connolly describes these resonance machines as being composed
of ``parties who hold overlapping political-economic theories'' (p. 68,
fT?). They do not necessarily have to believe the same thing - just
something compatible. The resonance machine then ``amplifies the sites
and modes of inculcation'' (p. 68) for each of the ideologies. It is
important to note here, then, that capitalist realism are *not* the same
thing; rather that neoliberalism is highly compatible with capitalist
realism, and that neoliberalism, capitalist realism, and the specific
brand of neoreactionaryism or neoconservatism we have seen in the past
ten years or so (since the 2008 financial crisis) have acted as
resonance machines for each other. These ideologies make each other more
possible, creating the conditions (or preconditions) for each other to
go further, or to metamorphose into a new state.

Neoliberalism brought with it a change in the nature of work and
everyday life, then. In contrast to the mundanity of Fordist work,
post-Fordist work ``promised to be flexible, exciting, fast-paced, based
on team-work, and full of variety''. For Fisher, post-Fordist labour is
marked by ```flexibility', `nomadism' and `spontaneity''', (2009, p.28),
a kind of decentralisation that becomes hard to resist because it
appears to be an uncontestable good in the face of the freedom and the
`new' of neoliberalism. This flexibility is a (re)territorialization of
the desire for freedom in the Fordist workplace. Although people
supposedly got what they wanted, it removed the stability and permanence
of anything. This instability and precarity destroyed traditional sites
of labour power (by removing the constancy of the Fordist workplace) and
could be summed up, Fisher argues, by the slogan `no long term' (p. 36).
Where once workers could build skills and move (albeit slowly) through
the hierarchies of an organisation, ``now they are required to
periodically re-skill as they move from institution to institution, from
role to role'' (p. 36). Nothing is ever - or can ever - be over.

Fisher suggests that this transformation of economic and social life
under capitalist realism also changes the nature of power - from
disciplinary society to ``Control society''. In his 1992 ``Postscript on
the Societies of Control'', Deleuze suggests that the ``societies of
control'' have taken hold, in which power and punishment are distributed
through the ``modulations'' of control (p.4). In contrast to enclosures,
which act as defined spaces, the modulations of the Control society -
expressed, for example, through post-Fordist neoliberal work - ensures
that things ``continuously change from one moment to the other'' (p. 4).
``Perpetual training'' replaces education; ``continuous control''
replaces examination (p. 5). Deleuze argues that in the disciplinary
societies, ``one was always starting again'', but that in the control
societies ``one is never finished with anything''. No longer is the aim
to meet the rigid, strictly-enforced structures of discipline - because
you cannot, you won't, it will always be infinitely delayed - but you
will still have to try. Fisher suggests that in the Control society,
``external surveillance is succeeded by internal policing'' (p. XXX)
whilst feedback mechanisms - their own kind of modulation - are governed
by the technological, able to change at a moment's notice. All that is
solid melts into air, *over, and over, and over.*

The true danger of the Control society that was brought about by
neoliberalism and post-Fordist work, then, is the way that it changed
the nature of desire. If, in the Control society of capitalist realism,
you can never be done with anything and are always governed by relations
of indefinite postponement - desires can never be completed. If I want
something, I may well get that thing, but I will never be able to
appreciate the realisation of that desire. Fisher suggests that this
leads to a kind of ``reflexive impotence'', a state in which people
``know things are bad, but\ldots{} know they can't do anything about
it'' (p. 21). An affect of boredom and resignation continues to
circulate, but no-one feels as if they can do anything about it. This
transforms desire from the productive, creative force that Deleuze and
Guattari argue it is in *Anti-Oedipus* (quote?) and instead makes all
desire necessarily the Lacanian `lack'. This sets the stage for the next
aspect of capitalist realism, which deals more significantly with the
nature of *how* Capital operates under capitalist realism: its creation
of new sources of, and strategies for accumulation, made to fill these
new desires which can never be fulfilled.

\emph{\emph{* Haven't spoken about how the regularisation that
neoliberalism does creates homogeneity which creates stagnation
hauntology and nostalgia }}

\subsection{New sites/strategies of accumulation }
\label{new-sitesstrategies-of-accumulation}

The new libidinal states created by neoliberalism's transformation of
the economy set the stage for entirely new sources of and strategies for
accumulation. If neoliberal desires gave birth to a Control society in
which desires can never be fulfilled, then this creates a rich new
ground for capitalist expansion, by creating products, services and
methods of surplus value extraction which could never have happened
before. To understand these new sources and strategies, we must first
understand how accumulation is thought to function within capitalism
more generally. In *Capital*, Marx (year, ch26) claims that ``primitive
accumulation'' functions as the Original Sin myth of capitalism, retold
by those aligned with the interests of capital:

\begin{quote}
In times long gone-by there were two sorts of people; one, the diligent,
intelligent, and, above all, frugal elite; the other, lazy rascals,
spending their substance, and more, in riotous living\ldots{} It came to
pass that the former sort accumulated wealth, and the latter sort had at
last nothing to sell except their own skins. And from this original sin
dates the poverty of the great majority that, despite all its labour,
has up to now nothing to sell but itself, and the wealth of the few that
increases constantly although they have long ceased to work.
\end{quote}

This myth serves as a rationale for the creation of private property and
the enclosure and privatisation of land: an intelligent elite (the
bourgeoisie) are able to claim that they have no particular *desire* to
colonise, enclose and privatise - they merely are compelled to by the
inaction of the lazy (the proletariat). Marx refers to this as a kind of
*primitive* or original accumulation (footnote about use of the word
primitive), whereby land is expropriated from the commons and an
exploited class is made to sell their labour for the first time in order
to provide for themselves and continue living. For Marx (still ch26)
then, primitive accumulation is ``the historical process of divorcing
the producer from the means of production'', which ``transforms\ldots{}
the social means of subsistence and of production into capital'' and
turns direct producers into wage labourers.

In her 19XX book *The Accumulation of Capital*, Rosa Luxemburg noted a
central problem with Marx's thesis that primitive accumulation is
sufficient for describing the functioning of capital: under Marx's view,
it appears that ``capitalist production would itself realise its entire
surplus value, and that it would use the capitalised surplus value
entirely for its own needs'' (309). Luxemburg shows that this cannot be
the case - as wage labourers receive less value than they create (a
necessary condition of capitalism's functioning, via the creation of
surplus value), this means that capitalism ``is unable to exist by
itself'', needing ``other economic systems as a medium and soil'' (p.
447). According to Luxemburg, then, accumulation becomes ``a
relationship between capital and a non-capitalist environment'' (p.
398). Herein lies the contradiction of capitalism, then: capitalism
``strives to become universal'', but is ``incapable of becoming a
universal form of production'' (p. 447) because of the requirement of an
`outside' to expand into, to make surplus value productive.

David Harvey casts Luxemburg's analysis of this problem of accumulation
as a theory of ``underconsumption'' - wage labourers are not able to
consume enough to make the logic of capital accumulation make sense. By
contrast, Harvey casts this problem as one of ``overaccumulation''.
Overaccumulation is a condition in which ``surpluses of capital\ldots{}
lie idle with no profitable outlets in sight'' (Harvey, 2003, p. 149).
Harvey's response to the problem of overaccumulation is to suggest that
accumulation by dispossession then occurs, whereby assets are released
at a low cost so that ``overaccumulated capital can seize hold of such
assets and immediately turn them to profitable use'' (Harvey, 2003, p.
149). Although accumulation by dispossession retains the idea of an
outside into which capital can expand from Luxemburg, Harvey astutely
notes that ``capitalism necessarily and always creates its own `other'''
(p. 1XX). Capitalism can either make use of a pre-existing outside - in
the form of colonial expansion, movement into sectors which have yet to
be effectively capitalised - or manufacture an outside through
dispossession.

Harvey suggests that primitive accumulation functioned as an example of
accumulation by dispossession - land was taken, enclosed, the population
who occupied the land were expelled, and a landless people was created
as the enclosed land became private property. Neoliberal privatisation
of public services and utilities (in the UK and other neoliberal
countries) - such as water, gas and electricity, the postal service, and
even housing) also function as accumulation by dispossesion. In each of
these circumstances, something that was held in common was then enclosed
or made private and access to or use of that thing became conditional.
Harvey also suggests that accumulation by dispossession can also occur
through flooding the market with ``cheap raw materials'', or ``the
devaluation of existing capital assets and labour power'' (150).

So how does capitalist realism deal with the creation of new sources of
and strategies for accumulation? Using accumulation by dispossession as
our guide, we can see that the heart of capitalist realist accumulation
can be seen its transformation of work and power. Accumulation in the
capitalist realist economy must therefore be beholden to the structures
of post-Fordist work, the control society, and desires which can never
be fulfilled. Key to the transformation of post-Fordist work, Fisher
argues, is the creation of a class of bureaucrats who can administer,
regulate, and maintain the new accumulation built upon flexibility. What
emerges in practice is a highly centralised network of managers and
bureaucrats who wield power but assume no responsibility for how they
use that power. These kinds of bureaucratic procedures appear to ``float
freely, independent of any external authority'' but this in turn means
that they face a heavy ``resistance to any amendment or questioning''
(2009, p. 55), able to make decisions only by ``refer{[}ring{]} to
decisions that have always-already been made'' (2009, p. 53). They are
agents of the Control society, deferring to a power which is
always-present and yet never really there. So one new aspect of
accumulation is present here: although bureaucrats have been present in
many different economic arrangements, the post-Fordist bureaucrat's job
is that of creating new sites of accumulation.

These `new' bureaucrats spend the majority of their labour creating
artefacts which hold a symbolic status within the realm of post-Fordist
neoliberal work. They are responsible for the creation or maintenance of
`aims and objectives', `outcomes' and `mission statements, functioning
as auditor, evaluator, producer and interpreter of symbols. This new
form of labour is ``geared towards the generation and massaging of
representations rather than the official goals of the work itself''
(2009, p. 46). In this new form of accumulation, anything can become a
site of enclosure or debt because this strategy of accumulation depends
upon ``a complex series of social semiotic signals'' (2009, p. 54).
Capital has again created its own outside, but this time, rather than
dispossessing land, labour or materials, it has dispossessed the
representational content of that which it accumulates. Put simply,
capital has transcended material artefacts: instead what now becomes
most relevant is the aesthetic, ideological or intellectual component of
an artefact. Capital begins to accumulate through this network of
representations, resulting in an economy in which ``real world effects
matter only insofar as they register at the level of (PR) appearance''
(2009, p. 47). In capitalist realism, then, accumulation becomes a
strategy of manipulating the appearance of reality rather than (or in
addition to) manipulating reality itself.  

There are two central problems with an accumulation that is centered on
the creation of representations: the creation of a sense of
`hyperreality' and the massive expansion in potential sites of
accumulation. In *Capitalist Realism*, Fisher argues that the transition
towards post-Fordist bureaucratic accumulation participates in a
Baudrillardian `hyperreality' (p. 52). Bureaucrats create
*representations* - which are necessarily contingent slices and
mediations of the world. However, representations which claim not to be
representations - which claim to ``present reality in an unmediated
way'' (p. 52) lead not to a direct encounter with reality but to a
``hemorrhaging of the Real'' (p. 52). This creates a hyperreality, in
which it becomes impossible to understand what is real in a way that
would exist without the creation of a representation, and what is real
only by way of the act of creating a representation. Fisher explains
this by way of reality tv shows, fly on the wall documentaries, and
political opinion polls - in each of these situations `reality' is not
simply accessed but is necessarily produced - and so the production of
the reality potentially gets in the way of experiencing authentic
reality.

This hyperreality is essential to the sense of necessity produced by
capitalist realism. There is a strong visual component to capitalist
realism: in order to make people believe that capitalism is the only
viable political system and it is futile to imagine or try to build
anything other than it, it is important to produce a large number of
hyperreal representations of the world. These hyperreal representations
try to make claims about the base nature of living in the world that
they are simply cannot. In late 2000s and early 2010s media, for
example, this took the form of media which was `gritty' or `grimdark'.
The creators of `gritty' media often claim that they are just
representing reality as it actually is, and pundits proclaim that their
work is incredibly authentic. In doing so, these creators show
themselves to conform to a capitalist realist view of the world: life is
``nasty, brutish and short'', after Hobbes, and humans are fundamentally
selfish creatures. Viewing the world as such, and creating media and
bureaucratic representations that claim to be the world `as it is'
serves the purpose of destroying the possibility of hope for something
*else*, something *more*. As Fisher says, ``we are integrated into a
control circuit that has our desires and preferences as its only mandate
- but those desires and preferences are returned to us, no longer as
ours, but as the desires of the big Other {[}capital{]}'' (p. 53).

In tandem with this move towards a representation-centered economy, the
potential sites of accumulation become massively expanded. If what
matters in this new accumulation is *representation*, not materiality,
then the constraint of the `old' accumulation - having its profits
limited by how much factories could physically produce - is removed.
This is in large part what has driven the massive flows of capital into
what might be thought of as `progressive' causes - venture capitalists
know that there is a market that will pay massively over the market rate
for a given commodity if it is ethically produced, is made in
sustainable ways or is perceived to be part of a `local' economy.
{[}\emph{\emph{I think I need to expand on this but it feels like
there's a whole /thing/ here - about how radical movements create new
`markets' for capitalism which commodified them and removes their
radical content. Think self-care, sustainable brands etc. The surplus
value of these products is in the feeling that you are doing something
good. But there remains no ethical consumption under
capitalism\ldots{}{]}}}

This transformation of accumulation is at the heart of the workings of
capitalist realism, precisely because the representation-led nature of
bureaucratically-driven capitalism means that desires can never be
realised. As explored in the previous section, in capitalist realism and
the Control society you can never be done with something - you are
always returning to the beginning, starting from the bottom,
indefinitely postponed. The representation-centric methods of
accumulation act as a form of ``dreamwork'', producing a ``confabulated
consistency which covers over anomalies and contradictions'' (p. 64),
whilst also creating a cultural anxiety with ``memory disorders''.
Fisher refers to capitalist realism as a ``condition\ldots{} of
ontological precarity'' in which ``forgetting becomes an adaptive
strategy'' (p. 60). People begin to exist in a cultural environment that
is a mosaic of contested representations, each painting a picture of a
perfect world, either smoothing over the inconsistencies inherent within
capitalist realism or becoming anxious about this very smoothing over.
Here, the hyperreal returns, as everyone knows at some level that they
are *doing* this smoothing over but finds themselves unable to *not*
smooth over the inconsistencies - ensuring that desire cannot never be
realised.

If this dreamwork and memory-anxiety goes on for long enough, it creates
a cultural affect of nostalgia - a longing for what was. Fisher argues
that contemporary culture is ``excessively nostalgic'' due to the fact
that it is ``incapable of generating any authentic novelty'' (p. 63).
The homogeneity engendered by neoliberalism, the temporal interruptions
created by the Control society, the creation of hyperreal
representations within capitalist realist accumulation, and the
`dreamwork' that sustaining these result in a stagnant culture. Fisher
suggests that this stagnant culture leads to a craving for ``familiar
cultural forms'' (p.63), being continuously drawn back to the security
of old things. This stagnant, overly nostalgic culture thus is incapable
to generate any novelty in itself, and must rely on the old and familiar
for the appearance of novelty. There is no particular difference, for
example, between modern vaporwave music and the smooth jazz and
corporate lounge music being produced in the 1980s and 1990s - except
that the smooth jazz and corporate lounge music of the 1980s and 1990s
were responding to their own cultural referents, rather than a past
culture's. In later work, this is what Fisher argues produces a profound
sense of hauntology. Hauntology acts as ``the agency of the virtual''
(p.18 GOML) - making clear that ``everything that exists is possible
only on the basis of a whole series of absences'' (p. 17). What is
present can only be present because of that which is not present. Fisher
extends this concept to the idea of capitalist realism and the collapse
of possibility by arguing that what is doing the haunting in hauntology
is not that which actually existed and once was (as in nostalgia) but
``the spectres of lost futures'' which never materialised (p. 21).
Whilst these futures that never came to pass might be `lost', they
sometimes surface themselves through material artefacts, traces that are
able to break through the fugue state of capitalist realist dreamwork.
Following these traces might prove essential to attempting to escape
capitalist realism, but this proves difficult because capitalist realism
has become a totalising system. It is an hegemony that transforms desire
and experience through these new methods of accumulation and the
transformation of social and economic life. In this next section, then,
we turn towards understanding how capitalist realism functions as an
hegemony, acts with plasticity, and is able to absorb and precorporate
resistance.

\subsection{The creation of hegemony and `common sense'}
\label{the-creation-of-hegemony-and-common-sense}

The final aspect of capitalist realism is its totalising, hegemonic
power. It is not enough that just some people cannot imagine any
meaningful alternative to capitalism: in line with Luxemburg's theory of
capital accumulation, it must become *universal* - no one should be able
to imagine an alternative to capitalism. Yet the Luxemburgian
contradiction still remains: capital strives to become universal yet
always requires a non-capitalist outside into which it can expand.
Harvey showed us in the previous section that the way to deal with this
problem is for capital to generate its *own* outside. In Harvey's
rendering, this is normally through devaluation, enclosure or flooding
the market with raw materials (another form of devaluation). In
contrast, Fisher suggests that capitalist realism engages in
*precorporation* in order to create its outside. Precorporation
describes capital's ability to pre-emptively shape emerging desires so
that they can be more easily absorbed by capitalism. Precorporation goes
one step further than traditional conceptions of hegemony, which might
argue that capital has an immense capacity to incorporate materials and
actors that were subversive. In contrast, it suggests that rather than
merely incorporating these materials and actors, it structures, formats
and shapes them from the very beginning so that they can more easily fit
into the structures of capital. Fisher uses Nirvana as his clearest
example of this: though `alternative' and `independent', Nirvana knew
well that ``nothing runs better on MTV than a protest against MTV'' (p.
13). Resistance to capital has become a necessary part of capital. Yet
it is important to understand precisely how this precorporation works
and how capitalist realism has become hegemonic.

One of the central aspects of capitalism that makes precorporation
possible is capital's plasticity. Developed by Catherine Malabou, the
concept of plasticity refers to aspects of any system that ``allows play
within the structure'' (Plastic Materialities p. 44), or that ``loosens
the frame's rigidity'' (ibid.). Plasticity then is an agent of
transformation that allows the preservation of an overall structure but
a recomposition of the elements contained within it. Capital - and
particularly neoliberal capital - have an huge capacity for plasticity,
needing only to retain capital's essential element - the accumulation of
surplus value. So long as accumulation remains possible, capitalism can
plasticly reconfigure itself and its elements to any possible form. This
plasticity enables precorporation: under capitalist realism, capital is
able to adjust itself, its form, and its perception of its form in order
to appear differently, so that aspects of resistance to capital that
might appear subversive present themselves in a way that is
always-already amenable to incorporation by capital. Capitalist realism
thus functions dialectically - its abstract is capital itself, its
negative is potential resistance, and its concrete is the incorporation
of that resistance into the capitalist system. We might understand this
process through deterritorialisation, too - genuine desires for
resistance become deterritorialized and reterritorialized with capital's
values and content.

Because capital seeks to become universal yet must generate an outside,
its plasticity and facilitation of `safe' resistance through
precorporation ensures its continued dominance. Precorporation thus acts
as a kind of governmentality. First developed by Foucault,
governmentality refers to both the ``rationalities, technologies,
programmes and identities of regimes of government'' (Dean, 2009, XX)
and the ``mentalities of government'' that must be created in order to
support these machineries and regimes. It widens our understanding of
governance as a practice to include different modes of thinking, and
highlights how these become embedded in language and other instruments
of power. The governmentalities that support both neoliberalism and
capitalist realism then describe how precorporated resistance and
desires become internalised, taken on by people in ways that make them
believe they are their own desires. They are biopolitical, acting as
``processes of power that seek to regulate and control life'' (Lemke,
2011, p. XX). Those aligned with the interests of capital create
structures, technologies and regimes of government which support
capital's interests. By interacting with these structures on a regular
basis, people who aren't aligned with the interests of capital gradually
develop the mentalities of government that a given form of government
seeks to create. Put simply, by interacting with structures of power,
one becomes liable to think more like the power structure demands.
Interacting with the power structure requires the acceptance of the
frame of reference of the power structure. \emph{\emph{(nb. This point
connects to hegemony and articulation) **I should also discuss
subjectification and bio politics here**}}

For Fisher, the sum total of precorporation and the biopolitical
internalisation of always-already incorporated ideas and resistance is
the creation of a sense of reflexive impotence. This reflexive impotence
is the knowledge that ``things are bad\ldots{} {[}but{]} they know they
can't do anything about it'' (Fisher, p. 24???). Fisher argues that this
feeling of powerlessness is not a mere observation but also is what
creates the feeling of powerlessness itself; knowing that you feel
powerless makes you powerless. Although Fisher does not go into a great
deal of depth on what creates this reflexive impotence (other than the
Control society), our understanding of precorporation, plasticity and
governmentality suggests that it is precisely through totalisation that
reflexive impotence occurs. Any potential resistance to capitalist
realism is subsumed by capitalist realism. If this happens for long
enough this creates the feeling that nothing can ever be changed. Add to
this the hyperreal produced by the Control society and it becomes clear
that not only can nothing ever be changed, but nothing has ever changed.
It is easy to see how a sense of powerlessness results.

It is important to note here that the force behind this powerlessness
becomes complete and totalising due to the creation of a capitalist
realist hegemony and the production of `common sense'. In Gramsci's
rendering, common sense refers to collective opinions that have gathered
`a social force' (Gramsci, 2007) behind them. As such, common sense is
produced and reproduced by hegemonic formations, able to exercise a
totalizing and coercive power over subjects of a state. In the case of
capitalist realism, the prevailing neoliberal common sense that there
can be no alternative to capitalism was reproduced by political and
civil society, changes in practice and everyday micropolitical
interactions. For Gramsci, hegemony is\ldots{}
\emph{\textbf{\emph{**{[}expand with detail from the Prison
Notebooks{]}**}}}

Whereas Gramsci's conception of hegemony is built upon the notion of the
institutional reproduction of power throughout civil society, Laclau and
Mouffe's (2001) refinement of the concept centers on the proliferation
of antagonisms around the potential meaning of ``floating signifiers''.
For Laclau and Mouffe, all practices that attempt to establish some
relation between elements can be considered an articulation, which
represents a claim about the nature or value of an element, or the
relationships between some elements. Hegemonic formations occur when a
social and political space becomes ``relatively unified'' (Laclau \&
Mouffe, 2001, p. 136) in its articulations. Capitalist realism, then, is
a hegemonic formation resting upon the articulations that: there is no
meaningful alternative to (neoliberal) capitalism; there is no
possibility of changing the current political, social and economic
order; and any negative feeling that arises as a result of these is
either a sign of something wrong with *you* or the hyperreality of
living. Yet Laclau and Mouffe make clear that even in a hegemonic
formation, there are always competing articulations. No articulation can
ever become entirely fixed forever. There is always the *possibility*of
resistance -- the only problem is that a hegemonic formation ``embraces
what opposes it'', attempting to incorporate aspects of novel
articulations in order to maintain its dominance. This is what accounts
for the dynamic, adaptive, and plastic qualities of capitalism -- if
something can be made subject to the overall hegemonic formation (that
is, can be made profitable within a market economy), then it can be
successfully incorporated.

Capitalist realism, then, describes the transformations that have
occurred to create massive instability and precarity through a change in
the political, social and economic order and a change in the nature of
work. It details how these transformations have created new sites,
sources of, and strategies for accumulation, and shows how these new
mechanisms of accumulation lead to a bureaucratic and powerless culture.
This culture slowly stagnates through its focus on representations, and
in doing so, begins to become a totalising hegemony, in which any
potential resistance to capitalist realism is always-already
re-incorporated into it. The cultural material from which people are
working under capitalist realism begins to stagnate, and there is a
sense that even the very idea of the future is retreating, as
hyper-reality paints a sense that the world has always been as it
currently is, and will always be that way. In the face of complete and
totalising power, people feel powerless and unable to change this
system. How could one even begin to imagine of the possibility of
change, when there are no successful examples of change and society
tells us again and again that it has always been as it is now?
Capitalist realism thus drives us back to the most essential posed by
Deleuze and Guattari's work, then:

How is it that there is always something new? How are novelty and change
possible? How can we account for a future that is different from, and
not merely predetermined by, the past? (Shaviro, 2007: 23)

Having clearly identified what capitalist realism *is*, *does*, and *how
it operates*, the rest of this thesis will explore these questions of
novelty and change in the empirical context of austerity, the children's
social care system and the charity or third sector.


\section{Capitalist realism in reality: austerity and the third sector}
\label{capitalist-realism-in-reality-austerity-and-the-third-sector}

As Fisher presented capitalist realism in *Capitalist Realism*, it was
largely a description of the past thirty years or so of policymaking and
political manoeuvring by neoliberals and others aligned with the
interests of capital. In many ways, though, capitalist realism truly
came to bear through the enactment of austerity policies across the
world in response to the 2008 financial crisis. Through austerity, we
can see capitalist realism in practice as it acts to try to create
entirely new forms of governance and precorporate any future resistance
to those forms of governance my transforming the governmentalities of
citizens. Although I will only discuss the United Kingdom in depth here,
what follows may be equally true of many other countries around the
world which have followed neoliberal policymaking ideals for years and
which experienced a wave of austerity policies in response to the global
financial crisis of 2008.

\subsection{What is austerity?}
\label{what-is-austerity}

In a speech at the Conservative Party spring conference in 2009, David
Cameron redoubled his party's commitment to an entirely new approach to
governance if they were elected. He claimed:

There are deep, dark clouds over our economy, our society, and our whole
political system. Steering our country through this storm; reaching the
sunshine on the far side cannot mean sticking to the same, wrong course.
We need a complete change of direction\ldots{} In this new world comes
the reckoning for Labour's economic incompetence. The age of
irresponsibility is giving way to the age of austerity. (Cameron, 2009)

This ```age of austerity'' was an attempt to signal an end to the
appearance of Keynesian economics that had reigned for the past fifteen
years or so (Blyth, 2013). Although these policies appeared Keynesian in
nature, they actually were an attempt to enact a progressive
neoliberalism, functioning as an ``ambidextrous state'' (Peck, 2010, p.
X) where one `hand' of the state supported social causes and the other
`hand' of the state developed punitive measures to veer people away from
these social welfare measures.

Austerity measures were developed in response to the 2008 global
financial crisis, supposedly as a way to bring public spending under
control after the aforementioned ``age of irresponsibility''. For
austerity to make sense, it casts the financial crisis as a ``sovereign
debt crisis'', in which the state spent too much on social welfare and
financial support for the poorest in society, spending `wastefully'. In
response to this, austerity measures were presented as a need for
``everyone {[}to{]} tighten\ldots{} their belts'' (Blyth, 2013, p. 13).
By falsely equating household debt with national debt and public
spending deficits, the Coalition government was able to position
austerity as a necessary set of measures to be taken after a shared
sense of `going too far'. Yet the issue with this is that the financial
crisis was never a sovereign debt crisis - it can be perfectly
acceptable for a state to be in a public spending deficit, as much
public infrastructure spending is costed on the basis of reaping the
benefits of long-term investments. At its core, the financial crisis was
a banking crisis that was built upon neoliberal deregulation of the
banks and a lack of legal and professional oversight of financial
institutions, so that unregulated credit could be easily given to those
who had little to no ability to pay it back - essentially, through
creating masses of personal debt. Austerity, then, was an answer without
a problem. **add in Lowndes and Pratchett 2012 which talks about key
policies of Coalition government**

When the Coalition government took power in 2010, the Conservatives and
Liberal Democrats quickly set about transforming the state to enact
austerity policies. Austerity affected local government, central
government, other public sector organisations and even third-sector
organisations in a multitude of ways. Central to this new age of
austerity were plans to cut the budgets of public sector organizations
and to incentivize projects and organizations that delivered `more for
less'. The Coalition government cut spending and changed policies in
order to ensure that a new logic of `value for money' reigned supreme.
Here, we can see the enactment of the austerity governmentality: the
machineries of governance change to reduce public spending, and changes
in logics and rationalities accompany it. Whilst no area of public
spending in the United Kingdom remained untouched by austerity policies,
one of the key areas of financial cuts and policy transformations was
services delivered to children and young people (Youdell \& McGimpsey,
2015).

\subsection{The impacts of austerity }
\label{the-impacts-of-austerity}

These cuts and policy transformations had both material and affective
impacts. Materially, this has led to a massive decline in funding for
public and third sector organizations. Small and medium-sized charities
experienced a 16\% and 17\% decline in income, respectively (over the
2008 - 2014 period). Although the narrative that the Coalition
government put out about austerity suggested that ``we're all in this
together'', Clifford (2017) has found larger declines in charity incomes
in more deprived areas. The `Big Society' policy agenda that accompanied
austerity suggested an enhanced role for voluntary sector organisations
would be important, but the massive reduction in income for mid-sized
organizations has been a huge barrier to this. In some cases this has
been due to a direct removal of public funding for the charity, but in
others it has been linked to a general decrease in charitable donations.
Jones et al. (2016) have highlighted the spatial disparities in the
effects of austerity on Voluntary and Community Sector (VCS)
organisations, showing how affluent areas have been able to maintain
healthy levels of VCS funding whilst deprived areas have not. Yet
affluent areas have comparatively less need for VCS organisations'
support in the context of austerity, intensifying the negative effects
of austerity in areas without high levels of charitable donation. VCS
organisations are being asked to do more with exponentially fewer
resources.

Particularly in the case of youth services, service provision has
disappeared or become particularly strained. Thousands of youth work
jobs were cut and youth centres closed across the country as a result of
the 71\% reduction in local authority spend in England since 201 [????? what yr} (YMCA
2020). The reduction in funding and loss of jobs also resulted in an
accelerated form of competitions between organisations, which tends to
favour larger, national organisations and which makes the development of
positive relationships in the third sector more difficult. As funds were
reduced and made increasingly conditional, decisions about the
allocation of resources began to be made on the basis of `value for
money', which in practice has seemed to mean ``the cheapest bid rather
than explicit commitment to support existing work which has demonstrated
effectiveness and commitment to local communities'' (Clayton, Donovan
and Merchant 2016: p. 732). Most significantly this ``austerity
localism'' has led to a breakdown in trust between the users of support
services and voluntary and third sector organisations. This increased
sense of distance between the state and organisations, and organisations
and the state is creating ``forms of disconnect between those in power
and those who feel on the receiving end of damaging decisions''
(Clayton, Donovan and Merchant 2016 p. 737).

Horton (2016) highlights austerity's role as an anticipatory politics,
an imagined-and-anticipated future that becomes embedded into
governmental discourses and highlights how this anticipated future helps
to build a `common sense' around the idea of austerity policy. Horton's
work focuses in particular on the experiential and affective aspects of
austerity, furthering Peck's (2012: 632) notion of exploring ``the
politics of everyday austerity at the street level\ldots{} experienced
in daily life, in workplaces, households and the public sphere'', trying
to understand what ``anticipated service withdrawal'' *feels* like.
Horton identifies a multitude of different affects around the
anticipated future of service withdrawal: feelings of ``low mood'',
``distrust'', anxieties about young peoples' futures, and a feeling that
people were ``just waiting'' because ``time {[}is{]} running out''
circulated in response to the horizon of an austere future.

Through its affective life, then, austerity creates a waiting subject
that anticipates a negative future. This anticipation of negative
affects can be seen as one of the defining factors of an
austerity-intensified capitalist realist hegemony: it transforms
experience and thus subjectivity. Recent work on austerity has paid
increased attention to its affective life, both to describe the affects
that circulate as a result of austerity and to understand how these
affects change the everyday life of those who experience them. For
example, Hitchen (2016: 102) focuses on austerity ``shape{[}s{]}
capacities to feel and act'', which draws attention to the
contradictions and complexities of austerity as it is actually
experienced. Through this lens, Hitchen suggests austerity can be
understood as a multiplicity that surfaces in different ways through
individuals' everyday lives, as a series of affective atmospheres that
``envelop and condition everyday moments and spaces'', that shapes
everyday life and future imaginaries or anticipations. These atmospheres
include ``frustration'', ``disappointment'', ``anxiety'', and ``fear''.
Importantly, these negative affects sit alongside changes to future
imaginaries or an understanding of the possibility of action. Hitchen
specifically highlights ``the presentation of absence'' as a limiting
factor to action, a perceived lack of possibility conditioning and
structuring people's capacities. This can lead to both an acceptance of
austerity, in order to attempt to ``get on with life'' and a paralysis
in the face of austerity, as people become exhausted with merely trying
to ``stay afloat''. In this way, then, austerity's affective atmospheres
``shape bodily capacities to act as they envelope, and therefore
influence, subject-object encounters'' (p. 117).

Austerity-intensified capitalist realism, then, acts in both the
background and the foreground, impacting every single part of life but
also failing to surface enough to become an active presence. Raynor
(2017: 195) also addresses the ``diffuse and disparate'' ways that the
contradictory present absence of austerity surfaces in people's lives.
For Raynor, mundane presences and absences bring austerity to the
forefront, like the unanswered phone or the empty flower bed. These
presences and absences take on a vital life and agency of their own,
making apparent the all-consuming power of austerity. Yet this is never
a power that can be taken to completion. Much like the Luxemburgian
contradiction to capitalism, that it must become universal but requires
an outside, austerity attempts to structure all experience but is
fractured, fragmentary, and partial. As Raynor states, under austerity,
things ``fall apart'' because austerity is always contradictory,
paradoxical, incomplete.

Hitchen (2019) picks up on this temporal aspect of austerity and the
ways it shapes experience. For Hitchen, these contradictory and
paradoxical affective atmospheres are both ``uncanny'' and ``paranoid''.
The extended experience of austerity becomes uncanny because of the way
the unknown and the known become entangled. As people begin to
experience a sense of the unknown as familiar, the uncanny emerges. The
unknown becomes more comfortable than the known, because it is always
bounded by the understanding that when the unknown becomes the known,
material circumstances will get worse; thus it is better to exist in the
unknown. Living with the uncanny begins to create a ``paranoid mode of
waiting'' simultaneously. Through the repetition of ``employee
engagement processes'' or other such bureaucratic capitalist realist
measures designed to engage in dreamwork or memory disorder, Hitchen
suggests that ``a temporality emerges that looks both forward and
backwards: forwards through the unknown knowledge that remains absent,
and backwards through knowledge imparted from all previous employee
engagement sessions'' (12). In doing so, paranoia is created and shapes
people's capacities and understanding of temporality.
Austerity-intensified capitalist realism, then, becomes an unbounded
temporality which threatens to consume all of time itself. Because there
is no clear start or end to austerity due to its rewriting of
temporality, austerity-intensified capitalist realism is able to secure
its hegemony through colonising the past, present, and future, and our
imaginaries of which of those are, were, or could be like.

Austerity-intensified capitalist realism exerts its hegemonic power
across subjectivity, materiality, affective atmospheres, and
temporality. The reason that austerity is able to intensify the effects
of capitalist realism is its self-optimising nature: it exhausts people
and their capacity to act against in. \citep{harrison_cant_2020} Harrison (2020) has built upon
this to explain why the public response to austerity has been so muted.
Harrison's theory, which blends together civic voluntarism, grievance
theory, and policy feedback theory, suggests that those most affected by
the lived realities of austerity are those who most lack the resources
to take action on austerity. For Harrison, austerity arrived at a time
of already decreasing levels of political participation. Because the
impacts of austerity have been unequally distributed, responses to it
have been also. On the one hand, those who have been affected most
severely by austerity lack the material and mental resources to
participate in oppositional activity - as we have seen, austerity
exhausts and disempowers people. On the other hand, those who are
minimally affected by austerity lack sufficient grievance with the
policy to mobilise opposition towards it, due to successful government
rhetoric around the economic necessity of austerity - ``we're all in
this together''. These factors combine to create significant barriers to
resistance to austerity as a policy, and ensure the maintenance of a
high-need, low-resource status quo - cementing austerity's impact.

In addition to this, it is worthwhile considering how austerity has
affected how people identify with their own subjectivity and class
positioning. \citep{jeffery_classificatory_2019} Jefferey et al. (2019) have shown that as a result of
austerity, people who might traditionally have identified as working
class instead feel a sense of ``shameful identification'' with that
identity or indeed a ``disidentification'' with it, based on a
displacement of that identity to others (who represent some kind of
negative Other). Jefferey et al. position these as ``classificatory
struggles'' after Tyler (2015), and show the experiential impacts of
almost a decade of austerity policies. Not only do people feel shame
around working class identity, but they have internalised this affect of
shame to the level of personal identity. In large part, this is due to
stigmatising media narratives that have come to dominate the public
consciousness - or as Gramsci might argue, create a new ``common
sense''. In analysing the response around the Grenfell Tower fire,
Shildrick (2018) develops the concept of ``poverty propaganda'', which
builds consent for a regressive and capital-supported class politics by
stigmatising working-class identity. Through TV shows such as *Benefits
Street* and *Life on the Dole*, alongside an austerity politics that
demonises ``shirkers'' and ``scrounger{[}s{]}'', the working class
subject becomes a point of shame, a haunting figure that stalks people's
experiences, intensifies the effectiveness of capitalist realism, and
which disempowers subjects almost completely.

\section{The digital turn and the production of vulnerability }
\label{the-digital-turn-and-the-production-of-vulnerability}

In line with capitalist realism, then, we have seen that austerity
creates widespread insecurity and precarity, completely restructures
social relations, and has a potent affective life. Moreover, it
transforms this precarity into common sense, demotivating people
affected by it, discouraging resistance to it, and making it appear as
if it is the only system that could ever exist, rewriting temporal
experiences. What is left, then, is to understand what the accumulative
activities of austerity are. As Blyth (201X) explains in *Austerity: The
History of a Dangerous Idea*, one of the central problems of a global
policy of austerity is essentially a crisis of accumulation - it becomes
impossible for capital to become productive. Someone has to spend in
order for someone else to save. Thus if all states try to cut their
growth at the same time, what happens is essentially a global stalemate.
Blyth invokes Keynes to explain this paradox of thrift: ``if we all save
at once there is no consumption to stimulate investment'' (p. 8). If an
entire country's economy is paying back debt at the same time, then the
only way capital can become productive is through exporting more to a
state that is still spending. If every country is paying back debt and
trying to bring down the amount of public spending at the same time,
capital has nowhere to go to become productive. Although austerity
hasn't (in most places) created any noticeable level of growth, it did
stabilise capital. The question becomes - if there is global austerity,
how was this crisis of accumulation being dealt with - and what are the
new sources of and strategies for accumulation that developed as a
response?

I would argue that it is the very creation of widespread insecurity and
precarity that has become the new site of accumulation, and that the new
strategy for accumulation in this site stems from the `digital turn'
that has occurred in particular over the last decade or so, in tandem
with austerity. First, I will explore the production of vulnerability as
a classification, and the way that in tandem with austerity policies,
vulnerability and the management of vulnerability has become a new (or
revitalised) economy. Then, having explored vulnerability as a site of
accumulation, I will explore `the digital turn' as a strategy for
accumulation in the context of these new `vulnerabilities'. This digital
turn is part of a very particular historical moment, as technological
and economic developments both needed to reach a specific threshold in
order to be able to create these new strategies.

\subsection{The production of vulnerability}
\label{the-production-of-vulnerability}

Charities that provide support services to people they perceive to be
vulnerable have had to completely change the way they operate as a
result of austerity policies. Not only have charities become
increasingly responsible for the provision of services due to the `Big
Society' policy agenda, but they also find themselves operating in an
even more competitive environment than previously found them in.
Charities find themselves operating in a heavily marketized and
financialized environment, competing against each other for `tenders'
for contracts to deliver vital services (Buckingham, 2012) or for
funding from grantmaking organisations, meaning they must follow these
organisations' agendas (Clayton, Donovan and Merchant, 2016). As Adams
(2013) has noted, market-driven governance such as those engendered by
neoliberalism, ``enable the needy to become a site for the production of
capital, generating profits for companies that spring into existence
after a disaster'' (p. 9). Following Lord (2019), I argue that the new
role that charities and social support organisations have taken on as a
result of austerity is one of these such enterprises. Austerity operates
by ``encouraging business intervention into areas where its presence was
traditionally limited'', adding market-driven rationales, and
``subordinat{[}ing{]} charity to business'' resulting in ``the insertion
of an ethos of private profit into charity work'' (p. XX).

Viewed in terms of accumulation, Lord's suggestion that charities have
had to become businesslike due to austerity is essentially claiming that
austerity has made ``the vulnerable'' into a site of accumulation *and*
that charities are the sole profit-making beneficiary of this new
commodity. Though both of these are somewhat true, I believe that by
separating out these two elements we can see that *more* than that is
happening. Charities are both the producer and consumer of ``the
vulnerable'' as commodity. In doing so, this means two important things:
they create and prepare the commodity of the vulnerable (for others to
make profit from) and they consume that commodity, spending contracted
funds, grant funds, or donated funds on `improving' the living situation
of ``the vulnerable'', but only to such an extent that they remain
vulnerable. Thus charities under austerity-intensified capitalist
realism act cyclically: they prepare the commodity of ``the
vulnerable'', allow others to consume ``the vulnerable'', and consume
``the vulnerable themselves'', and in doing so re-produce ``the
vulnerable''. The consumption of the category of ``the vulnerable'' by
charitable organisations is also that which produces ``the vulnerable''.

Under austerity, then, charities and other social justice organisations
have become an important part of the accumulative framework. Austerity
and other social injustices create a market for ``the vulnerable'' by
worsening social inequalities, and increasing levels of deprivation or
marginalisation. Marketized charities attempt to respond to this
increased need, but need to do so in a way that conforms to the
structures of capitalism and austerity-induced `value for money'
policies. As such, they contribute towards the continuation of the very
situations that sustain vulnerability. As will be explored in greater
depth later in this thesis, charities will end up doing things that are
counter to their ostensible purpose, but which are necessary for their
continued existence. In doing so, one of their primary activities - and
one of the primary mechanisms for accumulation under
austerity-intensified capitalist realism - becomes classification.

According to Bowker and Leigh Star (1999:10), a classification is ``a
spatial, temporal, or spatio-temporal segmentation of the world''. A
classification system is a ``set of boxes\ldots{} into which things can
be put to then do some kind of work - bureaucratic or knowledge
production''. Under austerity-intensified capitalist realism, charities
attempt to create classification systems which assert their ability to
perform well and suggest that their model of how to work with ``the
vulnerable'' is the best `value for money'. The difficulty here is that
when classifications are applied to ``human kinds'', they create
``looping effects'' (Hacking 1996) which change the characteristics and
behaviour of those being classified. Hacking (1996) explains that when
we create new ways of classifying people (such as ``vulnerable''), we
also ``change how we can think of ourselves\ldots{} our sense of
self-worth, even how we remember our own past''. For Hacking, this
generates a looping effect, because people who have been classified
begin to behave differently; classifications attempt to create
standardized segmentations of the world, but because of these changes in
self-conception and behaviour due to the act of being classified,
``kinds are modified, revised classifications are formed, and the
classified change again, loop upon loop'' (p. XX).

Campbell and Stark (2015) specifically deal with the looping effects of
the classification of vulnerability. In their paper, they describe how
the figure of the ``vulnerable human subject'' emerged in line with the
emergence of the field of bioethics. The figure of ``the vulnerable''
was seen to be a member of a racial minority group from an urban area
who was likely to have a low income and likely to have a minimal
education. Prior to the emergence of bioethics, people who were
incarcerated were frequently seen as an ideal population for long-term
research because of their perceived stability. After the emergence of
bioethics, ``children, prisoners and the `institutionalised mentally
infirm'\ldots{} {[}and{]} people at a socioeconomic disadvantage,
including racial minorities'' (p. 16) began to be seen as unfit for use
in research programs, due to the possibility of coercion or
exploitation. In surveying evidence of those involved in LSD trials both
at the time and through oral histories, Campbell and Stark show that
people are able to reorganise their experiences to fit with the new
classifications that have been developed to describe them, in order to
see their past in a different way. Put simply, those who were once not
``vulnerable'' came to understand their experiences in light of
``vulnerability'' once the classification had been applied to them.

Returning to the case of austerity-intensified capitalist realism, then,
this internalisation of the accumulative activity of classification
results in a further intensification of the power, stigmatisation, and
affects of powerlessness that circulate as a result of capitalist
realism. ``Vulnerability'' is a lens through which an individual's
capacities might be externally moderated or protected, creating
``vulnerable'' subjects who must remain passive, or be controlled.
Individuals who are supported by charities then might slowly begin to
understand themselves in the context of ``vulnerability'', and may begin
to limit their *own* capacities in response to this, or expect control,
paternalism, or passivity. This is the contradiction of charity work
under austerity: although ostensibly organizations appear to be helping
their service users, they only help them to the point that accumulative
activity can continue (i.e. their charity can continue to make money,
earn grants, or receive donations as a result of the work), and may
limit the capacities or functionings of their service users - or
encourage them to limit their own capacities or functionings.
Vulnerability is a new and enlarged site of accumulation under
austerity.

\subsection{The digital turn}
\label{the-digital-turn}

If the production of vulnerability is austerity-intensified capitalist
realism's new *site* of accumulation, then the turn towards `the
digital' - particularly in public sector and charitable organisations -
is austerity's new *strategy* for accumulation. When austerity policies
began to enacted in the UK in 2010, digital technologies had become
sufficiently mobile and pervasive that they warranted inclusion in the
government's strategies for how to deal with the country's national
debt. In June 2010, the newly elected Prime Minister David Cameron wrote
to Martha Lane Fox, the co-founder of lastminute.com, to ask her to
become the UK's first `Digital Champion'. Cameron tasked Lane Fox with:

\begin{itemize}
\item
  ``Contributing to the Government's work on behavioural change\ldots{}
  to encourage as many people as possible to get online in the lifetime
  of this Parliament'', and
\end{itemize}

\begin{itemize}
\item
  ``Advising Government on how efficiencies can best be realised through
  the online delivery of public services'', including putting public
  services online, redesigning services to make for more efficient
  delivery, and transforming the then-government website Directgov
  (Cameron, 2010).
\end{itemize}

Cameron viewed digital services as an important part of his ministry,
both for advancing the Big Society agenda and for bringing down the cost
of public services as part of his government's austerity program.
Cameron believed that digital public services would enable citizens to
make connections with one another, increase employability, increase
access to public services, and lead to consumer savings. As such, the
digital turn in public services and charitable organisations must be
viewed as an integral part of the austerity program.

The 2010s were the first point that a newly elected government could
easily make digital technologies a cornerstone of their program of
governance, as the infrastructure that underpins contemporary digital
technologies had become sufficiently mature. Prior to the 1990s, the
internet had largely been non-commercial, filled primarily with
researchers and hobbyists, but the dot com boom of the decade resulted
in its commercialisation (Srnicek, 2017). Even within its early
commercialisation, volunteer labour remained a key feature, as
demonstrated by the 14,000 people who willingly staffed the chat rooms
and bulletin boards of America Online (AOL) for years (Postigo, 2009).
When the dot com bubble eventually burst, its survivors had successfully
created an infrastructural base on which our modern digital technologies
have become installed (Srnicek, 2017).

For example, ahead of the company's Initial Public Offering in 2004,
Google launched its new mail service, Gmail. As part of Google's
personalisation endeavours to generate revenue (the way it had survived
the dot-com bubble), Gmail ``scan{[}ned{]} private correspondence to
generate advertising'' (Zuboff, 20XX). This was the first instance of
the extraction and exploitation of ``behavioural surplus'', in which the
by-products of an unrelated action (i.e. sending and receiving email)
could be used to accumulate further value. Throughout the 2000s, Google
and other companies developed on this approach in order to make the
creation and maintenance of free at the point of use digital
technologies profitable - eventually culminating in Google licensing its
Android mobile phone operating system to mobile phone creators for free
``to draw users into Google Search and other Google services'' (Zuboff,
20XX). As such, Google began to act as a platform, positioning
themselves as ``intermediaries that bring together different users'' and
profiting off of the behavioural surplus through targeted advertising.
By extracting capital through behavioural surplus and making services
free at the point of use, companies such as Google were able to bring
down the cost of mobile handsets, ensuring a wider reach across society.
As mobile devices became more pervasive, mobile data became faster and
more cheaply available (through an expansion of the 3G network (Ofcom,
2008)) and more local authorities began to roll out free public wifi
systems (Jefferies, The Guardian, 2011). This set the stage for the
digital public services to be of interest to Cameron's new government -
and a potential support for reducing public expenditure and enhancing
the `value for money' policy agenda.

Lane Fox's review of government online services concluded in September
2010 and made four key recommendations to government. These
recommendations were:

\begin{itemize}
\item
  To make Directgov the forntend for all government transactional
  services,
\item
  To make Directgov a wholesaler and retail shop front for all
  government services and content, including opening up APIs to third
  parties,
\item
  To create a central content publishing team to eliminate fragmentation
  and duplicated content within government, and
\item
  To appoint a CEO for a Digital within government which would have
  complete authority over user experience, and which would merge all
  previously existing digital teams
\end{itemize}

Lane Fox argued that in in the past few years (meaning from around 2007
onwards), ``there {[}was{]}\ldots{} a reinvention of the Internet and
the behaviour of users'' and that ``digital services are now more agile,
open, and cheaper'' (Lane Fox, 2010). By contrast, the UK government and
civil service's use of technology was ``not agile enough'' and hadn't
yet moved towards developing ``a service culture'' that put the needs of
users/citizens above the needs of other government departments. Lane Fox
suggested that making these changes could save the government between
£1.3 billion and £2.2 billion a year. By creating a service culture, she
suggested that those responsible for digital technologies inside of the
government would be able to challenge any policy and practice that
undermines good service design. Furthermore, to assist with this
process, she called for hiring ``business people with experience of
switching to digital service design'', and ``encouraging commercial and
charitable organisations to use government applications to create
services for their own consumers and clients'' (Lane Fox, 2010).

Lane Fox's recommendations were adopted almost entirely, setting the
stage for digital services to become a new strategy of accumulation
under capitalist realism and directly leading to the foundation of the
Government Digital Service (GDS), which is now responsible for the
design, development and deployment of all of the government's digital
services (should I cite ``A GDS story''?). The formation of GDS is
integral to the development of the new strategies of accumulation - not
necessarily for the organisation itself, but for the way that GDS'
practice developed and spread across the public and third sector.
Ultimately this is due to the way that the Big Society agenda
underpinned Lane Fox's report and the GDS itself; in making GDS a
wholesaler of government services and content, for example, Lane Fox was
proposing what became known as ``government as a platform'' - the idea
that a government might create an infrastructure that ``allows people
inside and outside of government to innovate'' (O'Reilly, 2011, p. 15),
or to ``encourage the private sector to build applications that
government didn't consider or doesn't have the resources to create''
(O'Reilly, 2011, p.31). The inclusion of charitable organisations in
Lane Fox's vision of what GDS would become, then, was an early sign that
charities would be expected to develop digital capacities and services
that could build upon government-provided infrastructure. GDS then would
become a template for the formation of digital transformation teams in
third sector organisations.

This `digital turn' in public services should not be understood as
purely about a move towards technologically mediated public services, or
Internet-enabled technologies which facilitate the use of these
services, but instead about the ``culture, processes, business models
and technologies of the internet era'' (Public Digital, 2017), as Public
Digital, a consultancy organisation composed of ex-GDS employees define
it. In this sense, the digital is about more than the creation or use of
digital technologies - it is about the ways that the availability of and
access to these technologies have transformed ways of living and
working. The most central change within this is the move towards service
design and `agile' ways of working, which notably were not present in
government or the third sector prior to the formation of digital teams.

`Agile' ways of working are those which conform to the twelve software
development principles established in the *Agile Manifesto* (2001). In
contrast to the `Waterfall' software development methodology, which
proceeds in linear and sequential blocks of activity from
requirements-gathering, to design, to delivery, testing, and maintenance
(does this need a ref? Probs), agile suggests an iterative and
continuous delivery process. This methodology ``welcome{[}s{]} changing
requirements'', suggests that teams should be mixed across different
types of expertise, promotes simplicity, and encourages
self-organisation and self-optimisation{[}fn: though there are many
debates to be had about whether or not agile processes *do* these
things, this is not within the scope of the chapter{]}. It is fitting
that agile methodologies would be brought into the civil service at the
outset of Cameron's first ministry, as its focus on self-organisation,
self-optimisation and flexibility are clearly rooted in neoliberal
ideals of self-governance, and as Gillies (2011) notes, agility acts a
governmentality designed to shape the subjectivity of the worker towards
always being a better worker, a more governable citizen. As such, it
clearly aligns with the instabilities created by capitalist realism.

Many definitions of service design employ a tautological definition -
describing it as the design of services (Downe, 201X; GDS, 2016). This
still leaves open what a service is. Downe (201X) defines a service as
``something that helps someone to do something'' (Downe, 201X) but this
cannot be easily operationalised. Goldstein et al. (2002) propose that
the service concept is the missing link in defining what the activity of
service design *is*, as the ``service in the mind'' can show the *how*,
the *what*, the *customer's needs* and the *strategic intent* of a
service, ensuring integration between each of this. Using this
definition, the design of a service becomes about more than just
creating supports that help people to do something, moving to a higher
level of abstraction that emphasises the number of components that need
to work together in order to design an effective service. Perhaps it is
most useful, though, to understand service design as the practice of
creating infrastructures that begins with people's material needs.
\emph{\emph{{[}this argument needs fleshing out but I'm not going to
just yet because I think it might even get cut{]}}}

Broadly thought of, though, service design as it exists within
government and the third sector though is merely an enacted version of
agile methodologies, with the `software' being developed merely a public
service (that happens to be digitally mediated). This can be seen in the
content of the GDS Design Principles, which broadly align to those of
the Agile Manifesto:

1. Start with user needs

2. Do less

3. Design with data

4. Do the hard work to make it simple

5. Iterate. Then iterate again

6. This is for everyone

7. Understand context

8. Build digital services, not websites

9. Be consistent, not uniform

10. Make things open: it makes things better

Just like agile methodologies, then, we can understand digital and
third-sector service design to be an instantiation of the neoliberal
governmentality.

\begin{itemize}
\item
  this last paragraph basically needs to say service design and agile
  are basically the method through which `digital' happens to the public
  and third-sector, GDS was the leader of practice for this, and as it
  comes into action in the third-sector it takes the commodity of the
  vulnerable and then seeks to accumulate *even more* through it - i.e.
  by developing services which respond to the needs of `vulnerable'
  `users' as they are spoken and which doesn't critique the structural
  power dynamics undermining it
\item
  methods and methodologies can often result in the cementing of
  people's needs into something static - biopolitical - make the arg of
  mine and Erkki's persona paper- seeking to universalise and make one
  service or system or infrastructure which fixes all
\item
  this becomes a new strategy of accumulation, closing the loop on
  capitalist realism - summarise - problem becomes - what is living
  under capitalist realism like, how can we begin to contest this if its
  hegemonic?
\end{itemize}
