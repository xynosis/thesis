\chapter{Introduction}
\label{ch:1}

\section{}
\label{sec:}

In the opening to Mark Fisher's 2009 book \emph{Capitalist Realism}, he invokes a phrase (attributed in-text to both Fredric Jameson and Slavoj Žižek) that has almost become a truism in our contemporary moment, that "it is easier to imagine the end of the world than it is to imagine the end of capitalism". Throughout the rest of the book, Fisher details his titular concept through reference to contemporary media, continental philosophy, and a history of late 20th Century work. Nowhere is he clearer, though, than this opening passage, in which he describes capitalist realism as:

\begin{quote}
    the widespread sense that not only is capitalism the only viable political and economic system, but also that it is now impossible even to imagine a coherent alternative to it. \citep{fisher_capitalist_2009}
\end{quote}

In addition to being a description of the socio-political system in 2009, then, capitalist realism functions as a metanarrative, a story about the kinds of stories that we can tell or even imagine. Capitalist realism is a story that we tell ourselves (or are told) about ourselves, the world, and our potential to change both of those things. As an hegemonic narrative, capitalist realism indulges in the fiction that it is the only story possible - that everything has broadly always been as it is now, that things will stay much the same into the future, and that any significant deviation from these is impossible, unworkable, bound to fail. 

Although Fisher's book was written in the midst of the 2008 financial crisis, it was prescient in many ways of the the political moment that succeeded it. Across much of the world, the 2008 financial crisis became the justification for a renewed programme of fiscal austerity [cite], and in the United Kingdom this led to some of the deepest cuts to the budgets of public services in the country's history \citep{lowndes_local_2012}. The financial crisis was a point of potential rupture; the hegemony of neoliberal public policy could have been disrupted by state actors understanding that the structure of the current socio-economic system wasn't working for the majority of people [cite?]. Instead, the choice to turn towards to a renewed programme of austerity showed states' commitments to the principles of capitalist realism: there is no alternative, and we must continually work to re-engineer the "preconditions of market vitality" \citep{connolly_fragility_2013}; not the distant, inactive state of neoliberal theory, but a state that is "selectively active" \citep{harvey_brief_2007} in order to reperpetuate capitalist realism.

This led to a fundamental transformation of public services, with significantly reduced budgets and  increased responsibilities \citep{clifford_charitable_2017, jones_uneven_2016}. Whilst no area of UK public services remained untouched by these austerity policies, one of the key areas of financial cuts and policy transformations were services delivered to children and young people. Youth services saw larger reductions in levels of funding than public services generally \citep{youdell_assembling_2015}, with thousands of youth work jobs cut, youth centres closed across the country, and a 71\% reduction in local authority spend since 2010. % cite YMCA 2020 
This was damaging to young recipients of these services - not just because funding for them disappeared practically overnight - but because professional relationships that might have been built over years became increasingly fraught as individual youth workers inadvertently became the face of this systems change \citep{clayton_distancing_2016}. 

Austerity was both a product of capitalist realism and an intensifier of it. For services with young people as their service users, it may have appeared as if austerity had simply looked like budget cuts and a call for a greater role for evidence and research in the commissioning of public services. In reality, though, the impacts of austerity have been primarily affective, making themselves known in a variety of ways: as neglect, abandonment, disruptions, or indifference  \citep{raynor_dramatising_2017} or as "a paranoid mode of waiting" \citep{hitchen_affective_2019}, for example. These affective atmospheres can saturate a person's imaginaries about themselves, the world around them, and the potential for things to be different. Our dominant political condition becomes one of scarcity and precarity \citep{berlant_cruel_2011}, and begins to shift the values, pracitces and experiences of young people and the organisations which work to support them. 

Taking this as a point of departure, this thesis is an exploration of what austerity \emph{did} at a human and affective level, how these transformed public services became complicit in the intensification of capitalist realism, and how the methods and tools of various modalities of design can be used to disrupt capitalist realism's position as our dominant cultural narrative. First and foremost, this thesis is a work of ethnography, action research and participatory design. From work with almost two hundred participants (across young people, youth workers, and charity managers), this thesis advances an explanation of the changes in practice engendered by austerity-intensified capitalist realism and the affects resulting from these, alongside a grounded theory describing how this system operates. Having explored how capitalist realism operates, the thesis then turns towards two design-led responses to capitalist realism in the third sector. These design projects (\emph{It's Our Future} and \emph{fractured signals} represent an iterative attempt to refine methods to disrupt how capitalist realism operates, and inform the central contribution of this thesis: speculative praxis. Speculative praxis is the use of speculative and critical design methods to facilitate "the ongoing creation of further relations of possibility". These relations of possibility show promise in destabilising the hegemony of capitalist realism.  % Cite Colebrook 2020 

% Do I say something about service design as a response to the production of vulnerability?


\section{Research questions}
To explore what austerity-intensified capitalist realism has looked like in practice, and how best to use the tools and methods of design to respond to these challenges, I developed three research questions: 
\begin{enumerate}
    \item What experiences and affects does capitalist realism elicit?
    \item How does capitalist realism operate to make these experiences and affects possible (or probable)?
    \item How can the tools and methods of design be used to respond to the challenges presented by capitalist realism?
\end{enumerate}
% Rationale?
% \section{Methodology}


% \section{Contributions}


\section{Thesis structure}
The broad outline of the thesis is as follows. Chapter 2 develops the theoretical basis behind capitalist realism, using it as a lens to understand the past few decades of socio-political change and connecting it to a history of austerity and the emergence of 'digital' and service design. Chapter 3 develops the methodological underpinnings of this work, drawing from literature and approaches across ethnography, action and design as distinct-yet-complementary modes of knowledge creation.  Chapter 4 sets the scene for the rest of the thesis, describing some of the themes that recur throughout and introducing some of the principal characters - and the semi-fictional organisation \emph{The Charity}. Chapters 5 and 6 deal with the substance of my first two research questions, attempting to understand the experiences and affects elicited by capitalist realism, and how capitalist realism works. Chapters 7 and 8 are chapters led by the development of my speculative design practice through two distinct projects, and are used to develop the ideas underpinning speculative praxis. Finally, chapter 9 is a conclusion and synthesis of the thesis, bringing together its distinct threads, highlighting my contributions, and drawing out the implications of some of the knowledge claims made throughout.

Chapter 2 (title) digs deeper into the concept of capitalist realism, fleshing out a more cogent theoretical basis for the concept. Drawing on \cite{shonkwiler_reading_2014}'s explanation of the three pillars of capitalist realism, I seek out a deeper philosophical underpinning for each aspect. I connect these ideas to the emergence of austerity and the increased marketisation of the third sector, and highlight how the production of vulnerability became a renewed site of capital accumulation as a result of austerity-intensified capitalist realism. I end the chapter with a discussion of the role that the emergence of the 'digital' sector has played in responding to this production of vulnerability, and describe the turn towards service design in the third sector. 

In Chapter 3 (Methodology), I draw on methodological literatures from across ethnography, action research and design to construct a methodological approach suitable to the research context I conducted my work in: third sector support services for young people perceived in some way to be 'vulnerable'. I draw on a common history to each of these approaches, and highlight the unique affordances of each approach. Following this, I detail my approach to data analysis, primarily relying upon Charmazian grounded theory and enriched by material-semiotic approaches such as Karen Barad's agential realism. Finally, the chapter ends with a discussion of the ethical and justice-focused considerations required to meaningfully conduct participatory work with oppressed and/or traumatised young people, and the way this translates to writing about this research. 

Chapter 4 (Setting the scene: finding slippages) introduces The Charity, a semi-fictional composite of the various organisations who I have worked with who is the primary object of study in this thesis. The chapter is structured around the introduction of three projects inside of The Charity, and the moments that I began to notice something strange and unexpected was happening in each - following my "ethnographer's nose twitch" (after \cite{star_ethnography_1999}). This chapter also introduces the recurring participants/ characters that feature throughout the thesis, and the various projects of The Charity that they work as a part of. The chapter effectively sets the scene for the two grounded theory chapters which follow.

In Chapter 5 (\emph{Living (and working) through austerity: everyday life for managers, workers, and care-experienced young people}), I detail the findings of my ethnographic work inside of The Charity, and explore what everyday life is like for care-experienced young people who have left local authority care, and what work is like for the workers who support them. I detail some of the changes in practice that occurred as a response to austerity, such as XYZ and ABC. I then turn towards understanding this experience from the perspective of a young person who makes use of these services

Chapter 6 (\textit{A grounded theory of justification, classification and performance practices}) is an exploration


%make clear that design chapters are taking a different and slightly more detahced approach 
Beginning the design-led portion of the thesis, Chapter 7 (\textit{It's Our Future: developing methods to resist justification practices}) describes a project commissioned by The Charity to understand what their service users really wanted for their futures, and my attempt to develop a suite of methods to enrich their visions of the future. The chapter describes the project's brief, the process of designing the intervention, the numerous challenges faced along the way (particularly through the continual re-appearance of justification practices), and the final methods used to elicit richer visions of the future and meaningfully resist capitalist realism. I analyse the outcomes of the event and evaluate the efficacy of the methods at reaching their stated aims, of developing young people's imaginaries beyond capitalist realism and finding meaningful ways to resist the influence of justification practices inside of charity projects such as this. 

Chapter 8 (fractured signals) is the final findings chapter of the thesis, detailing a design project ran in Spring 2021 to distil the methods used throughout this thesis. Due to the COVID-19 pandemic, original plans to refine these methods had to be abandoned, and so this project dealt explicitly with the theme of emergent events, unexpected changes, and the power that we all have to transform the world around us. The chapter details the design challenges I established at the outset of the project, and the design process of creating a set of speculative artefacts to be used in the intervention. I analyse the responses participants had to the project, and use these to present a set of methods and 

% Chapter 9 (Synthesis) reflects on what YYY to ZZZ. The chapter . It ends with a discussion of the overall research questions, a reflection on XXXX, and future research opportunities.

% I have constantly found it difficult to identify my disciplinary positioning; loved ones have joked on many occasions that my business card must just be an endless list of seemingly unrelated professions. I studied an undergraduate degree in ‘Flexible Combined Honours’ (English Literature, History, Politics, Geography), before pursuing a ‘Digital Civics’ masters and PhD (design, development, human computer-interaction, the social sciences). I have in many ways most neatly found an academic home in human geography and Science, Technology and Society studies, but even in those disciplines I tend to use languages or approaches that might be unfamiliar to their average readers. I am an ethnographer and so in many ways function as a sociologist, studying the patterns of interaction between different people, groups, services and systems. Yet I am also a designer, trying to ask how those interactions might be different and how they might be embedded in a different set of services or systems. My research site (and the majority of my adult working life) has been grounded around children and young people perceived to be ‘vulnerable’, and so knowledges and literatures from youth work and social work filter into what I do. I am an anarchist and activist and so theories of power, of movement-building, of mutual aid and self-care have always come to the forefront of my work. I have experienced trauma and thus I am attentive to the ways that personal and societal healing might arise through my work. 

% 	As such, it does not do my thesis and my research approach justice  to merely call it ‘interdisciplinary’. I am at once marginal to all of these disciplines and also attempting to integrate them. Because of this, I consider this thesis a work of ‘minor theory’. Coined by Cindi Katz, minor theory refers to the potential “rupturing effects” (Katz, 1986, p. 491) of theories and approaches that sit marginally to a dominant center.  Katz builds upon Deleuze and Guattari’s conception of minor literatures, which supposed that authors such as Kafka or Beckett were able to disrupt and change the languages they wrote in due to their ‘outsider’ status, having written significant works in a language that was not their first. Because they wrote in a ‘major’ language without having the experiences that the majority might have, Deleuze and Guattari suggested that they could “carry you away” or “send the major language racing” (1986, p. 105). To write minor literature then is to subvert the major language from within. Minor theory is similarly interstitial - working in a field “in which one is not at home - where one has become ‘deterritorialized’ but where one works that deterritorialization to its limits” (Katz, p. 490). As such, though human geography, STS, service design and interaction design are the nearest intersections that I find myself speaking towards, this thesis functions most clearly as a piece of minor theory, “intertwined in an exquisite and mobile tension” (Katz, p. 491) across these disciplines and attempting to integrate an approach that works across each. 



The very oppressive pervasiveness of capitalist realism means that even glimmers of alternative political and economic possibilities can have a disproportionately great effect. The tiniest event can tear a hole in the grey curtain of reaction which has marked the horizons of possibility under capitalist realism. From a situation in which nothing can happen, suddenly anything is possible again.


