\chapter{Introduction}
\label{}

\section{}
\label{sec:}

% I have constantly found it difficult to identify my disciplinary positioning; loved ones have joked on many occasions that my business card must just be an endless list of seemingly unrelated professions. I studied an undergraduate degree in ‘Flexible Combined Honours’ (English Literature, History, Politics, Geography), before pursuing a ‘Digital Civics’ masters and PhD (design, development, human computer-interaction, the social sciences). I have in many ways most neatly found an academic home in human geography and Science, Technology and Society studies, but even in those disciplines I tend to use languages or approaches that might be unfamiliar to their average readers. I am an ethnographer and so in many ways function as a sociologist, studying the patterns of interaction between different people, groups, services and systems. Yet I am also a designer, trying to ask how those interactions might be different and how they might be embedded in a different set of services or systems. My research site (and the majority of my adult working life) has been grounded around children and young people perceived to be ‘vulnerable’, and so knowledges and literatures from youth work and social work filter into what I do. I am an anarchist and activist and so theories of power, of movement-building, of mutual aid and self-care have always come to the forefront of my work. I have experienced trauma and thus I am attentive to the ways that personal and societal healing might arise through my work. 

% 	As such, it does not do my thesis and my research approach justice  to merely call it ‘interdisciplinary’. I am at once marginal to all of these disciplines and also attempting to integrate them. Because of this, I consider this thesis a work of ‘minor theory’. Coined by Cindi Katz, minor theory refers to the potential “rupturing effects” (Katz, 1986, p. 491) of theories and approaches that sit marginally to a dominant center.  Katz builds upon Deleuze and Guattari’s conception of minor literatures, which supposed that authors such as Kafka or Beckett were able to disrupt and change the languages they wrote in due to their ‘outsider’ status, having written significant works in a language that was not their first. Because they wrote in a ‘major’ language without having the experiences that the majority might have, Deleuze and Guattari suggested that they could “carry you away” or “send the major language racing” (1986, p. 105). To write minor literature then is to subvert the major language from within. Minor theory is similarly interstitial - working in a field “in which one is not at home - where one has become ‘deterritorialized’ but where one works that deterritorialization to its limits” (Katz, p. 490). As such, though human geography, STS, service design and interaction design are the nearest intersections that I find myself speaking towards, this thesis functions most clearly as a piece of minor theory, “intertwined in an exquisite and mobile tension” (Katz, p. 491) across these disciplines and attempting to integrate an approach that works across each. 

Research questions
What has the impact of austerity-intensified capitalist realism been on organisations who work with young people perceived to be vulnerable, and the young people themselves?
How can design practices and technologies build counterpower?
How do we sustainably embed these technologies or practices into situated practice?