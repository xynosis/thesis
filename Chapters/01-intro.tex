\chapter{Introduction}
\label{ch:1}

\section{Capitalist realism and "there is no alternative"}
\label{sec:1-1}

In the opening to Mark Fisher's 2009 book \emph{Capitalist Realism}, he invokes a phrase (attributed in-text to both Fredric Jameson and Slavoj Žižek) that has almost become a truism in our contemporary moment, that "it is easier to imagine the end of the world than it is to imagine the end of capitalism" \citep[p. 2]{fisher_capitalist_2009}. Throughout the rest of the book, Fisher details his titular concept through reference to contemporary media, continental philosophy, and a history of late 20th Century work. Nowhere is he clearer, though, than this opening passage, in which he describes capitalist realism as:
\begin{quote}
the widespread sense that not only is capitalism the only viable political and economic system, but also that it is now impossible even to imagine a coherent alternative to it. \citep[p. 2]{fisher_capitalist_2009}
\end{quote}
Elsewhere, Fisher pushes this idea further, referring to it as a "slow cancellation of the future" \citeyearpar[p. 5]{fisher_ghosts_2014}. In addition to being a description of the sociopolitical system in 2009, then, capitalist realism functions as a metanarrative, a story about the kinds of stories that we can tell or even imagine. Capitalist realism is a story that we tell ourselves (or are told) about ourselves, the world, and our potential to change both of those things. As an hegemonic narrative, capitalist realism indulges in the fiction that it is the only possible story about the world—that everything has broadly always been as it is now, that things will stay much the same into the future, and that any significant deviation from these is impossible, unworkable, bound to fail. 

Although Fisher's book was written in the midst of the 2008 financial crisis, it was prescient of the political moment that succeeded it. Across much of the world, the 2008 financial crisis became the justification for a renewed programme of fiscal austerity, and in the United Kingdom this led to some of the deepest cuts to the budgets of public services in the country's history \citep{lowndes_local_2012}. The financial crisis was a point of potential rupture; the hegemony of neoliberal public policy could have been disrupted by state actors understanding that the structure of the current socio-economic system wasn't working for the majority of people. Instead, the choice to turn towards to a renewed programme of austerity showed a commitment to the principles of capitalist realism: there is no alternative, and we must continually work to re-engineer the "preconditions of market vitality" \citep[p. 60]{connolly_fragility_2013}; not the distant, inactive state of neoliberal theory, but a state that is "selectively active" \citep[p. 21]{connolly_fragility_2013} in order to reperpetuate capitalist realism.

Austerity led to a fundamental transformation of public services, with significantly reduced budgets coming alongside increased responsibilities for service provision \citep{clifford_charitable_2017, jones_uneven_2016}.  Whilst no area of public services remained untouched by government austerity policies, one of the key areas of financial cuts and policy transformations were services delivered to children and young people. Youth services saw larger reductions in levels of funding than public services generally \citep{youdell_assembling_2015}, with thousands of youth work jobs cut, youth centres closed across the country, and a 71\% reduction in local authority spend on youth services since 2010 \citep{ymca_making_2020}. Against a backdrop of claiming to prioritise early intervention, this devaluation of youth work was incongruent. Youth work's primary focus is to "support a young person’s personal, social and educational development” \citep[p. 110]{ord_young_2022} through a person-centred practice based on voluntary participation. By reducing funding, destroying enduring professional relationships  \citep{clayton_distancing_2016} and recontextualising youth service provision as a product (improved outcomes for vulnerable young people), rather than a process (undirected informal education for any young people), austerity created the context for a generation of young people to viscerally feel Fisher's "slow cancellation of the future" \citeyearpar[p. 5]{fisher_ghosts_2014}.

As support disappeared, past experiences were also re-contextualised by the advent of austerity. The project of austerity was therefore both a product of capitalist realism and an intensifier of it. Austerity thus had a rich affective life, making itself known variously through presence and absence: as pessimism or trap \citep{coleman_austerity_2016}; as neglect, abandonment, disruptions, or indifference  \citep{raynor_dramatising_2017}; or as "a paranoid mode of waiting" \citep[p. 13]{hitchen_affective_2019}. These affective atmospheres can saturate a person's imaginaries about what is possible—for themselves, for the world around them, and for the potential for things to be different. The dominant political condition became one of scarcity and precarity \citep{berlant_cruel_2011}, and began to shift the values, practices and experiences of young people and the organisations which work to support them. 

\section{Overview, research questions, and contributions}
This thesis takes these profound economic, social, and affective shifts as its point of departure, as it seeks to understand austerity-intensified capitalist realism more deeply. At one level, then, this thesis is an exploration of austerity and capitalist realism, seeking to better understand the experiences and affects elicited by capitalist realism, and how this managed to become hegemonic—how it became possible that "there is no alternative". Yet this thesis also an exploration of what can be \textit{done} about the problem of austerity-intensified capitalist realism and its limitations of people's imaginaries. Drawing upon the tools and methods of design, this thesis also seeks to understand how it can be possible to effectively respond to the hegemony of austerity-intensified capitalist realism, and how it might be possible to create novelty and possibility out of "a situation in which nothing can happen" \citep[p. 81]{fisher_capitalist_2009}.

In this thesis, I use the methods of ethnography, action research and design to work with almost two hundred participants over three years (including young people, youth workers, and charity managers) to understand what experiences and affects were engendered by austerity-intensified capitalist realism, how these experiences and affects are made possible and sustained, and how the tools and methods of design can create novel possibilities for action. To explore this, this thesis poses three research questions: 
\begin{enumerate}
    \item What experiences and affects does capitalist realism elicit?
    \item How does capitalist realism operate to make these experiences and affects possible (or probable)?
    \item How can the tools and methods of design be used to respond to the challenges presented by capitalist realism?
\end{enumerate}
I answer these research questions throughout the following seven chapters, and reiterate these answers succinctly in the final chapter. Aligned to each of these questions, the central contributions of this thesis are:
\begin{enumerate}
    \item A deeper exploration of capitalist realism's theoretical and material basis, grounded in ethnographic observation rather than cultural theory,
    \item A grounded theory (of justification, classification, and discursive accumulation practices) which describes how capitalist realism sustains the negative experiences that it creates, and how it maintains hegemony through evaluation practices, and
    \item A method assemblage (speculative praxis) that can help to address the foreclosure of possibility that is presented by capitalist realism through design-centered experimental actions within a frame of action centered on activist praxis to facilitate "the ongoing creation of further relations of possibility" \citep[p. 320]{harrison_future_2020}, and destabilise capitalist realism's role as the dominant cultural narrative. 
\end{enumerate}

\section{Thesis structure}
The broad outline of the thesis is as follows. Chapter 2 develops the theoretical basis behind capitalist realism, using it as a lens to understand the past few decades of sociopolitical change and connecting it to the advent of fiscal austerity in the early 2010s. Chapter 3 develops the methodological underpinnings of this work, drawing from literature and approaches across ethnography, action and design as distinct-yet-complementary modes of knowledge creation. Chapter 4 sets the scene for the rest of the thesis, describing some of the themes that recur throughout, introducing \textit{The Charity} as a composite of the multiple organisations I worked with throughout this research, and some of the central figures within the organisation. Chapters 5 and 6 deal with the substance of my first two research questions, attempting to understand the experiences and affects elicited by capitalist realism, and how capitalist realism works. Chapters 7 and 8 are chapters led by the development of my speculative design practice through two distinct projects, and work towards the development of speculative praxis, answering my third research question. Finally, chapter 9 is a conclusion and synthesis of the thesis, bringing together its distinct threads, highlighting the answers to my research questions and my contributions to knowledge, identifying some limitations of the research, and suggesting future possibilities for research within this space.

Chapter \ref{ch:2}, "How does capitalist realism work? Austerity and the production of vulnerability", digs deeper into the concept of capitalist realism, providing a more cogent theoretical basis for the concept than is indicated in \citet{fisher_capitalist_2009}'s \textit{Capitalist Realism}. Drawing on \cite{shonkwiler_reading_2014}'s explanation of the three pillars of capitalist realism, I seek out a deeper theoretical underpinning for each component. First, I turn towards capitalist realism's economic, social, and affective life by detailing the ways that it creates instability through changing economics. Then, I describe how capitalist realism relies upon novel sources of and strategies for accumulation through the constant creation of a novel `outside' to expand into. Finally, I detail the transformation of these changes into a Gramscian `common sense' that creates a felt sense of powerlessness. I describe how austerity is both exemplary of and an intensifier of capitalist realism through reference to the third sector under austerity, and show how austerity's central accumulative activity rests upon the production of vulnerable subjects. This is the point of departure for the rest of the thesis, as I explore what this has looked like in practice in order to understand might can be done about it. 

In chapter \ref{ch:3}, "Methodology", I draw on methodological literatures from across ethnography, action research, and design to construct a methodological approach suitable to the research context I conducted my work in: third sector support services for young people perceived in some way to be `vulnerable'. I draw on a common history to each of these approaches, identify how each aspect of ethnography, action, and design supports me to answer my research questions, and indicate their complementary nature. Following this, I detail my approach to data analysis, primarily relying upon Charmazian grounded theory and enriched by material-semiotic approaches such as Karen Barad's agential realism. Finally, the chapter ends with a discussion of the ethical and justice-focused considerations required to meaningfully conduct participatory work with oppressed and/or traumatised young people, and the way this translates to writing about this research. 

Chapter \ref{ch:4}, "Setting the scene: finding slippages", introduces The Charity, a composite of the multiple organisations I have worked with throughout this research and who acts as the primary research site throughout this thesis. The chapter is structured around the introduction of three projects inside of The Charity, and the moments that I began to notice something strange and unexpected was happening in each—following my "ethnographer's nose twitch" (after \citet[p.  610]{star_ethnography_1999}). This chapter also introduces the recurring participants that feature throughout the thesis, and the projects of The Charity that they work as a part of. The chapter effectively sets the scene for the two grounded theory chapters which follow.

In chapter \ref{ch:5}, "Living (and working) through austerity: everyday life for frontline workers and care-experienced young people", I detail the findings of my ethnographic work inside of The Charity with frontline workers and care-experienced young people. I highlight how the the value for money agenda forces support organisations to change their perceived efficacy in relation to work, labour, time, or money. Focusing on the experience of having to deliver "too much work", I describe how workers have to do work that they aren't trained for, experience new forms of emotional labour, and are subject to the control techniques of their managers. As a result, workers feel anxious, confused, distrustful and powerless.  I then turn to understanding the experiences of young people, and identify how their agency is disregarded through inadequate forms of listening, and how this leads to young people not receiving the right support, leading to them feeling isolated, confused, and anxious. I end this chapter by questioning how these experiences are created and sustained by austerity-intensified capitalist realism.

Chapter \ref{ch:6} , "A grounded theory of justification, classification and discursive accumulation practices"  picks up the question posed by the previous chapter, and answers it with reference to the movement towards evaluation in the aftermath of austerity. I identify the call for a common measurement framework by the House of Commons Education Committee in 2011 as an inflection point that amplified the presence of evaluation processes within youth service provision. In the chapter, I detail the atomisation of practice into `outcomes' and `outputs' as discrete units of understanding that can be used to create an evaluation. I describe the work of doing evaluation through the lens of both unskilled and skilled evaluators, and show how each of these justify the work of their organisation differently. I refer to these subjective evaluation practices as "justification practices", which are underpinned by "classification practices" centered on identity, which seek to establish who various kinds of intervention work best for. Finally, I describe how justification and classification practices lead to a performativity within youth service provision that results in "discursive accumulation" practices, through which The Charity constructs "best practice" and attempts to position themselves as leaders within the sector by changing the presentation of their work rather than changing the actual practice within their work. 

%make clear that design chapters are taking a different and slightly more detahced approach 
Beginning the design-led portion of the thesis, chapter \ref{ch:7}, "\textit{It's Our Future}: developing methods to resist austerity-intensified capitalist realism", outlines a set of requirements for what a successful design response to the problems posed by austerity-intensified capitalist realism and justification, classification, and discursive accumulation practices must include. The chapter goes on to discuss \textit{It's Our Future}, a project designed in conjunction with The Charity that was an initial attempt to create methods that respond to the challenges of austerity-intensified capitalist realism. I present the methods that I designed through the use of a playful facilitation aid and focused on supporting young people to imagine futures together that would transform their lives. The chapter describes the project's brief, the process of designing the intervention, the numerous challenges faced along the way (particularly through the continual re-appearance of justification practices), and the final methods used to elicit richer visions of the future and mitigate, resist, or construct alternatives to capitalist realism. I evaluate the efficacy of the methods, and find that they are effective at helping young people to develop critical consciousness by reflecting on their lived experiences, but they are less effective at creating a sense of possibility about the future or resisting the functioning of justification, classification, and discursive accumulation practices. . 

Chapter \ref{ch:8}, "\textit{fractured signals}: sustaining and embedding resistance to capitalist realism through speculative praxis", details a design project that iterates further on the methods presented in the previous chapter. \textit{fractured signals} focuses on the use of speculative enactment and the creation of several diegetic prototypes which more successfully bring participants into the speculation suggested by the project. The chapter describes the project's constraints (particularly in the form of the COVID-19 pandemic, lockdowns, and recent advent of the Independent Review of Children's Social Care), presents the diegetic prototypes and their design processes, and presents the experiences of participants (people who work with care-experienced young people). I describe how the methods are successful at helping participants to develop critical consciousness in the workplace, envision futures that center alternative values, and help workers to center the experiences of the young people they support. Finally, I identify how and why this speculative activity works (through an analysis of Deleuzian deterritorialization) and present speculative praxis as a method assemblage for creating novel possibilities for action and reflection in the context of austerity-intensified capitalist realism.

In chapter \ref{ch:9}, "Conclusion", I reflect on the research that I have presented throughout. I return to my research questions and identify how I have answered each in turn, and articulate my contributions to knowledge for each question. I identify some limitations of the research that I have presented, including the difficulty of conducting work like this whilst organisations are actively pursuing service delivery, the impact of the COVID-19 pandemic, and question the usefulness of capitalist realism as ongoing analytic in the context of deep societal change. Finally, I detail some of the implications of my research contributions, and highlight some possibilities for future research, including tracking the presence of the dynamics that I have presented throughout this thesis in other spaces, the impacts of the contingency of evaluation data in a context of increasing datafication, and identifying how speculative activity becomes hegemonic. 

At the conclusion of \citet[pp. 80–81]{fisher_capitalist_2009}'s \textit{Capitalist Realism}, he remarks that:
\begin{quote}
The very oppressive pervasiveness of capitalist realism means that even glimmers of alternative political and economic possibilities can have a disproportionately great effect. The tiniest event can tear a hole in the grey curtain of reaction which has marked the horizons of possibility under capitalist realism. From a situation in which nothing can happen, suddenly anything is possible again.
\end{quote}
Ultimately, this thesis is an exploration of the pervasiveness of capitalist realism and the methods through which glimmers of alternatives can be created. 

