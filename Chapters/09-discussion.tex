\chapter{Discussion}
\label{}

\section{Introduction}
\label{sec:}
In the previous chapter, I detailed the development of speculative praxis, a methodology to support people to resist and act against austerity-intensified capitalist realism. Building on the insights developed in the \emph{It's Our Future} project, the \emph{fractured signals} project supported workers in and around the children's social care system to rethink their practice, imagine different futures, and begin to identify ways to work towards these futures. I refer to this approach as speculative praxis, a methodology or way of living that incorporates speculative design practice as a way to create ... [interventions which create new possibilities]?

In this chapter, I recap the contributions of this thesis, [implications]

% make a point here about how the thesis functions as minor theory


% grounded theory of justification practices

% spec praxis
%what you have shown, what you have shown it 

\section{Contributions}
This thesis has three central contributions:  

1. A deeper exploration of capitalist realism’s theoretical basis
2. A grounded theory (justification practices) which describes the experiences elicited by capitalist realism and how the third sector/care system has shifted to create and sustain these experiences
3. A methodology (speculative praxis) that can address some of the foreclosure of possibility in capitalist realism through experimental actions which facilitate “ongoing creation of further relations of possibility”

what justification practices mean 
how spec praxis can be used

implications for AI ADM etc

haunting 
time

% My methodology, which made use of ethnography, action-centered methods, and design, acted as 

% Traditionally, service design does [some bad bullshit thing that isn't very considerate of structural issues]. 

% something about commodification or infantilisation of lived experience 

implications
% •	Impacts of performativity
% •	Classification (quantification, measurement, standardization)
% •	Datafication
% o	automated systems (AI ML ADM) and implications of JPs etc on them
% •	classification and quantification —> change in anticipatory practices —> performativity, lack of trust —> control behaviours —> extraction —> negative affects
% •	stabilisation of a particular kind of knowledge regime; (looping effects, biopolitics, governmentality which intensify this, and how digital tech ties into it
% •	restriction of possibility: change in self-experience and ‘slow cancellation of the future’ etc
% o	negative self-talk which damages self-concept
% •	exponential looping effects on a systems level

% •	how this applies to other systems with similar dynamics: NHS, prisons, education, HE
% •	How do we design against these?
% By classifying young people as vulnerable, the type of engagement that can be done with them is limited and reflects the needs of organizations and funders moreso than YP. This results in the organization controlling more and more and becoming more anxious about a lack of control - but other data shows that 'not being in control'/being vulnerable may be key to recovering from some of these problems... Carceral technologies facilitate this, strengthening the ability of organizations to restrict possibilities, imaginaries and actions. The datafication required for the move towards these carceral technologies seeks to classify again; and seeks to move towards ADM systems to restrict the possible negative things that can happen to actors classified as vulnerable.
% Having seen the effects and affects of this new culture of evaluation on youth charities, I shall refer hereafter to evaluation as a technology, with affordances, disaffordances, intended users and functions. Just like any technology, however, evaluation processes are open to user appropriation, and I argue that the intensification of evaluation processes as a result of austerity led to a resurgence in methods of appropriation of them, which I refer to as justification practices.

% •	datafication 
% o	ramifications of classification and performativity on AI, ML, alg gov/ADM
% •	performativity 
% o	fake news, social media/identity management?
% •	control behaviours 
% o	slow cancellation of the future 





%discussion, maybe - point about other places jps are
%Although it is outside of the remit of this thesis, it is worth stepping outside of the third sector for a moment to consider the more general dynamics of justification practices and therefore other places that justification practices may work to stabilise austerity-intensified capitalist realism.  %The clearest examples are that of academisation in the school system and marketisation in Higher Education. In both of these systems, organisations with an ostensibly social function are rendered subject to funding that is contingent upon 'outcomes', with outputs and accreditations guiding their functioning. Consider how academy trusts spend huge sums of money on branding - to get their 'look and feel' right, like our friends at Changemakers. Consider how universities have for years had to engage in the Research Excellence Framework, submitting their work for external approval to verify its impact, and how in research years the Teaching Excellence Framework and Knowledge Excellence Framework have guided university decision-making and their ability to set fees at the highest level. This is much like how organisations who provide services to young people have had to engage in external review - for their funders, for decisionmakers, or at national events such as Benchmarking Forums. Burrows noted the hugely damaging impact of the prioritisation of h-indices as far back as 2012 (citation), and the contemporary academy is marked by impact case studies, 'innovative' partnerships and student teaching surveys governing promotion decisions. We have begun to see the impact of this upon the National Health Service, with maximimum appointment times in GP's surgeries and the metricisation of a practice's outcomes. So too have we seen it in policing (though not a social service), with a time of reduced provision coinciding with an increase in criminalising activity and the introduction of arrest quotas; and with border 'control' agents, one of the most carceral of systems, who are assessed on their ability to deal with (manufactured) threat.

%implications re datafication
%Justification practices are stabilising forces, attempts to stay afloat in an increasingly precarious world. They are an attempt to stay afloat by generating things which are static and appear empirical or universal. Yet the production of these static traces coincides with another cultural trend: the move towards datafication. The organisations engaged in justification practices necessarily engage in datafication as a means of survival, yet there is little consideration for the security, privacy or onward trajectory of the data they create. Why does that funder want you to create a database of clients with a tracker of their outcomes over time? Why do they want you to document your engagement methods as if you were writing a long and detailed recipe? Whether intentional or not, justification practices and their associated datafication facilitate a move towards the introduction of digital technologies into the youth and social work space; particularly, the introduction of automated technologies. In some cases these are automated or autonomous decision-making technologies, whilst in others they are machine learning or 'artificially intelligent' systems. What is true in all cases is that outcomes, however crudely recorded, are, in aggregate, being used to teach automated technologies what the care system and other spaces perceived to be 'vulnerable' look like, and how different kinds of practice intervenes in this. Yet justification practices just describe the stabilization work which make all of this possible. What is happening behind the scenes? Who is the 'man behind the curtain'? How do they work? What do they want? What is driving the changes that led to the creation of the Framework for Young People's Outcomes in the first place?



\section{Future research opportunities}



% •	Anticipatory practices/ future(justification practices): change in orientation towards probable futures changes practices in the present
% •	Technologies and techniques of control, what they are and their impacts – and lack of trust
	no change, no possibility of change, disorientation, on their own
	a need to re-engineer the possibility of other, plural futures.