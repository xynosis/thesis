\chapter{Conclusion}
\label{ch:9}

\section{Introduction}
\label{sec:}
In the previous chapter, I detailed the \textit{fractured signals} project. Building on the insights developed in the \emph{It's Our Future} project, the \emph{fractured signals} project supported workers who work with care-experienced young people to identify issues with their contemporary world, work towards the development of critical consciousness, imagine different futures, and begin to identify ways to work towards these futures. I refer to this approach as speculative praxis, and suggest that this can be used as a methodology to support people to mitigate, resist and create alternatives to austerity-intensified capitalist realism through targeted creative deterritorialization. In this chapter, I identify how speculative praxis constitutes a response to the problems posed by austerity-intensified capitalist realism and the justification, classification, and discursive accumulation practices it engenders. I begin this chapter by identifying how this thesis has answered the research questions presented at the outset, and review the thesis' central contributions. I dig deeper into the implications of these contributions and the concepts I have presented, including the applicability of these ideas outside of the space of the third sector/care provision, and suggest some directions for further work in the space. Finally, I present some limitations of this work and of capitalist realism's usefulness as an analytic in our contemporary context.

\section{Research questions}
In chapter \ref{ch:1}, I presented this thesis' research questions. These were:
\begin{itemize}
    \item What experiences and affects does capitalist realism elicit?
    \item How does capitalist realism operate to make these experiences and affects possible (or probable)?
    \item How can the tools and methods of design be used to respond to the challenges presented by capitalist realism?
    \item \end{itemize}
In this section, I will detail the answers that I have provided through this thesis to each of these questions in turn. 

\subsection{What experiences and affects does capitalist realism elicit?}
\label{subsec:9-2-1}
Although each chapter provides some insight on each of these questions, the first research question was primarily addressed in chapters \ref{ch:2} and \ref{ch:5}. In chapter \ref{ch:2}, I began by describing capitalist realism. I provided a richer theoretical basis for what capitalist realism is, how it operates, and its relevance to the sociopolitics of the 2010s through \citet{shonkwiler_reading_2014}’s understanding of the components of capitalist realism. These are capital’s ability to:
\begin{quote}
\begin{itemize}
\item  constantly revolutionize its sources of and strategies for accumulation, developing new configurations of activity;
\item have an economic, social and affective life that has vast consequences for our lived experiences;
\item and to transform this constant change and lived experience into a widely accepted brand of Gramscian 'common sense'.
\end{itemize}
\end{quote}
I explored each of these components of capitalist realism in turn, and argued that austerity is both exemplary of capitalist realism and an intensifier of it. First, I turned towards the economic, social, and affective life of capitalist realism, by detailing the instability created by changing economics. I surveyed the changes brought about to work and labour by Fordism and the advent of the disciplinary society (characterised by enclosure, repetition, and routine), before following the advent of neoliberalism and its transformation of work into post-Fordism and the control society (characterised by flexibility, spontaneity, and instability). I argued that these made possible new sources of and strategies for capital accumulation. Drawing on \citet{marx_capital_1889}, \citet{luxemburg_accumulation_2015}, and \citet{harvey_new_2003}, I described how capital requires the presence or creation of an 'outside' into which it can expand in order to maintain hegemony \citep{gramsci_selections_2007}. I described how \citet{fisher_capitalist_2009} presents this in capitalist realism as capital's ability to "precorporate" resistance to it, creating trends of resistance to capital that are easily incorporable into it. Finally, I presented the way that this transforms into a Gramscian "common sense" through the creation of an affect of reflexive impotence \citep[24]{fisher_capitalist_2009}, in which people recognise that "things are bad" but feel as if they cannot do anything about it. This felt sense of powerlessness leads to a "slow cancellation of the future" \citep[5]{fisher_ghosts_2014}, in which it feels impossible to imagine a future that is any different from our present reality.

I continued to explore capitalist realism with reference to the third sector under the fiscal austerity that emerged in the UK in response to the 2008 financial crisis. Austerity led to huge funding cuts for both public and third-sector organisations, and introduced a deepened conditionality to funding  which made organisations compete with each other to best demonstrate how they provide "value for money" \citep{clifford_charitable_2017, jones_uneven_2016, clayton_distancing_2016}. This engendered a shift in people's experiences, as affective atmospheres of frustration, disappointment and anxiety circulated \citep{hitchen_living_2016} and shaped people's capacities for action. These constituted austerity's economic, social, and affective shifts, and the way that these shifts create a "common sense" around the impossibility of change. I proposed that austerity's accumulative action was therefore in the production of vulnerability itself - that the new sources of accumulation were people who required support services  (which increased as a result of austerity), and the strategy for capital accumulation was through third sector organisations classifying people as vulnerable, limiting their capabilities to the point that their accumulative activity could continue. 

In chapter \ref{ch:5}, I explored the paradigm of third sector organisations under austerity-intensified capitalist realism in practice, through the case of \textit{The Charity} and their three projects Small Steps, Building Bridges, and Seabird, in order to explore the experiences and affects elicited by capitalist realism in a practical context. I conducted an ethnographic study of frontline support workers, managers, and young people perceived to be vulnerable within these projects. I highlighted how the value for money agenda forces support organisations to change their perceived efficacy in relation to work, labour, time, or money. In The Charity, these formed three consistent experiences for workers - there not being enough labour to do the amount of work there is, there not being enough time to do the amount of work there is, or there not being enough money to source a sufficient amount of labour. In this chapter, I focused on only the first of these in the interests of length - what workers' experiences of there being 'too much work' was. I identified how this led to workers doing work that they aren't trained for, experiencing a need for a high amount of emotional labour, and being subject to the control techniques of their managers to ensure they get the work done. I then turned to understanding the experiences of young people perceived to be vulnerable within this, and identified how their agency was disregarded through inadequate forms of listening, and how this leads to young people not receiving the right support, leading to them feeling isolated, confused, and anxious. I identified a level of consistency between the affects experienced by workers, young people, and managers: they felt anxious, confused, distrustful and powerless. Managers and workers struggle to build trusting relationships with each other, whilst workers and young people struggle in a similar way. Managers use control techniques on their workers to ensure work gets done, whilst workers use similar techniques on young people to offer limited support or make them 'engage'. Young people feel isolated, disoriented, "broken down" and powerless, whilst workers feel individualised, anxious, in "panic mode", and unable to leave their jobs. The answer to my first research question is that austerity-intensified capitalist realism elicits experiences that are grounded in the scarcity created by the value for money agenda, and results in the creation of affects centered on separation, confusion, and instability.

\subsection{How does capitalist realism operate to make these experiences and affects possible (or probable)?}
\label{subsec:9-2-2}
I answered my second research question in chapter \ref{ch:6}, where I detailed the mechanics of the systems that sustain austerity-intensified capitalist realism. To answer the question of how capitalist realism makes these experiences and affects possible (or probable), I continued my ethnographic research with The Charity. I identified the call for a "common measurement framework" for organisations engaged in youth service provision \citep{house_of_commons_education_committee_services_2011} as a significant inflection point, as it both set a standard for youth service provision going forward and amplified a set of measurement and evaluation practices that were already circulating at the time. This renewed focus on evaluation led to service provision being more deeply understood in terms of "outcomes" and "outputs" and a greater use of tools and technologies such as the Outcomes Star and theories of change to make sense of programmes. In this chapter, I presented the grounded theory of justification, classification, and discursive accumulation practices as a descriptor of the changes brought about by this renewed focus on evaluation, and as the central mechanism through which austerity-intensified capitalist realism makes the aforementioned experiences and affects probable. Justification practices describe how the work of evaluation (whether through skilled or unskilled workers) becomes focused on the act of justifying the efficacy of the programme, not on the act of authentically evaluating it. Skilled evaluators tell an edited, clarified story of the programme, presenting the work done in its best possible light. Unskilled evaluators rely on data that may have been collected with little consideration of context, questioning, or bias. Both present a justification of the programme's work rather than an effective evaluation of it.

Justification practices occur alongside classification practices, and both represent an attempt to make youth service provision appear closer to a scientific practice. Justification practices describe what changes can be made in a given project and why, whilst classification practices seek to establish who this intervention could be successful for, and what different types of intervention might have in common. Classification practices are the central mechanism through which the figure of the vulnerable subject is produced. Under austerity-intensified capitalist realism, a dual classification system occurs, distinguishing between the types of vulnerable people that exist and the types of intervention that might work to support these different types of vulnerability. Classifications may exist based on age, type of need, or lived experiences, and types of intervention are imagined to be entirely separate even though someone may have overlapping needs or experiences. Classification based on identity (or kinds of lived experiences) is common within the sector, and can lead to young people internalising these classifications, the stigma attached to them, and limiting their own capabilities in turn. 

The professionalisation of youth service provision that has happened as a result of the move towards "value for money" and evaluation has resulted in an increased performativity within the third sector and led to accumulative activity that is discursive in nature. This discursive accumulation can be near-endless, as the surplus value that can be extracted from accumulation is performative, and does not need to directly relate to a material thing. One form of this discursive accumulation is the focus on the construction of 'best practice' through co-production and innovation, despite projects often failing to live up to their ideals of co-production. Despite the performance of being co-productive, The Charity often sidelined co-production in favour of reaching their other organisational objectives. This kind of discursive accumulation - based on the appearance of following a recent practice-based trend within the third sector - can lead an organisation to be considered 'innovative'. Yet the focus on 'innovative' practice is part of the creation of economic, social, and affective instability within capitalist realism itself, as the pull to innovate encourages organisations to continually undertake new kinds of projects that they may have no expertise in, ensuring a looping effect: organisations take on too much work, this leads to a lack of capacity, workers feel overworked, the experiences and affects presented in section \ref{subsec:9-2-1} result, and then the need to innovate (to justify the organisation's practice) returns.  The answer to my second research question is that austerity-intensified capitalist realism makes these experiences and affects possible (or probable) through the move towards the "value for money" agenda and youth service provision centered on the production of evaluations. These lead to an atomisation of youth service provision into discrete outcomes and outputs and evaluations that describe how practice creates these outcomes and outputs, which act as justifications of the programme's practice to funders. In order to establish for whom this practice works, potential participants are classified on the basis of their identities. Finally, these justification and classification practices result in performativity and a focus on discursive accumulation at the expense of high-quality practice. 

\subsection{How can the tools and methods of design be used to respond to the challenges presented by capitalist realism?}




The tools and methods of design can be used to respond to the challenges presented by capitalist realism by...


\section{Contributions}
Broadly, the three central contributions of this thesis align to the answers to the research questions I have just presented. This thesis' three central contributions are: 
%i'm not sure about this first one
\begin{itemize}
    \item A deeper exploration of capitalist realism's theoretical and material basis, providing a richer picture of its impacts that is grounded in ethnographic observation rather than cultural theory,
    \item  A grounded theory (of justification, classification, and discursive accumulation practices) which describes how these experiences are elicited by capitalist realism, and how the third sector/care system has shifted to both create and sustain these experiences, and
    \item A methodology (in the form of speculative praxis) that can address the foreclosure of possibility presented by capitalist realism, through experimental actions which facilitate "ongoing creation of further relations of possibility" 
\end{itemize}



what justification practices mean 
how spec praxis can be used

implications for AI ADM etc

haunting 
time

% My methodology, which made use of ethnography, action-centered methods, and design, acted as 

% Traditionally, service design does [some bad bullshit thing that isn't very considerate of structural issues]. 

% something about commodification or infantilisation of lived experience 

\section{Implications}
% •	Impacts of performativity
% •	Classification (quantification, measurement, standardization)
% •	Datafication
% o	automated systems (AI ML ADM) and implications of JPs etc on them
% •	classification and quantification —> change in anticipatory practices —> performativity, lack of trust —> control behaviours —> extraction —> negative affects
% •	stabilisation of a particular kind of knowledge regime; (looping effects, biopolitics, governmentality which intensify this, and how digital tech ties into it
% •	restriction of possibility: change in self-experience and ‘slow cancellation of the future’ etc
% o	negative self-talk which damages self-concept
% •	exponential looping effects on a systems level

% •	how this applies to other systems with similar dynamics: NHS, prisons, education, HE
% •	How do we design against these?
% By classifying young people as vulnerable, the type of engagement that can be done with them is limited and reflects the needs of organizations and funders moreso than YP. This results in the organization controlling more and more and becoming more anxious about a lack of control - but other data shows that 'not being in control'/being vulnerable may be key to recovering from some of these problems... Carceral technologies facilitate this, strengthening the ability of organizations to restrict possibilities, imaginaries and actions. The datafication required for the move towards these carceral technologies seeks to classify again; and seeks to move towards ADM systems to restrict the possible negative things that can happen to actors classified as vulnerable.
% Having seen the effects and affects of this new culture of evaluation on youth charities, I shall refer hereafter to evaluation as a technology, with affordances, disaffordances, intended users and functions. Just like any technology, however, evaluation processes are open to user appropriation, and I argue that the intensification of evaluation processes as a result of austerity led to a resurgence in methods of appropriation of them, which I refer to as justification practices.

% •	datafication 
% o	ramifications of classification and performativity on AI, ML, alg gov/ADM
% •	performativity 
% o	fake news, social media/identity management?
% •	control behaviours 
% o	slow cancellation of the future 





%discussion, maybe - point about other places jps are
%Although it is outside of the remit of this thesis, it is worth stepping outside of the third sector for a moment to consider the more general dynamics of justification practices and therefore other places that justification practices may work to stabilise austerity-intensified capitalist realism.  %The clearest examples are that of academisation in the school system and marketisation in Higher Education. In both of these systems, organisations with an ostensibly social function are rendered subject to funding that is contingent upon 'outcomes', with outputs and accreditations guiding their functioning. Consider how academy trusts spend huge sums of money on branding - to get their 'look and feel' right, like our friends at Changemakers. Consider how universities have for years had to engage in the Research Excellence Framework, submitting their work for external approval to verify its impact, and how in research years the Teaching Excellence Framework and Knowledge Excellence Framework have guided university decision-making and their ability to set fees at the highest level. This is much like how organisations who provide services to young people have had to engage in external review - for their funders, for decisionmakers, or at national events such as Benchmarking Forums. Burrows noted the hugely damaging impact of the prioritisation of h-indices as far back as 2012 (citation), and the contemporary academy is marked by impact case studies, 'innovative' partnerships and student teaching surveys governing promotion decisions. We have begun to see the impact of this upon the National Health Service, with maximimum appointment times in GP's surgeries and the metricisation of a practice's outcomes. So too have we seen it in policing (though not a social service), with a time of reduced provision coinciding with an increase in criminalising activity and the introduction of arrest quotas; and with border 'control' agents, one of the most carceral of systems, who are assessed on their ability to deal with (manufactured) threat.

%implications re datafication
%Justification practices are stabilising forces, attempts to stay afloat in an increasingly precarious world. They are an attempt to stay afloat by generating things which are static and appear empirical or universal. Yet the production of these static traces coincides with another cultural trend: the move towards datafication. The organisations engaged in justification practices necessarily engage in datafication as a means of survival, yet there is little consideration for the security, privacy or onward trajectory of the data they create. Why does that funder want you to create a database of clients with a tracker of their outcomes over time? Why do they want you to document your engagement methods as if you were writing a long and detailed recipe? Whether intentional or not, justification practices and their associated datafication facilitate a move towards the introduction of digital technologies into the youth and social work space; particularly, the introduction of automated technologies. In some cases these are automated or autonomous decision-making technologies, whilst in others they are machine learning or 'artificially intelligent' systems. What is true in all cases is that outcomes, however crudely recorded, are, in aggregate, being used to teach automated technologies what the care system and other spaces perceived to be 'vulnerable' look like, and how different kinds of practice intervenes in this. Yet justification practices just describe the stabilization work which make all of this possible. What is happening behind the scenes? Who is the 'man behind the curtain'? How do they work? What do they want? What is driving the changes that led to the creation of the Framework for Young People's Outcomes in the first place?



\section{Future work}



% •	Anticipatory practices/ future(justification practices): change in orientation towards probable futures changes practices in the present
% •	Technologies and techniques of control, what they are and their impacts – and lack of trust
	no change, no possibility of change, disorientation, on their own
	a need to re-engineer the possibility of other, plural futures.

 SPEX PRAXIS AS ALT TO SERVICE DESIGN