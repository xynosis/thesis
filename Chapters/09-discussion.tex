\chapter{Conclusion}
\label{ch:9}

\section{Introduction}
\label{sec:}
In the previous chapter, I detailed the \textit{fractured signals} project. Building on the insights developed in the \emph{It's Our Future} project, the \emph{fractured signals} project supported workers who work with care-experienced young people to identify issues with their contemporary world, work towards the development of critical consciousness, imagine different futures, and begin to identify ways to work towards these futures. I refer to this approach as speculative praxis, and suggest that this can be used as a methodology to support people to mitigate, resist and create alternatives to austerity-intensified capitalist realism through targeted creative deterritorialization. In this chapter, I identify how speculative praxis constitutes a response to the problems posed by austerity-intensified capitalist realism and the justification, classification, and discursive accumulation practices it engenders. I begin this chapter by identifying how this thesis has answered the research questions presented at the outset, and review the thesis' central contributions. I dig deeper into the implications of these contributions and the concepts I have presented, including the applicability of these ideas outside of the space of the third sector/care provision, and suggest some directions for further work in the space. Finally, I present some limitations of this work and of capitalist realism's usefulness as an analytic in our contemporary context.

\section{Research questions and contributions to knowledge}
In chapter \ref{ch:1}, I presented this thesis' research questions. These were:
\begin{itemize}
    \item What experiences and affects does capitalist realism elicit?
    \item How does capitalist realism operate to make these experiences and affects possible (or probable)?
    \item How can the tools and methods of design be used to respond to the challenges presented by capitalist realism?
    \item \end{itemize}
In this section, I will detail the answers that I have provided through this thesis to each of these questions in turn, and the contributions to knowledge that I have made.

\subsection{What experiences and affects does capitalist realism elicit?}
\label{subsec:9-2-1}
Although each chapter provides some insight on each of these questions, the first research question was primarily addressed in chapters \ref{ch:2} and \ref{ch:5}. In chapter \ref{ch:2}, I began by describing capitalist realism. I provided a richer theoretical basis for what capitalist realism is, how it operates, and its relevance to the sociopolitics of the 2010s through \citet{shonkwiler_reading_2014}’s understanding of the components of capitalist realism. These are capital’s ability to:
\begin{quote}
\begin{itemize}
\item  constantly revolutionize its sources of and strategies for accumulation, developing new configurations of activity;
\item have an economic, social and affective life that has vast consequences for our lived experiences;
\item and to transform this constant change and lived experience into a widely accepted brand of Gramscian 'common sense'.
\end{itemize}
\end{quote}
I explored each of these components of capitalist realism in turn, and argued that austerity is both exemplary of capitalist realism and an intensifier of it. First, I turned towards the economic, social, and affective life of capitalist realism, by detailing the instability created by changing economics. I surveyed the changes brought about to work and labour by Fordism and the advent of the disciplinary society (characterised by enclosure, repetition, and routine), before following the advent of neoliberalism and its transformation of work into post-Fordism and the control society (characterised by flexibility, spontaneity, and instability). I argued that these made possible new sources of and strategies for capital accumulation. Drawing on \citet{marx_capital_1889}, \citet{luxemburg_accumulation_2015}, and \citet{harvey_new_2003}, I described how capital requires the presence or creation of an 'outside' into which it can expand in order to maintain hegemony \citep{gramsci_selections_2007}. I described how \citet{fisher_capitalist_2009} presents this in capitalist realism as capital's ability to "precorporate" resistance to it, creating trends of resistance to capital that are easily incorporable into it. Finally, I presented the way that this transforms into a Gramscian "common sense" through the creation of an affect of reflexive impotence \citep[24]{fisher_capitalist_2009}, in which people recognise that "things are bad" but feel as if they cannot do anything about it. This felt sense of powerlessness leads to a "slow cancellation of the future" \citep[5]{fisher_ghosts_2014}, in which it feels impossible to imagine a future that is any different from our present reality.

I continued to explore capitalist realism with reference to the third sector under the fiscal austerity that emerged in the UK in response to the 2008 financial crisis. Austerity led to huge funding cuts for both public and third-sector organisations, and introduced a deepened conditionality to funding  which made organisations compete with each other to best demonstrate how they provide "value for money" \citep{clifford_charitable_2017, jones_uneven_2016, clayton_distancing_2016}. This engendered a shift in people's experiences, as affective atmospheres of frustration, disappointment and anxiety circulated \citep{hitchen_living_2016} and shaped people's capacities for action. These constituted austerity's economic, social, and affective shifts, and the way that these shifts create a "common sense" around the impossibility of change. I proposed that austerity's accumulative action was therefore in the production of vulnerability itself - that the new sources of accumulation were people who required support services  (which increased as a result of austerity), and the strategy for capital accumulation was through third sector organisations classifying people as vulnerable, limiting their capabilities to the point that their accumulative activity could continue. 

In chapter \ref{ch:5}, I explored the paradigm of third sector organisations under austerity-intensified capitalist realism in practice, through the case of \textit{The Charity} and their three projects Small Steps, Building Bridges, and Seabird, in order to explore the experiences and affects elicited by capitalist realism in a practical context. I conducted an ethnographic study of frontline support workers, managers, and young people perceived to be vulnerable within these projects. I highlighted how the value for money agenda forces support organisations to change their perceived efficacy in relation to work, labour, time, or money. In The Charity, these formed three consistent experiences for workers - there not being enough labour to do the amount of work there is, there not being enough time to do the amount of work there is, or there not being enough money to source a sufficient amount of labour. In this chapter, I focused on only the first of these in the interests of length - what workers' experiences of there being 'too much work' was. I identified how this led to workers doing work that they aren't trained for, experiencing a need for a high amount of emotional labour, and being subject to the control techniques of their managers to ensure they get the work done. I then turned to understanding the experiences of young people perceived to be vulnerable within this, and identified how their agency was disregarded through inadequate forms of listening, and how this leads to young people not receiving the right support, leading to them feeling isolated, confused, and anxious. I identified a level of consistency between the affects experienced by workers, young people, and managers: they felt anxious, confused, distrustful and powerless. Managers and workers struggle to build trusting relationships with each other, whilst workers and young people struggle in a similar way. Managers use control techniques on their workers to ensure work gets done, whilst workers use similar techniques on young people to offer limited support or make them 'engage'. Young people feel isolated, disoriented, "broken down" and powerless, whilst workers feel individualised, anxious, in "panic mode", and unable to leave their jobs. The answer to my first research question is that austerity-intensified capitalist realism elicits experiences that are grounded in the scarcity created by the value for money agenda, and results in the creation of affects centered on separation, confusion, and instability.

The contribution that I have made through answering this research question is a deeper exploration of capitalist realism's theoretical and material basis, providing a richer picture of its impacts that is grounded in ethnographic observation rather than cultural theory. Capitalist realism is a useful lens through which to understand the sociopolitics of the late 2000s and the 2010s, and its initial grounding in cultural theory and anecdotal observation in the work of \citet{fisher_capitalist_2009} has led to a dearth of its use within more ethnographic academic traditions such as geography or sociology. By more deeply theorising about \textit{how} capitalist realism came to dominance and collecting primary data about the experiences of people living under capitalist realism, there is an opportunity for it to become a more prominent analytic through which to view this period of time. Although a good amount of work already exists drawing upon capitalist realism as an analytic of media and cultural outputs, such as advertising \citep{schudson_advertising_2013}, movies, \citep{flisfeder_love_2017}, literature \cite{jr_for_2022}, and television \citep{la_berge_capitalist_2010}, there is comparatively little published on capitalist realism as it pertains to social policy, service provision, or austerity. The insights that I have provided in this thesis on  capitalist realism's intellectual ancestry and the experiences it engenders suggest a wider applicability of capitalist realism as an analytic. It also helps to provide a unifying frame for much research that may otherwise appear distinct - for example, research about the experiences of workers responding to an economic change (in the form of capitalist realism) has a direct relationship to research about young people receiving poor support services or finding it difficult to imagine a future for themselves.  

\subsection{How does capitalist realism operate to make these experiences and affects possible (or probable)?}
\label{subsec:9-2-2}
I answered my second research question in chapter \ref{ch:6}, where I detailed the mechanics of the systems that sustain austerity-intensified capitalist realism. To answer the question of how capitalist realism makes these experiences and affects possible (or probable), I continued my ethnographic research with The Charity. I identified the call for a "common measurement framework" for organisations engaged in youth service provision \citep{house_of_commons_education_committee_services_2011} as a significant inflection point, as it both set a standard for youth service provision going forward and amplified a set of measurement and evaluation practices that were already circulating at the time. This renewed focus on evaluation led to service provision being more deeply understood in terms of "outcomes" and "outputs" and a greater use of tools and technologies such as the Outcomes Star and theories of change to make sense of programmes. In this chapter, I presented the grounded theory of justification, classification, and discursive accumulation practices as a descriptor of the changes brought about by this renewed focus on evaluation, and as the central mechanism through which austerity-intensified capitalist realism makes the aforementioned experiences and affects probable. Justification practices describe how the work of evaluation (whether through skilled or unskilled workers) becomes focused on the act of justifying the efficacy of the programme, not on the act of authentically evaluating it. Skilled evaluators tell an edited, clarified story of the programme, presenting the work done in its best possible light. Unskilled evaluators rely on data that may have been collected with little consideration of context, questioning, or bias. Both present a justification of the programme's work rather than an effective evaluation of it.

Justification practices occur alongside classification practices, and both represent an attempt to make youth service provision appear closer to a scientific practice. Justification practices describe what changes can be made in a given project and why, whilst classification practices seek to establish who this intervention could be successful for, and what different types of intervention might have in common. Classification practices are the central mechanism through which the figure of the vulnerable subject is produced. Under austerity-intensified capitalist realism, a dual classification system occurs, distinguishing between the types of vulnerable people that exist and the types of intervention that might work to support these different types of vulnerability. Classifications may exist based on age, type of need, or lived experiences, and types of intervention are imagined to be entirely separate even though someone may have overlapping needs or experiences. Classification based on identity (or kinds of lived experiences) is common within the sector, and can lead to young people internalising these classifications, the stigma attached to them, and limiting their own capabilities in turn. 

The professionalisation of youth service provision that has happened as a result of the move towards "value for money" and evaluation has resulted in an increased performativity within the third sector and led to accumulative activity that is discursive in nature. This discursive accumulation can be near-endless, as the surplus value that can be extracted from accumulation is performative, and does not need to directly relate to a material thing. One form of this discursive accumulation is the focus on the construction of 'best practice' through co-production and innovation, despite projects often failing to live up to their ideals of co-production. Despite the performance of being co-productive, The Charity often sidelined co-production in favour of reaching their other organisational objectives. This kind of discursive accumulation - based on the appearance of following a recent practice-based trend within the third sector - can lead an organisation to be considered 'innovative'. Yet the focus on 'innovative' practice is part of the creation of economic, social, and affective instability within capitalist realism itself, as the pull to innovate encourages organisations to continually undertake new kinds of projects that they may have no expertise in, ensuring a looping effect: organisations take on too much work, this leads to a lack of capacity, workers feel overworked, the experiences and affects presented in section \ref{subsec:9-2-1} result, and then the need to innovate (to justify the organisation's practice) returns.  The answer to my second research question is that austerity-intensified capitalist realism makes these experiences and affects possible (or probable) through the move towards the "value for money" agenda and youth service provision centered on the production of evaluations. These lead to an atomisation of youth service provision into discrete outcomes and outputs and evaluations that describe how practice creates these outcomes and outputs, which act as justifications of the programme's practice to funders. In order to establish for whom this practice works, potential participants are classified on the basis of their identities. Finally, these justification and classification practices result in performativity and a focus on discursive accumulation at the expense of high-quality practice. 

The contribution that I have made through answering this research question is a grounded theory which describes how these negative experiences are elicited by capitalist realism, and how the third sector has shifted to both create and sustain these experiences. Justification, classification, and discursive accumulation practices each bear relevance to work that is focused on the financialisation of the third sector more generally, and particularly of the application of reforms based on evaluation, "value for money", and new public management ideas within the third sector \citep{mcmullin_challenging_2021}. These reforms focused on creating competition in the hope of creating efficiency and a high level of performance in the third sector, and evaluating this performance through a process that purports to be value-neutral, objective, and self-evidently good. This is based on a construction of evidence and evaluation that imagines already-established social problems waiting in service of optimal solutions, rather than understanding these problems to be actively and socially constructed \citep{greenhalgh_evidence_2009}. My work and the new materialist approaches to knowledge (such as Karen Barad's agential realism \citep[45]{barad_meeting_2007}) that I have drawn upon can help to foreground these evaluation practices as being about the subjective creation of representations of reality rather than a direct reflection of it. My research takes this new materialist premise (of evidence being subjectively produced and entangled in the networks of power that create it) and contributes research through direct ethnographic observation that shows just how contingent and subjectively-produced these evaluations are. 

\subsection{How can the tools and methods of design be used to respond to the challenges presented by capitalist realism?}
I answered my third research question in chapters \ref{ch:7} and \ref{ch:8}, in which I described two design responses to the challenges presented by capitalist realism. In chapter \ref{ch:7}, I identified what a design response against austerity-intensified capitalist realism would have to be composed of. For workers, I suggested that a successful design response might reduce workloads, support them to do work that they are skilled at, or provide support for workers to build more trusting relationships with managers. For young people perceived to be vulnerable, I proposed that a successful design response might prioritise their agency, listen authentically and follow through on what has been heard, or ensure that they receive the right support. Additionally, I suggested that for these design responses to be successful, they would have to disrupt the functioning of justification, classification, or discursive accumulation practices, or would have to find a way to work around these. As such, a design response to austerity-intensified capitalist realism must find ways to act against the prioritisation of evaluation, the atomisation of projects into outcomes and outputs, the stratification of support based on discrete classifications, or the production of 'best practice' through discursive accumulation. Above all else, a design response should address the "slow cancellation of the future \citep[5]{fisher_ghosts_2014} engendered by capitalist realism and create a sense of possibility about the potential of different futures.

In chapter \ref{ch:7}, I went on to describe the \textit{It's Our Future} project, which was designed in conjunction with The Charity. I described the specific constraints upon the project and the limitations this placed upon the exploration of some of these methods - centrally, the re-presentation of justification practices, and the transformation of the project from a digital intervention to an in-person one. I presented the method set that I designed, which focused on the use of idea, summary, and scenario cards, as part of a playful facilitation aid. The methods focused on understanding what would transform young people's lives,  what potential histories might have informed the way that things are right now, to envision and dream a possible future in which those changes have been made, and to identify practical ways of how to build these in the present. The methods were effective at helping young people to develop critical consciousness by reflecting on their lived experiences, but required a greater emphasis on skills, resources, and  creating a sense of possibility about the future, as the dreaming and building the future activities were more limited in scope than they had originally intended to be. Finally, the project was ultimately subsumed by justification, classification and discursive accumulation practices as The Charity continued to center themselves and the image of them as sector leaders and the guardian of the voice of young people perceived to be vulnerable.

After the project, I reflected on these methods and refined them further in a document called \textit{Design Strategies against Justification Practices}, which separated the methods we had used in the project into eight strategies for identifying problems and making alternatives, and four principles for doing this work well. I used this as the starting point for the \textit{fractured signals} project. In contrast to \textit{It's Our Future}'s focus on young people, \textit{fractured signals} focused on people who worked with young people perceived to be vulnerable, and supporting them to reflect on their participation and co-production practices in view of the upcoming \textit{Independent Review of Children's Social Care for England} in the context of the future-focused method set I had developed. Drawing on the ideas of speculative enactment \citep{elsden_speculative_2017} and diegetic prototypes \citep{kirby_future_2010}, I proceeded to create a number of objects that would draw participants into the fiction of the project and more deeply immerse them in the potential futures they were envisioning. \textit{fractured signals} helped participants to develop critical consciousness in the workplace, envision futures that center alternative values, and helped workers to center the experiences of young people perceived to be vulnerable.  In an attempt to understand how \textit{It's Our Future} and \textit{fractured signals} worked as design responses to austerity-intensified capitalist realism, I proposed that the methodology used to create them can be considered to be a speculative praxis. This speculative praxis focuses on bringing together speculative design practice and activist praxis. Speculative design practice creates a frame of activity (centered on a possible future, past, or alternate world) that in turn engenders novel possibilities for action and reflection. When deployed in service of an activist cause (i.e. attempting to act outside of the hegemony of austerity-intensified capitalist realism), the experimental actions and reflections participants make can lead to the development of new ideas or capabilities around social change. Speculative praxis is how the tools and methods of design can be used to respond to the challenges presented by capitalist realism.

The contribution that I have made through answering this research question is a methodology (in the form of speculative praxis) that can help to address the foreclosure of possibility that is presented by capitalist realism through experimental actions that facilitate "ongoing creation of further relations of possibility" \citep[320]{harrison_future_2020}. My research contributes to work within the critical and speculative design community by more deeply situating critical and speculative design work within paradigms centered on activism and social change. Rather than merely creating individual objects that provoke reflection on possible futures \citep{auger_speculative_2013, dunne_design_2001, dunne_speculative_2013}, I draw on the notion of speculative enactment \citep{elsden_speculative_2017}, speculative interventions \citep{disalvo_irony_2016} and diegetic prototypes \citep{kirby_future_2010} to place speculation within a semi-fictional narrative frame that causes participants to take seriously the capabilities they embody within the frame of the speculation and connect these back to their everyday lives. In essence, the contribution that I have made here consists of creating ways for participants to mentally visit spaces of potential and possibility and take some of this possibility back into their everyday lives with them. Additionally, I have also provided an alternative basis for work centered on speculation with reference to Deleuze and Guattari's ideas of deterritorialization and (re)territorialization. Understanding speculation through deterritorialization and (re)territorializaton helps to place capitalist realism's hegemony in context. Capitalist realism is able to retain hegemony through its "generalised decoding of flows" \citep[153]{deleuze_anti-oedipus:_1983} and the (re)territorialization of capitalistic values which engender commodification and alienation. Speculative activity works in the same way, with different values entering during the (re)territorialization. As such, my research suggests that capital is able to maintain its hegemony as it is a widespread speculation that is no longer believed to be a speculation. Speculative praxis suggests that alternative speculations could come to prominence, weakening or disrupting capital's hegemony. 

At the conclusion of Fisher's \textit{Capitalist Realism}, he proposes that the hegemony of capitalist realism means that even small acts that create a sense of novelty can have disproportionately large effects:
\begin{quote}
The very oppressive pervasiveness of capitalist realism means that even glimmers of alternative political and economic possibilities can have a disproportionately great effect. The tiniest event can tear a hole in the grey curtain of reaction which has marked the horizons of possibility under capitalist realism. From a situation in which nothing can happen, suddenly anything is possible again.    
\end{quote}
In this way, one of the central contributions of this thesis has been a deeper exploration of how to actively create these glimmers of alternative possibilities, to "tear a hole in the grey curtain of reaction", and to create new horizons of possibility. This thesis has acted as a holistic exploration of capitalist realism, understanding more deeply what it is, how it works, and what can be done about it.

\section{Limitations}

Although I have strived throughout this research to continually adjust my methodology and data analysis processes to attempt to minimise the limitations of this work, there remain several that are relevant to mention at this stage. A central limitation of this work is that though I have been exploring the impact of austerity on the provision of support services for young people perceived to be vulnerable, it was outside of the scope of this work to conduct direct historical work around the advent of austerity inside of these organisations. As such, there are few insights that are specifically around how post-austerity youth and social work was different to pre-austerity youth and social work. In some places, this did emerge in interviews with participants that have worked in the sector for a long time. This was less significant in my work than it could have been due to my understanding of how these issues build over time. As explored in chapter \ref{ch:2}, the ideas and forces that led to capitalist realism and evaluation-centric youth and social work did not appear out of nowhere in response to the financial crisis and the advent of austerity - these merely amplified existing ideas and social forces. As such, the fact that I do not focus on pre-austerity youth and social work (or even youth and social work during 'early austerity') is less significant as I am focused on my participants experiences in the present as they related to these forces. 

%Though I developed my methodology over time throughout the project, I remained relatively confident from the outset that this project would be focused on qualitative analysis of ethnographic, action-centred, and design-oriented work. I scoped a project that would be suitable for the use of this methodology, but it is possible that by letting my methodology guide the work in this way, I may have missed some opportunities for the research to take a divergent direction that would have focused 
%* qual work - not represenative, not generalisable, 'small sample size'
%* theory building tapered direction

Throughout my research, it has been difficult to maintain relationships with organisations who are facing their own set of concerns alongside creating a space for my PhD research. As such, there were likely some 'missed opportunities' for the research that could have also yielded generative data around the functioning of capitalist realism and how best to support these kinds of organisations. A notable actor here was Small Steps, who were forced to close during my research due to lack of funding. No doubt, a closer view of the inside of an organisation who is being forced to close due to lack of funding would have been incredibly insightful for understanding how capitalist realism works. Due to the messiness and complexity of the situation (with Small Steps wanting more and more of my time to help avoid the closure of the organisation), I had to step away from my engagement with them in order to not get overwhelmed and continue paying attention to the other organisations I was working with. If this project was being conducted by a wider team, this would have been a brilliant - if difficult, and sad - opportunity to understand the full trajectory of how capitalist realism can affect organisations providing support services. Similarly, it is possible that my relationships with participants in this project and my perceived power as a researcher affected the kind of data I was able to gather. I developed close relationships with many staff members throughout this project, and it is possible that they were more willing to tell me about critical issues due to this relationship. Similarly, I was perceived positively by the younger participants of this project, who generally felt that they could trust and rely on me. I remained critical about the power I exerted throughout, both with staff and young people. I continually re-established consent during and after times that more critical or sensitive issues were spoken about, and some of the more critical or sensitive things that emerged during my research have been omitted out of respect for participants even though they were willing for that information to feature.

I began this research in 2018 and finished writing my thesis in 2023. Particularly in 2020 and 2021 (critical years for the development of this work), my research plans had to shift significantly due to the advent of the COVID-19 pandemic and the associated public health measures that were taken to protect people. Practically, this meant that follow-up work to \textit{It's Our Future} did not take place in the way that I originally intended, and the follow-up project that did take place (\textit{fractured signals}) had to do so remotely. This did not lead to a drastic transformation of the research itself, as I believe the outcome of the work is still broadly similar to what it would have been otherwise, but it is important to note that this disruption took place. It is fitting perhaps, considering the nature of \textit{fractured signals} (being concerned with possible futures and alternate timelines), that there is an alternate timeline in which the project would not have taken place.

Finally, the central limitation of this project concerns capitalist realism's usefulness as an ongoing analytic. As I have mentioned throughout this chapter, capitalist realism is certainly a useful analytic for understanding the late 2000s and the 2010s, and the way that social forces shaped people's capabilities during this time. Due to the advent of the COVID-19 pandemic, the lockdown measures that followed, a worsening lack of trust in government experienced by many due to government mismanagement during the pandemic, and a deepening of carceral, surveillance-centered, and capability-limiting systems, it is worth questioning whether capitalist realism \textit{remains} as useful as a way to understanding our contemporary sociopolitics. Certainly, capitalist realism at least remains present as part of a resonance machine with many of these new, more prominent forces, (as expressed in chapter \ref{ch:2} about neoliberalism), but it may not remain the central structuring force as it perhaps was throughout the 2010s. Future work should explore the ongoing relevance of capitalist realism as an analytic. Despite this, however, speculative praxis should remain a powerful methodology for counteracting whatever the dominant capability/possibility-limiting power dynamic of the day is.

\section{Implications and possibilities for future work}
There are a number of implications of the research presented in this thesis which I have not focused on in any depth throughout to retain focus on the central line of argument around what capitalist realism is, how it operates, and how design can be used to work against it. In this section, I present a number of implications of this research, alongside a sense of future work that could be undertaken to understand these issues more deeply. 

\subsection{Where else do these dynamics exist?}
It occurred to me relatively early on within my research that the insights that I was generating through my ethnography of The Charity and grounded theory of justification, classification, and discursive accumulation practices were highly applicable to areas beyond my focus. In this thesis, I have exclusively focused on a set of organisations working within the third sector to provide support services to children and young people perceived to be vulnerable. Yet looking at these dynamics more abstractly, it is possible to see areas where these dynamics might also exist, and where an analysis of capitalist realism in this space may also be fruitful. Considering these dynamics in abstract briefly, justification practices can be seen in any system where: 
\begin{itemize}
    \item funding is made conditional and competitive,
    \item the conditionality of that funding rests upon the measurement or evaluation of some aspect of a practice, and
    \item the conditionality of funding removes possibilities from actors, thus creating a situation in which ensuring a positive evaluation becomes integral. 
\end{itemize}
This has been a pattern within social policy in the United Kingdom over the past thirteen years under the governance of the Conservative Party.  The academisation of schools in England comes to a mind as a clear example. Under the academies policy (brought in by a Labour government in the year 2000 but intensified by the Conservative governments of the last thirteen years), schools may divest themselves from local authority control and funding and gain "independence" to operate and spend their funding how they choose as long as they are broadly in line with the National Curriculum. All schools in England are assessed by Ofsted (the Office for Standards in Education, Children's Services and Skills), who inspect educational providers and assess their "quality" according to a number of criteria. Although academies are notionally independent, many closely follow national educational guidelines due to a fear of veering too far away from Ofsted criteria (who have the power to make sweeping changes if they deem the school to require improvement due to perceived low standards of teaching, facility provision, or educational attainment). \citet{kauko_evaluation_2015} notes how in fear of poor Ofsted outcomes, school managers institute frequent observations of teachers - as one teacher noted, "if you're not good or oustanding, they are after your life". Whilst these dynamics are not identical to what I have presented from the case of the third sector, they are exceedingly similar. In this case, whilst funding is not directly conditional, the ability to run a school freely is. As such, managers bring in control techniques designed to limit the freedom of teachers and make them operate as closely as possible to national policy/Ofsted standards, which in turn may limit teachers' abilities to work with students in the way that is most useful for them. There is a greater focus on the production of a positive evaluation (through Ofsted inspections) or positive outcomes (through high grades) than there is on the freedom to use educational practice in the way that is most appropriate and useful for students and teachers. An exploration of the specific dynamics of capitalist realism and how justification, classification, and discursive accumulation practices manifest in primary and secondary education would be fruitful for understanding another part of the system that ostensibly supports young people. Similarly, an understanding of what speculative praxis might mean or require in the primary and secondary education space would be valuable.

It would also be generative to explore the impact of capitalist realism with reference to the marketisation of Higher Education. This is perhaps the closest analogue to that of the third sector. \citet[355]{burrows_living_2012} identified how the academy was becoming increasingly metricised and falling under "quantified control" at exactly the same time that academics were beginning to feel more frequently "exhausted, stressed, overloaded, suffering from insomnia, feeling anxious, experiencing feelings of shame, aggression, hurt, [and] guilt". In that paper, Burrows identifies the increasing volume of measurement technologies that are being applied to the Higher Education sector - citation metrics (such as the h-index), research quality assessment (such as the Research Assessment Exercise or the Research Excellence Framework), teaching quality assessment (in the form of the National Student Survey, university league tables). Since this paper, teaching quality is also assessed through the Teaching Excellence Framework (with a direct link to tuition fee funding), and the applicability of academic knowledge to wider society and business through the Knowledge Exchange Framework. The increasing metricisation of academic knowledge production, of its "impact" and "public engagement", is emblematic of the same paradigm that leads to the undertaking of evaluation within the third sector. It is clear also that similar management control practices (through inadequate workload models) and affects are circulating. As such, it would be useful to understand the current landscape of the measurement of academic knowledge production through the lens of capitalist realism, justification, classification, and discursive accumulation. These same forces may also exist in other areas of social policy that have been impacted similarly after the past decade or so - such as in the National Health Service, through the introduction of maximum appointment times and the metricisation of practice outcomes; or through an intensified criminalisation of everyday life through police arrest quotas. 

\subsection{What are the impacts of the contingency of evaluation data?}

Data and technologies
% By classifying young people as vulnerable, the type of engagement that can be done with them is limited and reflects the needs of organizations and funders moreso than YP. This results in the organization controlling more and more and becoming more anxious about a lack of control - but other data shows that 'not being in control'/being vulnerable may be key to recovering from some of these problems... Carceral technologies facilitate this, strengthening the ability of organizations to restrict possibilities, imaginaries and actions. The datafication required for the move towards these carceral technologies seeks to classify again; and seeks to move towards ADM systems to restrict the possible negative things that can happen to actors classified as vulnerable.
% •	datafication 
% o	ramifications of classification and performativity on AI, ML, alg gov/ADM
% •	control behaviours 
% o	automated systems (AI ML ADM) and implications of JPs etc on them
% •	classification and quantification —> change in anticipatory practices —> performativity, lack of trust —> control behaviours —> extraction —> negative affects
% •	stabilisation of a particular kind of knowledge regime; (looping effects, biopolitics, governmentality which intensify this, and how digital tech ties into it
% •	restriction of possibility: change in self-experience and ‘slow cancellation of the future’ etc
% o	negative self-talk which damages self-concept
% •	exponential looping effects on a systems level


%Justification practices are stabilising forces, attempts to stay afloat in an increasingly precarious world. They are an attempt to stay afloat by generating things which are static and appear empirical or universal. Yet the production of these static traces coincides with another cultural trend: the move towards datafication. The organisations engaged in justification practices necessarily engage in datafication as a means of survival, yet there is little consideration for the security, privacy or onward trajectory of the data they create. Why does that funder want you to create a database of clients with a tracker of their outcomes over time? Why do they want you to document your engagement methods as if you were writing a long and detailed recipe? Whether intentional or not, justification practices and their associated datafication facilitate a move towards the introduction of digital technologies into the youth and social work space; particularly, the introduction of automated technologies. In some cases these are automated or autonomous decision-making technologies, whilst in others they are machine learning or 'artificially intelligent' systems. What is true in all cases is that outcomes, however crudely recorded, are, in aggregate, being used to teach automated technologies what the care system and other spaces perceived to be 'vulnerable' look like, and how different kinds of practice intervenes in this. Yet justification practices just describe the stabilization work which make all of this possible. What is happening behind the scenes? Who is the 'man behind the curtain'? How do they work? What do they want? What is driving the changes that led to the creation of the Framework for Young People's Outcomes in the first place?

\subsection{How do speculations become hegemonic?}


Discourses of innovation and projections of the future

% Traditionally, service design does [some bad bullshit thing that isn't very considerate of structural issues]. 

What the potentials for a sustained speculative praxis are...
    fractals co-op

0


 \section{Concluding remarks}

 This thesis has

