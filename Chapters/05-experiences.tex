\chapter{Living (and working) through austerity: everyday life for managers, workers, and care-experienced young people}
\label{5}

\section{Introduction}
\label{sec:}

In chapter 2, I explained that one of the key features of austerity and Cameron's 'Big Society' was an outsourcing of governmental responsibility under the guise of 'freedom' but without any material or economic resources to back this. Charities then increasingly provided what were once public services as a way to 'plug the gap' in provision. It is simple to say 'these changes occurred'. Yet how does social policy change get done at the organisational and individual levels? How the elements of a policy system shift, change and transform as a result of a policy change cannot be predicted. In this chapter, I ask what actually happened because of austerity. In lieu of being present in my research sites at the introduction of austerity policies, I take a retrospective approach, having spoken with managers, workers and young people across the youth charity sector.


From observations, interviews and workshops, I assemble here a phenomenology of life in the care system. What are the lived experiences of those who live and work in the care system? What does everyday look and feel like? What are the affective and material consequences of the introduction of austerity? In this chapter, I identify how the experiences and affects that occur at one level of the care system operate at each level, occurring fractally. The pattern at each level is not the same, but similar dynamics of trust, control, disempowerment, anxiety, burnout, and isolation occur. In this chapter I try to let the voices of those who are a part of the care system speak as much for themselves as possible; I offer my voice as that of a critical narrator. Whilst the structure of experiences put together here is my own, I have assembled it in such a way that I hope it is as reflective of the experiences told to me by the people I have worked with.

% [probably insert diagram here]

\section{Can I speak to the manager?}
\label{}
As the people who spend most of their time deep within the overall care system, the managers and directors of charities and services who provide support young people bear a lot of responsibility. To invert the famous adage from Spiderman's 'Uncle Ben', with this responsibility comes great power, over how their organisation operates and why they operate in their context. Through the course of my work with PseudonymCharity, I worked with a multitude of different managers and directors. Though all of these operated at different levels and had differing levels of overall responsibility, they all occupied a central point in the chain of command of their section of the organisation.

Despite these distinctions, most of these managers exhibited similar patterns of behaviour. They would start their post or their branch of the organisation with a full heart and an earnest belief in the need for the thing they are doing. Then, over time, the realities of running a charity would take them further and further from these goals and mean that they spend longer dealing with management and operations. As they lose sight of their staff and what it means to be engaged in frontline practice, they start to focus more on what is needed to keep the organisation running. Slowly, their values shift: once, they were most interested in ensuring that what is best for young people happens; now, they are most interested in ensuring what is best for young people happens (if it's safe for the future of their organisation). Ultimately, this creates a toxic workplace where expectations are constantly changing and workers find it harder and harder to enact their job roles.

Let's take Michael as an example. He started his role in PseudonymCharity around about the time I began this research. He had joined the organisation from another, similar organisation, where one of his former colleagues told me that he was known for "doing what's right and getting things done". People spoke about Michael with a kind of passion - he knew what was up and he knew what needed to happen. When I first met him, a few months into his role, he still wore the proud enthusiasm of someone who knows that they are doing what's right: he was critical of PseudonymCharity and was keen on doing whatever he could to make sure the bureaucracy didn't get in the way of what young people were telling him needed to happen. He viewed the Building Bridges project, which he championed, as a kind of 'trojan horse', a way to break into the rest of the organisation and transform it. The individual Bridges would variously act as 'local bombs', 'fireworks' or 'explosions' that could set off a chain reaction of change within the organisation. I was captivated by the same passion Michael's former colleagues spoke of: he knew what was up and knew how to get it done.

Over time, though, this started to slip. Early into Building Bridges, Michael assembled a team for a planning workshop at PseudonymCharity's Bristol offices. He thanked everyone for coming, and explained the reasoning behind the meeting:
"I just need to get together some of the content for this program. I want to work with other people on this - you brilliant and amazing people - but it's difficult to collaborate when you've got no time in your diary. I keep putting hour slots in my calendar like 'be creative' but then I get to it and I'm like... ah."

Michael's lofty ideas of collaboration and centering young people's experiences were slowly turning to nothing as he was consumed with the busy realities of a national management job. Trying to plan his creativity in isolated segments was leaving him without any opportunity to collaborate, and though he managed to reach out and form a team around him on this occasion, it wouldn't last. Over the course of the next year, Michael began to struggle trusting anyone other than himself to do things. At Michael's behest, Karen would spend weeks creating content for Building Bridges' residentials, and then a few days before each residential, Michael would tell her that she had missed something vital or that this "wasn't really what he was looking for". What he would inevitably end up telling Karen to do was more like Karen had originally planned, but was told wasn't quite right by Michael. By failing to trust in her capability as a competent and qualified professional, Michael doubled her workload - planning and replanning - and demonstrated his inability to trust anyone to complete tasks on their own.

This lack of trust in the ability of others was matched by a need to control at every turn, most notable near the end of Michael's time at PseudonymCharity. We were reaching the end of the first year of Building Bridges, and we were getting ready to redesign the program for the next cohort. Over the course of a four hour meeting, Karen and I presented our plans for the redesign: we would start as soon as possible, working with current workers and young people, supporting them in redesigning the program and understanding what worked and what didn't. We would complete the redesign by September, setting us up for a [November] start to the program. After we had presented our ideas in their entirety, accompanied by rationales, with little intervention from Michael throughout, it became apparent that he was not going to be swayed. Although he had sat patiently and listened, he had not actually paid attention: he had entered this meeting knowing that he was going to get his desired outcome (not doing a redesign process; beginning recruitment immediately) - but knew that we felt differently, and so had offered a space for us to feel heard without actually being heard.

This space of apparent listening acts as a technique of control, a way that someone with organisational or structural power can enact their power in the face of resistance. Karen and I had discussed this possibility at a break in the meeting, though, so we were able to hold the line slightly stronger. Eventually, Karen agreed with Michael's plan, telling me afterwards that she was worried that he felt that she wasn't doing her job properly and didn't want to come across as less competent or willing to do her work. I held the line slightly longer, explaining how it went against all of the values of the program to not go through a participatory redesign process. The meeting came to a point of increased tension, and Michael's facade slipped. He turned to me, frustrated and flustered:
"Look, I agree with you 100\% Kieran. There would be nothing more that I would like than going through a long redesign process, hearing everything we can from everyone involved and making sure we learn from what we're doing. But I'm getting more and more pressure for some tangible output. They want to see where all this money's going. And if we just - I'm not suggesting it's doing nothing, but they'll see it like that - if we just do nothing for a few months, there are going to be serious and hard-to-answer questions asked to me about what we're doing with all the money they give us."

The spell was broken: Michael had never wanted to not do this redesign process, but was put in the position of blocking it by way of his structural position in the organisation. It was not that he couldn't be swayed, but that he in some way feared the consequences for him if the plan his bosses wanted him to follow wasn't followed. The tension rose higher and higher, and then it dissipated. In this moment of possibility, where all masks, spells, facades and pretenses were dropped, Michael scrambled to re-establish control. To this day, I am sure he saw what happened next as a courtesy to make up for his role in making us do something that we all knew wasn't right for the program - but he suggested we go for a walk in the nearby park together, "to cool off". When I have spoken about this with Karen since, she remembers this as "him taking us for an ice cream in the park, like he was our dad after a big argument or something". Sadly, there was no ice cream, but Karen's impression of the event is somehow closer to the truth of what was happening in that moment. Michael was both trying to make up for his role in the creation of a potential harm, but also trying to re-establish control, to settle our tensions, our nerves, to talk things out. In essence, though he was using his professional training to realise that we needed to decompress, what he actually did was try to quell our resistance.

Through these few moments I have recounted with Michael, we can see the key elements of manegerial existence under austerity: starting with good intentions, experiencing a high degree of pressure in their role, a lack of trust in others which becomes an inability to collaborate with others, and the development of mechanisms, techniques and behaviours of control. These same practices can be seen with Joanne from Small Steps or Michelle from Changemakers. Joanne started her Small Steps branch because she felt there was a lack of specialised youth work services for care-experienced young people; Michelle worked with Changemakers because she saw the withdrawal of government provision for young people with experience of homelessness at the beginning of austerity and was adamant there needed to be some continuity. They both started with the values of co-production at the heart of their work. Over time, Joanne had to withdraw from frontline youth work delivery, to sit on panels and boards of all sorts, advertising the work of Small Steps, trying to spread their practices elsewhere. Michelle, on the other hand, spent so much time trying to secure the existence of Changemakers that the majority of her work became applying opportunistically for funding. This meant that her values of co-production which had previously guided so much of her practice had to become sidelined - she didn't have time for that, Changemakers needs more funding or it will stop existing! Soon, core values begin to operate only at a performative and discursive level - claiming to be participatory but having little practice to support this.

After the good intentions begin to erode, a lack of trust begins to filter through - despite Joanne wanting to share practice, she instead needed to be secretive, guarding certain parts of the organisation's work. Instead of sharing best practice openly, a competitive environment encourages managers to view their work in financial terms. In a project which Small Steps was commissioned to work on (detailed in the next chapter), a local authority manager summed this viewpoint best:
"I seriously do not come from a commercial standpoint at all with this... but it's worth more if people have to pay to see [the film]. We live in a commercial world... if we wanted access to the film and someone else had made it, we would pay for it. And that's the way the world is! And, they would pay. If someone wants to see this film, you and I should be paid for it." [footnote: here 'you' refers to *Small Steps; '*I' refers to the local authority]

According to this manager, 'the way the world is' (read: commercial) demands that people should pay for access to things. This is an extension of this lack of trust - the view that not only can we not trust people, but actually in order to access our good practice, they should pay for it. It goes without saying that this is a heavily financialised viewpoint that would, in recent history, be anathema to youth and social work practice. Yet now 'we live in a commercial world'.

The lack of trust is extended towards the organisation's own workers, too. Early into Changemakers' independence, they hired Sue as their funding officer. Sue was deeply experienced in the local charity funding sector: she had worked for several local charities in this same role and even had links to funders herself, so even had the ability to had grant applications read speculatively by funders before submission. She was a huge asset to the organisation. After having dropped off some documents at Changemakers' offices one afternoon, we ended up in a protracted discussion about the state of the organisation as I hadn't worked closely with them recently. Eventually, Sue shared with me that she hadn't seen a single funding application since her first week working at the organisation:
"It was about something to do with the website, so I said to Michelle, I'll happily have a look at it, but just so you know, I don't know that much about digital technology and it's not my specialty, so you might need to have another look over it after just to check, but I'm happy to redo anything. And so I did this one application - I can't remember where it was for. And then she takes it off of me, and says 'no, it's fine, I'll just do it then' - and I haven't had a single application since. She just does it all herself. And I'm not being funny but I'm good at what I do. I still work part-time for [another organisation] and they have no trouble with their funding. But Michelle just won't let us touch anything because being open and all that, I asked her to double check something on an area I don't know much about in me first week! It's ridiculous."

Without knowledge of how to write grant proposals, Michelle had taken on the labour of searching for and writing grant proposals, despite having hired a worker whose entire skillset is in this area. Yet the idea of trusting someone else to do this organisation-critical work was too scary for Michelle. This is how a lack of trust turns into control behaviours - if managers cannot trust their workers to do anything, then they need to limit their range of potential actions. Do this. Now do that. I know you did this yesterday, but now I want you to do that. Untrusting managers working in a highly precarious space who take on the entire labour of organisational management and development become despotic, commanding and controlling their employees to follow their whims, so they can get on with the work of making the organisation continue to exist. Yet all the while, these managers are becoming further divorced from meaningful frontline practice, abstracted from the lived experiences of those are trying to help. Ultimately, they are scared: imagining a future where their organisation might not exist. As this possible (non)future begins to seem more probable, their control behaviours tighten. Yet as their control behaviours tighten, this (non)future begins to instantiate itself, as workers leave the organisation, expertise goes unused, work goes undelegated, and trust disintegrates.

\section{Work, work, work, work, work}
The changes in practices and behaviour exhibited by managers in the care system have found their way to workers who are not managers in the care system, too. It is not as simple as classifying an individual as a 'manager' or a 'worker'; often, people will occupy different roles with different people. As such, workers may also exhibit the good intentions, high pressure, lack of trust, control behaviours, and commercialism of managers. Yet workers are also subject to a different set of changes in behaviour, practices, and experiences which are a product of existing within the working environment created by managers. These include a culture of deferentialism, dishonesty, and needing to seem busy; an environment that is constantly filled with change but marked by a lack of capacity; and a desire to leave the organisation which is harming them so much but an inability to do so out of concern for the organisation.

To illustrate these three branches, I will (re)introduce you to four of the people I have worked most closely with throughout my research: Beth, Karen, Colin, and Ed. Beth worked in the Small Steps branch; Karen worked as part of Building Bridges; Colin and Ed both worked in Changemakers. All four of these workers operated at least one level underneath management - they might have held some managerial responsibilities, but ultimately they were not 'managers'. Variously, Beth, Karen, Colin and Ed have acted as comrades, co-conspirators, collaborators and colleagues. I have worked with them at the frontline of their practice; and they have worked with me in shaping my research continuously. We have seen each other at our best and worst, and I offer my voice here mostly as a way to elevate and bring to light the commonalities and patterns in their experiences.

The techniques and behaviours of control used by managers elicit a cultural shift within their organisation: workers begin to act differently. A team that was once united on all fronts slowly starts to erode. Concerns about the way a project was operating stay unvoiced as workers become less honest with their managers and more afraid of negative outcomes if they express dissent. In turn, the workers begin to learn to be deferential and obedient. They worry about how their manager may treat them if they act honestly, so they do whatever their manager says - or at least, appear to do so. They know the pressures of the care system, and so they do whatever they can to appear busy at all times - they're already at capacity, and so they will do anything to avoid more work. If you seem busy, you don't get more busy. Yet you also spend so much time seeming busy that you lose time on what really matters - the frontline practice you value so much.

This culture of deferentialism, dishonesty and needing to seem busy is the logical endpoint of a working environment that prevents the building of trust. If managers enlist ever-more sophisticated techniques of control and find themselves unable to be honest, vulnerable or authentic with their workers, then it becomes impossible to build trusting relationships. Towards the end of my time with Changemakers, I interviewed Ed and Colin about how the organisation was operating at the time as I had a feeling that issues with the organisation were getting in the way of the project we were working on. At this point, I had no sensitivity to justification practices, but followed my "ethnographer's nose twitch" (Leigh Star, 20XX), my hunch that something was amiss. Colin laid out Changemakers' current situation with remarkable clarity:

"At the moment, we try to do too much. We need to clarify what we want to achieve. Get a vision, some aims, a goal. We need that. It's also the size of us though... you know, we're tiny. And then if you have issues, you don't feel comfortable explaining them to Michelle. It's just you two. I feel like I can't be honest with her... hell it's even hard to disagree with her."

In the following interview, with no knowledge of what Colin had said, Ed echoed him:

"Really, we're doing too much. We're doing more now than we ever were but with less capacity. Training? That was never really what we were about! We need to understand what we want to do. I don't know if we do the best right now... I don't know how to put into words what I even do. I'd never want to see the organisation go away. I believe in Changemakers. but we need to keep our core values. We're doing too much right now. Michelle thinks she can get everything done - she thinks she knows what she wants. But there's so much pressure... she just agrees to everything - but then we're the ones who have to do it."

Both Ed and Colin felt like Changemakers was doing too much, that they needed to take a step back and clarify what their purpose was, and that the organisation was taking on too much work without any focus - which ended up becoming a burden on Ed and Colin, who felt that they couldn't disagree with her, but who were struggling to get everything done. If a manager like Michelle wants something done, then you agree to it, because it's hard to disagree with her - you don't feel comfortable explaining to her that it's a bad idea, or that you don't have time - you just tell her you'll get it done. Then, when you're struggling to get it done, you don't tell her - you just pretend you're further along than you are, or that something else came up. You start making up increasingly sophisticated justifications. You lie to her. You are constantly busy. Just in case you're not busy enough, you make sure that you're performing your busy-ness. You make a show of just how busy you are.

Deferentialism, dishonesty and the busy culture manifest differently in different projects. In the wider organisation, though, [Sohila] explained that it was "full of people being busy to be seen to be busy". For the central office, this took the form of standing meetings that were never taken out of people's calendars that people were required to attend. In [Sohila]'s words, these meetings were "a fucking waste of time" in which nothing ever got done. In Small Steps, a worker who had been offered an opportunity to take part in a national policy event approached Joanne to ask if he could go. I can only describe [Duncan]'s tone as that of a child asking his mum if he could go to the cinema after school. Joanne replied to him like a naughty child:

"Well, [Duncan], you've already been to [X] things in London so far this year, you need to make sure each thing you go to actually matters."

You may also recall how Michael took Karen and I to the park "like he was our dad after a big argument or something": these cultures necessarily invoke a paternalistic relationship, where any mistake leads to admonishment.

Because managers are constantly changing things about the organisation's function, values and modes of operation, workers have to attempt to exist in a state of constant change. Ed and Colin, for example, felt that the organisation was attempting to do too much, taking on projects that it did not have capacity for or skillset to deliver on. The disconnect between managers and workers mean that projects emerge with little consideration of the workforce's ability to enact it. In the first instance, this can lead to people doing work that they are not trained for or which they do not have time for. In the second instance, this leads to some workers doing everything - having no clear and definable role but doing some of everything. In the last instance, this leads to a profound lack of worker identity - which can make it difficult to focus on what your role actually is, to use these skills in further work, and to seek relevant professional development opportunities.

Mostly these changing experiences of work seem to emerge from sophisticated worker management practices. Managers know that their organisation needs to win a certain funding bid, so it makes a proposal with little consideration of its ability to deliver on the project - that becomes secondary. It then reallocates workers from project to project or task to task in order to make sure that it can fulfil on its proposal. Colin, a trained social worker and employed as a Regional Campaigns Officer, has to write frequent blogs. Beth, an experienced youth worker, is handed control of the entire county's care-experienced participation work. Karen, a trained social worker and experienced youth group facilitator, has to become an evaluator, program manager, travel co-ordinator, supervisor. Ed, a just-graduated marketing student, is forced to become a social media manager, designer, website editor, personal assistant, and receptionist. Workers at Small Steps are moved from project to project with ease. Last month you were trying to support young people to get into employment? Now you're trying to help their mental health. Next month? You'll be trying to get them interested in art galleries and museums. The work goes where the money is.

I have found myself enfolded into these change processes time and time again, even when I have specifically avoided them. Much of the research that I will detail in the following chapters represents a partial enfolding of these processes and my work: they adapt to the priorities of the partner organisation, and then sometimes they adapt and adapt again. As Colin once explained to me, "they'll get every pound of flesh if you let them". Sometimes, these have been small pieces of work - a logo here, a survey there. Other times, they have been significantly larger - like the time I was asked to redesign Changemakers entire mode of service delivery, or to make a theory of change that explained their work and the impact that it hoped to make. These kinds of work go above and beyond what I had expected to do with these organisations, even taking into account my embedded positions with them.

Workers move between projects freely, fulfilling whatever gap in capacity there is. People's relative skillsets do not matter: the work needs doing, so someone has to do it. Everyone becomes a multi-role worker. The workforce remains composed of the same people, hopping around from task to task. In Changemakers this lack of capacity was mostly marked by busy-ness and lack of identity, but at Small Steps it instead took the form of the way funds were used. [Laura]'s job was funded from one project, but she was made to work on two other projects too. In turn, other workers who were funded to work on other projects could share some of the delivery of this one project with [Laura]. People end up doing a 'bit of everything'.

Both of these states of affairs are not necessarily a problem, and represent a creative response to the rigidity of funding criteria. Yet they both contain the possibility of problems. Beth was constantly at breaking point, struggling to meet the demands of multiple projects at once. Both Ed and Colin felt that they couldn't explain what their job role was. This lack of identity made their actual work harder on several occasions, preventing them explaining what they did - or what the organisation did, for that matter. As a matter of course, I asked Colin what he did and what the organisation was about when I first met him. He struggled to answer. I asked him periodically throughout our work together, using it as a proxy for organisational clarity. Once, he attempted an answer, but he wasn't sure if it was right. I met up with him several months after he left Changemakers and he still didn't know how to explain what he did. He felt that this had been part of the problem with the organisation - he didn't even know how to explain what he did, so how could he explain what the organisation did?

A lack of trust and the use of control techniques make workers deferential, dishonest and busy, whilst a lack of clear job description leads to workers doing everything but being unsure what they should be doing. It is no surprise, then, that the pressures of working in this environment makes many workers want to leave their jobs. Yet this desire to leave is arrested by a reluctance to leave out of a sense of responsibility to the organisation or the young people workers are supporting. When I spoke to Ed about the current state of the organisation, he explained that he "really, really want[ed] another job", but was worried about what would happen to Changemakers if he (or Colin) left. Ed was aware that by doing a bit of everything, all of the workers had become essential to the actual functioning of the organisation, even if Michelle didn't realise this. On more than one occasion, Karen had wanted to leave her role, but as the person responsible for planning and delivery of residentials in Building Bridges, she knew that she was the project: if she left, the project would stop happening, and all of the important relationships she had built with young people and workers would fall away. Ed left Changemakers eventually, but he felt deeply conflicted about it to the end. His new role was with another charity in the building, and when he left Michelle joked that he would be down every other day to help them with tech support. Ed nervously laughed, but once he left there was no going back. He couldn't willingly subject himself to that again.

What of those who don't leave? If you're not being trusted, but have a great deal of pressure and expectation put onto you, what happens? You become anxious and paranoid, consumed by the contradictions of wanting to leave but feeling like you should stay. You keep lying to your manager, trying to take everything on yourself. And you might end up like Beth - constantly unable to concentrate because she was always going over the million things she had to do in her head. It is a delicate balance, and it is easy to disturb. One morning, I came in and found Beth flustered, constantly going between her computer, a phone call and other people's offices, trying to do three things at once. I asked her if she was alright, and she just responded "I've gone panic mode", and left the room. I spoke about that day to her after the fact and she explained:

"That morning, I had so much going on, I just couldn't deal with any of it. You know, I got in that morning, I had a cup of coffee and just burst into tears. But hey, some days are just like that".

It had become so normal for Beth to be in this constant state of alertness that it was just another day to her. The cultures created by managers through their lack of trust, techniques of control and high pressure has a profound affective toll. Workers are ground down further and further until they are barely able to function - when going 'panic mode' is an everyday affair. Care work is already a deeply affective labour: it requires a great deal of emotional labour (in the classical sense) (cite) in order to manage the day-to-day happenings and routine ups-and-downs of caring for other people as a profession. Yet the changes brought about by austerity normalise a heightened affective labour, where workers are used to being stressed, pressured and untrusting; wanting to leave their jobs but finding themselves unable to.

\section{Won't somebody think of the children?}
If managers and workers in the care system are struggling so much in this context, what are young people's actual experiences of receiving care under austerity? How have these same processes affected them? In the words of Mary, a care-experienced young person who was a part of Building Bridges, "the whole system is shit". Referring to the whole system as 'shit' is an easy shorthand for identifying the interrelated failures of the care system from the point of view of a care-experienced young person. This is not to say that there are no positive experiences or good moments in the care system: there are certainly tender, loving moments, points of solidarity, genuine care, great workers and transformative times. These moments, however, tend to be exceptions rather than the rule. By virtue of having been taken into care, care-experienced young people are in a situation where they are more susceptible to multiple forms of harm. The support infrastructures contained within the care system fail to help young people work through, process or escape these harms, instead compounding them or creating new harms which take prominence.

I am not suggesting that the young people I have worked with would not have been exposed to harms if it had not been for being taken into care; after all, there were circumstances surrounding their initial removal from their birth families which will have posed some degree of potential harm. I am, however, suggesting that a system which is meant to act as a supportive infrastructure is failing to meet its stated aims. These failures fall along two central categories: disregarding young people's agency, and failing to provide appropriate support. Both of these types of failure have negative impacts on young people's lives.

\subsection{"They think they know everything": disregarding young people's agency}
Care-experienced young people frequently feel like they're not being listened to. This can happen on any scale, from the tiniest interpersonal interaction to the systemic violence of ignoring lived experiences or making poor care decisions. By not listening to care-experienced young people at such a basic level, workers, foster carers and other professionals in a young person's life can easily set them up for harmful experiences and negative affects. Systemically refusing to listen to care-experienced young people is the root betrayal at the heart of a lot of the issues with the care system. Why would you trust someone who doesn't listen to you? Why would you share parts of your life with someone who never follows the course of action you want them to, or explains why that might not be possible? Why build a relationship with someone who will treat you like a natural resource to be extracted from?

The damage of not being listened to is not just in the act of listening, but of what this has the capacity to do to a young person's sense of agency. By consistently disrespecting a young person's efforts to assert themselves, professionals in the care system work against some of the positive outcomes they might be trying to work towards with a young person. Workers may be 'talking the talk' and attempting to get young people to listen to others, be more patient and empathetic, and understand and talk through their emotions whilst also not 'walking the walk' themselves, failing to model what good listening might look like and what it means to respect the agency of others. I focus here on three ways that professionals fail to model good listening:

•	by not paying attention to what young people have said,
•	by listening extractively, stopping a young person when they have 'got what they needed', and
•	by failing to follow through on agreed-upon actions.

\subsubsection{Not paying attention}
The most basic way of performing bad listening and disrespecting young people's agency is by not sufficiently paying attention to them when they are speaking. Sky talked me through what that might look like in practice. She described a time where she was with her personal advisor and they were thinking through what she would like to do in the future:

"We were going through my pathway plan and... probably weeks after? They had written it up, they sent it back to me and there was loads of incorrect information."

Sky had left the session thinking that everything they had agreed was said and done and would need no further follow-up. Yet by the information being recorded inaccurately, she was forced to enact the labour of returning to this, telling her PA that the details recorded were wrong, and fighting to make sure the record was accurate. Because care-experienced young people work with many professionals at a time, ensuring the official record ('their file') is accurate is important to ensure everyone they are working with is on the same page. Not paying attention to what Sky wanted could have resulted in other professionals approaching her with support options she didn't need.

L.TUKZOMBIE had a more extreme experience, where not being listened to meant that inappropriate support was actually put into place:
"You feel like you're not being listened to because there's no one there to chat to, no one there to speak and open up to and actually try and get the help from. It makes me feel lonely and upset because I feel like I've done something wrong. Like it's my fault, why I'm not getting the right support. And also I've told the same people what support I need, and yet they've gone through and put the wrong support in place for different things that I didn't even ask for."

L.TUKZOMBIE had been clear with his professionals about what he needed, and why. He knew his needs well and was trying to ensure the people around him could proactively support him. Yet by professionals not paying attention to what he wanted and needed, L.TUKZOMBIE had to access support services which weren't right for him and didn't help him with the needs he knew he had. It is worth noting that it is possible that the professionals L.TUKZOMBIE was working with had an idea that he may have benefited from the support that he felt he didn't need - yet the important distinction is that they did not also make him feel listened to with his own concerns. If his workers were to put both pieces of support into place, then L.TUKZOMBIE may have been more receptive to the service he felt he didn't need, because the needs he knew he had were being met too.

\subsubsection{Extractive listening}

When listening does occur, it may not be the genuine and reciprocal listening that people want, and may instead be an extractive practice. This can mean listening up to a certain point, stopping a young person talking, or only paying attention the bits of what they say that the professional feels matters - again highlighting this tension between self-identification of needs and professional practice.

Danny gave an example from a child protection meeting he was a part of, where they were reviewing his current care arrangements:
"During this meeting, I was trying to say how I felt about everything, and the senior social worker at the time, y'know, just wasn't that interested. I think if I express my views there's a different direction that should have happened. I think it's important that young people are listened to... social workers and that only come in for fifty minutes and they think they know everything."
Danny felt disregarded by the senior social worker because they acted like they didn't care about his views, even though he was the one most significantly affected by any outcome from the meeting. For Danny, this was an essential moment, too, that he felt may have changed the direction of his care journey if he had been listened to. By not engaging in meaningful listening practices, workers run the risk of enacting harms upon young people or disengaging them from the youth or social work practice they are engaging in. Danny highlights how the limited contact young people and workers get means that it makes little sense for a worker to assume they know best: you cannot speak to someone for 'fifty minutes' and then 'know everything'.

Ricardo had experienced something similar to this with previous social workers, who seemed to come in, get what they want and then leave:

"When I've had social workers in the past, when they've come in and out... they like to just tick boxes. They ask you certain questions - and you start giving a detailed description and then they'll cut you off! They're like 'okay, that's fine'. They've got what they've needed to gather from you, and then they move on... they just take what they need to sign their paperwork sometimes."
Ricardo highlights how the professionals he has worked with are not engaged in a holistic listening practice - they may be listening for specific events, words, or safeguarding concerns. He ties this to their 'paperwork' - and suggests that they are visiting just so they can get enough information to 'sign' it. Though Ricardo was not specific about what he meant in that instance, later that day he learned how large the caseload of personal advisors and social workers may sometimes be. He reflected on his earlier thoughts, explaining:

“I didn’t realise just how many people a personal advisor has to look after. It’s like, 25, 30 people. You’ve only got so much time… and all the paperwork that comes with it as well. We’d like to think sitting down with young people is more like an emotional thing, but there’s so much paperwork in it… PAs and social workers can’t do everything."

Ricardo highlights how extractive listening practices might be motivated by a high caseload of young people and the bureaucracy associated with this. In Ricardo's construction, paperwork and a high caseload seem to get in the way of the more 'emotional' parts of youth and social work.

\subsubsection{Following through}
It isn't enough to just make someone feel like you have listened to them - it is also important to show that you have listened by further action. Just like a good apology is supported by future actions that are congruent , it can't be considered good listening unless you follow through. Not actioning on a given piece of listening is just as bad as not listening in the first place, because it feels like you haven't valued what you heard - or that you didn't really listen to what was implied by what was said.
Alucard felt like his Leaving Care Service never followed through on what he said:

"Lots of the time when I've been at the Sidings [I haven't felt listened to]. It's like, listen to me, hear what I'm saying rather than just say 'yeah, wait for your worker to come in'. I went down to speak to my worker about the stuff for my flat [that I had just moved into] and she weren't there, so I spoke to Duty (the staff member on duty) and then I spoke to my worker recently and she had no idea I'd been down there three times".

Being told to 'wait for your worker' or failing to pass on messages can be hugely damaging - it can cause important information to get lost and weakens the already-frail bonds of trust that many care-experienced young people have with their services. In the case of Alucard, this was particularly frustrating as he had been getting annoyed with his worker as he thought she was the one who hadn't progressed the issues he was facing with the furniture in his flat. Poor flows of communication and action through support services can lead to a feeling of resentment which might be unearned.

The experience of information 'getting lost' was familiar to Ricardo. He explained that even when information has been accurately recorded, and it had been a positive listening experience, sometimes information disappears:

"You can mention something to a PA, or a social worker, and it goes to someone above, and then it's 'well let's sit back and wait a few weeks, maybe a month, we'll see what's happening'. And before you know it gets lost somewhere, or you forget."

In Ricardo's experience, his PA and social worker was not who he had an issue with - rather, the system behind that worker would cause his issue to get lost. He gave an example at a later date where he was trying to do some research into some legal entitlements that he had as someone who had left the care of his local authority. His PA was supportive of him finding out more about this, as she felt he deserved all of the support that he was legally entitled to, and was heartened by his proactiveness in seeking this out. Yet once she got back to the office and started asking around about how he might be able to access this entitlement, things seemed to fall into an administrative black hole - different pressures took precedence, and he was unable to find out about that service. By not following through on the results of listening - or not being able to follow through on the results of listening - young people are left in the same material position as not having been listened to at all.

\subsubsection{Powerlessness and trust}
Not being listened to by people whose job is ostensibly to support them can make care-experienced young people disengage from pieces of youth and social work that may be of material benefit to them. Imagine you are a care-experienced young person attempting to access some mental health support that you know you need and that you have been told exists in your local area. You manage to pluck up some considerable courage to ask for some help in accessing this service, but you find that your worker isn't interested. Either they note down the generic 'needs help with mental health' in their notes, or they have preconceived notions about the support that you do need, and so disregard your in-depth research you have already done. You know you need help with this specific issue and it is this person's specialty; yet your worker had already decided you could benefit from a course of cognitive behavioural therapy and started putting the wheels in motion last week.

Or suppose you get one of the 'good ones', willing to fight for you and put their job at risk because they have seen just how passionate you are about this and admire the effort you have put in so far. You leave your session with them feeling pretty content that things are going to go well. The next day, you get an email:

"Hi, I'm really sorry to have to say this but my manager said that therapist is oversubscribed right now and we can't put you on the waiting list until we've put you through some CBT. Is that okay? I know it's not what you wanted but it's all we can do right now."

You feel left down. Dejected. Disappointed. You put in all that effort, and for what? They don't have it on your file, but you've already been through CBT once and you just didn't find it useful. You appreciated some of the strategies but you feel like you need a more open space for you to talk about the struggles you're having at the moment.

These profound examples of not being listened to display a lack of care for the agency of care-experienced young people. Trying to get your voice heard in the first place can takes a huge amount of labour and time. Repeated experiences of not being listened to - through a lack of attention, extractive listening practices or by not being able to action things - can devalue the act of 'speaking up' in the first place. If you know you will be met by apathy, ignorance or resistance when you try to speak up about your needs, you are much less likely to speak up in the first place.

This can make care-experienced young people feel extremely powerless in a system which already works to limit their power or which may not recognise the civic contributions of children and young people (cite some CG lit). This powerlessness can make young people feel as if it is impossible to change their personal, social or political realities. In the context of austerity, this is even more concerning. As mentioned in chapter 2, Harrison's theory of the public response to austerity (2020) argues that those most affected by the lived realities of austerity lack the resources to take action on austerity. The experiences of care-experienced young people in not being listened to add another dynamic to this - learning that things won't change when you try to change them stops you trying to change them.

Many of the care-experienced young people who have taken part in this research have shown this feeling of powerlessness. At a Corporate Parenting Board I attended with Small Steps where care-experienced young people were leading the meeting, a young person was excited to talk about some of the issues they had been facing recently because they wanted to get the knowledge into the hands of "someone who can actually make change". The idea of 'someone else' being the person who can make change became a source of tension with one Bridge in Building Bridges' first year, as their group mission focused on trying to improve the quality of accommodation provided to care leavers. After Emrys spoke passionately about how this could improve both her and Symone's life as they would benefit from doing it, Symone became visibly angry. "There's just no point though, is there. 'Cause nothing's gonna change, so why we even bothering trying to make it happen?"

At its core, this powerlessness is related to the lack of trust that has been constant throughout this chapter - in other young people, workers, decision-makers, services, and the government. If you learn that you can't change things when they are bad, your expectations start from a much lower basis. Why trust a service that supports you? They're not going to listen anyway, so why spend any of your effort on trying to engage? It's easier to sit comfortably inside yourself. Powerlessness is sustained through a lack of trust, then, and workers often do little to warrant any change in this relationship. Young people who are perceived to be vulnerable have told me on multiple different occasions just how little they trust relationships with others. One young person at ACT told me they couldn't depend on workers for anything, because "if we depend on them now, what will they do when they're not there?" Young people who accessed Changemakers echoed this at a more personal scale, explaining how they felt more comfortable when they had less friends, because it meant they didn't have to depend on anyone. George, a young man who had recently experienced homelessness, put it most eloquently - "good friends are hard to come by".

These experiences are written into the fabric of precarious youth experiences. Sean was in residential care when he was younger, and when we first met him, spent most of his time withdrawn, high, hiding under the hood of his hoody. On our second residential, he had started to trust that we would be there for a longer time and were genuinely interested in what he had to say. That night, he explained to us how different this was for him than his previous experiences:

"When you first go into a care home - like a residential care home - you're followed everywhere. There's no trust. The doors are locked, things are locked. You don't need to be that untrusting. It's like because I'm in care you think I'm gonna rob everything. If you don't trust me like that, why the hell should I trust you? Sure, some people try to take advantage, but workers think everyone's trying to take the piss."

By not listening to the experiences of care-experienced young people, we cement a relationship of powerlessness and trust that is difficult to change.

\subsection{"Trapped in a spiral circle": not getting the right support}
As L.TUKZOMBIE has shown us, not listening to young people can result in delivering inappropriate services. In the worst case, this may result in the provision of services that are entirely inappropriate for a specific young person. The best case scenario may simply be a relevant service that a young person is not yet ready for. In either case, care-experienced young people fail to receive services that they need, when they need them, because professionals in their lives are not listening to them and respecting their needs. In the previous section, we saw how a misalignment between what care-experienced young people want and need can produce a dissonance between care practice and lived experience which inflates a sense of powerlessness and breaks down relations of trust. In this section, I turn to the specifics of what being unable to access appropriate services does to young people's lives.

Many young people made an equivalence between 'the right support' and the support that they had identified as being appropriate for them. For a given care-experienced young person, not getting 'the right support' can thus be equated with having support enforced upon them or being unable to access support they know will help them. In the first instance, tensions reign between what a worker feels may help a young person and what the young person feels will help them. In the second instance, young people are met with an insurmountable blockade which prevents them accessing support. Most frequently, this is due to sudden nadirs in support provision, where a life transition (e.g. from placement to placement, from care to leaving care) prevents access to support. In other cases, this may be due to exceptionally limited criteria for support, where only moments of crisis are enough to gain access to support. What matters is not necessarily what happens, but the impression that a young person is left with from that experience. What is the damage of not being able to access 'the right support'?

\subsubsection{Enforced support/getting the wrong support}
If a young person isn't being listened to, decisions get made about their lives by others. This might be birth parents, foster carers, child protection workers, social workers, adopted parents, personal advisors or some other 'adult' deemed to have responsibility for them. Disrespecting a young person's agency in this way can lead to these decisions being sharply out of line of both what will support that young person and what support they will engage with. As such, these decisions may be actively harmful for them. Young people see the gap between what they want and need and what workers think they want and need, and are sharply aware that things would go better for many young people if they were listened to and respected. Mary explained that the abstracted form of this tended to look like:

"'You're a child and we're going to make that decision for you, because we know best'. But if they took more time to sit back and listen... a lot of placements wouldn't break down".

Mary makes the link between forced support and placement breakdown. Repeated placement breakdown was spoken about by many young people as a source of disappointment and ill-feeling. Danny explained how damaging this could be:

"Sometimes carers don't want them or it doesn't work out... [that makes life] really unstable for [young people], and they're gonna get used to that... and they're going to get into trouble because they're not settled and not getting the love and support they need".

Danny explicitly talks about placement breakdown as being 'not wanted' or things 'not working out', and highlights how the instability this creates can stop care-experienced young people getting meaningful support. Though I will return to the difficulties that are created by inconsistent and unstable care journeys shortly, it is worth noting here that young people draw direct lines between not being listened, placement breakdown, and getting the wrong support.

In this same discussion, Danny detailed why it was important for young people to be at the forefront of shaping their own care, by explaining that he knew he was getting the wrong support when he has:

"Lost control. If it becomes restrictive... then it's not support. It's just being told what to do when to do it. It's not actually support because support is helping you to do something."

Support has to help you do something; therefore, 'the right support' can never be enforced on someone because then they are not being supported. Putting support in place that is restrictive and which doesn't facilitate young people to take control of their care is as good as enacting harm, as it neglects to provide support and expends effort trying to center someone else's power. One might think of how Celen wasn't given an opportunity to say "I don't want to do my life story work now", or how Symone wasn't able to tell her local authority that the accommodation they were trying to place her in wasn't suited to her needs. Their power was devalued, their voices were sidelined.

\subsubsection{Not getting support}0
If people around aren't listening to you, but also aren't putting in some 'support' of their own, then what happens? In most cases, this means simply not receiving support. There has been a marked rise in this due to the reduction in support capacity and provision since the advent of austerity. Although there are many reasons someone may not receive support - which may tie into the lack of trust and sense of powerlessness referred to in the previous section - there are two main reasons someone may not receive support that they may want to access. The first is the other side of what Danny referred to as 'restrictive' support - support services with very tightly defined access criteria. The second is the sudden removal of existing support at points of transition during a care-experienced young person's life.

The clearest example of these limited access criteria is that of Child and Adolescent Mental Health Services (CAMHS). DeNiro explained that in his local area:

"It's awful. It's really bad. You basically need to be suicidal to be referred to CAMHS."

This was a sentiment that was echoed by most young people. The resounding view was that CAMHS was overrun and had no capacity to deal with the cases on its waiting list. This view was so common that I heard it from young people and workers alike. A worker for a local authority project explained the size of the problem to me starkly - even if CAMHS had enough workers to deal with the size of its waiting list, the scale of the mental health crisis would mean they would need more workers by the end of the day. As such, the criteria for accessing CAMHS has become more restrictive, and the approaches taken by professionals more solutionist.

Cognitive behavioural therapy is seen as the catch-all response to mental health issues (ref) despite its limited efficacy for many different types of people. Because of the increased possibility of a history of trauma, care-experienced young people may more easily fall into the category of 'not really getting that much from CBT'. Specialist mental health services are required, but capacity is so greatly reduced that there seems to be no chance of this happening soon. Quinn felt this often led to young people getting the wrong support because when someone reaches out for support with their mental health, they are not often met with a positive response. In his case, it took 18 months to get into CAMHS and when he did, he found himself met by a mental illness diagnosis rather than support. He emphasized that it was so important:

"Not to diagnose straight away because that's a brand for life. Just tell me, 'you have traits of this, and this is how we'll help you'."

Because of this lack of capacity, mental health professionals jump to diagnosis of mental health conditions or require mental health crises to become involved in the first place. Choosing to diagnose a young person with a mental illness before helping them to learn more about symptom clusters they experience shows one aspect of what this increased pressure is doing to the (mental) healthcare system. This mentality wraps back round to social welfare provision too. When Colin left Changemakers and went on to work for the local authority, he explained to me that in his new role as a personal advisor, he was being told off for "spending too much time with young people". He painted a vivid picture:

"It's just all crisis management. I've got no opportunity to use any of me social work skills to help actively develop them or anything. It's just 'can you buy me this', 'I've got no heating', 'I've been feeling shit lately'. I want to do some proper social work or youth work with them but I'm not allowed."

As frontline workers become crisis managers, they become unable to engage in a relational youth or social work practice. Instead, they become the agents of procedure, becoming like the social worker that Ricardo mentioned who just wants to get their 'paperwork signed'. This is consistent across a number of social welfare services that care-experienced young people may access. One only needs to think of the harsh sanctions of the DWP if people fail to jump through the increasingly complex hoops required to maintain a Universal Credit application (ref).

Restrictive support is paralleled by sudden losses of support. Moments of transition can be hard for anyone because they represent moments of significant change, often in ways that alter our relationship with or orientation towards the world. These transition periods have taken on a greater significance in the context of austerity because they have become more plentiful: higher workloads, reduced capacity, and poor working conditions mean that social workers and other professionals in the care system are leaving their roles with greater frequency. In the session at the beginning of the first residential of Building Bridges, Quinn and Cameron told us that they knew someone who had had fourteen social workers in a single year. Cameron explained that this meant young people had to continuously rebuild relationships of trust. This also applies in the context of continuously moving care placements: new placements mean new relationships to build and the loss of a previously supportive - or at least familiar - environments.

In situations like this, support can quickly disappear, just when new support should be put in place to help deal with these harder moments. Mary elucidated what this might look like in practice with what she called 'the suitcase scenario': a young person is told that they have to move placements with very short notice, and instead of packing their belongings into a suitcase, they are made to pack all of their worldly possessions into a series of binbags. The founder of the backpack company Madlug, Dave Linton, has been vocal about how this state of affairs can have significant impacts on a young person's sense of value, worth and dignity (ref). 'The suitcase scenario' as Mary paints it is a clear example of how the care system fails to support its young people at some of its most important moments.

The 2017 Children and Social Work Act introduced the provision of statutory support for young people who have left care up until the age of 25 (previously, the 'cut-off' age was 21). Though this is better acknowledges the all-encompassing nature of the care-experience by acknowledging statutory responsibilities beyond the care-experience and having just left care, it still does not even begin to acknowledge the way that care-experience affects an individual's entire life. The first pledge of the 2019 Care-Experienced Conference (a gathering of 141 care-experienced people from across the UK) emphasises that "the care experience is a continuous life time process and does not simply cease at 16, 18 or 25 years of age". When policy and practice treats it as if it does, care-experienced people suffer. Mary was adamant with me that arbitrary age cut-offs for support made no sense and disrespected people's needs:

"Sure, the cut off point for support has been moved to 25. But it's not enough, 'cause if you're a parent, just 'cause your child turns 21, doesn't mean you're like 'move out'. You're there for 'em."

Making arbitrary transition points both distinguishes care-experienced people from their non care-experienced counterparts, and can make a care-experienced person feel as if the state wishes to absolve themselves of their responsibility. If something happens later in the life of a child raised with its birth family, they can go back to their birth parents and ask questions, seek clarification, or receive support. By enforcing any age-based cut-off, the state attempts to position its responsibilities as completed. Emrys reflected on Mary's discussion and explained that she felt that would be important, going forward - she was glad that the age of support had been raised to 25, but acknowledged that:

"Some people do need more help than others. You need to be able to say you don't need the help right now - but further down the line, to still ask for and receive that help. Just cause you're 25 doesn't mean you don't need it."

In the context of frequent transition, all of the young people I spoke with were clear that 'consistency is key' to having a positive experience in the care system. Most notably, workers and placements need to be consistent, as they provide the backdrop to much of a young person’s experience. Cameron and Mary both echoed the importance of having a social worker be the "one person" who could be "the most consistent thing" in a care-experienced young person's life. Perhaps some of the reasons that so many care-experienced young people feel that "all social workers are shit" is because so often, this is not the case. When workers and placements change frequently, a person’s sense of stability begins to erode. The withdrawal of support in this already-unstable environment (being taken into care, new placements, new workers, 'leaving care', reaching 25) devalues the experiences of care-experienced people, and forces them to expend a great deal of emotional labour in order to build new bonds of trust. Constantly having to retell your story - to justify who you are, how you like to interact with people, what you want to do in life - is draining, and eventually, a young person may cease to want to, withdrawing from building those new relationships because of the feeling that it will all disappear anyway.

\subsection{The impressions}
We have seen that inappropriate support is frequently put into place in the care system; but what do these experiences of ill-suited support leave care-experienced young people with? What does it teach them about care, and how does it disrupt or otherwise alter their conception of what the world is and how it works? In other words, what are the impressions they leave upon care-experienced young people? Just as not being listened to negatively impacts a person's ability to feel powerful and to engage in trusting relationships, not getting the right support has profound negative impacts and affects. [add sentence]

At the core of many of these experiences are feelings of isolation, loneliness and abandonment. Scarlett describes how it feels like the whole world is on her shoulders - it is a huge weight on her that she has to deal with on their own. In this isolation, any problems that come her way can be dealt with by her and her only. She links this to the abandonment that some may feel by virtue of being in the care system - explaining how when someone says they will be there for you and they are not, that it's like "a sense of abandonment, all over again". Danny made this same comparison, explaining that it feels like you are "let down again, and you're constantly let down". Billy drew a connection to how this can make people "stop still"; if you don't know people who can help you, or if you "don't have the connections you need to move on to planning your goal", then you end up in a state of arrested development. Having inappropriate support can make care-experienced young people feel as if they are on their own, and, as Scarlett mentions, it can make people feel as if they are not worthy of that support in the first place. This unworthiness and isolation can take its toll and may compound some young people's trauma, reminding them of feelings of neglect that may have been significant for them.

A sense of disorientation and confusion pervades many of these descriptions of inappropriate support. Scarlett feels "trapped in a spiral circle"; Mary doesn't know whether she's "coming or going"; Sky is "constantly on edge"; a young person from Changemakers felt as if he was "'gan round in circles" every time he tried to access support. All of these images portray a sense of paradox, inactivity, and anxiety. In Scarlett's spiral circle, no one will "put out their hand" to meaningfully help - instead approaching with empty platitudes such as saying "things will be better". Actions speak louder than words. When Danny can't access support, he feels "lost" in a system. Mary not knowing whether she's coming or going shows how being unable to get appropriate support may involve a degree of constraint - being shoved in one direction, and then the other, with no definite way out. There is an anxiousness which pervades these states of being - a sense that though things are confusing or paused right now, that there is an oncoming crisis which might disrupt the balance of the "chaos" and make everything tumble down. Not getting the right support necessitates confusion and anxiety. (maybe add BwO or Cruel Optimism).

All of these experiences are punctuated by negative affects. Billy calls attention to the embodied nature of this, highlighting how it can "break people down". Jill feels frustrated; Mary feels angry and depressed; L.TUKZOMBIE experiences pain and is scared. The idea of being broken down is visceral and makes clear how this feels - like the parts of a person being de-composed, broken apart into fragments. Viewed in this light, it is easy to understand how this could "put you in an even worse position than you started off", as L.TUKZOMBIE said. Being unable to access appropriate support can make you feel worse about yourself, and as I already mentioned, this can affect young people's feelings of self-worth. If you keep being unable to access appropriate services, you might start to assume it's your fault. I saw this time and again in how the care-experienced young people who were a part of Building Bridges tried to minimize themselves and their experiences, or talk about themselves negatively: L.TUKZOMBIE said that he was the only nasty thing in the woods; Cameron was constantly worried about annoying the others (whilst also fighting to be the loudest in the room); Jill felt like she was "too fat" to complete a physical task; (probably discuss B Brown here).

The experiences of care-experienced young people in not being listened to and being unable to access 'the right support' has had vast negative impacts. Care-experienced people may struggle to trust others (and themselves). They might not recognise or be able to use their own power, or feel powerless at the hands of others. They may feel a sense of isolation, abandonment or unworthiness. They might feel like they have been 'stopped still' whilst being disoriented or anxious. They may feel a great deal of hurt and pain, particularly in relation to reaching out for help. Perhaps most insidious is contained within Danny's mention of "feeling like a tickbox", not yet mentioned in this analysis. What does it mean to 'feel like a tickbox'? Unlike the multitude of emotions and experiences that people can have, tick-boxes have a binary state - they are either ticked or not. If the people whose job it is to provide care and support services for young people are making them feel like 'tickboxes', then care work has been undermined, and bureaucratic functions which should aid or support care workers have taken over.

As mentioned at the beginning of this chapter, the affects, experiences and practices which circulate in the care system do so fractally: they can be found at every level manifesting in only subtly different ways. Managers and workers are unable to build trusting relationships, whilst workers and young people struggle in the same way. Managers use techniques of control to make their workers behave a certain way, whilst workers use similar techniques on young people. Young people feel isolated, disoriented, 'broken down' and powerless; workers feel individualised, anxious, in 'panic mode' and unable to leave their jobs; managers take work onto themselves because they feel they cannot trust their workers, placing increasing pressure on those working for them and becoming disconnected from the good intentions that made them start their project in the first place. As such, we must ask ourselves - what is it that is affecting the care system so profoundly as to alter its entire functioning, at the affective, experiential and structural levels? To understand this we must return to the changes made to charity funding in 2011 and take a closer look at evaluation processes and the role they play in youth and social work.
% +Discussion-y bit which draws them together?

