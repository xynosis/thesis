\chapter{fractured signals: }
\label{}

\section{Towards fractured signals}
\label{sec:}

As the kinds of speculative methods I am developing here show, however, there is always a possibility of something else happening. In November 2020, the funding pot that was originally being used to print the It’s Our Future manifesto became available again. Because of the disruptions that had occurred across the year due to the measures taken to protect against the dangers of COVID-19,  there was some excess funding available for further work on projects. Through this, I was able to work on the project that became fractured signals, and finally get to refine the methods developed within It’s Our Future and Design Strategies against Justification Practices. Our starting point for fractured signals was much simpler than the starting point for It’s Our Future – quite simply, our brief this time was ‘how can we use the methods developed in It’s Our Future, iterating on them so that we learn from what did and didn’t work last time, and adapt them to the new context we find ourselves in, in which the majority of life is undertaken digitally and remotely?’

With a much smaller team – just myself and artist and digital content creator Daniel Parry – we set about reviewing the data from It’s Our Future with a critical eye on what worked, what didn’t work how we wanted it to, and what wouldn’t make sense or be possible within the context of the current COVID-19 restrictions.  We identified some key design considerations. First, the project would have to ensure that it had robust data collection mechanisms, as although we had designed the cards for this purpose in It’s Our Future, it was hard to understand them without any of the incredibly rich context that was present within discussions. Secondly, we planned to go further with some of our design concepts than we were able to when we were working with The Charity as a collaborating partner. Finally, the methods used would learn from the shortcomings of the methods developed within It’s Our Future and Design Strategies against Justification Practices and would have to be something that people could embed into their daily practice if they wanted – not just a one-off event. 

At this early stage, we were fairly certain that we would be running some kind of virtual event akin to It’s Our Future, utilising some of the abandoned plans for the digital version of the engagement. In particular, we hoped that we would be able to pick back up on the two-phase interaction and infrastructuring across multiple social media platforms onto a central organising platform. Instead of having it be a single event, though, we felt that it would be most appropriate to spread the process across two weeks, in order to give people more time to get more comfortable with the process and reflect more deeply at every stage. Throughout It’s Our Future we felt that we had to rush the process, not giving participants enough time to reflect on their ideas and meaningfully connect them. Expanding the time that the process took was a key factor for us in the design of fractured signals. 

We wanted to build upon the interaction set that we had developed in It’s Our Future, and decided that the tarot card design was useful as it spoke in a familiar visual language of the future. As explained earlier, though, the well-designed nature of the cards made some participants reluctant to write on them. We decided to explore this urge by giving participants of fractured signals the opposite experience – cards that were designed to feel special, as if they deserved recognition and care, and which could not be written on. The goal of using these cards would still be the same, however – to create spaces for ambiguous and fragmentary reflection. As such, we felt it would be most appropriate to create cards which represented the major arcana of a regular tarot deck. The major arcana, often referred to as ‘the fool’s journey’, refers to a set of 22 cards which comprise some essential aspects of human experience and which are thought of in different settings as referring to the experiences humans must go through in order to achieve enlightenment or meaningful self-knowledge. 

Inspired by other decks of tarot cards, we knew it was important the visual style used here was ambiguous yet somehow familiar – containing enough elements to give people something to begin interpretation from, but few enough elements to retain ambiguity. We also wanted to ensure that the cards we created contained threads to the research from It’s Our Future, as creating ambiguous understandings of research that had already happened was a key intention behind the project. To do this, we identified 22 substantive themes within the It’s Our Future research, drawing from the idea cards and manifesto primarily, and we mapped these against the 22 cards of the major arcana. We researched their typical meanings, and paired themes from the research with meanings of cards in the major arcana to give us a new deck of tarot cards which occupied the interpretative space of both things (detailed in table x).

% add table

 
In order to arrive at the visual style for the cards used in fractured signals, Daniel and I held a design session in which we explored traditional tarot iconography around each of the cards of the major arcana, and explored how these were expressed in different decks of tarot cards. We then thought about how the semantic field or imagery associated with the substantive area from the It’s Our Future research. For example, the card ‘The Explorer’ corresponds to ‘The World’ in the major arcana of the tarot. ’The World’ card is typically associated with completion, integration, accomplishment and travel. It is the completion of the Fool’s Journey -  it represents finishing what was started in ‘The Fool’ card. As ‘The Fool’ is a card of new beginnings, though, there is a sense that the cards are caught in a cycle of eternal recurrence, where The Fool will ascend to the level of self-knowledge requisite in The World, only to begin again, starting anew. For us, The Fool became ‘The Fresh Start’, a card associated with moving away, being somewhere different, a chance to find yourself in a new place. ‘The Explorer’ on the other hand depicts the same character, subtly changed, embedded in a series of portraits at themselves. The card hints at the movement through time, the beginning of another cycle – but as an explorer, someone who goes to new places, demonstrates a level of commitment and confidence to experiencing the new. The Explorer is The Fresh Start with all of the knowledge of the Fool’s journey. 

Figure x. Design day for the major arcana.

By adapting the cards in this way, we were able to craft a visual language that associated several unalike ideas and made them tangible together. ‘The Cyborg’, for example, is typically ‘The Emperor’, a card about paternalism, authority and the establishment. This level of control spoke to the ways that young people who attended It’s Our Future felt about technology – that it was something removing their control and making them feel as if they had no agency to control or change. In making ‘The Emperor’ into ‘The Cyborg’, we begin to suggest that perhaps technology has become our new rulers, or that the establishment has simply taken on the clothes and framing of the technological in order to find new ways to enact their control. This expanded visual semantic field enhanced the range of meanings of the card and enabled us to create yet more ambiguity within these as resources for reflection. The fractured signals cards ensure that the meaning of the tarot is comprehensible within the contemporary context of the experiences and wants of young people. 

In order to address our needs for robust data collection mechanisms, we began to think about other tools and technologies which contain or participate in a visual language of future-forecasting. Our initial ideas for this gave us a huge list of artefacts, tools and technologies which people use to tell the future or reflect on the present, which we categorised by the way that they operate (depicted in table x). As I noted in the previous chapter on the use of Gabber, the best kinds of data collection mechanisms are also tools for self-reflection, so we felt it appropriate that we potentially think about our primary data collection device as a scrying tool. Inspired by Gabber’s voice recording capabilities, we begin to think about the design of a scrying tool that could be used to record reflections on experiences that occurred through the research process and which could be sent via the internet to us as the research team, but that marked an important reflection experience for the participant.

Table x. A typology of methods of foretelling or understanding the future. (add in fortune telling/seers, dowsing, runes)

As we began to design the scrying tool, however, the policy context in which we were operating massively changed. On [date], the UK government announced an independent review of children’s social care in England. The ‘Care Review’ (as it is colloquially known) would see [insert details]. All of a sudden, a research project designed to encourage care-experienced young people to reflect on their experiences, imagine better futures and gain skills that would help them to make those futures material felt like either the best thing or the worst thing. On the one hand, it could be brilliantly timed – a well-needed amplification tool that supports care-experienced young people through what would be a very difficult time for them, particularly with the amount of public focus there might be on potentially traumatising experiences. On the other hand, though, participating in my research – merely a well-meaning ally – could be a whole host of extra trauma and too much of an ask at this time. After consulting with friends who had formed the Reclaim Care collective, I decided to change the focus of the research. Together, we felt that it was potentially too much to ask of people at the beginning of the Care Review when that dominated discussion. Instead, what might be useful was a project which attempted to make workers – who will be enlisted as gatekeepers throughout the process of the Care Review – reflect on their participation, co-production and allyship practices. This gave fractured signals new life and purpose as it could close the loop of my research and adequately address the experiences of the youth and social workers in the care system whose experiences underpin the very idea of justification practices.  

Within these new plans, then, the primary goal became finding ways of using the methods and design strategies I had developed throughout It’s Our Future and Design Strategies against Justification Practices to make workers reconsider their practices within how they center the experiences of care-experienced young people. It became clear to me soon after the Care Review was announced that this was going to be something that happened increasingly often over the next year as the Review takes place, and it would be up to workers to find the right ways to center the experiences of young people where appropriate, and equally up to them to protect them from waves of institutional bureaucracy at other times. fractured signals thus became a tool to make workers reconsider their practice at a time when it was more important than ever that they got things right. 

Developing from the four activities within It’s Our Future and the eight strategies and four principles in Design Strategies against Justification Practices, fractured signals focused on eighteen critical questions and instructions which structured the design process. These questions and instructions are presented within table x, alongside a comparison to the activities, strategies and principles they developed from. 

% another table

Having identified the discrete methods and activities we were hoping to weave throughout the research process of fractured signals, we set about thinking about how to enact these within the context of a research project that had to take place in a digitally-driven, decentralized way and which had to conform to any and all coronavirus safety restrictions in place at the time. As we were running the project just after the UK had entered its third coronavirus-related lockdown, we were planning the project to make no use of in-person group interactions. Our activities, then, would be some combination of digital or physical; online or offline; individual or group; and synchronous or asynchronous. We began developing our activities with these four modalities in mind. For example, for the  ‘connecting and building trust within the group’ activity, we envisioned a synchronous, digital, online group – something like the Zoom workshops that have taken place so frequently across the past year. On the other hand,  for the ‘ensuring the process creates spaces for self-reflection and situated sensemaking’ activity, we envisioned an asynchronous, physical, online individual interaction, where individuals would interact with the reflection device physically on their own, but where their reflections would be digitally recorded and sent to a digital platform where they could review their recordings. 

\section{fractured signals: a guide for new futureweavers}

The final fractured signals project was composed of a set of artefacts – the ‘dreamthreads’ (the tarot cards), a ‘divining board’ (which introduced  participants to some basic card spreads of the tarot), a ‘signalfinder’ (which acted as both a box for the cards and a passive amplifier for participants’ phones, and two documents – the It’s Our Future manifesto and the ‘guide for new futureweavers’ which was designed to introduce them to the world of fractured signals and their newfound powers (the artefacts are all pictured together in figure x).   Participants would interact with the artefacts by calling a phone number and being delivered instructions by a cryptic voice at the end of the phone. They would call the number every working day for two weeks, and be greeted by a member of fractured signals, a far-future, multi-timeline organisation who was reaching back in time to individuals with the power to ‘futureweave’ to help them learn to harness their powers and build new and better worlds for the young people they work with. 

Figure x. The fractured signals artefacts. From left to right: 

\section{Setting the stage: the DOCUMENT and the introduction letter}

Before the research process started, each participant was posted a letter from fractured signals, with ‘the DOCUMENT’ enclosed (pictured in figure x). The letter explained that ‘FRACTURED SIGNALS’ had detected a ‘TEMPORAL ANOMALY’ in the reader’s timeline that was in some way tied to the document – which was the It’s Our Future manifesto. The letter explained that ‘something or someone has taken you off of your track’, though left it deliberately ambiguous as to what that may have been. In reality, this could easily have referred to the lack of follow up engagement to It’s Our Future, or the COVID-19 pandemic and the measures taken to keep people safe – but by keeping it ambiguous, we hoped to let participants’ imagination fill in the gaps. In doing so, we hoped that they would begin to construct a narrative of why these futures failed to arrive, and the narratives participants constructed would blur the boundaries of fiction and reality and thus create a new sense of possibility. 

% image of letter

\subsection{A Guide for New Futureweavers}

A week after the letter arrived with participants, the rest of the artefacts were delivered to them. A Guide for New Futureweavers structured the artefacts and helped participants to gain a sense of what each of them were and how to use them. The Guide began with a description of the multiple futures that fractured signals exists across, before detailing who or what fractured signals actually was, and what futureweaving is. Then, the Guide described each of the artefacts in depth – giving potential meanings for each of the dreamthreads, showing how to use the divining board, and how to operate the signalfinder. 

The opening pages describe the deeply contingent futures that fractured signals exist across (shown in figure x).  My intent here was to begin troubling the notion of time and linearity itself, and start creating a sense of possibility, building on the idea that the singular, capitalist realist notion of the future is just one of many possible futures. In doing so, I worked on the strategies of ‘imagining new or different worlds and truly inhabiting them’, ‘making our current world seem as strange as it is’, and ‘making different future worlds seem possible’.  After this, the Guide introduced fractured signals as a far-future multi-timeline organization that may or may not exist (shown and described in figure x).

Figure x.  The introductory pages to A Guide for New Futureweavers, which talked participants through a different idea of what the future could be. 

Figure x. The pages of A Guide for New Futureweavers introducing fractured signals.

As an organization, fractured signals deliberately invoked its own fictionality, calling participants to consider the gap between reality, fiction and possibility.  [I should flesh this out a little more]

\subsection{The dreamthreads}

Figure x. Each of the dreamthreads. 

The dreamthreads are the aspect of the project that changed the least from start to finish. They comprised a set of twenty two cards (pictured in figure x) that were to be used in a multitude of different ways throughout the project. The simplest way is reflecting on them on their own. As each of them represented a fragment of meaning from It’s Our Future (or a thread of a dream, hence the name), they could be reflected on in isolation, and participants could think about what the card represented to them and what potential meanings it might have. They could also be used in a problem-led sense, where if participants were having trouble with an issue in their work, they would be able to draw a card at random and reflect on the advice it suggested to them.
	The dreamthreads could also be used in combination with the divining board, which suggests some potential contexts for each of the cards to be considered within (pictured in figure x). These were adapted from common tarot spreads, which give a logic and consistency to how randomly chosen tarot cards can be made to make sense. Examples include ‘past, present, future’, and ‘context, problem, lesson’. Each card pulled in order would give guidance about what the querent might need to do. 

Do I detail the fact that these work because of the nature of self reflection here??

Figure x. The divining board.

\subsection{The signalfinder}

The last artefact delivered to participants was the signalfinder, a small wooden box that was designed both to house the cards and act as a passive speaker for participants’ phones when they called the phone number they were given to receive guidance from fractured signals (pictured in figure x). The box went through a number of designs before landing on this final design, mostly requiring adjustments because of issues with the living hinge around the side. In its final design, the dreamthreads sit comfortably inside of the box, and most sizes of modern mobile phone rest easily on the phone stand on the top of the box. Assuming that the phone’s speakers are on the base (as most are), the spaces in the living hinge at the front create a small degree of amplification for sound coming out of the phone’s speakers. Whilst this was one of our central aims for the design of the signalfinder, the amplification is very small and not likely to make a huge difference for participants. 

The key aspect of the signalfinder for the fractured signals project is the way it helps to build the world of the story. In the realms of the design fiction, the signalfinder is the object that the fractured signals organization need the participants to use in order to locate them through the different possible timestreams. In the narrative, the signalfinder amplifies ‘resonant frequencies’ through the timestreams and makes it easier for fractured signals to locate participants. By making the signalfinder into a passive speaker, we played with this idea of resonance and tried to connect the idea of temporal resonance to audial resonance. Mostly, we hoped that the ritual of taking the cards out of the signalfinder, setting up the stand, placing the phone on top of the stand and calling fractured signals would help participants to feel like important parts of the world of fractured signals. 

Figure x. The signalfinder.

\subsection{Twilio-based phone calls}
The majority of participant interaction with the project consisted of using different artefacts, guided by the disembodied polyvocal voices of fractured signals through a series of phone calls. Every working day for two weeks, participants were invited to call the phone number they were given, and they would speak to members of fractured signals (produced by Amazon Polly, a text-to-speech service that is part of AWS and integrated into Twilio). The two weeks each had a different function: in the first week, the ‘fragmentation’, participants were invited to use the artefacts to consider different ways that things they consider fixed, constant, and unchanging actually contain a deep sense of contingency. In the second week, the ‘weaving’, participants were invited to envision positive futures based on their reflections in the fragmentation and their understanding of the ‘document’ (the It’s Our Future manifesto). The phone calls acted both as a mechanism of narrative development and as a method of workshop delivery, instructing participants to perform certain actions and reflect (aloud, if possible). Each day focused on a different aspect of this, detailed in table x.  
In order to run the voice-based interactions, I wrote some code to interface with Twilio that would read different lines of a predetermined script based on the day of the week and record and respond to user interactions. Does this need more detail? It feels like it does… The full script can be found in appendix x.

% another table
\section{Partcipants' experiences}

\section{Sustaining a speculative praxis}